\documentclass[class=report,crop=false]{standalone}
\usepackage[screen]{../exo7book}

\begin{document}

%====================================================================
\chapitre{Polynômes}
%====================================================================

\insertvideo{dDKI3jkMjfw}{partie 1. Définitions}

\insertvideo{CnMrf9aW-LU}{partie 2. Arithmétique des polynômes}

\insertvideo{qCMnvqc2t8A}{partie 3. Racine d'un polynôme, factorisation}

\insertvideo{wf-eEQPBX0Y}{partie 4. Fractions rationnelles}

\insertfiche{fic00007.pdf}{Polynômes}

\insertfiche{fic00008.pdf}{Fractions rationnelles}



%%%%%%%%%%%%%%%%%%%%%%%%%%%%%%%%%%%%%%%%%%%%%%%%%%%%%%%%%%%%%%%%
\section*{Motivation}



Les polynômes sont des objets très simples mais aux propriétés extrêmement riches.
Vous savez déjà résoudre les équations de degré $2$ : $aX^2+bX+c=0$.
Savez-vous que la résolution des équations de degré $3$,
$aX^3+bX^2+cX+d=0$, a fait l'objet de luttes acharnées dans l'Italie
du \textsc{\romannumeral 16}\textsuperscript{e} siècle ?
Un concours était organisé avec un prix pour chacune de trente équations
de degré $3$ à résoudre. Un jeune italien, Tartaglia, trouve la formule générale des solutions
et résout les trente équations en une seule nuit ! Cette méthode que Tartaglia voulait garder secrète
sera quand même publiée quelques années plus tard comme la \og méthode de Cardan\fg.

Dans ce chapitre, après quelques définitions des concepts de base,
nous allons étudier l'arithmétique des polynômes. Il y a une grande analogie
entre l'arithmétique des polynômes et celles des entiers.
On continue avec un théorème fondamental de l'algèbre :
\og Tout polynôme de degré $n$ admet $n$ racines complexes. \fg{}
On termine avec les fractions rationnelles : une fraction rationnelle
est le quotient de deux polynômes.

\medskip

\noindent{\bf Dans ce chapitre $\Kk$ désignera l'un des corps $\Qq$, $\Rr$ ou $\Cc$.}


%%%%%%%%%%%%%%%%%%%%%%%%%%%%%%%%%%%%%%%%%%%%%%%%%%%%%%%%%%%%%%%%
\section{Définitions}

%---------------------------------------------------------------
\subsection{Définitions}


\begin{definition}
Un \defi{polynôme}\index{polynome@polynôme} à coefficients dans $\Kk$
est une expression de la forme
$$P(X) = a_n X^n + a_{n-1} X^{n-1} + \cdots + a_2 X^2 + a_1 X + a_0,$$
avec $n\in \Nn$ et $a_0,a_1,\ldots,a_n \in \Kk$.

L'ensemble des polynômes est noté $\Kk[X]$.

\begin{itemize}
\item Les $a_i$ sont appelés les \defi{coefficients} du polynôme.

\item Si tous les coefficients $a_i$ sont nuls, $P$ est appelé le \defi{polynôme nul}, il est noté $0$.

\item  On appelle le \defi{degré}\index{degre@degré} de $P$ le plus grand entier $i$ tel que $a_i\neq0$ ;
on le note $\deg P$. Pour le degré du polynôme nul on pose par convention $\deg(0)=-\infty$.

\item Un polynôme de la forme $P=a_0$ avec $a_0\in\Kk$ est appelé un
  \defi{polynôme constant}. Si $a_0\neq0$, son degré est $0$.
\end{itemize}
\end{definition}

\begin{exemple}
\sauteligne
\begin{itemize}
  \item $X^3-5X+\frac 34$ est un polynôme de degré $3$.
  \item $X^n+1$ est un polynôme de degré $n$.
  \item $2$ est un polynôme constant, de degré $0$.
\end{itemize}
\end{exemple}


%---------------------------------------------------------------
\subsection{Opérations sur les polynômes}

\begin{itemize}
  \item \textbf{\'Egalité.}
Soient $P=a_nX^n+a_{n-1}X^{n-1}+\cdots + a_1X+a_0$ et
$Q=b_nX^n+b_{n-1}X^{n-1}+\cdots+b_1X+b_0$
deux polynômes à coefficients dans $\Kk$.
$$P=Q \quad \iff \quad   \forall i \quad a_i=b_i$$
et on dit que $P$ et $Q$ sont égaux.

  \item \textbf{Addition.} Soient  $P=a_nX^n+a_{n-1}X^{n-1}+\cdots + a_1X+a_0$ et
$Q=b_nX^n+b_{n-1}X^{n-1}+\cdots+b_1X+b_0$.

On définit :
$$P+Q = (a_n+b_n)X^n+(a_{n-1}+b_{n-1})X^{n-1}+\cdots
+(a_1+b_1)X+(a_0+b_0)$$

  \item \textbf{Multiplication.}
Soient  $P=a_nX^n+a_{n-1}X^{n-1}+\cdots + a_1X+a_0$ et
$Q=b_mX^m+b_{m-1}X^{m-1}+\cdots+b_1X+b_0$. On définit
$$P \times Q= c_rX^r+c_{r-1}X^{r-1}+\cdots
+c_1X+c_0$$
$$\text{  avec } r=n+m \text{ et } c_k=\sum_{i+j=k}a_ib_j  \text{ pour }k\in \{0,\ldots, r\}.$$

  \item \textbf{Multiplication par un scalaire.}
Si $\lambda \in \Kk$ alors $\lambda \cdot P$ est le polynôme dont le $i$-ème coefficient
est $\lambda a_i$.
\end{itemize}


\begin{exemple}
\sauteligne
\begin{itemize}
  \item Soient $P=aX^3+bX^2+cX+d$ et $Q=\alpha X^2+\beta X + \gamma$.
Alors $P+Q =  aX^3+(b+\alpha)X^2+(c+\beta)X+(d+\gamma)$,
$P\times Q = (a\alpha)X^5+(a\beta+b\alpha)X^4+(a\gamma+b\beta+c\alpha)X^3
+(b\gamma+c\beta+d\alpha)X^2+(c\gamma+d\beta)X + d\gamma$.
Enfin $P=Q$ si et seulement si $a=0$, $b=\alpha$, $c=\beta$ et $d=\gamma$.

  \item La multiplication par un scalaire $\lambda \cdot P$ équivaut à multiplier
le polynôme constant $\lambda$ par le polynôme $P$.
\end{itemize}

\end{exemple}




L'addition et la multiplication se comportent sans problème :
\begin{proposition}
Pour $P,Q,R \in \Kk[X]$ alors
\begin{itemize}
  \item $0+P=P$, \quad $P+Q=Q+P$, \quad $(P+Q)+R=P+(Q+R)$ ;
  \item $1\cdot P = P$, \quad $P\times Q=Q \times P$, \quad $(P \times Q) \times R=P \times (Q \times R)$ ;
  \item $P\times (Q+R)=P\times Q + P \times R$.
\end{itemize}
\end{proposition}


Pour le degré il faut faire attention :
\begin{proposition}
Soient $P$ et $Q$ deux polynômes à coefficients dans $\Kk$.

\mybox{$\deg(P\times Q)=\deg P + \deg Q$}

\mybox{$\deg(P+Q) \le \max(\deg P, \deg Q)$}
\end{proposition}


On note $\Rr_n[X]= \big\{P \in \Rr[X] \mid \deg P \le n\big\}$.
Si $P,Q \in \Rr_n[X]$ alors $P+Q \in \Rr_n[X]$.


%---------------------------------------------------------------
\subsection{Vocabulaire}

Complétons les définitions sur les polynômes.

\begin{definition}
\sauteligne
\begin{itemize}
\item Les polynômes comportant un seul terme non nul (du type $a_kX^k$) sont
appelés \defi{monômes}\index{monome@monôme}.

\item Soit $P=a_nX^n+a_{n-1}X^{n-1}+\cdots + a_1X+a_0,$ un polynôme avec
$a_n\neq0$. On appelle \defi{terme dominant} le monôme $a_nX^n$. Le coefficient
$a_n$ est appelé le \defi{coefficient dominant} de $P$.

\item Si le coefficient dominant est $1$, on dit que $P$ est un \defi{polynôme
unitaire}.
\end{itemize}
\end{definition}

\begin{exemple}
$P(X)=(X-1)(X^n+X^{n-1}+\cdots + X+1)$.
On développe cette expression :
$P(X)= \big(X^{n+1}+X^{n}+\cdots + X^2+X\big) - \big(X^n+X^{n-1}+\cdots + X+1\big) = X^{n+1} - 1$.
$P(X)$ est donc un polynôme de degré $n+1$, il est unitaire et
est somme de deux monômes : $X^{n+1}$ et $-1$.
\end{exemple}


\begin{remarque*}
Tout polynôme est donc une somme finie de monômes.
\end{remarque*}

%---------------------------------------------------------------
%\subsection{Mini-exercices}

\begin{miniexercices}
\sauteligne
\begin{enumerate}
  \item Soit $P(X)=3X^3-2$, $Q(X)=X^2+X-1$, $R(X)=aX+b$.
Calculer $P+Q$, $P\times Q$, $(P+Q)\times R$ et $P\times Q \times R$.
Trouver $a$ et $b$ afin que le degré de $P-QR$ soit le plus petit possible.

  \item Calculer $(X+1)^5-(X-1)^5$.

  \item Déterminer le degré de $(X^2+X+1)^n - aX^{2n}-bX^{2n-1}$ en fonction de $a,b$.

  \item Montrer que si $\deg P \neq \deg Q$ alors $\deg (P+Q)= \max(\deg P,\deg Q)$.
Donner un contre-exemple dans le cas où $\deg P = \deg Q$.

  \item Montrer que si $P(X)=X^n+a_{n-1}X^{n-1}+ \cdots$ alors le coefficient devant $X^{n-1}$ de
$P(X-\frac{a_{n-1}}{n})$ est nul.
\end{enumerate}
\end{miniexercices}

%%%%%%%%%%%%%%%%%%%%%%%%%%%%%%%%%%%%%%%%%%%%%%%%%%%%%%%%%%%%%%%%
\section{Arithmétique des polynômes}

Il existe de grandes similitudes entre l'arithmétique dans $\Zz$ et
l'arithmétique dans $\Kk[X]$. Cela nous permet d'aller assez vite
et d'omettre certaines preuves.

%---------------------------------------------------------------
\subsection{Division euclidienne}

\begin{definition}
Soient $A,B \in\Kk[X]$, on dit que $B$ \defi{divise}\index{divisibilite@divisibilité} $A$ s'il existe  $Q\in\Kk[X]$ tel que $A=BQ$.
On note alors $B|A$.
\end{definition}

On dit aussi que $A$ est multiple de $B$ ou que $A$ est divisible par $B$.

Outre les propriétés évidentes comme $A|A$, $1|A$ et $A|0$ nous avons :
\begin{proposition}
Soient $A,B, C\in\Kk[X]$.
\begin{enumerate}
\item Si $A|B$ et $B|A$, alors il existe $\lambda \in\Kk^*$ tel que
  $A=\lambda B$.
\item Si $A|B$ et $B|C$ alors $A|C$.
\item Si $C|A$ et $C|B$ alors  $C|(AU+BV)$, pour tout $U,V \in\Kk[X]$.
\end{enumerate}
\end{proposition}




\begin{theoreme}[Division euclidienne des polynômes]
Soient $A,B \in\Kk[X]$, avec $B \neq 0$, alors il existe un
unique polynôme $Q$ et il existe un unique polynôme $R$ tels que :
\mybox{$A=BQ+R \quad \text{ et } \quad \deg R < \deg B$.}
\end{theoreme}

$Q$ est appelé le \defi{quotient}\index{quotient} et $R$ le \defi{reste}\index{reste} et cette écriture
est la \defi{division euclidienne}\index{division euclidienne} de $A$ par $B$.

Notez que la condition $\deg R < \deg B$ signifie $R=0$
ou bien $0 \le \deg R < \deg B$.

Enfin $R=0$ si et seulement si $B|A$.

\begin{proof}
~\\
\textbf{Unicité.}
 Si $A=BQ+R$ et $A=BQ'+R'$, alors $B(Q-Q')=R'-R$. Or $\deg(R'-R) < \deg B$.
Donc $Q'-Q=0$. Ainsi $Q=Q'$, d'où aussi $R=R'$.


\textbf{Existence.}
On montre l'existence par récurrence sur le degré de $A$.
\begin{itemize}
  \item Si $\deg A=0$ et $\deg B>0$, alors $A$ est une constante, on pose $Q=0$ et $R=A$.
  Si $\deg A=0$ et $\deg B=0$, on pose $Q=A/B$ et $R=0$.
  \item On suppose l'existence vraie lorsque $\deg A \le n-1$.
Soit $A=a_nX^n+\cdots+a_0$ un polynôme de degré $n$ ($a_n \neq 0$).
Soit $B =b_mX^m+\cdots+b_0$ avec $b_m \neq 0$.
Si $n<m$ on pose $Q=0$ et $R=A$.

Si $n \ge m$ on écrit $A=B \cdot \frac{a_n}{b_m}X^{n-m} + A_1$
avec $\deg A_1 \le n-1$. On applique l'hypothèse de récurrence à $A_1$ :
il existe $Q_1,R_1 \in \Kk[X]$ tels que $A_1=BQ_1+R_1$ et $\deg R_1 < \deg B$.
Il vient :
$$A= B \left( \frac{a_n}{b_m}X^{n-m} + Q_1 \right) + R_1.$$
Donc $Q=\frac{a_n}{b_m}X^{n-m} + Q_1$ et $R=R_1$ conviennent.
\end{itemize}
\end{proof}



\begin{exemple}
On pose une division de polynômes comme on pose une division euclidienne de deux entiers.
Par exemple si $A=2X^4-X^3-2X^2+3X-1$ et $B=X^2-X+1$.
Alors on trouve $Q= 2X^2+X-3$ et $R=-X+2$.
On n'oublie pas de vérifier qu'effectivement $A=BQ+R$.

\myfigure{1.2}{
\tikzinput{fig_polynome01}
}

\end{exemple}

\begin{exemple}
Pour $X^4-3X^3+X+1$ divisé par $X^2+2$ on trouve un quotient égal à $X^2-3X-2$ et
un reste égale à $7X+5$.

\myfigure{1.2}{
\tikzinput{fig_polynome02}
}
\end{exemple}




%---------------------------------------------------------------
\subsection{pgcd}

\begin{proposition}
\label{prop_pgcd1}
Soient  $A,B \in\Kk[X]$, avec $A \neq 0$ ou $B \neq 0$.
Il existe un unique polynôme unitaire de plus grand degré qui
divise à la fois $A$ et $B$.
\end{proposition}

Cet unique polynôme est appelé le \defi{pgcd}\index{pgcd} (plus grand commun diviseur) de $A$ et $B$
que l'on note $\pgcd(A,B)$.

\begin{remarque*}
\sauteligne
\begin{itemize}
  \item $\pgcd(A,B)$ est un polynôme unitaire.
  \item Si $A|B$ et $A \neq 0$, $\pgcd(A,B)=\frac{1}{\lambda}A, $ où $\lambda$
est le coefficient dominant de $A$.
  \item Pour tout $\lambda \in K^*$, $\pgcd( \lambda A,B)= \pgcd(A,B)$.
  \item Comme pour les entiers : si $A=BQ+R$ alors $\pgcd(A,B) = \pgcd(B,R)$.
  C'est ce qui justifie l'algorithme d'Euclide.
\end{itemize}
\end{remarque*}


\textbf{Algorithme d'Euclide.}
\index{algorithme d'Euclide}

Soient $A$ et $B$ des polynômes, $B \neq 0$.

On calcule les divisions euclidiennes successives,

$$\begin{array}{ll}
A = B Q_1+R_1 \quad \quad & \deg R_1 < \deg B \\
B = R_1 Q_2 + R_2  & \deg R_2 < \deg R_1 \\
R_1=R_2Q_3+R_3 & \deg R_3 < \deg R_2 \\
\vdots & \\
R_{k-2}=R_{k-1}Q_{k}+R_k  & \deg R_k < \deg R_{k-1} \\
R_{k-1}=R_kQ_{k+1} & \\
\end{array}$$

Le degré du reste diminue à chaque division.
On arrête l'algorithme lorsque le reste est nul.
Le pgcd est le dernier reste non nul $R_k$ (rendu unitaire).


\begin{exemple}
Calculons le pgcd de  $A=X^4-1$ et $B=X^3-1$.
On applique l'algorithme d'Euclide :
$$\begin{array}{rcl}
X^4-1 & = & (X^3-1) \times X + X-1 \\
X^3-1 & = & (X-1)\times (X^2+X+1) + 0 \\
\end{array}$$
Le pgcd est le dernier reste non nul, donc $\pgcd(X^4-1, X^3-1)=X-1$.
\end{exemple}


\begin{exemple}
Calculons le pgcd de  $A=X^5+X^4+2X^3+X^2+X+2$ et $B=X^4+2X^3+X^2-4$.
{\small
$$\begin{array}{rcl}
X^5+X^4+2X^3+X^2+X+2 & = & (X^4+2X^3+X^2-4) \times (X-1)  + 3X^3+2X^2+5X-2\\
X^4+2X^3+X^2-4 & = & (3X^3+2X^2+5X-2)\times \frac19(3X+4) - \frac{14}{9}(X^2+X+2) \\
3X^3+2X^2+5X-2 & = & (X^2+X+2) \times (3X-1) + 0 \\
\end{array}$$
}
Ainsi $\pgcd(A,B)=X^2+X+2$.
\end{exemple}




\begin{definition}
Soient  $A,B \in\Kk[X]$. On dit que $A$ et $ B$ sont \defi{premiers entre eux}\index{polynome@polynôme!premiers entre eux} si  $\pgcd(A,B)=1$.
\end{definition}

Pour $A,B$ quelconques on peut se ramener à des polynômes premiers entre eux :
si  $\pgcd(A,B)=D$ alors $A$ et $B$ s'écrivent : $A=DA'$, $B=DB'$ avec $\pgcd(A',B')=1$.



%---------------------------------------------------------------
\subsection{Théorème de Bézout}

\begin{theoreme}[Théorème de Bézout]
\index{theoreme@théorème!de Bézout}
\label{thm_Bezout}
Soient $A, B\in \Kk[X]$ des polynômes avec $A\neq 0$ ou $B\neq 0$.
On note $D=\pgcd(A,B)$.
Il existe deux polynômes $U, V\in \Kk[X]$ tels que $AU+BV=D$.
\end{theoreme}

Ce théorème découle de l'algorithme d'Euclide et plus spécialement de sa remontée
comme on le voit sur l'exemple suivant.

\begin{exemple}
Nous avons calculé $\pgcd(X^4-1, X^3-1) = X-1$.
Nous remontons l'algorithme d'Euclide, ici il n'y avait qu'une ligne :
$X^4-1  =  (X^3-1) \times X + X-1$, pour en déduire
$X-1 = (X^4-1)\times 1 + (X^3-1) \times (-X)$.
Donc $U=1$ et $V=-X$ conviennent.
\end{exemple}

\begin{exemple}
Pour $A=X^5+X^4+2X^3+X^2+X+2$ et $B=X^4+2X^3+X^2-4$ nous avions trouvé $D= \pgcd(A,B)=X^2+X+2$.
En partant de l'avant dernière ligne de l'algorithme d'Euclide on a d'abord :
 $B  =  (3X^3+2X^2+5X-2)\times \frac19(3X+4) - \frac{14}{9} D$
donc
$$- \frac{14}{9} D = B - (3X^3+2X^2+5X-2)\times \frac19(3X+4).$$
La ligne au-dessus dans l'algorithme d'Euclide était :
$A = B \times (X-1)  + 3X^3+2X^2+5X-2$.
On substitue le reste pour obtenir :
$$- \frac{14}{9} D = B - \big(A-B\times(X-1)\big)\times \frac19(3X+4).$$
On en déduit
$$- \frac{14}{9}D = -A\times \frac19(3X+4) + B\big(1+(X-1)\times \frac19(3X+4)\big)$$
Donc en posant $U=\frac{1}{14}(3X+4)$ et $V= -\frac{1}{14}\big(9+(X-1)(3X+4)\big)=-\frac{1}{14}(3X^2+X+5)$
on a $AU+BV=D$.
\end{exemple}


Le corollaire suivant s'appelle aussi le théorème de Bézout.
\begin{corollaire}
Soient $A$ et $B$ deux polynômes. $A$ et $B$ sont premiers entre eux si
et seulement s'il existe deux polynômes $U$ et $V$ tels que $AU+BV=1$.
\end{corollaire}

\begin{corollaire}
Soient $A, B, C\in \Kk[X]$ avec $A\neq 0$ ou $B\neq 0$.
Si $C|A$ et $C|B$ alors $C|\pgcd(A,B)$.
\end{corollaire}

\begin{corollaire}[Lemme de Gauss]
\index{lemme!de Gauss}
Soient $A, B, C\in \Kk[X]$.
Si $A|BC$ et $\pgcd(A,B)=1$ alors $A|C$.
\end{corollaire}



%---------------------------------------------------------------
\subsection{ppcm}

\begin{proposition}
Soient $A, B\in \Kk[X]$ des polynômes non nuls, alors il existe un unique
polynôme unitaire $M$ de plus petit degré tel que $A|M$ et $B|M$.
\end{proposition}

Cet unique polynôme est appelé le \defi{ppcm}\index{ppcm}
(plus petit commun multiple) de $A$ et $B$
qu'on note $\ppcm(A,B)$.

\begin{exemple}
$\ppcm \big( X(X-2)^2(X^2+1)^4, (X+1)(X-2)^3(X^2+1)^3 \big) = X(X+1)(X-2)^3(X^2+1)^4$.
\end{exemple}


De plus le ppcm est aussi le plus petit au sens de la divisibilité :
\begin{proposition}
Soient $A, B\in\Kk[X]$ des polynômes non nuls et
$M=\ppcm(A,B)$. Si $C\in\Kk[X]$ est un polynôme tel que $A|C$ et $B|C$, alors $M|C$.
\end{proposition}



%---------------------------------------------------------------
%\subsection{Mini-exercices}

\begin{miniexercices}
\sauteligne
\begin{enumerate}
  \item Trouver les diviseurs de $X^4+2X^2+1$ dans $\Rr[X]$, puis dans $\Cc[X]$.

  \item Montrer que $X-1|X^n-1$ (pour $n\ge 1$).

  \item Calculer les divisions euclidiennes de $A$ par $B$ avec $A=X^4-1$, $B=X^3-1$.
Puis $A = 4X^3+2X^2-X-5$ et $B = X^2+X$ ; $A= 2X^4-9X^3+18X^2-21X+2$ et $B=X^2-3X+1$ ;
$A=X^ 5-2X^4+6X^3$ et $B=2X^3+1$.

  \item Déterminer le pgcd de $A=X^5+X^3+X^2+1$ et $B=2X^3+3X^2+2X+3$.
Trouver les coefficients de Bézout $U,V$.
Mêmes questions avec $A=X^5-1$ et $B=X^4+X+1$.

  \item Montrer que si $AU+BV=1$ avec $\deg U < \deg B$ et $\deg V < \deg A$
  alors les polynômes $U,V$ sont uniques.
\end{enumerate}
\end{miniexercices}

%%%%%%%%%%%%%%%%%%%%%%%%%%%%%%%%%%%%%%%%%%%%%%%%%%%%%%%%%%%%%%%%
\section{Racine d'un polynôme, factorisation}

%---------------------------------------------------------------
\subsection{Racines d'un polynôme}

\begin{definition}
Soit $P=a_nX^n+a_{n-1}X^{n-1}+\cdots+a_1X+a_0\in\Kk[X]$.
Pour un élément $x \in \Kk$, on note $P(x)=a_nx^n+\cdots+a_1x+ a_0$.
On associe ainsi au polynôme $P$ une \defi{fonction polynôme}
(que l'on note encore $P$)
$$P: \Kk\to \Kk, \quad x \mapsto P(x)=a_nx^n+\cdots+a_1x+ a_0.$$
\end{definition}



\begin{definition}
Soit $P\in\Kk[X]$ et $\alpha\in\Kk$. On dit que
$\alpha$ est une \defi{racine}\index{racine} (ou un \defi{zéro}) de $P$ si $P(\alpha)=0$.
\end{definition}

\begin{proposition}
$$P(\alpha)=0 \quad \iff \quad X-\alpha \text{ divise } P$$
\end{proposition}

\begin{proof}
Lorsque l'on écrit la division euclidienne de $P$ par $X-\alpha$ on obtient
$P=Q\cdot(X-\alpha)+R$ où $R$ est une constante car $\deg R < \deg (X-\alpha) =1$.
Donc $P(\alpha)=0 \iff R(\alpha) =0 \iff R=0 \iff X-\alpha | P$.
\end{proof}

\begin{definition}
  Soit $k\in \Nn^*$. On dit que $\alpha$ est une \defi{racine de multiplicité $k$}\index{multiplicite@multiplicité}
 de $P$ si $(X-\alpha )^k$ divise $P$ alors que $(X- \alpha )^{k+1}$ ne divise pas $P$.
Lorsque $k=1$ on parle d'une \defi{racine simple}, lorsque $k=2$ d'une \defi{racine double}, etc.
\end{definition}

On dit aussi que $\alpha$ est une \defi{racine d'ordre $k$}.

\begin{proposition}
\label{prop:racmul}
Il y a équivalence entre :
\begin{itemize}
  \item[(i)] $\alpha$ est une racine de multiplicité $k$ de $P$.

  \item[(ii)] Il existe  $Q \in\Kk[X]$ tel que $P=(X-\alpha)^kQ,$ avec $Q(\alpha) \neq 0$.

  \item[(iii)] $P(\alpha)= P'(\alpha)=\cdots=P^{(k-1)}(\alpha)=0$ et $P^{(k)}(\alpha) \neq 0$.
\end{itemize}
\end{proposition}
La preuve est laissée en exercice.

\begin{remarque*}
Par analogie avec la dérivée d'une fonction,
si $P(X) = a_0+a_1X+\cdots+a_nX^n\in\Kk[X]$ alors le polynôme
$P'(X) = a_1+2a_2X+\cdots+na_nX^{n-1}$ est le \defi{polynôme dérivé}
de $P$.
\end{remarque*}

%---------------------------------------------------------------
\subsection{Théorème de d'Alembert-Gauss}

Passons à un résultat essentiel de ce chapitre :

\begin{theoreme}[Théorème de d'Alembert-Gauss]
Tout polynôme à coefficients complexes de degré $n \ge 1$
a au moins une racine dans $\Cc$.
Il admet exactement $n$ racines si on compte chaque racine
avec multiplicité.
\end{theoreme}

Nous admettons ce théorème.

\begin{exemple}
Soit $P(X)=aX^2+bX+c$ un polynôme de degré $2$ à coefficients réels : $a,b,c \in \Rr$ et $a\neq 0$.
\begin{itemize}
  \item Si $\Delta = b^2-4ac > 0$ alors $P$ admet $2$ racines réelles distinctes $\frac{-b+\sqrt{\Delta}}{2a}$
et $\frac{-b-\sqrt{\Delta}}{2a}$.
  \item Si $\Delta < 0$ alors $P$ admet $2$ racines complexes distinctes $\frac{-b+\ii\sqrt{|\Delta|}}{2a}$
et $\frac{-b-\ii\sqrt{|\Delta|}}{2a}$.
  \item Si $\Delta = 0$ alors $P$ admet une racine réelle double $\frac{-b}{2a}$.
\end{itemize}
En tenant compte des multiplicités on a donc toujours exactement $2$ racines.
\end{exemple}

\begin{exemple}
$P(X)=X^n-1$ admet $n$ racines distinctes.

Sachant que $P$ est de degré $n$ alors par le théorème de d'Alembert-Gauss
on sait qu'il admet $n$ racines comptées avec multiplicité.
Il s'agit donc maintenant de montrer que ce sont des racines simples.
Supposons --par l'absurde-- que $\alpha \in \Cc$ soit une racine de multiplicité $\ge 2$.
Alors $P(\alpha)=0$ et $P'(\alpha)=0$.
Donc $\alpha^n-1=0$ et $n\alpha^{n-1}=0$. De la seconde égalité on déduit $\alpha=0$,
contradictoire avec la première égalité. Donc toutes les racines sont simples.
Ainsi les $n$ racines sont distinctes.
(Remarque : sur cet exemple particulier on aurait aussi pu calculer les racines
qui sont ici les racines $n$-ième de l'unité.)
\end{exemple}


Pour les autres corps que les nombres complexes nous avons le résultat
plus faible suivant :
\begin{theoreme}
Soit $P\in\Kk[X]$ de degré $n\ge 1$. Alors
$P$ admet au plus $n$ racines dans $\Kk$.
\end{theoreme}

\begin{exemple}
$P(X)=3X^3-2X^2+6X-4$.
Considéré comme un polynôme à coefficients dans $\Qq$ ou $\Rr$,
$P$ n'a qu'une seule racine (qui est simple) $\alpha = \frac23$ et il se décompose en
$P(X)=3(X-\frac23)(X^2+2)$.
Si on considère maintenant $P$ comme un polynôme à coefficients dans $\Cc$ alors
$P(X)=3(X-\frac23)(X-\ii\sqrt2)(X+\ii\sqrt2)$ et admet $3$ racines simples.
\end{exemple}


%---------------------------------------------------------------
\subsection{Polynômes irréductibles}



\begin{definition}
Soit  $P \in\Kk[X]$ un polynôme de degré $\ge 1$, on dit que $P$ est 
\defi{irréductible}\index{irreductibilite@irréductibilité} si
pour tout $Q \in\Kk[X]$ divisant $P$, alors, soit $Q \in\Kk^*$, soit il existe $\lambda \in\Kk^*$ tel
que $Q=\lambda P$.
\end{definition}


\begin{remarque*}
\sauteligne
\begin{itemize}
  \item Un polynôme irréductible  $P$ est donc un polynôme non constant
dont les seuls diviseurs de $P$ sont les constantes ou $P$ lui-même
(à une constante multiplicative près).

  \item La notion de polynôme irréductible pour l'arithmétique de $\Kk[X]$
correspond à la notion de nombre premier pour l'arithmétique de $\Zz$.

  \item Dans le cas contraire, on dit que $P$ est \defi{réductible}\index{reductibilite@réductibilité} ;
il existe alors des polynômes $A, B$ de $\Kk[X]$ tels que $P=AB$, avec
 $\deg A \ge 1$ et  $\deg B \ge 1$.
\end{itemize}
\end{remarque*}

\begin{exemple}
\sauteligne
\begin{itemize}
  \item Tous les polynômes de degré 1 sont irréductibles. Par conséquent il y a
  une infinité de polynômes irréductibles.

  \item $X^2-1=(X-1)(X+1)\in\Rr[X]$ est réductible.

  \item $X^2+1=(X-\ii)(X+\ii)$ est réductible dans $\Cc[X]$ mais est irréductible dans $\Rr[X]$.

  \item $X^2-2=(X-\sqrt2)(X+\sqrt2)$ est réductible dans $\Rr[X]$ mais est irréductible dans $\Qq[X]$.
\end{itemize}
\end{exemple}




Nous avons l'équivalent du lemme d'Euclide de $\Zz$ pour les polynômes :
\begin{proposition}[Lemme d'Euclide]
\index{lemme!d'Euclide}
Soit $P\in\Kk[X]$ un polynôme irréductible et soient $A, B\in\Kk[X]$.
Si $P|AB$ alors $P|A$ ou $P|B$.
\end{proposition}

\begin{proof}
Si $P$ ne divise pas $A$ alors $\pgcd(P,A)=1$ car $P$ est irréductible.
Donc, par le lemme de Gauss, $P$ divise $B$.
\end{proof}


%---------------------------------------------------------------
\subsection{Théorème de factorisation}


\begin{theoreme}
Tout polynôme non constant $A\in\Kk[X]$ s'écrit comme un produit de polynômes
irréductibles unitaires :
$$A= \lambda  P_1^{k_1}P_2^{k_2} \cdots P_r^{k_r}$$
 où $\lambda\in\Kk^*$, $r \in \Nn^*$, $k_i \in \Nn^*$
et les $P_i$ sont des polynômes irréductibles distincts.

De plus cette décomposition est unique à l'ordre près des facteurs.
\end{theoreme}

Il s'agit bien sûr de l'analogue de la décomposition d'un nombre en facteurs premiers.




%---------------------------------------------------------------
\subsection{Factorisation dans $\Cc[X]$ et $\Rr[X]$}



\begin{theoreme}
Les polynômes irréductibles de $\Cc[X]$ sont les polynômes de degré $1$.

Donc pour $P\in\Cc[X]$ de degré $n\ge1$ la factorisation s'écrit
$P=\lambda (X-\alpha_1)^{k_1}(X- \alpha_2)^{k_2}\cdots(X- \alpha_r)^{k_r},$
où $\alpha_1,...,\alpha_r$ sont les racines distinctes de $P$ et
$k_1,...,k_r$ sont leurs multiplicités.
\end{theoreme}

\begin{proof}
Ce théorème résulte du théorème de d'Alembert-Gauss.
\end{proof}




\begin{theoreme}
Les polynômes irréductibles de $\Rr[X]$
sont les polynômes de degré $1$ ainsi que les polynômes de degré $2$ ayant
un discriminant $\Delta<0$.

Soit $P\in\Rr[X]$ de degré $n\ge1$. Alors
la factorisation s'écrit
$P=\lambda(X-\alpha_1)^{k_1}(X-\alpha_2)^{k_2}\cdots(X-\alpha_r)^{k_r}
Q_1^{\ell_1}\cdots Q_s^{\ell_s},$
où  les $\alpha_i$ sont exactement les racines réelles distinctes
de multiplicité $k_i$ et les $Q_i$ sont des polynômes irréductibles de degré $2$ :
$Q_i=X^2+\beta_iX+\gamma_i$ avec $\Delta = \beta_i^2-4\gamma_i<0$.
\end{theoreme}

\begin{exemple}
$P(X)=2X^4(X-1)^3(X^2+1)^2(X^2+X+1)$ est déjà décomposé en facteurs irréductibles dans $\Rr[X]$
alors que sa décomposition dans $\Cc[X]$
est $P(X)=2X^4(X-1)^3(X-\ii)^2(X+\ii)^2(X-j)(X-j^2)$
où $j=e^{\frac{2\ii\pi}{3}}=\frac{-1+\ii\sqrt3}{2}$.
\end{exemple}

\begin{exemple}
Soit $P(X)=X^4+1$.
\begin{itemize}
  \item Sur $\Cc$. On peut d'abord décomposer $P(X)=(X^2+\ii)(X^2-\ii)$.
Les racines de $P$ sont donc les racines carrées complexes de $\ii$ et $-\ii$.
Ainsi $P$ se factorise dans $\Cc[X]$ :
$$P(X)=\big(X-\tfrac{\sqrt2}{2}(1+\ii)\big)\big(X+\tfrac{\sqrt2}{2}(1+\ii)\big)\big(X-\tfrac{\sqrt2}{2}(1-\ii)\big)
\big(X+\tfrac{\sqrt2}{2}(1-\ii)\big).$$

  \item Sur $\Rr$. Pour un polynôme à coefficient réels, si $\alpha$ est une racine
alors $\bar \alpha$ aussi. Dans la décomposition ci-dessus on regroupe les facteurs ayant des racines conjuguées,
cela doit conduire à un polynôme réel :
\begin{align*}
P(X)&=\left[\big(X-\tfrac{\sqrt2}{2}(1+\ii)\big)\big(X-\tfrac{\sqrt2}{2}(1-\ii)\big)\right]
\left[\big(X+\tfrac{\sqrt2}{2}(1+\ii)\big)\big(X+\tfrac{\sqrt2}{2}(1-\ii)\big)\right]\\
&= \big[X^2+\sqrt2X+1\big]\big[X^2-\sqrt2X+1\big],  
\end{align*}
qui est la factorisation dans $\Rr[X]$.
\end{itemize}
\end{exemple}




%---------------------------------------------------------------
%\subsection{Mini-exercices}

\begin{miniexercices}
\sauteligne
\begin{enumerate}
  \item Trouver un polynôme $P(X) \in \Zz[X]$ de degré minimal tel que : $\frac 12$ soit une racine simple,
$\sqrt 2$ soit une racine double et $\ii$ soit une racine triple.

  \item Montrer cette partie de la proposition \ref{prop:racmul} :
\og$P(\alpha)=0$ et $P'(\alpha)=0$  $\iff$ $\alpha$ est une racine de multiplicité $\ge 2$\fg.

  \item Montrer que pour $P\in\Cc [X]$ :
\og$P$ admet une racine de multiplicité $\ge 2$ $\iff$ $P$ et $P'$ ne sont pas premiers entre eux\fg.

  \item Factoriser $P(X) = (2X^2+X-2)^2(X^4-1)^3$ et $Q(X)=3(X^2-1)^2(X^2-X+\frac14)$ dans $\Cc[X]$.
En déduire leur pgcd et leur ppcm. Mêmes questions dans $\Rr[X]$.

  \item Si $\pgcd(A,B)=1$ montrer que $\pgcd(A+B,A\times B)=1$.

  \item Soit $P\in \Rr[X]$ et $\alpha \in \Cc\setminus\Rr$ tel que $P(\alpha)=0$.
Vérifier que $P(\bar \alpha)=0$. Montrer que $(X-\alpha)(X-\bar\alpha)$ est un polynôme irréductible de $\Rr[X]$
et qu'il divise $P$ dans $\Rr[X]$.
\end{enumerate}
\end{miniexercices}


%%%%%%%%%%%%%%%%%%%%%%%%%%%%%%%%%%%%%%%%%%%%%%%%%%%%%%%%%%%%%%%%
\section{Fractions rationnelles}

\begin{definition}
Une \defi{fraction rationnelle}\index{fraction rationnelle} à coefficients dans $\Kk$ est une expression
de la forme
$$F=\frac{P}{Q}$$
où $P,Q \in \Kk[X]$ sont deux polynômes et $Q \neq 0$.
\end{definition}


Toute fraction rationnelle se décompose comme une somme de fractions rationnelles
élémentaires que l'on appelle des \og éléments simples \fg. Mais les éléments simples
sont différents sur $\Cc$ ou sur $\Rr$.

%---------------------------------------------------------------
\subsection{Décomposition en éléments simples sur $\Cc$}


\begin{theoreme}[Décomposition en éléments simples sur $\Cc$]
Soit $P/Q$ une fraction rationnelle avec $P,Q \in \Cc[X]$, $\pgcd(P,Q)=1$ et
$Q=(X-\alpha_1)^{k_1}\cdots(X-\alpha_r)^{k_r}$.
Alors il existe une et une seule écriture :
$$\begin{array}{rl}
\displaystyle \frac{P}{Q} \  =  \ \  E  & + \
 \displaystyle  \frac{a_{1,1}}{(X-\alpha_1)^{k_1}}+\frac{a_{1,2}}{(X-\alpha_1)^{k_1-1}}+\cdots
+\ \frac{a_{1,k_1}}{(X-\alpha_1)} \\[4mm]
  & \displaystyle+ \frac{a_{2,1}}{(X-\alpha_2)^{k_2}}+\cdots
+\ \frac{a_{2,k_2}}{(X-\alpha_2)} \\[3mm]
 & + \ \cdots
\end{array}$$
\end{theoreme}

Le polynôme $E$ s'appelle la \defi{partie polynomiale}\index{partie polynomiale} 
(ou \defi{partie entière}).
Les termes $\frac{a}{(X-\alpha)^i}$ sont les \defi{éléments simples}\index{element simple@élément simple} sur $\Cc$.


\begin{exemple}
\sauteligne
\begin{itemize}
  \item Vérifier que $\frac{1}{X^2+1} = \frac{a}{X+\ii} + \frac{b}{X-\ii}$ avec $a=\frac12 \ii$, $b=-\frac12\ii$.
  \item Vérifier que  $\frac{X^4-8X^2+9X-7}{(X-2)^2(X+3)}= X+1 + \frac{-1}{(X-2)^2} + \frac{2}{X-2} + \frac{-1}{X+3}$.
\end{itemize}
\end{exemple}


Comment se calcule cette décomposition ?
En général on commence par déterminer la partie polynomiale.
Tout d'abord si $\deg Q > \deg P$ alors $E(X)=0$.
Si $\deg P \le \deg Q$ alors effectuons la division euclidienne de $P$ par $Q$ :
$P=QE+R$ donc $\frac{P}{Q} = E + \frac{R}{Q}$
où $\deg R < \deg Q$.
La partie polynomiale est donc le quotient de cette division.
Et on s'est ramené au cas d'une fraction $\frac{R}{Q}$ avec $\deg R < \deg Q$.
Voyons en détails comment continuer sur un exemple.

\begin{exemple}
Décomposons la fraction $\displaystyle\frac{P}{Q}= \frac{X^5-2X^3+4X^2-8X+11}{X^3-3X+2}$.

\begin{itemize}
  \item \textbf{Première étape : partie polynomiale.}
On calcule la division euclidienne de $P$ par $Q$ :
$P(X) = (X^2+1)Q(X)+ 2X^2-5X+9$.
Donc la partie polynomiale est $E(X)=X^2+1$ et la fraction s'écrit
$\frac{P(X)}{Q(X)} =X^2+1 + \frac{2X^2-5X+9}{Q(X)}$.
Notons que pour la fraction $\frac{2X^2-5X+9}{Q(X)}$ le degré du numérateur
est strictement plus petit que le degré du dénominateur.

  \item \textbf{Deuxième étape : factorisation du dénominateur.}
$Q$ a pour racine évidente $+1$ (racine double) et $-2$ (racine simple) et se factorise donc ainsi
$Q(X)= (X-1)^2(X+2)$.

  \item \textbf{Troisième étape : décomposition théorique en éléments simples.}
Le théorème de décomposition en éléments simples nous dit qu'il existe une unique décomposition :
$\frac{P(X)}{Q(X)}= E(X)+ \frac{a}{(X-1)^2} + \frac{b}{X-1} + \frac{c}{X+2}$.
Nous savons déjà que $E(X)=X^2+1$, il reste à trouver les nombres $a,b,c$.

  \item \textbf{Quatrième étape : détermination des coefficients.}
Voici une première façon de déterminer $a,b,c$.
On récrit la fraction $\frac{a}{(X-1)^2} + \frac{b}{X-1} + \frac{c}{X+2}$
au même dénominateur et on l'identifie avec $\frac{2X^2-5X+9}{Q(X)}$ :
$$\frac{a}{(X-1)^2} + \frac{b}{X-1} + \frac{c}{X+2} =
\frac{(b+c)X^2+(a+b-2c)X+2a-2b+c}{(X-1)^2(X+2)}$$
qui doit être égale à
$$\frac{2X^2-5X+9}{(X-1)^2(X+2)}.$$
On en déduit $b+c=2$, $a+b-2c=-5$ et $2a-2b+c=9$.
Cela conduit à l'unique solution $a=2$, $b=-1$, $c=3$.
Donc
$$\frac{P}{Q} = \frac{X^5-2X^3+4X^2-8X+11}{X^3-3X+2} = X^2+1 + \frac{2}{(X-1)^2} + \frac{-1}{X-1} + \frac{3}{X+2}.$$
Cette méthode est souvent la plus longue.

  \item \textbf{Quatrième étape (bis) : détermination des coefficients.}
Voici une autre méthode plus efficace.

Notons $\frac{P'(X)}{Q(X)} =  \frac{2X^2-5X+9}{(X-1)^2(X+2)}$ dont la décomposition théorique est :
$\frac{a}{(X-1)^2} + \frac{b}{X-1} + \frac{c}{X+2}$

Pour déterminer $a$ on multiplie la fraction $\frac{P'}{Q}$ par $(X-1)^2$ et on évalue en $x=1$.

Tout d'abord en partant de la décomposition théorique on a :
$$F_1(X)= (X-1)^2 \frac{P'(X)}{Q(X)} = a + b(X-1) + c\frac{(X-1)^2}{X+2} \quad
\text{ donc } \quad F_1(1)=a$$

D'autre part
$$F_1(X)=(X-1)^2 \frac{P'(X)}{Q(X)}= (X-1)^2\frac{2X^2-5X+9}{(X-1)^2 (X+2)}
= \frac{2X^2-5X+9}{X+2}$$
donc $F_1(1)=2.$
On en déduit $a=2$.

On fait le même processus pour déterminer $c$ : on
multiplie par $(X+2)$ et on évalue en $-2$. On calcule
$F_2(X) = (X+2)\frac{P'(X)}{Q(X)} = \frac{2X^2-5X+9}{(X-1)^2} = a\frac{X+2}{(X-1)^2} + b\frac{X+2}{X-1} + c$
de deux façons et lorsque l'on évalue $x=-2$ on obtient d'une part $F_2(-2) = c$ et
d'autre part $F_2(-2) = 3$. Ainsi $c=3$.

Comme les coefficients sont uniques tous les moyens sont bons pour les déterminer.
Par exemple lorsque l'on évalue la décomposition théorique
$\frac{P'(X)}{Q(X)}= \frac{a}{(X-1)^2} + \frac{b}{X-1} + \frac{c}{X+2}$
en $x=0$, on obtient :
$$\frac{P'(0)}{Q(0)} = a - b + \frac c2$$
Donc $\frac{9}{2} = a - b + \frac c2$. Donc $b=a + \frac c2 - \frac{9}{2}=-1$.
\end{itemize}
\end{exemple}

%---------------------------------------------------------------
\subsection{Décomposition en éléments simples sur $\Rr$}

\begin{theoreme}[Décomposition en éléments simples sur $\Rr$]
Soit $P/Q$ une fraction rationnelle avec $P,Q \in \Rr[X]$, $\pgcd(P,Q)=1$.
Alors $P/Q$ s'écrit de manière unique comme somme :
\begin{itemize}
  \item d'une partie polynomiale $E(X)$,
  \item d'éléments simples du type $\frac{a}{(X-\alpha)^i}$,
  \item d'éléments simples du type $\frac{aX+b}{(X^2+\alpha X + \beta)^i}$.
\end{itemize}
Où les $X-\alpha$ et $X^2+\alpha X + \beta$ sont les facteurs irréductibles de $Q(X)$
et les exposants $i$ sont inférieurs ou égaux à la puissance correspondante dans cette factorisation.
\end{theoreme}

\begin{exemple}
Décomposition en éléments simples de
$\frac{P(X)}{Q(X)}=\frac{3X^4+5X^3+8X^2+5X+3}{(X^2+X+1)^2(X-1)}$.
Comme $\deg P < \deg Q$ alors $E(X)=0$.
Le dénominateur est déjà factorisé sur $\Rr$ car $X^2+X+1$ est irréductible.
La décomposition théorique est donc :
$$\frac{P(X)}{Q(X)} = \frac{aX+b}{(X^2+X+1)^2}+\frac{cX+d}{X^2+X+1}+\frac{e}{X-1}.$$

Il faut ensuite mener au mieux les calculs pour déterminer les coefficients afin d'obtenir :
$$\frac{P(X)}{Q(X)} = \frac{2X+1}{(X^2+X+1)^2}+\frac{-1}{X^2+X+1}+\frac{3}{X-1}.$$
\end{exemple}



%---------------------------------------------------------------
%\subsection{Mini-exercices}

\begin{miniexercices}
\sauteligne
\begin{enumerate}
  \item Soit $Q(X)=(X-2)^2(X^2-1)^3(X^2+1)^4$. Pour $P \in \Rr[X]$ quelle est la forme théorique
de la décomposition en éléments simples sur $\Cc$ de $\frac PQ$ ? Et sur $\Rr$ ?

  \item Décomposer les fractions suivantes en éléments simples sur $\Rr$ et $\Cc$ :
$\frac{1}{X^2-1}$ ; $\frac{X^2+1}{(X-1)^2}$ ; $\frac{X}{X^3-1}$.

  \item Décomposer les fractions suivantes en éléments simples sur $\Rr$ :
$\frac{X^2+X+1}{(X-1)(X+2)^2}$ ; $\frac{2X^2-X}{(X^2+2)^2}$ ; $\frac{X^6}{(X^2+1)^2}$.

  \item Soit $F(X) = \frac{2X^2+7X-20}{X+2}$.
  Déterminer l'équation de l'asymptote oblique en $\pm \infty$.
  \'Etudier la position du graphe de $F$ par rapport à cette droite.

\end{enumerate}
\end{miniexercices}


\auteurs{
\begin{itemize}
  \item Rédaction : Arnaud Bodin ; relecture : Stéphanie Bodin
  \item Basé sur des cours de Guoting Chen et Marc Bourdon
\end{itemize}
}

\finchapitre
\end{document}

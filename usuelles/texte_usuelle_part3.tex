
%%%%%%%%%%%%%%%%%% PREAMBULE %%%%%%%%%%%%%%%%%%


\documentclass[12pt]{article}

\usepackage{amsfonts,amsmath,amssymb,amsthm}
\usepackage[utf8]{inputenc}
\usepackage[T1]{fontenc}
\usepackage[francais]{babel}


% packages
\usepackage{amsfonts,amsmath,amssymb,amsthm}
\usepackage[utf8]{inputenc}
\usepackage[T1]{fontenc}
%\usepackage{lmodern}

\usepackage[francais]{babel}
\usepackage{fancybox}
\usepackage{graphicx}

\usepackage{float}

%\usepackage[usenames, x11names]{xcolor}
\usepackage{tikz}
\usepackage{datetime}

\usepackage{mathptmx}
%\usepackage{fouriernc}
%\usepackage{newcent}
\usepackage[mathcal,mathbf]{euler}

%\usepackage{palatino}
%\usepackage{newcent}


% Commande spéciale prompteur

%\usepackage{mathptmx}
%\usepackage[mathcal,mathbf]{euler}
%\usepackage{mathpple,multido}

\usepackage[a4paper]{geometry}
\geometry{top=2cm, bottom=2cm, left=1cm, right=1cm, marginparsep=1cm}

\newcommand{\change}{{\color{red}\rule{\textwidth}{1mm}\\}}

\newcounter{mydiapo}

\newcommand{\diapo}{\newpage
\hfill {\normalsize  Diapo \themydiapo \quad \texttt{[\jobname]}} \\
\stepcounter{mydiapo}}


%%%%%%% COULEURS %%%%%%%%%%

% Pour blanc sur noir :
%\pagecolor[rgb]{0.5,0.5,0.5}
% \pagecolor[rgb]{0,0,0}
% \color[rgb]{1,1,1}



%\DeclareFixedFont{\myfont}{U}{cmss}{bx}{n}{18pt}
\newcommand{\debuttexte}{
%%%%%%%%%%%%% FONTES %%%%%%%%%%%%%
\renewcommand{\baselinestretch}{1.5}
\usefont{U}{cmss}{bx}{n}
\bfseries

% Taille normale : commenter le reste !
%Taille Arnaud
%\fontsize{19}{19}\selectfont

% Taille Barbara
%\fontsize{21}{22}\selectfont

%Taille François
\fontsize{25}{30}\selectfont

%Taille Pascal
%\fontsize{25}{30}\selectfont

%Taille Laura
%\fontsize{30}{35}\selectfont


%\myfont
%\usefont{U}{cmss}{bx}{n}

%\Huge
%\addtolength{\parskip}{\baselineskip}
}


% \usepackage{hyperref}
% \hypersetup{colorlinks=true, linkcolor=blue, urlcolor=blue,
% pdftitle={Exo7 - Exercices de mathématiques}, pdfauthor={Exo7}}


%section
% \usepackage{sectsty}
% \allsectionsfont{\bf}
%\sectionfont{\color{Tomato3}\upshape\selectfont}
%\subsectionfont{\color{Tomato4}\upshape\selectfont}

%----- Ensembles : entiers, reels, complexes -----
\newcommand{\Nn}{\mathbb{N}} \newcommand{\N}{\mathbb{N}}
\newcommand{\Zz}{\mathbb{Z}} \newcommand{\Z}{\mathbb{Z}}
\newcommand{\Qq}{\mathbb{Q}} \newcommand{\Q}{\mathbb{Q}}
\newcommand{\Rr}{\mathbb{R}} \newcommand{\R}{\mathbb{R}}
\newcommand{\Cc}{\mathbb{C}} 
\newcommand{\Kk}{\mathbb{K}} \newcommand{\K}{\mathbb{K}}

%----- Modifications de symboles -----
\renewcommand{\epsilon}{\varepsilon}
\renewcommand{\Re}{\mathop{\text{Re}}\nolimits}
\renewcommand{\Im}{\mathop{\text{Im}}\nolimits}
%\newcommand{\llbracket}{\left[\kern-0.15em\left[}
%\newcommand{\rrbracket}{\right]\kern-0.15em\right]}

\renewcommand{\ge}{\geqslant}
\renewcommand{\geq}{\geqslant}
\renewcommand{\le}{\leqslant}
\renewcommand{\leq}{\leqslant}

%----- Fonctions usuelles -----
\newcommand{\ch}{\mathop{\mathrm{ch}}\nolimits}
\newcommand{\sh}{\mathop{\mathrm{sh}}\nolimits}
\renewcommand{\tanh}{\mathop{\mathrm{th}}\nolimits}
\newcommand{\cotan}{\mathop{\mathrm{cotan}}\nolimits}
\newcommand{\Arcsin}{\mathop{\mathrm{Arcsin}}\nolimits}
\newcommand{\Arccos}{\mathop{\mathrm{Arccos}}\nolimits}
\newcommand{\Arctan}{\mathop{\mathrm{Arctan}}\nolimits}
\newcommand{\Argsh}{\mathop{\mathrm{Argsh}}\nolimits}
\newcommand{\Argch}{\mathop{\mathrm{Argch}}\nolimits}
\newcommand{\Argth}{\mathop{\mathrm{Argth}}\nolimits}
\newcommand{\pgcd}{\mathop{\mathrm{pgcd}}\nolimits} 

\newcommand{\Card}{\mathop{\text{Card}}\nolimits}
\newcommand{\Ker}{\mathop{\text{Ker}}\nolimits}
\newcommand{\id}{\mathop{\text{id}}\nolimits}
\newcommand{\ii}{\mathrm{i}}
\newcommand{\dd}{\mathrm{d}}
\newcommand{\Vect}{\mathop{\text{Vect}}\nolimits}
\newcommand{\Mat}{\mathop{\mathrm{Mat}}\nolimits}
\newcommand{\rg}{\mathop{\text{rg}}\nolimits}
\newcommand{\tr}{\mathop{\text{tr}}\nolimits}
\newcommand{\ppcm}{\mathop{\text{ppcm}}\nolimits}

%----- Structure des exercices ------

\newtheoremstyle{styleexo}% name
{2ex}% Space above
{3ex}% Space below
{}% Body font
{}% Indent amount 1
{\bfseries} % Theorem head font
{}% Punctuation after theorem head
{\newline}% Space after theorem head 2
{}% Theorem head spec (can be left empty, meaning ‘normal’)

%\theoremstyle{styleexo}
\newtheorem{exo}{Exercice}
\newtheorem{ind}{Indications}
\newtheorem{cor}{Correction}


\newcommand{\exercice}[1]{} \newcommand{\finexercice}{}
%\newcommand{\exercice}[1]{{\tiny\texttt{#1}}\vspace{-2ex}} % pour afficher le numero absolu, l'auteur...
\newcommand{\enonce}{\begin{exo}} \newcommand{\finenonce}{\end{exo}}
\newcommand{\indication}{\begin{ind}} \newcommand{\finindication}{\end{ind}}
\newcommand{\correction}{\begin{cor}} \newcommand{\fincorrection}{\end{cor}}

\newcommand{\noindication}{\stepcounter{ind}}
\newcommand{\nocorrection}{\stepcounter{cor}}

\newcommand{\fiche}[1]{} \newcommand{\finfiche}{}
\newcommand{\titre}[1]{\centerline{\large \bf #1}}
\newcommand{\addcommand}[1]{}
\newcommand{\video}[1]{}

% Marge
\newcommand{\mymargin}[1]{\marginpar{{\small #1}}}



%----- Presentation ------
\setlength{\parindent}{0cm}

%\newcommand{\ExoSept}{\href{http://exo7.emath.fr}{\textbf{\textsf{Exo7}}}}

\definecolor{myred}{rgb}{0.93,0.26,0}
\definecolor{myorange}{rgb}{0.97,0.58,0}
\definecolor{myyellow}{rgb}{1,0.86,0}

\newcommand{\LogoExoSept}[1]{  % input : echelle
{\usefont{U}{cmss}{bx}{n}
\begin{tikzpicture}[scale=0.1*#1,transform shape]
  \fill[color=myorange] (0,0)--(4,0)--(4,-4)--(0,-4)--cycle;
  \fill[color=myred] (0,0)--(0,3)--(-3,3)--(-3,0)--cycle;
  \fill[color=myyellow] (4,0)--(7,4)--(3,7)--(0,3)--cycle;
  \node[scale=5] at (3.5,3.5) {Exo7};
\end{tikzpicture}}
}



\theoremstyle{definition}
%\newtheorem{proposition}{Proposition}
%\newtheorem{exemple}{Exemple}
%\newtheorem{theoreme}{Théorème}
\newtheorem{lemme}{Lemme}
\newtheorem{corollaire}{Corollaire}
%\newtheorem*{remarque*}{Remarque}
%\newtheorem*{miniexercice}{Mini-exercices}
%\newtheorem{definition}{Définition}




%definition d'un terme
\newcommand{\defi}[1]{{\color{myorange}\textbf{\emph{#1}}}}
\newcommand{\evidence}[1]{{\color{blue}\textbf{\emph{#1}}}}



 %----- Commandes divers ------

\newcommand{\codeinline}[1]{\texttt{#1}}

%%%%%%%%%%%%%%%%%%%%%%%%%%%%%%%%%%%%%%%%%%%%%%%%%%%%%%%%%%%%%
%%%%%%%%%%%%%%%%%%%%%%%%%%%%%%%%%%%%%%%%%%%%%%%%%%%%%%%%%%%%%



\begin{document}

\debuttexte

%%%%%%%%%%%%%%%%%%%%%%%%%%%%%%%%%%%%%%%%%%%%%%%%%%%%%%%%%%
\diapo

\change

Nous allons enrichir notre bibliothèque de fonctions avec
six nouvelles fonctions :

\change

tout d'abord le cosinus hyperbolique.


A la fonction cosinus classique nous avons associé arccosinus,

ici nous verrons que l'on peut associer au cosinus hyperbolique 
une bijection réciproque qui s'appelle l'argument cosinus hyperbolique.

\change

Nous ferrons le même travail pour le sinus hyperbolique

\change

et la tangente hyperbolique.

\change

On terminera par une liste de formules à retenir.




%%%%%%%%%%%%%%%%%%%%%%%%%%%%%%%%%%%%%%%%%%%%%%%%%%%%%%%%%%
\diapo

Voici la définition du cosinus hyperbolique.

Pour $x$ un nombre réel on définit 

$\ch x = \frac{e^x+e^{-x}}{2}$

où $e^x$ désigne bien sûr $exp(x)$.

Notez l'analogie avec la formule d'Euler 
pour le cosinus classique 
$\cos x = \frac{e^{ix}+e^{-ix}}{2}$


\change

On considère l'application qui à $x$ associe $\ch x$. Cela définit une fonction de $\Rr$ dans $\Rr$.

Pour avoir une fonction bijective, il faut restreindre les intervalles
de départ et d'arrivée :
ainsi la restriction $\ch_| : [0,+\infty[ \to [1,+\infty[$ est une bijection

\change

La bijection réciproque est une nouvelle fonction qui s'appelle l'argument cosinus hyperbolique, 

et que l'on note $\Argch$; c'est une fonction qui va de $[1,+\infty[$ 
dans $[0,+\infty[$.


\change


Voici tout d'abord le graphe de la fonction cosinus hyperbolique 
sur $\Rr$
tout entier.

Grâce à la définition, on vérifie que $\ch(-x)=\ch(x)$,

ainsi $\ch$ est une fonction paire. 

Vous voyez que le minimum est atteint en $x=0$ avec une valeur de $1$ et 

lorsque $x\to+\infty$, $\ch x \to +\infty$.



\change

Si maintenant on restreint l'intervalle de départ aux seuls $x\ge0$,
alors nous avons une bijection.
Et voici le graphe de $\Argch$.

C'est une fonction qui n'est définit que pour $x\ge 1$.
En $x=1$ elle vaut $0$ et elle croit vers $+\infty$ 
lorsque $x$ tend vers $+\infty$
comme le logarithme.


%%%%%%%%%%%%%%%%%%%%%%%%%%%%%%%%%%%%%%%%%%%%%%%%%%%%%%%%%%
\diapo

Ces fonctions apparaissent naturellement dans la résolution de 
problèmes simples, en particulier issus de la physique.
Par exemple lorsqu'un fil est suspendu entre deux poteaux 

[[amener corde]]

alors la courbe dessinée est une \defi{chaînette} 
dont l'équation fait intervenir le cosinus hyperbolique
et un paramètre $a$ 
(qui dépend de la longueur du fil et de l'écartement des poteaux)

L'équation de la courbe est donnée par 
$y = a\ch \left( \frac x a \right)$


%%%%%%%%%%%%%%%%%%%%%%%%%%%%%%%%%%%%%%%%%%%%%%%%%%%%%%%%%%
\diapo

Pour $x\in \Rr$, le \defi{sinus hyperbolique} est :

$\sh x = \frac{e^x-e^{-x}}{2}$

\change

En tant que fonction le sinus hyperbolique est une fonction de $\Rr$ dans 
$\Rr$ qui est une bijection.

\change

Sa bijection réciproque est l'argument sinus hyperbolique, 
notée $\Argsh$ 


\change

Voici le graphe du sinus hyperbolique.

Comme $\sh(-x)=-\sh(x)$ alors
la fonction $\sh$ est impaire.

Elle tend vers $-\infty$ en $-\infty$ et $+\infty$ en $+\infty$.

\change

Voici la graphe de argument sinus hyperbolique.
C'est aussi une fonction impaire.


%%%%%%%%%%%%%%%%%%%%%%%%%%%%%%%%%%%%%%%%%%%%%%%%%%%%%%%%%%
\diapo

Voici dans le même repère les graphes des fonctions ch et sh.

La fonction ch est une fonction paire, son graphe -en rouge- 
est symétrique par rapport à l'axe des $y$.

La fonction sh est impaire, son graphe est symétrique par rapport à l'origine.

$\sh x$ est toujours inférieur à $\ch x$ mais lorsque $x \to +\infty$
les deux fonctions se comportent toutes deux comme $e^x/2$.


\change

Voici les graphes des bijections réciproques,

la fonction $\Argsh$ est définie sur $\Rr$ alors
que $\Argch x$ n'est défini que pour $x \ge 1$.


%%%%%%%%%%%%%%%%%%%%%%%%%%%%%%%%%%%%%%%%%%%%%%%%%%%%%%%%%%
\diapo


Voici une proposition qui regroupe quelques résultats.
Je vous encourage à faire les démonstrations par vous même.


Tout d'abord la formule $\ch^2 x - \sh^2 x =1$

qu'il faut bien sûr mettre en parallèle avec $\cos^2 + \sin^2 =1$.



\change


Pour les dérivées, la situation est très simple et symétrique :
la dérivée de $\ch$ est $\sh$ la dérivée de $\sh$ est $\ch$.


\change 

 $\Argsh$ est une fonction strictement croissante et continue.
 
 C'est bien sûr une bijection et sa réciproque est $\sh$.
 
 \change
 
$\Argsh$ est dérivable  et on calcule sa dérivée comme dérivée de la 
bijection réciproque de $\sh x$, on obtient $\Argsh'x=\frac{1}{\sqrt{x^2+1}}$.

\change


Enfin une formule utile $\Argsh x = \ln\big(x+ \sqrt{x^2+1}\big)$.
On vérifie facilement cette égalité par dérivation.

  
Il existe de telles formules pour le cosinus hyperbolique, 
nous y reviendrons.
  

%%%%%%%%%%%%%%%%%%%%%%%%%%%%%%%%%%%%%%%%%%%%%%%%%%%%%%%%%%
\diapo


Par analogie avec la tangente usuelle la tangente hyperbolique 
est le quotient du sinus hyperbolique par le cosinus hyperbolique :

$\displaystyle \tanh x = \frac{\sh x}{\ch x}$

\change

En tant que fonction $x \mapsto \tanh x$  va de $\Rr$ vers l'intervalle $]-1,1[$

et c'est une bijection.


Sa bijection réciproque va donc de $]-1,1[\to\Rr$


et par définition on l'appelle $\Argth$.

\change

Voici le graphe de $\tanh$ c'est une fonction continue, dérivable,
strictement croissante.

La limite en $-\infty$ est $-1$, la limite en $+\infty$ est $+1$.
On a deux asymptotes horizontales.

Sa bijection réciproque est aussi continue, dérivable,
strictement croissante. Et bien sûr, en tant que bijection réciproque,
on obtient deux asymptotes verticales en $-1$ et $+1$.

\change


%%%%%%%%%%%%%%%%%%%%%%%%%%%%%%%%%%%%%%%%%%%%%%%%%%%%%%%%%%
\diapo

Comme pour les formules trigonométriques classiques, il y a des formules 
de trigonométrie hyperbolique. Et comme pour les formules trigonométriques classiques
elles sont à apprendre !

On commence par la formule de base :

$$\ch^2 x - \sh^2 x = 1$$

Pour le prouver, il s'agit juste de remplacer $\ch x$ et $\sh x$
par les formules des définitions, puis de simplifier !

\change

Ensuite voici les formules d'addition.

\begin{align*}
\ch(a+b)&=\ch a\cdot\ch b + \sh a\cdot\sh b\\
\sh(a+b)&=\sh a\cdot\ch b  +  \sh b\cdot\ch a\\
\tanh (a+b)&=\frac{\tanh a + \tanh b}{1+\tanh a\cdot\tanh b}\\
\end{align*}

Remarquez qu'elles sont plus simples que pour cos, sin tan, 
car il n'y a que des signes "+".

\change

Et voici maintenant les formules des dérivées, qui elles aussi sont assez simples :

\begin{align*}
\ch'x&= \sh x\\
\sh'x&=\ch x\\
\tanh' x &= 1-\tanh^2x=\frac{1}{\ch^2x}\\
\end{align*}


%%%%%%%%%%%%%%%%%%%%%%%%%%%%%%%%%%%%%%%%%%%%%%%%%%%%%%%%%%
\diapo

On termine par les formules pour les fonctions réciproques. 

Tout d'abord les dérivées :


\begin{align*}
\Argch'x&=\frac{1}{\sqrt{x^2-1}} \quad (x>1)\\
\Argsh'x&=\frac{1}{\sqrt{x^2+1}} \\
\Argth'x&=\frac{1}{1-x^2} \quad (|x|<1)\\
\end{align*}

\change


Enfin voici les formules appelées formules logarithmiques.
Elles expriment les fonctions hyperboliques inverses en fonction du logarithme.


\begin{align*}
\Argch x&= \ln\big(x+ \sqrt{x^2-1}\big) \quad (x\ge1)\\
\Argsh x&= \ln\big(x+ \sqrt{x^2+1}\big)  \quad (x \in \Rr) \\
\Argth x&= \frac12\ln\left(\frac{1+x}{1-x}\right) \quad (-1<x<1) \\
\end{align*}

Je vous encourage à les démontrer !



%%%%%%%%%%%%%%%%%%%%%%%%%%%%%%%%%%%%%%%%%%%%%%%%%%%%%%%%%%
\diapo

Pour cette leçon il faut bien connaître les formules et ensuite 
les appliquer !




\end{document}

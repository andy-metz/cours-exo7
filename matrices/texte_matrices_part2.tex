
%%%%%%%%%%%%%%%%%% PREAMBULE %%%%%%%%%%%%%%%%%%


\documentclass[12pt]{article}

\usepackage{amsfonts,amsmath,amssymb,amsthm}
\usepackage[utf8]{inputenc}
\usepackage[T1]{fontenc}
\usepackage[francais]{babel}


% packages
\usepackage{amsfonts,amsmath,amssymb,amsthm}
\usepackage[utf8]{inputenc}
\usepackage[T1]{fontenc}
%\usepackage{lmodern}

\usepackage[francais]{babel}
\usepackage{fancybox}
\usepackage{graphicx}

\usepackage{float}

%\usepackage[usenames, x11names]{xcolor}
\usepackage{tikz}
\usepackage{datetime}

\usepackage{mathptmx}
%\usepackage{fouriernc}
%\usepackage{newcent}
\usepackage[mathcal,mathbf]{euler}

%\usepackage{palatino}
%\usepackage{newcent}


% Commande spéciale prompteur

%\usepackage{mathptmx}
%\usepackage[mathcal,mathbf]{euler}
%\usepackage{mathpple,multido}

\usepackage[a4paper]{geometry}
\geometry{top=2cm, bottom=2cm, left=1cm, right=1cm, marginparsep=1cm}

\newcommand{\change}{{\color{red}\rule{\textwidth}{1mm}\\}}

\newcounter{mydiapo}

\newcommand{\diapo}{\newpage
\hfill {\normalsize  Diapo \themydiapo \quad \texttt{[\jobname]}} \\
\stepcounter{mydiapo}}


%%%%%%% COULEURS %%%%%%%%%%

% Pour blanc sur noir :
%\pagecolor[rgb]{0.5,0.5,0.5}
% \pagecolor[rgb]{0,0,0}
% \color[rgb]{1,1,1}



%\DeclareFixedFont{\myfont}{U}{cmss}{bx}{n}{18pt}
\newcommand{\debuttexte}{
%%%%%%%%%%%%% FONTES %%%%%%%%%%%%%
\renewcommand{\baselinestretch}{1.5}
\usefont{U}{cmss}{bx}{n}
\bfseries

% Taille normale : commenter le reste !
%Taille Arnaud
%\fontsize{19}{19}\selectfont

% Taille Barbara
%\fontsize{21}{22}\selectfont

%Taille François
\fontsize{25}{30}\selectfont

%Taille Pascal
%\fontsize{25}{30}\selectfont

%Taille Laura
%\fontsize{30}{35}\selectfont


%\myfont
%\usefont{U}{cmss}{bx}{n}

%\Huge
%\addtolength{\parskip}{\baselineskip}
}


% \usepackage{hyperref}
% \hypersetup{colorlinks=true, linkcolor=blue, urlcolor=blue,
% pdftitle={Exo7 - Exercices de mathématiques}, pdfauthor={Exo7}}


%section
% \usepackage{sectsty}
% \allsectionsfont{\bf}
%\sectionfont{\color{Tomato3}\upshape\selectfont}
%\subsectionfont{\color{Tomato4}\upshape\selectfont}

%----- Ensembles : entiers, reels, complexes -----
\newcommand{\Nn}{\mathbb{N}} \newcommand{\N}{\mathbb{N}}
\newcommand{\Zz}{\mathbb{Z}} \newcommand{\Z}{\mathbb{Z}}
\newcommand{\Qq}{\mathbb{Q}} \newcommand{\Q}{\mathbb{Q}}
\newcommand{\Rr}{\mathbb{R}} \newcommand{\R}{\mathbb{R}}
\newcommand{\Cc}{\mathbb{C}} 
\newcommand{\Kk}{\mathbb{K}} \newcommand{\K}{\mathbb{K}}

%----- Modifications de symboles -----
\renewcommand{\epsilon}{\varepsilon}
\renewcommand{\Re}{\mathop{\text{Re}}\nolimits}
\renewcommand{\Im}{\mathop{\text{Im}}\nolimits}
%\newcommand{\llbracket}{\left[\kern-0.15em\left[}
%\newcommand{\rrbracket}{\right]\kern-0.15em\right]}

\renewcommand{\ge}{\geqslant}
\renewcommand{\geq}{\geqslant}
\renewcommand{\le}{\leqslant}
\renewcommand{\leq}{\leqslant}

%----- Fonctions usuelles -----
\newcommand{\ch}{\mathop{\mathrm{ch}}\nolimits}
\newcommand{\sh}{\mathop{\mathrm{sh}}\nolimits}
\renewcommand{\tanh}{\mathop{\mathrm{th}}\nolimits}
\newcommand{\cotan}{\mathop{\mathrm{cotan}}\nolimits}
\newcommand{\Arcsin}{\mathop{\mathrm{Arcsin}}\nolimits}
\newcommand{\Arccos}{\mathop{\mathrm{Arccos}}\nolimits}
\newcommand{\Arctan}{\mathop{\mathrm{Arctan}}\nolimits}
\newcommand{\Argsh}{\mathop{\mathrm{Argsh}}\nolimits}
\newcommand{\Argch}{\mathop{\mathrm{Argch}}\nolimits}
\newcommand{\Argth}{\mathop{\mathrm{Argth}}\nolimits}
\newcommand{\pgcd}{\mathop{\mathrm{pgcd}}\nolimits} 

\newcommand{\Card}{\mathop{\text{Card}}\nolimits}
\newcommand{\Ker}{\mathop{\text{Ker}}\nolimits}
\newcommand{\id}{\mathop{\text{id}}\nolimits}
\newcommand{\ii}{\mathrm{i}}
\newcommand{\dd}{\mathrm{d}}
\newcommand{\Vect}{\mathop{\text{Vect}}\nolimits}
\newcommand{\Mat}{\mathop{\mathrm{Mat}}\nolimits}
\newcommand{\rg}{\mathop{\text{rg}}\nolimits}
\newcommand{\tr}{\mathop{\text{tr}}\nolimits}
\newcommand{\ppcm}{\mathop{\text{ppcm}}\nolimits}

%----- Structure des exercices ------

\newtheoremstyle{styleexo}% name
{2ex}% Space above
{3ex}% Space below
{}% Body font
{}% Indent amount 1
{\bfseries} % Theorem head font
{}% Punctuation after theorem head
{\newline}% Space after theorem head 2
{}% Theorem head spec (can be left empty, meaning ‘normal’)

%\theoremstyle{styleexo}
\newtheorem{exo}{Exercice}
\newtheorem{ind}{Indications}
\newtheorem{cor}{Correction}


\newcommand{\exercice}[1]{} \newcommand{\finexercice}{}
%\newcommand{\exercice}[1]{{\tiny\texttt{#1}}\vspace{-2ex}} % pour afficher le numero absolu, l'auteur...
\newcommand{\enonce}{\begin{exo}} \newcommand{\finenonce}{\end{exo}}
\newcommand{\indication}{\begin{ind}} \newcommand{\finindication}{\end{ind}}
\newcommand{\correction}{\begin{cor}} \newcommand{\fincorrection}{\end{cor}}

\newcommand{\noindication}{\stepcounter{ind}}
\newcommand{\nocorrection}{\stepcounter{cor}}

\newcommand{\fiche}[1]{} \newcommand{\finfiche}{}
\newcommand{\titre}[1]{\centerline{\large \bf #1}}
\newcommand{\addcommand}[1]{}
\newcommand{\video}[1]{}

% Marge
\newcommand{\mymargin}[1]{\marginpar{{\small #1}}}



%----- Presentation ------
\setlength{\parindent}{0cm}

%\newcommand{\ExoSept}{\href{http://exo7.emath.fr}{\textbf{\textsf{Exo7}}}}

\definecolor{myred}{rgb}{0.93,0.26,0}
\definecolor{myorange}{rgb}{0.97,0.58,0}
\definecolor{myyellow}{rgb}{1,0.86,0}

\newcommand{\LogoExoSept}[1]{  % input : echelle
{\usefont{U}{cmss}{bx}{n}
\begin{tikzpicture}[scale=0.1*#1,transform shape]
  \fill[color=myorange] (0,0)--(4,0)--(4,-4)--(0,-4)--cycle;
  \fill[color=myred] (0,0)--(0,3)--(-3,3)--(-3,0)--cycle;
  \fill[color=myyellow] (4,0)--(7,4)--(3,7)--(0,3)--cycle;
  \node[scale=5] at (3.5,3.5) {Exo7};
\end{tikzpicture}}
}



\theoremstyle{definition}
%\newtheorem{proposition}{Proposition}
%\newtheorem{exemple}{Exemple}
%\newtheorem{theoreme}{Théorème}
\newtheorem{lemme}{Lemme}
\newtheorem{corollaire}{Corollaire}
%\newtheorem*{remarque*}{Remarque}
%\newtheorem*{miniexercice}{Mini-exercices}
%\newtheorem{definition}{Définition}




%definition d'un terme
\newcommand{\defi}[1]{{\color{myorange}\textbf{\emph{#1}}}}
\newcommand{\evidence}[1]{{\color{blue}\textbf{\emph{#1}}}}



 %----- Commandes divers ------

\newcommand{\codeinline}[1]{\texttt{#1}}

%%%%%%%%%%%%%%%%%%%%%%%%%%%%%%%%%%%%%%%%%%%%%%%%%%%%%%%%%%%%%
%%%%%%%%%%%%%%%%%%%%%%%%%%%%%%%%%%%%%%%%%%%%%%%%%%%%%%%%%%%%%



\begin{document}

\debuttexte


%%%%%%%%%%%%%%%%%%%%%%%%%%%%%%%%%%%%%%%%%%%%%%%%%%%%%%%%%%%
\diapo

\change
Dans la leçon précédente, nous avons vu comment additionner deux matrices, 
ainsi que multiplier une matrice par un scalaire. Nous poursuivons à présent 
par une leçon consacrée à la multiplication des matrices entre elles.

\change
Nous commencerons par définir cette multiplication.
% 
% \change
% et nous en donnerons quelques exemples.

\change
Nous verrons ensuite les propriétés du produit de matrices, qui sont parfois assez 
différentes de celles du produit des nombres réels.

\change
Nous définirons la matrice identité, qui joue un rôle analogue à celui du nombre 1 pour les réels.

\change
Nous verrons également comme définir la puissance d'une matrice, ainsi qu'une formule du binôme pour les matrices.


%%%%%%%%%%%%%%%%%%%%%%%%%%%%%%%%%%%%%%%%%%%%%%%%%%%%%%%%%%
\diapo

Le produit $AB$ de deux matrices $A$ et $B$ est défini si et seulement si le nombre de colonnes de 
$A$ est égal au nombre de lignes de $B$. Soient $A=(a_{ij})$ une matrice $n\times p$ et $B=(b_{ij})$ une matrice $p\times q$. 

\change
Alors le produit $C=AB$ est une matrice $n\times q$

\change
 dont les coefficients $c_{ij}$ sont définis par :

$
c_{ij} = \sum_{k=1}^p a_{ik}b_{kj}
$

\change
On peut écrire le coefficient de façon plus développée, à savoir : 
$$c_{ij}=a_{i1}b_{1j}+a_{i2}b_{2j}+ \dots + 
a_{ik}b_{kj}+ \dots + a_{ip}b_{pj}.$$ 

On voit ici apparaître tous les éléments de la ligne numéro $i$ de la matrice $A$, et ici les éléments de la colonne numéro $j$ de $B$.

\change
Il est commode de disposer les calculs de la façon suivante.

Avec cette disposition, on considère d'abord la ligne de la matrice $A$ située à gauche du coefficient $c_{ij}$ que l'on veut calculer (ligne représentée par des $\times $ bleus dans $A$), c'est-à-dire la ligne numéro $i$. Et aussi la colonne de la matrice $B$ située au-dessus du coefficient que l'on veut calculer (colonne représentée par des $\times$ mauve dans $B$), c'est-à-dire la colonne numéro $j$.

On calcule le produit du premier coefficient de la ligne par le premier coefficient 
de la colonne ($a_{i1} \times b_{1j}$), que l'on ajoute au produit du deuxième coefficient de la ligne par le deuxième coefficient 
de la colonne ($a_{i2} \times b_{2j}$), que l'on ajoute au produit du troisième\ldots

On voit donc qu'il est indispensable que cette ligne et cette colonne aient le même nombre $p$ de coefficients.


%%%%%%%%%%%%%%%%%%%%%%%%%%%%%%%%%%%%%%%%%%%%%%%%%%%%%%%%%%%
\diapo

Passons tout de suite à un exemple. Considérons les deux matrices $A$ et $B$ suivantes. 
La matrice $A$ est de taille $2\times 3$ et la matrice $B$ de taille $3\times 2$. 
Il est donc possible de calculer le produit $AB$ car $A$ a $3$ colonnes et $B$, $3$ lignes.

\change
On dispose d'abord le produit correctement : la matrice obtenue sera de taille $2\times2$. Puis on calcule les coefficients un par un,
en commençant par le premier coefficient $c_{11} $

\change
 $c_{11}= 1\times 1\ +\ 2\times(-1)\ +\ 3\times1=2$

\change
De même, $c_{12}=1\times2+2\times1+3\times1=7$.

\change
Et on calcule de la même manière $c_{21}$, on trouve $3$

\change
et $c_{22}=11$.



%%%%%%%%%%%%%%%%%%%%%%%%%%%%%%%%%%%%%%%%%%%%%%%%%%%%%%%%%%
\diapo
Un exemple intéressant est le produit d'un vecteur ligne par un vecteur colonne. On considère un vecteur ligne $u = \begin{pmatrix} a_1 & a_2 & \cdots & a_n \end{pmatrix} $ et un vecteur colonne $v = \begin{pmatrix} b_1 \\ b_2 \\ \vdots \\ b_n \end{pmatrix}$.

\change
Alors $u \times v$ est une matrice de taille $1\times 1$ dont l'unique coefficient est
$a_1 b_1 + a_2 b_2 + \cdots + a_n b_n$.

\change
Ce nombre s'appelle le \defi{produit scalaire} des vecteurs $u$ et $v$.


De manière générale, calculer le coefficient $c_{ij}$ dans le produit $A\times B$
revient donc à calculer le produit scalaire des vecteurs formés 
par la $i$-ème ligne de $A$ et la $j$-ème colonne de $B$.


%%%%%%%%%%%%%%%%%%%%%%%%%%%%%%%%%%%%%%%%%%%%%%%%%%%%%%%%%%%
\diapo

Il faut faire attention au fait que le produit entre deux matrices 
se comporte parfois de manière très différente du produit entre deux nombres réels. Voici donc quelques pièges à éviter.

Tout d'abord, le produit de matrices n'est pas commutatif en général. En effet, il se peut que $AB$ soit défini mais pas $BA$, ou que $AB$ et $BA$ 
soient tous deux définis mais pas de la même taille. 
Mais même dans le cas où $AB$ et $BA$ sont définis et de la même taille, on a en général $AB\neq BA$. 

\change
Par exemple, ces deux matrices sont carrées de taille $2\times2$. On peut donc calculer à la fois $AB$ et $BA$, 
et on obtient à chaque fois une matrice carrée  $2\times2$. 


\change
Mais on voit en faisant le calcul que $AB$ est cette matrice

\change
qui est différente de la matrice $BA$.

\change
Deuxième piège classique à éviter : $AB=0$ n'implique pas $A=0$ ou $B=0$.

\change
C'est-à-dire que l'on peut trouver deux matrices $A$ et $B$ dont le produit $AB$ est nul, 
mais telles que ni $A$ ni $B$ ne soit nulle, comme c'est le cas sur cet exemple.

\change

Troisième piège à éviter du même style : $AB=AC$ n'implique pas $B=C$.
On peut avoir à la fois $AB  = AC$ et $B \neq C$ : on ne peut donc pas "simplifier" par $A$.

% %%%%%%%%%%%%%%%%%%%%%%%%%%%%%%%%%%%%%%%%%%%%%%%%%%%%%%%%%%
% \diapo
% 
% Troisième piège à éviter : $AB=AC$ n'implique pas $B=C$.
% 
% 
% \change
% Par exemple si on considère les trois matrices carrées $2\times2$ suivantes, alors on peut vérifier facilement que les produits $AB$ et $AC$ coïncident, bien que les matrices $B$ et $C$ soient clairement différentes.


%%%%%%%%%%%%%%%%%%%%%%%%%%%%%%%%%%%%%%%%%%%%%%%%%%%%%%%%%%%
\diapo
Malgré les difficultés soulevées juste avant, le produit vérifie les propriétés suivantes :

Tout d'abord,  $A (BC) = (AB) C$, on dit que le produit est associatif, 
 
\change
 $A(B+C) = AB + AC$ \ et \ $(B+C) A = BA + CA$ : le produit est distributif par rapport à la somme,

\change
enfin, $A\cdot 0 = 0$ \ et \ $0\cdot A= 0$, où $0$ désigne la matrice nulle.
 
% 
% %%%%%%%%%%%%%%%%%%%%%%%%%%%%%%%%%%%%%%%%%%%%%%%%%%%%%%%%%%
% \diapo
% Démontrons la première propriété, c'est-à-dire l'associativité du produit.
% 
% Posons $A=(a_{ij}) \in M_{n,p}(\Kk)$, $B=(b_{ij})\in M_{p,q}(\Kk)$ 
% et $C=(c_{ij})\in M_{q,r}(\Kk)$.  
% 
% \change
% Prouvons que $A(BC) = (AB) C$ en montrant que les matrices $A(BC)$ et $(AB) C$ ont les mêmes coefficients.
% 
% \change
% Le coefficient d'indice $(i,k)$ de la matrice $AB$ est 
% $x_{ik}={\displaystyle \sum_{\ell=1}^{p}}a_{i \ell}b_{\ell k}$.
% 
% \change
% Le coefficient d'indice $(i,j)$ de la matrice $(AB)C$ est donc
% $$ \sum_{k=1}^{q}x_{ik}c_{kj}=\sum_{k=1}^{q}
% \left ( \sum_{\ell=1}^{p}a_{i\ell}b_{\ell k} \right )c_{kj}.$$
% 
% \change
% Par ailleurs, le coefficient d'indice $(\ell,j)$ de la matrice $BC$ est 
% $y_{\ell j}={\displaystyle \sum_{k=1}^{q}}b_{\ell k}c_{kj}$. 
% 
% \change
% Le coefficient d'indice $(i,j)$ de la matrice $A(BC)$ est donc
% $$\sum_{\ell=1}^{p}a_{i\ell}\left (  \sum_{k=1}^{q}b_{\ell k}c_{kj}\right ).$$
% 
% \change
% A présent, comme dans $\Kk$ la multiplication est distributive et associative, 
% les coefficients de $(AB)C$ et $A(BC)$ coïncident. 
% 
% Les autres démonstrations se font comme celle de l'associativité. 


%%%%%%%%%%%%%%%%%%%%%%%%%%%%%%%%%%%%%%%%%%%%%%%%%%%%%%%%%%%
\diapo

La matrice carrée suivante s'appelle \defi{la matrice identité} : 
 \[
 I_n = \left(
 \begin{array}{cccc}
1 & 0 & \dots & 0\\
0& 1& \dots & 0\\
 \vdots& \vdots & \ddots  & \vdots\\
0 & 0 & \dots &1
\end{array}
\right)
 \]

Ses éléments diagonaux sont égaux à $1$ et tous ses autres éléments sont égaux à $0$.
Elle se note $I_n$ ou simplement $I$.

\change
On peut formaliser cela en introduisant le symbole de Kronecker.
Si $i$ et $j$ sont deux entiers, on appelle \defi{symbole de Kronecker}, et on note $\delta_{i,j}$, le réel qui vaut $0$ si $i$ est différent de $j$, et $1$ si $i$ est égal à $j$. 

\change
Alors le terme général de la matrice identité $I_{n}$ est $\delta_{i,j}$  avec $i$ et $j$ entiers, compris entre $1$ et $n$.


%%%%%%%%%%%%%%%%%%%%%%%%%%%%%%%%%%%%%%%%%%%%%%%%%%%%%%%%%%%
\diapo

Dans le calcul matriciel, la matrice identité joue un rôle
analogue à celui du nombre $1$ pour les réels.
C'est l'élément neutre pour la multiplication. En
d'autres termes :

Si $A$ est une matrice $n \times p$, alors 
$$ I_n \cdot A = A \qquad \text{et} \qquad A \cdot I_p = A.$$

\change

Nous allons détailler la preuve.
Soit  $A \in M_{n,p}(\Kk)$ de terme général $a_{ij}$. Montrons que $AI_p=A$, où la matrice unité d'ordre $p$ est donc telle que tous les éléments de la diagonale principale sont égaux à $1$, les autres étant tous nuls. 

\change
La matrice produit $AI_{p}$ est une matrice appartenant à $M_{n,p}(\Kk)$
dont le terme général $c_{ij}$ est donné par la formule 
$c_{ij}={\displaystyle \sum_{k=1}^{p}a_{ik}\delta_{kj}}$, où $\delta_{kj}$ est le symbole de Kronecker. Dans cette somme, $i$ et $j$ sont fixés et $k$ prend toutes les valeurs comprises entre $1$ et $p$. 

\change
Si $k\neq j$ alors $\delta_{kj}=0$,

\change
et si $k=j$ alors $\delta_{kj}=1$.

\change
Donc dans cette somme, tous les termes correspondant à des valeurs de $k$ différentes de $j$ 
sont nuls et il reste donc $c_{ij}=a_{ij}\delta_{jj}=a_{ij}1=a_{ij}$.

\change
Donc les matrices $AI_{p}$ et $A$ ont le même terme général : elles sont donc égales.

\change
L'égalité $I_{n}A=A$ se démontre de la même façon.


%%%%%%%%%%%%%%%%%%%%%%%%%%%%%%%%%%%%%%%%%%%%%%%%%%%%%%%%%%
\diapo

Dans l'ensemble $M_{n}(\Kk)$ des matrices carrées de taille $n \times n$ à
coefficients dans $\Kk$,  la multiplication des matrices est une  
opération interne : si $A,B \in M_n(\Kk)$ alors $AB \in M_n(\Kk)$.

\change
En particulier, on peut multiplier une matrice carrée par elle-même :
on note $A^2 = A \times A$, $A^3 = A \times A \times A$.

\change
On peut ainsi définir les puissances successives d'une matrice :

$A^p = \underbrace{A \times A \times \cdots \times A}_{p \text{ facteurs}}$.

\change

Et par définition $A^0$ est la matrice identité.


%%%%%%%%%%%%%%%%%%%%%%%%%%%%%%%%%%%%%%%%%%%%%%%%%%%%%%%%%%%
\diapo

On cherche à calculer $A^{p}$ où $A$ est la matrice suivante.

\change
On commence par calculer $A^{2}$, $A^3$ et $A^{4}$

\change
et on obtient les trois matrices suivantes. 

\change
\change

Regardons de plus près ces quatre matrices $A$, $A^{2}$, $A^3$ et $A^{4}$. On remarque qu'elles ont toutes la même forme, avec des zéros au mêmes endroits ici et là. Le coefficient $(1,1)$ vaut toujours $1$. Le coefficient $(2,2)$ prend les valeurs $1$ et $-1$. Le coefficient en bas à droite prend successivement les valeurs 2, 4, 8, 16, et le coefficient en haut à droite prend ces mêmes valeurs moins une unité, c'est-à-dire 1, 3, 7, 15.

\change
L'observation de ces premières puissances permet donc de penser que la formule
pourrait bien être : $A^{p}$ qui vaut cette matrice avec sur la diagonale $1,\  (-1)^{p}, \ 2^p $ et en haut à droite $2^p-1$. 

Il ne s'agit pour l'instant que d'une conjecture, c'est-à-dire d'une assertion que l'on pense être vraie, et qu'il nous faut démontrer. 


%%%%%%%%%%%%%%%%%%%%%%%%%%%%%%%%%%%%%%%%%%%%%%%%%%%%%%%%%%
\diapo

Démontrons ce résultat par récurrence.

\change
Tout d'abord l'initialisation. Ce résultat est vrai pour $p=0$, puisqu'on trouve l'identité. 

\change
A présent l'hérédité. On suppose cette formule  vraie pour un entier $p$ fixé et on va en déduire qu'elle est alors vraie aussi pour $p+1$.

\change
On a 
$A ^{p+1}=A ^{p} \times A $

\change
et donc d'après l'hypothèse de récurrence, on obtient ce produit.

\change
En calculant ce produit, on obtient pour $A ^{p+1}$ la valeur suivante, c'est-à-dire qu'on retrouve bien cette formule, 
mais au rang $p+1$.

\change
Donc, d'après le principe de récurrence, la propriété est démontrée. 


%%%%%%%%%%%%%%%%%%%%%%%%%%%%%%%%%%%%%%%%%%%%%%%%%%%%%%%%%%
\diapo

On termine par la formule de binôme de Newton pour les matrices, que vous pouvez passer en première lecture.


Comme la multiplication n'est pas commutative, les identités binomiales usuelles sont fausses. 
En particulier, $(A+B)^2$ ne vaut en général pas $A^2+2AB+B^2$, mais on sait seulement que 
$(A+B)^2= A^2+{\color{myred}AB+BA}+B^2.$

\change
La proposition suivante permet de calculer $(A+B)^{p}$ dans le cas particulier où les matrices $A$ et $B$ commutent.

Dire que deux matrices $A$ et $B$ \defi{commutent}, c'est dire que $AB=BA$. 

\change
Alors, pour tout entier $p \ge 0$, on a la formule 
$$(A+B)^{p}= \sum_{k=0}^{p} \binom{p}{k} A^{p-k}B^{k}$$
où $\binom{p}{k}$ désigne le coefficient usuel du binôme.

La démonstration de cette proposition est similaire à celle de la formule du binôme pour $(a+b)^p$, avec $a,b \in \Rr$.


%%%%%%%%%%%%%%%%%%%%%%%%%%%%%%%%%%%%%%%%%%%%%%%%%%%%%%%%%%%
\diapo

Calculons les puissances de la matrice $A$ suivante, qui a des $1$ sur la diagonale, et des zéros sous la diagonale.

\change
On pose $N$ la matrice $A$ moins l'identité, et qui a donc des zéros sur la diagonale et les mêmes coefficients que $A$ ailleurs.

\change
Alors on obtient pour $N^2$ cette matrice qui a également des zéros ici

\change
pour $N^3$ encore plus de zéros, 

\change
et finalement on obtient que $N^4$ est nulle.

On dit que la matrice $N$ est nilpotente c'est-à-dire il existe un entier $k$ tel que $N^{k}=0$.

%%%%%%%%%%%%%%%%%%%%%%%%%%%%%%%%%%%%%%%%%%%%%%%%%%%%%%%%%%
\diapo
Comme on a $A=I+N$ et les matrices $N$ et $I$ commutent (car la matrice identité 
commute avec toutes les matrices), on peut appliquer la formule du binôme de Newton. 

\change

On obtient ainsi
$
A^{p} = \sum_{k=0}^{p} \binom{p}{k} N^{k} I^{p-k}$

\change
Et comme toutes les puissance de l'identité valent l'identité, ceci vaut simplement $  \sum \binom{p}{k} N^{k} $
 
\change
Or $N^{k}=0$ dès que $k \geq 4$, et il ne reste donc plus dans la somme que les termes pour $k$ variant de $0$ à $3$.

\change
En développant on obtient $I+pN+\tfrac{p(p-1)}{2!}N^{2}+ \tfrac{p(p-1)(p-2)}{3!}N^{3}$.

\change
ce qui donne pour $A^p$ la formule suivante.





%%%%%%%%%%%%%%%%%%%%%%%%%%%%%%%%%%%%%%%%%%%%%%%%%%%%%%%%%%%
\diapo

Entraînez-vous avec ces exercices pour vérifier que vous avez bien compris le cours.



\end{document}

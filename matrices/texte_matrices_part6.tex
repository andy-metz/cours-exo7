
%%%%%%%%%%%%%%%%%% PREAMBULE %%%%%%%%%%%%%%%%%%


\documentclass[12pt]{article}

\usepackage{amsfonts,amsmath,amssymb,amsthm}
\usepackage[utf8]{inputenc}
\usepackage[T1]{fontenc}
\usepackage[francais]{babel}


% packages
\usepackage{amsfonts,amsmath,amssymb,amsthm}
\usepackage[utf8]{inputenc}
\usepackage[T1]{fontenc}
%\usepackage{lmodern}

\usepackage[francais]{babel}
\usepackage{fancybox}
\usepackage{graphicx}

\usepackage{float}

%\usepackage[usenames, x11names]{xcolor}
\usepackage{tikz}
\usepackage{datetime}

\usepackage{mathptmx}
%\usepackage{fouriernc}
%\usepackage{newcent}
\usepackage[mathcal,mathbf]{euler}

%\usepackage{palatino}
%\usepackage{newcent}


% Commande spéciale prompteur

%\usepackage{mathptmx}
%\usepackage[mathcal,mathbf]{euler}
%\usepackage{mathpple,multido}

\usepackage[a4paper]{geometry}
\geometry{top=2cm, bottom=2cm, left=1cm, right=1cm, marginparsep=1cm}

\newcommand{\change}{{\color{red}\rule{\textwidth}{1mm}\\}}

\newcounter{mydiapo}

\newcommand{\diapo}{\newpage
\hfill {\normalsize  Diapo \themydiapo \quad \texttt{[\jobname]}} \\
\stepcounter{mydiapo}}


%%%%%%% COULEURS %%%%%%%%%%

% Pour blanc sur noir :
%\pagecolor[rgb]{0.5,0.5,0.5}
% \pagecolor[rgb]{0,0,0}
% \color[rgb]{1,1,1}



%\DeclareFixedFont{\myfont}{U}{cmss}{bx}{n}{18pt}
\newcommand{\debuttexte}{
%%%%%%%%%%%%% FONTES %%%%%%%%%%%%%
\renewcommand{\baselinestretch}{1.5}
\usefont{U}{cmss}{bx}{n}
\bfseries

% Taille normale : commenter le reste !
%Taille Arnaud
%\fontsize{19}{19}\selectfont

% Taille Barbara
%\fontsize{21}{22}\selectfont

%Taille François
\fontsize{25}{30}\selectfont

%Taille Pascal
%\fontsize{25}{30}\selectfont

%Taille Laura
%\fontsize{30}{35}\selectfont


%\myfont
%\usefont{U}{cmss}{bx}{n}

%\Huge
%\addtolength{\parskip}{\baselineskip}
}


% \usepackage{hyperref}
% \hypersetup{colorlinks=true, linkcolor=blue, urlcolor=blue,
% pdftitle={Exo7 - Exercices de mathématiques}, pdfauthor={Exo7}}


%section
% \usepackage{sectsty}
% \allsectionsfont{\bf}
%\sectionfont{\color{Tomato3}\upshape\selectfont}
%\subsectionfont{\color{Tomato4}\upshape\selectfont}

%----- Ensembles : entiers, reels, complexes -----
\newcommand{\Nn}{\mathbb{N}} \newcommand{\N}{\mathbb{N}}
\newcommand{\Zz}{\mathbb{Z}} \newcommand{\Z}{\mathbb{Z}}
\newcommand{\Qq}{\mathbb{Q}} \newcommand{\Q}{\mathbb{Q}}
\newcommand{\Rr}{\mathbb{R}} \newcommand{\R}{\mathbb{R}}
\newcommand{\Cc}{\mathbb{C}} 
\newcommand{\Kk}{\mathbb{K}} \newcommand{\K}{\mathbb{K}}

%----- Modifications de symboles -----
\renewcommand{\epsilon}{\varepsilon}
\renewcommand{\Re}{\mathop{\text{Re}}\nolimits}
\renewcommand{\Im}{\mathop{\text{Im}}\nolimits}
%\newcommand{\llbracket}{\left[\kern-0.15em\left[}
%\newcommand{\rrbracket}{\right]\kern-0.15em\right]}

\renewcommand{\ge}{\geqslant}
\renewcommand{\geq}{\geqslant}
\renewcommand{\le}{\leqslant}
\renewcommand{\leq}{\leqslant}

%----- Fonctions usuelles -----
\newcommand{\ch}{\mathop{\mathrm{ch}}\nolimits}
\newcommand{\sh}{\mathop{\mathrm{sh}}\nolimits}
\renewcommand{\tanh}{\mathop{\mathrm{th}}\nolimits}
\newcommand{\cotan}{\mathop{\mathrm{cotan}}\nolimits}
\newcommand{\Arcsin}{\mathop{\mathrm{Arcsin}}\nolimits}
\newcommand{\Arccos}{\mathop{\mathrm{Arccos}}\nolimits}
\newcommand{\Arctan}{\mathop{\mathrm{Arctan}}\nolimits}
\newcommand{\Argsh}{\mathop{\mathrm{Argsh}}\nolimits}
\newcommand{\Argch}{\mathop{\mathrm{Argch}}\nolimits}
\newcommand{\Argth}{\mathop{\mathrm{Argth}}\nolimits}
\newcommand{\pgcd}{\mathop{\mathrm{pgcd}}\nolimits} 

\newcommand{\Card}{\mathop{\text{Card}}\nolimits}
\newcommand{\Ker}{\mathop{\text{Ker}}\nolimits}
\newcommand{\id}{\mathop{\text{id}}\nolimits}
\newcommand{\ii}{\mathrm{i}}
\newcommand{\dd}{\mathrm{d}}
\newcommand{\Vect}{\mathop{\text{Vect}}\nolimits}
\newcommand{\Mat}{\mathop{\mathrm{Mat}}\nolimits}
\newcommand{\rg}{\mathop{\text{rg}}\nolimits}
\newcommand{\tr}{\mathop{\text{tr}}\nolimits}
\newcommand{\ppcm}{\mathop{\text{ppcm}}\nolimits}

%----- Structure des exercices ------

\newtheoremstyle{styleexo}% name
{2ex}% Space above
{3ex}% Space below
{}% Body font
{}% Indent amount 1
{\bfseries} % Theorem head font
{}% Punctuation after theorem head
{\newline}% Space after theorem head 2
{}% Theorem head spec (can be left empty, meaning ‘normal’)

%\theoremstyle{styleexo}
\newtheorem{exo}{Exercice}
\newtheorem{ind}{Indications}
\newtheorem{cor}{Correction}


\newcommand{\exercice}[1]{} \newcommand{\finexercice}{}
%\newcommand{\exercice}[1]{{\tiny\texttt{#1}}\vspace{-2ex}} % pour afficher le numero absolu, l'auteur...
\newcommand{\enonce}{\begin{exo}} \newcommand{\finenonce}{\end{exo}}
\newcommand{\indication}{\begin{ind}} \newcommand{\finindication}{\end{ind}}
\newcommand{\correction}{\begin{cor}} \newcommand{\fincorrection}{\end{cor}}

\newcommand{\noindication}{\stepcounter{ind}}
\newcommand{\nocorrection}{\stepcounter{cor}}

\newcommand{\fiche}[1]{} \newcommand{\finfiche}{}
\newcommand{\titre}[1]{\centerline{\large \bf #1}}
\newcommand{\addcommand}[1]{}
\newcommand{\video}[1]{}

% Marge
\newcommand{\mymargin}[1]{\marginpar{{\small #1}}}



%----- Presentation ------
\setlength{\parindent}{0cm}

%\newcommand{\ExoSept}{\href{http://exo7.emath.fr}{\textbf{\textsf{Exo7}}}}

\definecolor{myred}{rgb}{0.93,0.26,0}
\definecolor{myorange}{rgb}{0.97,0.58,0}
\definecolor{myyellow}{rgb}{1,0.86,0}

\newcommand{\LogoExoSept}[1]{  % input : echelle
{\usefont{U}{cmss}{bx}{n}
\begin{tikzpicture}[scale=0.1*#1,transform shape]
  \fill[color=myorange] (0,0)--(4,0)--(4,-4)--(0,-4)--cycle;
  \fill[color=myred] (0,0)--(0,3)--(-3,3)--(-3,0)--cycle;
  \fill[color=myyellow] (4,0)--(7,4)--(3,7)--(0,3)--cycle;
  \node[scale=5] at (3.5,3.5) {Exo7};
\end{tikzpicture}}
}



\theoremstyle{definition}
%\newtheorem{proposition}{Proposition}
%\newtheorem{exemple}{Exemple}
%\newtheorem{theoreme}{Théorème}
\newtheorem{lemme}{Lemme}
\newtheorem{corollaire}{Corollaire}
%\newtheorem*{remarque*}{Remarque}
%\newtheorem*{miniexercice}{Mini-exercices}
%\newtheorem{definition}{Définition}




%definition d'un terme
\newcommand{\defi}[1]{{\color{myorange}\textbf{\emph{#1}}}}
\newcommand{\evidence}[1]{{\color{blue}\textbf{\emph{#1}}}}



 %----- Commandes divers ------

\newcommand{\codeinline}[1]{\texttt{#1}}

%%%%%%%%%%%%%%%%%%%%%%%%%%%%%%%%%%%%%%%%%%%%%%%%%%%%%%%%%%%%%
%%%%%%%%%%%%%%%%%%%%%%%%%%%%%%%%%%%%%%%%%%%%%%%%%%%%%%%%%%%%%



\begin{document}

\debuttexte


%%%%%%%%%%%%%%%%%%%%%%%%%%%%%%%%%%%%%%%%%%%%%%%%%%%%%%%%%%%
\diapo

\change
Dans cette partie, nous présentons quelques types de matrices particulièrement intéressants, 
et les opérations qui leurs sont associées. Les définitions sont visuellement très simples, 
et nous verrons leurs propriétés mathématiques.

\change
Tout d'abord les matrices triangulaires, avec un cas particulier très important:
les matrices diagonales.

\change
Ensuite, l'opération de transposition, qui échange ce qui est au-dessus de la diagonale avec ce qui 
en-dessous.

\change
La trace d'une matrice, un nombre qui mesure certaines propriétés d'une matrice,

\change
et deux autres classes de matrices: les matrices \emph{symétriques}, qui sont symétriques 
de part et d'autre de la diagonale, ...

\change
... et les antisymétriques, dont les deux côtés de la diagonale sont opposés.

%%%%%%%%%%%%%%%%%%%%%%%%%%%%%%%%%%%%%%%%%%%%%%%%%%%%%%%%%%
\diapo

On considère pour le moment uniquement des matrices \emph{carrées}, c-à-d ayant un même nombre 
$n$ de colonnes et de lignes. On peut alors parler de \defi{diagonale}, 
qui est la ligne partant de l'élément $11$ à l'élément $n n$.

\change
Une matrice carrée $A$ est dite \defi{triangulaire inférieure} si ses éléments strictement 
\emph{au-dessus} de la diagonale sont nuls, ce qui se traduit par: chaque fois que $i < j$
on a $a_{ij} = 0$. 

\change
Voici la forme générale d'une matrice triangulaire inférieure: les coefficients strcitement au-dessus de la diagonale sont nuls,

sur la diagonale et en-dessous les coefficients sont quelconques. 
Notons que ces coefficients peuvent 
être nuls ou non, mais a priori on n'en sait rien.

\change
De la même manière, on dit que $A$ est \defi{triangulaire supérieure} si ses éléments strictement 
\emph{en-dessous} de la diagonale sont nuls, ce qui se traduit par: $i > j \implies a_{ij} = 0$. 

\change
Voici la forme générale d'une matrice triangulaire supérieure.


%%%%%%%%%%%%%%%%%%%%%%%%%%%%%%%%%%%%%%%%%%%%%%%%%%%%%%%%%%%
\diapo

Voici des exemples concrets de matrices triangulaires.
Les deux premières matrices sont triangulaires inférieures, et la 3ème est triangulaire supérieure.

\change
Les matrices triangulaires se rencontrent notamment dans la résolution de certains systèmes linéaires.
Comme elles sont plus simples, elles ont des propriétés spécifiques comme celle-ci: 

\emph{Une matrice triangulaire est inversible 
si et seulement si ses éléments diagonaux sont tous non nuls}.

C'est donc très simple de dire si une telle matrice est ou non inversible.

[Sur exemple] Toutes ces matrices sont triangulaires, les coefficients sur la diagonale sont non nuls, donc elles sont toutes trois inversibles.


%%%%%%%%%%%%%%%%%%%%%%%%%%%%%%%%%%%%%%%%%%%%%%%%%%%%%%%%%%
\diapo

Une matrice peut être \emph{à la fois} triangulaire inférieure et supérieure, elle est dite alors
\defi{diagonale}. Dans ce cas tous les éléments au-dessus et en-dessous de la diagonale
sont nuls: $i\neq j \ \Longrightarrow \ a_{ij} = 0$

\change
Voici deux exemples. Remarquez que la première matrice possède aussi un zéro \emph{sur}
la diagonale, ce qui est évidemment autorisé.

\change
Les matrices diagonales jouent un rôle particulier et ont aussi des propriétés particulières.
Par exemple, il est très facile de calculer la puissance $p$ d'une matrice diagonale: 
il suffit d'élever chaque élément de la diagonale à la puissance $p$.

%%%%%%%%%%%%%%%%%%%%%%%%%%%%%%%%%%%%%%%%%%%%%%%%%%%%%%%%%%%
\diapo

Revenons au cas général d'une matrice $A$ de taille $n \times p$.

\change
La matrice \defi{transposée} de $A$, notée $A^T$, est une matrice de taille $p \times n$ dont les 
indices $i$ et $j$ des coefficients ont été échangés.


Le coefficient $a_{21}$ de $A$ se retrouve ici,
le coefficient $a_{n1}$ de $A$ se retrouve ici,
le coefficient $a_{2p}$ de $A$ se retrouve là.

Les coefficients de la diagonale restent sur la diagonale.

\change
De façon générale le coefficient $a_{ij}$ se retrouve à la place $(j,i)$ dans la matrice transposée.


La transposée de $A$ est en quelque sorte l'image miroir de $A$, par rapport à sa diagonale.

\change
On peut aussi dire que la $i$-ème ligne de $A$ devient la $i$-ème colonne de $A^{T}$ 
(ou réciproquement la $j$-ème colonne de $A$ devient la $j$-ème ligne de $A^{T}$).
On dit que la transposition échange lignes et colonnes.

\change
Attention, il existe diverses notations pour la transposée, avec $t$ majuscule ou minuscule,
à gauche ou à droite (mais toujours en exposant).

%%%%%%%%%%%%%%%%%%%%%%%%%%%%%%%%%%%%%%%%%%%%%%%%%%%%%%%%%%
\diapo

Voici 3 exemples de calcul de transposée. 

On notera que la transposée d'une matrice carrée 
est encore carrée, 

la transposée d'un matrice $3\times 2$ est un matrice $2\times 3$,

la transposée d'un vecteur-ligne est un vecteur-colonne. 

\change
La transposée est donc une fonction qui à une matrice associe une autre matrice. 
Ses propriétés notables sont que: $(A + B)^T = A^T + B^T$ :
la transposée d'une somme est la somme des transposées.

\change
$(\alpha \, A)^T = \alpha \, A^T $
%Par conséquent on voit que la fonction transposée est une application \defi{linéaire};

\change
Si on fait deux fois l'opération de transposition alors on retombe sur la matrice de départ 


\change
$(AB)^T = B^T A^T$ 
La transposée d'un produit est le produit des transposée : mais notez bien
que l'ordre est inverse (comme pour l'inverse d'un produit de matrice).


\change
Si $A$ est inversible, alors $A^T$ l'est aussi et l'inverse de la transposée est
la transposée de l'inverse.

%%%%%%%%%%%%%%%%%%%%%%%%%%%%%%%%%%%%%%%%%%%%%%%%%%%%%%%%%%%
\diapo

Revenons au cas d'une matrice carrée, de taille $n \times n$.

\change
Les éléments situés sur sa diagonale (c-à-d $i=j$) sont appelées éléments diagonaux,
et il y en exactement $n$.

\change
Ils forment un sous-ensemble de la matrice appelé \defi{diagonale principale}

\change
On définit alors une fonction, appelée \defi{trace} qui à la matrice carrée $A$ 
associe la somme de ses éléments diagonaux.

$\tr A = a_{11} + a_{22} + \cdots +a_{nn}$

\change
Voici un exemple: les éléments diagonaux sont $2$ et $5$, et la trace vaut donc $7$.

\change
Ici les éléments diagonaux sont $1$, $2$ et $-10$, et la trace vaut $-7$.

%%%%%%%%%%%%%%%%%%%%%%%%%%%%%%%%%%%%%%%%%%%%%%%%%%%%%%%%%%
\diapo

Voici quelques propriétés importantes de la fonction trace.
\begin{enumerate}
\item $\tr(A + B)$ = $\tr A$ + $\tr B$
\item $\tr(\alpha A)$ = $\alpha$ $\tr A$ pour tout $\alpha \in \Kk$
\item $\tr(A^T)$ = $\tr A $
\end{enumerate}

La dernière propriété est la plus importante :
$\tr(AB)$ = $\tr(BA)$

Les démonstrations de ces 3 propriétés sont simples,
par contre je vous encourage à travailler la démonstration
de la formule $\tr(AB)$ = $\tr(BA)$.



%%%%%%%%%%%%%%%%%%%%%%%%%%%%%%%%%%%%%%%%%%%%%%%%%%%%%%%%%%%
\diapo

Étudions maintenant les matrices carrées et leurs transposées.
Une matrice est dite \defi{symétrique} si elle est égale à sa transposée:
$A^T = A$.

\change
Autrement dit, le coefficient $a_{ij}$ est égal au coefficient $a_{ji}$
situé symétriquement par rapport à la diagonale.

\change
Voici deux exemples de matrices symétriques. 

Par exemple ici et là $0$,

ici et là $5$,

ici et là $-1$.

On remarque qu'il n'y a aucune contrainte sur les coefficients diagonaux.



%%%%%%%%%%%%%%%%%%%%%%%%%%%%%%%%%%%%%%%%%%%%%%%%%%%%%%%%%%%
\diapo

On peut construire une matrice symétrique à partir d'une matrice $B$ quelconque
(même pas forcément carrée), en multipliant $B$ par sa transposée, à gauche ou à droite :

les matrices $B \cdot B^T$ et $B^T \cdot B$ sont symétriques. 

Attention ces deux matrices sont en général différentes, mais toutes deux symétriques.


\change
La preuve est très simple, et utilise le théorème précédent sur la transposition.
\[
(BB^T)^T = (B^T)^T B^T = BB^T 
\]
C'est le même raisonnement pour l'autre produit.

%%%%%%%%%%%%%%%%%%%%%%%%%%%%%%%%%%%%%%%%%%%%%%%%%%%%%%%%%%%
\diapo

Une matrice est dite \defi{antisymétrique} si elle est l'opposée de sa transposée:
$A^T = - A$.

\change
Autrement dit, le coefficient $a_{ij}$ est égal à $-a_{ji}$
situé symétriquement par rapport à la diagonale.

\change
Voici deux exemples.



$-4$ ici, $+4$ là.

$-2$ ici, $+2$ là.

$+5$ ici, $-5$ là.

\change
On remarquera que les éléments diagonaux d'une matrice antisymétrique sont 
toujours tous nuls. En effet, si $i=j$, on a $a_{ii} = - a_{ii}$)

%%%%%%%%%%%%%%%%%%%%%%%%%%%%%%%%%%%%%%%%%%%%%%%%%%%%%%%%%%%
\diapo

Entraînez-vous avec ces exercices pour vérifier que vous avez bien compris le cours.

\end{document}

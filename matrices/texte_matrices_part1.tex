
%%%%%%%%%%%%%%%%%% PREAMBULE %%%%%%%%%%%%%%%%%%


\documentclass[12pt]{article}

\usepackage{amsfonts,amsmath,amssymb,amsthm}
\usepackage[utf8]{inputenc}
\usepackage[T1]{fontenc}
\usepackage[francais]{babel}


% packages
\usepackage{amsfonts,amsmath,amssymb,amsthm}
\usepackage[utf8]{inputenc}
\usepackage[T1]{fontenc}
%\usepackage{lmodern}

\usepackage[francais]{babel}
\usepackage{fancybox}
\usepackage{graphicx}

\usepackage{float}

%\usepackage[usenames, x11names]{xcolor}
\usepackage{tikz}
\usepackage{datetime}

\usepackage{mathptmx}
%\usepackage{fouriernc}
%\usepackage{newcent}
\usepackage[mathcal,mathbf]{euler}

%\usepackage{palatino}
%\usepackage{newcent}


% Commande spéciale prompteur

%\usepackage{mathptmx}
%\usepackage[mathcal,mathbf]{euler}
%\usepackage{mathpple,multido}

\usepackage[a4paper]{geometry}
\geometry{top=2cm, bottom=2cm, left=1cm, right=1cm, marginparsep=1cm}

\newcommand{\change}{{\color{red}\rule{\textwidth}{1mm}\\}}

\newcounter{mydiapo}

\newcommand{\diapo}{\newpage
\hfill {\normalsize  Diapo \themydiapo \quad \texttt{[\jobname]}} \\
\stepcounter{mydiapo}}


%%%%%%% COULEURS %%%%%%%%%%

% Pour blanc sur noir :
%\pagecolor[rgb]{0.5,0.5,0.5}
% \pagecolor[rgb]{0,0,0}
% \color[rgb]{1,1,1}



%\DeclareFixedFont{\myfont}{U}{cmss}{bx}{n}{18pt}
\newcommand{\debuttexte}{
%%%%%%%%%%%%% FONTES %%%%%%%%%%%%%
\renewcommand{\baselinestretch}{1.5}
\usefont{U}{cmss}{bx}{n}
\bfseries

% Taille normale : commenter le reste !
%Taille Arnaud
%\fontsize{19}{19}\selectfont

% Taille Barbara
%\fontsize{21}{22}\selectfont

%Taille François
\fontsize{25}{30}\selectfont

%Taille Pascal
%\fontsize{25}{30}\selectfont

%Taille Laura
%\fontsize{30}{35}\selectfont


%\myfont
%\usefont{U}{cmss}{bx}{n}

%\Huge
%\addtolength{\parskip}{\baselineskip}
}


% \usepackage{hyperref}
% \hypersetup{colorlinks=true, linkcolor=blue, urlcolor=blue,
% pdftitle={Exo7 - Exercices de mathématiques}, pdfauthor={Exo7}}


%section
% \usepackage{sectsty}
% \allsectionsfont{\bf}
%\sectionfont{\color{Tomato3}\upshape\selectfont}
%\subsectionfont{\color{Tomato4}\upshape\selectfont}

%----- Ensembles : entiers, reels, complexes -----
\newcommand{\Nn}{\mathbb{N}} \newcommand{\N}{\mathbb{N}}
\newcommand{\Zz}{\mathbb{Z}} \newcommand{\Z}{\mathbb{Z}}
\newcommand{\Qq}{\mathbb{Q}} \newcommand{\Q}{\mathbb{Q}}
\newcommand{\Rr}{\mathbb{R}} \newcommand{\R}{\mathbb{R}}
\newcommand{\Cc}{\mathbb{C}} 
\newcommand{\Kk}{\mathbb{K}} \newcommand{\K}{\mathbb{K}}

%----- Modifications de symboles -----
\renewcommand{\epsilon}{\varepsilon}
\renewcommand{\Re}{\mathop{\text{Re}}\nolimits}
\renewcommand{\Im}{\mathop{\text{Im}}\nolimits}
%\newcommand{\llbracket}{\left[\kern-0.15em\left[}
%\newcommand{\rrbracket}{\right]\kern-0.15em\right]}

\renewcommand{\ge}{\geqslant}
\renewcommand{\geq}{\geqslant}
\renewcommand{\le}{\leqslant}
\renewcommand{\leq}{\leqslant}

%----- Fonctions usuelles -----
\newcommand{\ch}{\mathop{\mathrm{ch}}\nolimits}
\newcommand{\sh}{\mathop{\mathrm{sh}}\nolimits}
\renewcommand{\tanh}{\mathop{\mathrm{th}}\nolimits}
\newcommand{\cotan}{\mathop{\mathrm{cotan}}\nolimits}
\newcommand{\Arcsin}{\mathop{\mathrm{Arcsin}}\nolimits}
\newcommand{\Arccos}{\mathop{\mathrm{Arccos}}\nolimits}
\newcommand{\Arctan}{\mathop{\mathrm{Arctan}}\nolimits}
\newcommand{\Argsh}{\mathop{\mathrm{Argsh}}\nolimits}
\newcommand{\Argch}{\mathop{\mathrm{Argch}}\nolimits}
\newcommand{\Argth}{\mathop{\mathrm{Argth}}\nolimits}
\newcommand{\pgcd}{\mathop{\mathrm{pgcd}}\nolimits} 

\newcommand{\Card}{\mathop{\text{Card}}\nolimits}
\newcommand{\Ker}{\mathop{\text{Ker}}\nolimits}
\newcommand{\id}{\mathop{\text{id}}\nolimits}
\newcommand{\ii}{\mathrm{i}}
\newcommand{\dd}{\mathrm{d}}
\newcommand{\Vect}{\mathop{\text{Vect}}\nolimits}
\newcommand{\Mat}{\mathop{\mathrm{Mat}}\nolimits}
\newcommand{\rg}{\mathop{\text{rg}}\nolimits}
\newcommand{\tr}{\mathop{\text{tr}}\nolimits}
\newcommand{\ppcm}{\mathop{\text{ppcm}}\nolimits}

%----- Structure des exercices ------

\newtheoremstyle{styleexo}% name
{2ex}% Space above
{3ex}% Space below
{}% Body font
{}% Indent amount 1
{\bfseries} % Theorem head font
{}% Punctuation after theorem head
{\newline}% Space after theorem head 2
{}% Theorem head spec (can be left empty, meaning ‘normal’)

%\theoremstyle{styleexo}
\newtheorem{exo}{Exercice}
\newtheorem{ind}{Indications}
\newtheorem{cor}{Correction}


\newcommand{\exercice}[1]{} \newcommand{\finexercice}{}
%\newcommand{\exercice}[1]{{\tiny\texttt{#1}}\vspace{-2ex}} % pour afficher le numero absolu, l'auteur...
\newcommand{\enonce}{\begin{exo}} \newcommand{\finenonce}{\end{exo}}
\newcommand{\indication}{\begin{ind}} \newcommand{\finindication}{\end{ind}}
\newcommand{\correction}{\begin{cor}} \newcommand{\fincorrection}{\end{cor}}

\newcommand{\noindication}{\stepcounter{ind}}
\newcommand{\nocorrection}{\stepcounter{cor}}

\newcommand{\fiche}[1]{} \newcommand{\finfiche}{}
\newcommand{\titre}[1]{\centerline{\large \bf #1}}
\newcommand{\addcommand}[1]{}
\newcommand{\video}[1]{}

% Marge
\newcommand{\mymargin}[1]{\marginpar{{\small #1}}}



%----- Presentation ------
\setlength{\parindent}{0cm}

%\newcommand{\ExoSept}{\href{http://exo7.emath.fr}{\textbf{\textsf{Exo7}}}}

\definecolor{myred}{rgb}{0.93,0.26,0}
\definecolor{myorange}{rgb}{0.97,0.58,0}
\definecolor{myyellow}{rgb}{1,0.86,0}

\newcommand{\LogoExoSept}[1]{  % input : echelle
{\usefont{U}{cmss}{bx}{n}
\begin{tikzpicture}[scale=0.1*#1,transform shape]
  \fill[color=myorange] (0,0)--(4,0)--(4,-4)--(0,-4)--cycle;
  \fill[color=myred] (0,0)--(0,3)--(-3,3)--(-3,0)--cycle;
  \fill[color=myyellow] (4,0)--(7,4)--(3,7)--(0,3)--cycle;
  \node[scale=5] at (3.5,3.5) {Exo7};
\end{tikzpicture}}
}



\theoremstyle{definition}
%\newtheorem{proposition}{Proposition}
%\newtheorem{exemple}{Exemple}
%\newtheorem{theoreme}{Théorème}
\newtheorem{lemme}{Lemme}
\newtheorem{corollaire}{Corollaire}
%\newtheorem*{remarque*}{Remarque}
%\newtheorem*{miniexercice}{Mini-exercices}
%\newtheorem{definition}{Définition}




%definition d'un terme
\newcommand{\defi}[1]{{\color{myorange}\textbf{\emph{#1}}}}
\newcommand{\evidence}[1]{{\color{blue}\textbf{\emph{#1}}}}



 %----- Commandes divers ------

\newcommand{\codeinline}[1]{\texttt{#1}}

%%%%%%%%%%%%%%%%%%%%%%%%%%%%%%%%%%%%%%%%%%%%%%%%%%%%%%%%%%%%%
%%%%%%%%%%%%%%%%%%%%%%%%%%%%%%%%%%%%%%%%%%%%%%%%%%%%%%%%%%%%%



\begin{document}

\debuttexte


%%%%%%%%%%%%%%%%%%%%%%%%%%%%%%%%%%%%%%%%%%%%%%%%%%%%%%%%%%%
\diapo

\change
Cette leçon est la première d'une série de leçons consacrées aux matrices. Les matrices sont des tableaux de nombres. La résolution d'un
certain nombre de problèmes d'algèbre linéaire se ramène
à des manipulations sur les matrices. Ceci est vrai en particulier 
pour la résolution des systèmes linéaires.  

\change
Dans cette leçon, nous allons tout d'abord donner la définition de matrice,

\change
puis nous verrons quelques exemples particuliers de matrices,

\change
et enfin, nous définirons une première opération sur les matrices, à savoir l'addition.

%%%%%%%%%%%%%%%%%%%%%%%%%%%%%%%%%%%%%%%%%%%%%%%%%%%%%%%%%%
\diapo

Dans tout ce chapitre, $\Kk$ désigne un corps. On peut penser à $\Qq$, $\Rr$ ou $\Cc$.
Mais pour les exemples ce sera souvent $\Rr$ le corps des nombres réels.

\change
Une matrice $A$ est un tableau rectangulaire  d'éléments de $\Kk$.

\change
Un tel tableau est représenté de la manière suivante.

\change
La matrice $A$ est dite de \defi{taille} $n \times p$ si le tableau 
possède $n$ lignes et $p$ colonnes. 

\change
Les nombres du tableau sont appelés les \defi{coefficients} de $A$. 

\change
Le coefficient situé à  la $i$-ème ligne et à la $j$-ème colonne est noté $a_{i,j}$,

\change
et on note aussi $A= \big(a_{i,j}\big)_{\substack{1\leq i \leq n \\ 1\leq j \leq p}}$

%%%%%%%%%%%%%%%%%%%%%%%%%%%%%%%%%%%%%%%%%%%%%%%%%%%%%%%%%%%
\diapo

Voici un exemple de matrice.

\change

$A$ est une matrice de taille $2\times 3$ 

\change
avec, par exemple, $a_{1,1}$ c'est-à-dire l'élément situé en ligne 1 colonne 1, qui vaut 1,

\change
 et $a_{2,3}=7$. 
 
\change
Encore quelques définitions.

Deux matrices sont \defi{égales} lorsqu'elles ont la même taille et que les coefficients correspondants sont égaux. 
  
\change
L'ensemble des matrices à $n$ lignes et $p$ colonnes à coefficients dans $\Kk$ est noté $M_{n,p}(\Kk)$. 

Lorsque le corps $\Kk$ est le corps des réels, les éléments de $M_{n,p}(\mathbb{R})$ sont appelés matrices réelles.

%%%%%%%%%%%%%%%%%%%%%%%%%%%%%%%%%%%%%%%%%%%%%%%%%%%%%%%%%%
\diapo

Voici à présent quelques types de matrices intéressantes.

Tout d'abord, si $n=p$, c'est-à-dire si le nombre de lignes est le même que le nombre de colonnes, la matrice est dite matrice carrée.

On note $M_{n}(\Kk)$ au lieu de $M_{n,n}(\Kk)$.

\change
 Les éléments $a_{1,1}, a_{2,2}, \ldots, a_{n,n}$ forment la diagonale principale de la matrice.

\change
Une matrice qui n'a qu'une seule ligne, c'est-à-dire avec $n=1$, est appelée \defi{matrice ligne} ou \defi{vecteur ligne}. On la note 
$$A=\begin{pmatrix} 
a_{1,1}& a_{1,2}&  \ldots & a_{1,p}\cr
\end{pmatrix}.$$

\change
De même, une matrice qui n'a qu'une seule colonne ($p=1$) est appelée \defi{matrice
colonne} ou \defi{vecteur colonne}. On la note ainsi.

\change
La matrice (de taille $n\times p$) dont tous les coefficients sont des zéros 
est appelée la \defi{matrice nulle} et est notée $0_{n,p}$ ou plus simplement $0$. 
Dans le calcul matriciel, la matrice nulle joue le rôle du nombre $0$ pour les réels.

%%%%%%%%%%%%%%%%%%%%%%%%%%%%%%%%%%%%%%%%%%%%%%%%%%%%%%%%%%%
\diapo
Voyons à présent comment on peut additionner deux matrices entre elles.

Soient $A$ et $B$ deux matrices ayant la même taille $n\times p$. Leur \defi{somme} $C=A+B$ est la matrice de taille $n\times p$ définie par 
\[c_{ij}=a_{ij}+b_{ij}.\]

En d'autres termes, on somme coefficients par coefficients.

\change
Considérons par exemple les deux matrices $A$ et $B$ de taille $2\times2$ suivantes, et calculons leur somme. 

\change

$3+0=3$, \quad $-2+5=3$, 

$1+2=3$, \quad $7-1=6$.

\change
Par contre, si on considère à présent cette matrice $B'$ de taille $2\times1$, alors la somme de $A$ et $B'$ n'est pas définie, puisque les deux matrices n'ont pas la même taille.



%%%%%%%%%%%%%%%%%%%%%%%%%%%%%%%%%%%%%%%%%%%%%%%%%%%%%%%%%%
\diapo

Le produit d'une matrice $A=\big(a_{ij}\big)$ de $M_{n,p}(\Kk)$ 
par un scalaire $\alpha \in \Kk$ est la matrice
$\big(\alpha a_{ij}\big)$ formée en 
multipliant chaque coefficient de $A$ par $\alpha$. Elle est notée $\alpha \cdot A$ .

\change
$$
\text{Si} \qquad 
A  = \begin{pmatrix} 
1& 2 & 3\cr 
0& 1& 0\cr
\end{pmatrix}
\qquad \text{et} \qquad \alpha = 2
$$
alors la matrice $\alpha A$ est obtenue en multipliant tous les coefficients de $A$ par 2. Elle vaut donc
$$
\begin{pmatrix} 
2& 4 & 6\cr 
0& 2& 0\cr
\end{pmatrix}.$$

%%%%%%%%%%%%%%%%%%%%%%%%%%%%%%%%%%%%%%%%%%%%%%%%%%%%%%%%%%%
\diapo

La matrice $(-1)A$ est l'\defi{opposée} de $A$ et est notée $-A$.

\change
La \defi{différence} $A-B$ est définie par $A+$ l'opposé de $B$, c'est-à-dire $A + (-B)$.

\change
Considérons les deux matrices $A$ et $B$ de taille $2\times3$ suivantes. 
Alors leur différence $A-B$ est obtenue en soustrayant coefficients par coefficients. 
Par exemple $2-(-1)=3$, $-1-4=-5$ ou encore $2-3=-1$.

%%%%%%%%%%%%%%%%%%%%%%%%%%%%%%%%%%%%%%%%%%%%%%%%%%%%%%%%%%
\diapo

L'addition et la multiplication par un scalaire se comportent sans surprises. 

Donnons nous des matrices $A$, $B$ et $C$ de même taille, et deux scalaires $\alpha$ et $\beta$ dans $\Kk$

\change
alors $A + B = B + A$ : la somme est commutative,

\change
 $A + (B+C) = (A + B) + C$ : la somme est associative,

\change
 $A + 0 = A$ : la matrice nulle est l'élément neutre de l'addition,

\change
 $(\alpha + \beta )A =\alpha A + \beta A$,
    
\change 
$\alpha (A+B)=\alpha A + \alpha B$.

\change
Démontrons par exemple le quatrième point. Le terme général de la matrice $(\alpha + \beta ) A $ est égal à 
$(\alpha + \beta)a_{ij}$. 

\change
D'après les règles de calcul dans $\Kk$,  $(\alpha + \beta)a_{ij}$ est égal à  $\alpha a_{ij}+ \beta a_{ij}$

\change
qui est bien le terme général de la matrice $\alpha A + \beta A$.


%%%%%%%%%%%%%%%%%%%%%%%%%%%%%%%%%%%%%%%%%%%%%%%%%%%%%%%%%%%
\diapo

Entrainez-vous avec ces exercice pour vérifier que vous avez bien compris le cours.

\end{document}

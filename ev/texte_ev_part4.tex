
%%%%%%%%%%%%%%%%%% PREAMBULE %%%%%%%%%%%%%%%%%%


\documentclass[12pt]{article}

\usepackage{amsfonts,amsmath,amssymb,amsthm}
\usepackage[utf8]{inputenc}
\usepackage[T1]{fontenc}
\usepackage[francais]{babel}


% packages
\usepackage{amsfonts,amsmath,amssymb,amsthm}
\usepackage[utf8]{inputenc}
\usepackage[T1]{fontenc}
%\usepackage{lmodern}

\usepackage[francais]{babel}
\usepackage{fancybox}
\usepackage{graphicx}

\usepackage{float}

%\usepackage[usenames, x11names]{xcolor}
\usepackage{tikz}
\usepackage{datetime}

\usepackage{mathptmx}
%\usepackage{fouriernc}
%\usepackage{newcent}
\usepackage[mathcal,mathbf]{euler}

%\usepackage{palatino}
%\usepackage{newcent}


% Commande spéciale prompteur

%\usepackage{mathptmx}
%\usepackage[mathcal,mathbf]{euler}
%\usepackage{mathpple,multido}

\usepackage[a4paper]{geometry}
\geometry{top=2cm, bottom=2cm, left=1cm, right=1cm, marginparsep=1cm}

\newcommand{\change}{{\color{red}\rule{\textwidth}{1mm}\\}}

\newcounter{mydiapo}

\newcommand{\diapo}{\newpage
\hfill {\normalsize  Diapo \themydiapo \quad \texttt{[\jobname]}} \\
\stepcounter{mydiapo}}


%%%%%%% COULEURS %%%%%%%%%%

% Pour blanc sur noir :
%\pagecolor[rgb]{0.5,0.5,0.5}
% \pagecolor[rgb]{0,0,0}
% \color[rgb]{1,1,1}



%\DeclareFixedFont{\myfont}{U}{cmss}{bx}{n}{18pt}
\newcommand{\debuttexte}{
%%%%%%%%%%%%% FONTES %%%%%%%%%%%%%
\renewcommand{\baselinestretch}{1.5}
\usefont{U}{cmss}{bx}{n}
\bfseries

% Taille normale : commenter le reste !
%Taille Arnaud
%\fontsize{19}{19}\selectfont

% Taille Barbara
%\fontsize{21}{22}\selectfont

%Taille François
\fontsize{25}{30}\selectfont

%Taille Pascal
%\fontsize{25}{30}\selectfont

%Taille Laura
%\fontsize{30}{35}\selectfont


%\myfont
%\usefont{U}{cmss}{bx}{n}

%\Huge
%\addtolength{\parskip}{\baselineskip}
}


% \usepackage{hyperref}
% \hypersetup{colorlinks=true, linkcolor=blue, urlcolor=blue,
% pdftitle={Exo7 - Exercices de mathématiques}, pdfauthor={Exo7}}


%section
% \usepackage{sectsty}
% \allsectionsfont{\bf}
%\sectionfont{\color{Tomato3}\upshape\selectfont}
%\subsectionfont{\color{Tomato4}\upshape\selectfont}

%----- Ensembles : entiers, reels, complexes -----
\newcommand{\Nn}{\mathbb{N}} \newcommand{\N}{\mathbb{N}}
\newcommand{\Zz}{\mathbb{Z}} \newcommand{\Z}{\mathbb{Z}}
\newcommand{\Qq}{\mathbb{Q}} \newcommand{\Q}{\mathbb{Q}}
\newcommand{\Rr}{\mathbb{R}} \newcommand{\R}{\mathbb{R}}
\newcommand{\Cc}{\mathbb{C}} 
\newcommand{\Kk}{\mathbb{K}} \newcommand{\K}{\mathbb{K}}

%----- Modifications de symboles -----
\renewcommand{\epsilon}{\varepsilon}
\renewcommand{\Re}{\mathop{\text{Re}}\nolimits}
\renewcommand{\Im}{\mathop{\text{Im}}\nolimits}
%\newcommand{\llbracket}{\left[\kern-0.15em\left[}
%\newcommand{\rrbracket}{\right]\kern-0.15em\right]}

\renewcommand{\ge}{\geqslant}
\renewcommand{\geq}{\geqslant}
\renewcommand{\le}{\leqslant}
\renewcommand{\leq}{\leqslant}

%----- Fonctions usuelles -----
\newcommand{\ch}{\mathop{\mathrm{ch}}\nolimits}
\newcommand{\sh}{\mathop{\mathrm{sh}}\nolimits}
\renewcommand{\tanh}{\mathop{\mathrm{th}}\nolimits}
\newcommand{\cotan}{\mathop{\mathrm{cotan}}\nolimits}
\newcommand{\Arcsin}{\mathop{\mathrm{Arcsin}}\nolimits}
\newcommand{\Arccos}{\mathop{\mathrm{Arccos}}\nolimits}
\newcommand{\Arctan}{\mathop{\mathrm{Arctan}}\nolimits}
\newcommand{\Argsh}{\mathop{\mathrm{Argsh}}\nolimits}
\newcommand{\Argch}{\mathop{\mathrm{Argch}}\nolimits}
\newcommand{\Argth}{\mathop{\mathrm{Argth}}\nolimits}
\newcommand{\pgcd}{\mathop{\mathrm{pgcd}}\nolimits} 

\newcommand{\Card}{\mathop{\text{Card}}\nolimits}
\newcommand{\Ker}{\mathop{\text{Ker}}\nolimits}
\newcommand{\id}{\mathop{\text{id}}\nolimits}
\newcommand{\ii}{\mathrm{i}}
\newcommand{\dd}{\mathrm{d}}
\newcommand{\Vect}{\mathop{\text{Vect}}\nolimits}
\newcommand{\Mat}{\mathop{\mathrm{Mat}}\nolimits}
\newcommand{\rg}{\mathop{\text{rg}}\nolimits}
\newcommand{\tr}{\mathop{\text{tr}}\nolimits}
\newcommand{\ppcm}{\mathop{\text{ppcm}}\nolimits}

%----- Structure des exercices ------

\newtheoremstyle{styleexo}% name
{2ex}% Space above
{3ex}% Space below
{}% Body font
{}% Indent amount 1
{\bfseries} % Theorem head font
{}% Punctuation after theorem head
{\newline}% Space after theorem head 2
{}% Theorem head spec (can be left empty, meaning ‘normal’)

%\theoremstyle{styleexo}
\newtheorem{exo}{Exercice}
\newtheorem{ind}{Indications}
\newtheorem{cor}{Correction}


\newcommand{\exercice}[1]{} \newcommand{\finexercice}{}
%\newcommand{\exercice}[1]{{\tiny\texttt{#1}}\vspace{-2ex}} % pour afficher le numero absolu, l'auteur...
\newcommand{\enonce}{\begin{exo}} \newcommand{\finenonce}{\end{exo}}
\newcommand{\indication}{\begin{ind}} \newcommand{\finindication}{\end{ind}}
\newcommand{\correction}{\begin{cor}} \newcommand{\fincorrection}{\end{cor}}

\newcommand{\noindication}{\stepcounter{ind}}
\newcommand{\nocorrection}{\stepcounter{cor}}

\newcommand{\fiche}[1]{} \newcommand{\finfiche}{}
\newcommand{\titre}[1]{\centerline{\large \bf #1}}
\newcommand{\addcommand}[1]{}
\newcommand{\video}[1]{}

% Marge
\newcommand{\mymargin}[1]{\marginpar{{\small #1}}}



%----- Presentation ------
\setlength{\parindent}{0cm}

%\newcommand{\ExoSept}{\href{http://exo7.emath.fr}{\textbf{\textsf{Exo7}}}}

\definecolor{myred}{rgb}{0.93,0.26,0}
\definecolor{myorange}{rgb}{0.97,0.58,0}
\definecolor{myyellow}{rgb}{1,0.86,0}

\newcommand{\LogoExoSept}[1]{  % input : echelle
{\usefont{U}{cmss}{bx}{n}
\begin{tikzpicture}[scale=0.1*#1,transform shape]
  \fill[color=myorange] (0,0)--(4,0)--(4,-4)--(0,-4)--cycle;
  \fill[color=myred] (0,0)--(0,3)--(-3,3)--(-3,0)--cycle;
  \fill[color=myyellow] (4,0)--(7,4)--(3,7)--(0,3)--cycle;
  \node[scale=5] at (3.5,3.5) {Exo7};
\end{tikzpicture}}
}



\theoremstyle{definition}
%\newtheorem{proposition}{Proposition}
%\newtheorem{exemple}{Exemple}
%\newtheorem{theoreme}{Théorème}
\newtheorem{lemme}{Lemme}
\newtheorem{corollaire}{Corollaire}
%\newtheorem*{remarque*}{Remarque}
%\newtheorem*{miniexercice}{Mini-exercices}
%\newtheorem{definition}{Définition}




%definition d'un terme
\newcommand{\defi}[1]{{\color{myorange}\textbf{\emph{#1}}}}
\newcommand{\evidence}[1]{{\color{blue}\textbf{\emph{#1}}}}



 %----- Commandes divers ------

\newcommand{\codeinline}[1]{\texttt{#1}}

%%%%%%%%%%%%%%%%%%%%%%%%%%%%%%%%%%%%%%%%%%%%%%%%%%%%%%%%%%%%%
%%%%%%%%%%%%%%%%%%%%%%%%%%%%%%%%%%%%%%%%%%%%%%%%%%%%%%%%%%%%%



\begin{document}

\debuttexte


%%%%%%%%%%%%%%%%%%%%%%%%%%%%%%%%%%%%%%%%%%%%%%%%%%%%%%%%%%%
\diapo

\change

Continuons notre exploration des sous-espaces vectoriels.

\change

Nous allons définir les combinaisons linéaires

\change 

et ainsi caractériser les sous-espaces vectoriels

\change

On termine avec l'intersection de deux sous-espaces vectoriels
qui reste un sous-espace vectoriel.

%%%%%%%%%%%%%%%%%%%%%%%%%%%%%%%%%%%%%%%%%%%%%%%%%%%%%%%%%%
\diapo

Fixons  $v_1, v_2, \ldots, v_n$, $n$ vecteurs d'un espace vectoriel $E$

On appelle \defi{combinaison linéaire} des vecteurs $v_1, v_2, \ldots, v_n$
tout vecteur de la forme  
 $$\lambda_1 v_1+\lambda_2v_2+ \cdots + \lambda_nv_n$$
 
\change

Les scalaires $\lambda_1, \lambda_2, \ldots , \lambda_n$ sont des éléments de $\Kk$
    et s'appelle les \defi{coefficients} de la combinaison linéaire

\change

Cela généralise une notion que vous connaissez déjà : dans le cas particulier où
$n=1$, alors $u=\lambda_1 v_1$ et on dit que $u$ est \defi{colinéaire} à $v_1$. 
 
 

%%%%%%%%%%%%%%%%%%%%%%%%%%%%%%%%%%%%%%%%%%%%%%%%%%%%%%%%%%%
\diapo

Voici quelques exemples simples.


Tout d'abord dans le $\Rr$-espace vectoriel $\Rr^3$, 
le vecteur $(3,3,1)$ est combinaison linéaire des vecteurs 
 $(1,1,0)$ et $(1,1,1)$, 
 tout simplement car on a l'égalité  
 $$(3,3,1)=2(1,1,0)+(1,1,1).$$
Le coefficient $\lambda_1$ vaut $2$ et le coefficient $\lambda_2$ vaut $1$.

\change

Par contre dans $\Rr^2$, le vecteur
$u=(2,1)$ \emph{n'est pas} colinéaire au vecteur $v_1=(1,1)$
car s'il l'était, il existerait un réel $\lambda_1$ tel que $u=\lambda_1 v_1$, ce qui est impossible.

\change

Passons maintenant à l'espace vectoriel des fonctions réelles.
Fixons quatre fonctions $f_0$, $f_1$, $f_2$ et $f_3$
 définies par : 
 $$f_0(x)=1, \;\;f_1(x)=x,\;\; f_2(x)=x^2,\;\; f_3(x)=x^3.$$
 
\change

 Alors la fonction $f$ définie par 
 $$f(x)=x^3-2x ^2-7x-4$$
 est combinaison linéaire des fonctions $f_0, f_1, f_2, f_3$ puisque l'on a l'égalité 
 $$f=f_{3}-2f_2-7f_1-4f_0.$$ 
 
 Et plus généralement n'importe quel polynômes de degré $\le 3$ peut s'écrire 
 comme combinaison linéaire de ces quatre fonctions.
 

%%%%%%%%%%%%%%%%%%%%%%%%%%%%%%%%%%%%%%%%%%%%%%%%%%%%%%%%%%
\diapo

Passons à un exemple en détails.

Soient $u = \left(\begin{smallmatrix}1\\ 2\\ -1\end{smallmatrix}\right)$ et $v =
   \left(\begin{smallmatrix}6\\4\\2\end{smallmatrix}\right)$ deux vecteurs de $\R^3$

   \change
   
  Il s'agit de montrer que $w =
   \left(\begin{smallmatrix}9\\ 2\\ 7\end{smallmatrix}\right)$ est combinaison linéaire de $u$ et $v$
   
   \change
   
   Cela signifie que l'on doit trouver deux réels $\lambda$ et $\mu$ tels que $w=\lambda u + \mu v$
  
  \change 
  
  Ce qui s'écrit $\left(\begin{matrix}9\\2\\7\end{matrix}\right) 
 =  \lambda \left(\begin{matrix}1\cr 2\cr -1\end{matrix}\right) + \mu \left(\begin{matrix}6\\4\\2\end{matrix}\right)$
 
 \change
 
 qui vaut  $  =  \left(\begin{matrix}\lambda\\ 2\lambda\\ -\lambda\end{matrix}\right) 
     + \left(\begin{matrix}6\mu\\ 4\mu\\ 2\mu\end{matrix}\right)$
   
   \change 
   
 ou encore $ =  \left(\begin{matrix}\lambda + 6\mu\\ 2\lambda + 4\mu\\ -\lambda + 2\mu\end{matrix}\right)$  
   
   \change 
  
  Trouver $\lambda$ et $\mu$ tels que $w=\lambda u + \mu v$ équivaut donc à résoudre le système 
  à $3$ équations et $2$ inconnues 
  $\left\{
\begin{array}{rcl}
9 & = & \lambda + 6\mu\\ 2 & = & 2\lambda + 4\mu\\ 7 & = & -\lambda + 2\mu 
\end{array}\right.
$

  \change
  
  On résout ce système et on trouve  $(\lambda = -3, \mu = 2)$,
  
  \change
  
ainsi  $w$ est bien combinaison linéaire de $u$ et $v$

  \change 
  
  et on a 

  $\left(\begin{smallmatrix}9\\2\\7\end{smallmatrix}\right)
=  -3\left(\begin{smallmatrix}1\\2\\-1\end{smallmatrix}\right) 
+ 2\left(\begin{smallmatrix}6\\4\\2\end{smallmatrix}\right)$


%%%%%%%%%%%%%%%%%%%%%%%%%%%%%%%%%%%%%%%%%%%%%%%%%%%%%%%%%%%
\diapo

Continuons avec un exemple un peu différent.

Soient $u = \left(\begin{smallmatrix}1\\ 2\\ -1\end{smallmatrix}\right)$ et $v =
  \left(\begin{smallmatrix}6\\4\\2\end{smallmatrix}\right)$

  \change
  
Nous allons montrer que $ w = \left(\begin{smallmatrix}4\\ -1\\ 8\end{smallmatrix}\right)$
  n'est pas combinaison linéaire de $u$ et $v$
  
  \change

  Etre combinaison linéaire s'écrit $w=\lambda u + _mu v$.
  
  \change
  
  ce qui équivaut au système 

  $
\left\{\begin{array}{rcl}
4 & = &\lambda + 6 \mu\\ -1 & = &2\lambda + 4\mu\\ 8 & = & -\lambda + 2\mu
\end{array}\right.$
  
  \change
  
  Ici on calcule que ce système n'a aucune solution.
  
  Donc il n'existe pas $\lambda,\mu \in \Rr$ tels que $w=\lambda u + \mu v$
  
  

%%%%%%%%%%%%%%%%%%%%%%%%%%%%%%%%%%%%%%%%%%%%%%%%%%%%%%%%%%
\diapo

Les combinaisons linéaires donnent une caractérisation
d'être (ou pas) un sous-espace vectoriel.


Soient $E$ un $\Kk$-espace vectoriel et $F$ une partie non vide de $E$.


$F$ est un sous-espace vectoriel de $E$ si et seulement si
$$\lambda u + \mu v \in F \qquad \text{pour tous } u,v \in F \quad \text{ et tous } \lambda, \mu \in \Kk.$$

\change


Autrement dit $F$ est un sous-espace vectoriel de $E$ si et seulement si toute combinaison linéaire de deux éléments 
de $F$ appartient à $F$.

%%%%%%%%%%%%%%%%%%%%%%%%%%%%%%%%%%%%%%%%%%%%%%%%%%%%%%%%%%%
\diapo

Montrons à présent que l'intersection de deux sous-espaces vectoriels est un sous-espace vectoriel.

Précisément si $F,G$ sont deux sous-espaces vectoriels d'un $\Kk$-espace vectoriel $E$.
alors l'intersection $F \cap G$ est aussi un sous-espace vectoriel de $E$.

\change

On démontrerait de même que l'intersection $F_1 \cap F_2 \cap F_3 \cap \cdots \cap F_n$ 
d'une famille quelconque de sous-espaces vectoriels de $E$ est un sous-espace 
vectoriel de $E$. 

\change

Passons à la preuve : soient $F$ et $G$ deux sous-espaces vectoriels de $E$. 

Il y a trois points à vérifier :

$0_E \in F$, $0_E\in G$ car $F$ et $G$ sont des sous-espaces vectoriels de $E$ ;
  donc $0_E \in F \cap G$.
  
\change

Montrons que $F \cap G$ est stable par addition :

Soient $u$ et $v$ deux vecteurs de  $F \cap G$.
Comme $F$ est un sous-espace vectoriel, alors $u,v \in F$ implique $u+v\in F$.
  De même $u,v \in G$ implique $u+v \in G$. Donc $u+v \in F \cap G$.
  
\change

Montrons que $F \cap G$ est stable par multiplication par un scalaire.


Soient $u \in F\cap G$ et $\lambda  \in \Kk$. 
Comme $F$ est un sous-espace vectoriel, 
  alors $u \in F$ implique $\lambda u \in F$. 
  De même $u \in G$ implique $\lambda u \in G$. 
  Donc $\lambda u \in F \cap G$.  

  
\change


On vient de vérifier que $F\cap G$ est un sous-espace vectoriel de $E$.

\change

On aurait pu regrouper ces deux points en montrant que
pour tout $u,v\in F$ toute combinaison linéaire $\lambda u + \mu v$ reste dans $F$.


%%%%%%%%%%%%%%%%%%%%%%%%%%%%%%%%%%%%%%%%%%%%%%%%%%%%%%%%%%%
\diapo


Soit $\mathcal{D}$ le sous-ensemble de $\Rr^3$ défini par :
$$\mathcal{D}= \big\{ (x,y,z) \in \Rr^3\mid x+3y+z =0 \;\; \text{ et } \;\; x-y+2z=0 \big\}.$$

Est-ce que $\mathcal{D}$ est sous-espace vectoriel de $\Rr^3$ ?

\change


L'ensemble $\mathcal{D}$ est l'intersection de $F$ et $G$, 
les sous-ensembles de $\Rr^3$ définis par :
$$F=\big\{ (x,y,z) \in \Rr^3\mid x+3y+z =0 \big\}$$

\change

et $$G=\big\{ (x,y,z) \in \Rr^3 \mid x-y+2z =0 \big\}$$


\change


Ce sont deux plans passant par l'origine, donc des sous-espaces vectoriels de $\Rr^3$.

\change

Ainsi $\mathcal{D} =F \cap G$ est un sous-espace vectoriel de $\Rr^3$, c'est une droite vectorielle.

 
 %%%%%%%%%%%%%%%%%%%%%%%%%%%%%%%%%%%%%%%%%%%%%%%%%%%%%%%%%%%
\diapo

On termine par un avertissement : 

La réunion de deux sous-espaces vectoriels de $E$ n'est pas en général 
un sous-espace vectoriel de $E$.

\change

Prenons par exemple $E$ l'espace vectoriel $\Rr^2$. Considérons les sous-espaces vectoriels
$F=\big\{(x,y)\mid x=0\big\}$ et $G=\big\{(x,y)\mid y=0\big\}$.

Alors $F\cup G$ n'est pas un sous-espace vectoriel de $\Rr^2$.

\change

Par exemple, $(0,1)$ est un élément de $F$,
$(1,0)$ est un élément de $G$, donc ces deux vecteurs sont dans $F\cup G$.

\change

Mais leur somme, le vecteur $(1,1)$  n'est pas dans $F\cup G$.

\change

L'union de deux sous-espaces vectoriels n'est pas en général 
un sous-espace vectoriel.



%%%%%%%%%%%%%%%%%%%%%%%%%%%%%%%%%%%%%%%%%%%%%%%%%%%%%%%%%%%
\diapo

Voici deux petits exercices pour vous entraîner !

\end{document}

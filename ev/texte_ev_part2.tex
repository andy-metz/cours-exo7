
%%%%%%%%%%%%%%%%%% PREAMBULE %%%%%%%%%%%%%%%%%%


\documentclass[12pt]{article}

\usepackage{amsfonts,amsmath,amssymb,amsthm}
\usepackage[utf8]{inputenc}
\usepackage[T1]{fontenc}
\usepackage[francais]{babel}


% packages
\usepackage{amsfonts,amsmath,amssymb,amsthm}
\usepackage[utf8]{inputenc}
\usepackage[T1]{fontenc}
%\usepackage{lmodern}

\usepackage[francais]{babel}
\usepackage{fancybox}
\usepackage{graphicx}

\usepackage{float}

%\usepackage[usenames, x11names]{xcolor}
\usepackage{tikz}
\usepackage{datetime}

\usepackage{mathptmx}
%\usepackage{fouriernc}
%\usepackage{newcent}
\usepackage[mathcal,mathbf]{euler}

%\usepackage{palatino}
%\usepackage{newcent}


% Commande spéciale prompteur

%\usepackage{mathptmx}
%\usepackage[mathcal,mathbf]{euler}
%\usepackage{mathpple,multido}

\usepackage[a4paper]{geometry}
\geometry{top=2cm, bottom=2cm, left=1cm, right=1cm, marginparsep=1cm}

\newcommand{\change}{{\color{red}\rule{\textwidth}{1mm}\\}}

\newcounter{mydiapo}

\newcommand{\diapo}{\newpage
\hfill {\normalsize  Diapo \themydiapo \quad \texttt{[\jobname]}} \\
\stepcounter{mydiapo}}


%%%%%%% COULEURS %%%%%%%%%%

% Pour blanc sur noir :
%\pagecolor[rgb]{0.5,0.5,0.5}
% \pagecolor[rgb]{0,0,0}
% \color[rgb]{1,1,1}



%\DeclareFixedFont{\myfont}{U}{cmss}{bx}{n}{18pt}
\newcommand{\debuttexte}{
%%%%%%%%%%%%% FONTES %%%%%%%%%%%%%
\renewcommand{\baselinestretch}{1.5}
\usefont{U}{cmss}{bx}{n}
\bfseries

% Taille normale : commenter le reste !
%Taille Arnaud
%\fontsize{19}{19}\selectfont

% Taille Barbara
%\fontsize{21}{22}\selectfont

%Taille François
\fontsize{25}{30}\selectfont

%Taille Pascal
%\fontsize{25}{30}\selectfont

%Taille Laura
%\fontsize{30}{35}\selectfont


%\myfont
%\usefont{U}{cmss}{bx}{n}

%\Huge
%\addtolength{\parskip}{\baselineskip}
}


% \usepackage{hyperref}
% \hypersetup{colorlinks=true, linkcolor=blue, urlcolor=blue,
% pdftitle={Exo7 - Exercices de mathématiques}, pdfauthor={Exo7}}


%section
% \usepackage{sectsty}
% \allsectionsfont{\bf}
%\sectionfont{\color{Tomato3}\upshape\selectfont}
%\subsectionfont{\color{Tomato4}\upshape\selectfont}

%----- Ensembles : entiers, reels, complexes -----
\newcommand{\Nn}{\mathbb{N}} \newcommand{\N}{\mathbb{N}}
\newcommand{\Zz}{\mathbb{Z}} \newcommand{\Z}{\mathbb{Z}}
\newcommand{\Qq}{\mathbb{Q}} \newcommand{\Q}{\mathbb{Q}}
\newcommand{\Rr}{\mathbb{R}} \newcommand{\R}{\mathbb{R}}
\newcommand{\Cc}{\mathbb{C}} 
\newcommand{\Kk}{\mathbb{K}} \newcommand{\K}{\mathbb{K}}

%----- Modifications de symboles -----
\renewcommand{\epsilon}{\varepsilon}
\renewcommand{\Re}{\mathop{\text{Re}}\nolimits}
\renewcommand{\Im}{\mathop{\text{Im}}\nolimits}
%\newcommand{\llbracket}{\left[\kern-0.15em\left[}
%\newcommand{\rrbracket}{\right]\kern-0.15em\right]}

\renewcommand{\ge}{\geqslant}
\renewcommand{\geq}{\geqslant}
\renewcommand{\le}{\leqslant}
\renewcommand{\leq}{\leqslant}

%----- Fonctions usuelles -----
\newcommand{\ch}{\mathop{\mathrm{ch}}\nolimits}
\newcommand{\sh}{\mathop{\mathrm{sh}}\nolimits}
\renewcommand{\tanh}{\mathop{\mathrm{th}}\nolimits}
\newcommand{\cotan}{\mathop{\mathrm{cotan}}\nolimits}
\newcommand{\Arcsin}{\mathop{\mathrm{Arcsin}}\nolimits}
\newcommand{\Arccos}{\mathop{\mathrm{Arccos}}\nolimits}
\newcommand{\Arctan}{\mathop{\mathrm{Arctan}}\nolimits}
\newcommand{\Argsh}{\mathop{\mathrm{Argsh}}\nolimits}
\newcommand{\Argch}{\mathop{\mathrm{Argch}}\nolimits}
\newcommand{\Argth}{\mathop{\mathrm{Argth}}\nolimits}
\newcommand{\pgcd}{\mathop{\mathrm{pgcd}}\nolimits} 

\newcommand{\Card}{\mathop{\text{Card}}\nolimits}
\newcommand{\Ker}{\mathop{\text{Ker}}\nolimits}
\newcommand{\id}{\mathop{\text{id}}\nolimits}
\newcommand{\ii}{\mathrm{i}}
\newcommand{\dd}{\mathrm{d}}
\newcommand{\Vect}{\mathop{\text{Vect}}\nolimits}
\newcommand{\Mat}{\mathop{\mathrm{Mat}}\nolimits}
\newcommand{\rg}{\mathop{\text{rg}}\nolimits}
\newcommand{\tr}{\mathop{\text{tr}}\nolimits}
\newcommand{\ppcm}{\mathop{\text{ppcm}}\nolimits}

%----- Structure des exercices ------

\newtheoremstyle{styleexo}% name
{2ex}% Space above
{3ex}% Space below
{}% Body font
{}% Indent amount 1
{\bfseries} % Theorem head font
{}% Punctuation after theorem head
{\newline}% Space after theorem head 2
{}% Theorem head spec (can be left empty, meaning ‘normal’)

%\theoremstyle{styleexo}
\newtheorem{exo}{Exercice}
\newtheorem{ind}{Indications}
\newtheorem{cor}{Correction}


\newcommand{\exercice}[1]{} \newcommand{\finexercice}{}
%\newcommand{\exercice}[1]{{\tiny\texttt{#1}}\vspace{-2ex}} % pour afficher le numero absolu, l'auteur...
\newcommand{\enonce}{\begin{exo}} \newcommand{\finenonce}{\end{exo}}
\newcommand{\indication}{\begin{ind}} \newcommand{\finindication}{\end{ind}}
\newcommand{\correction}{\begin{cor}} \newcommand{\fincorrection}{\end{cor}}

\newcommand{\noindication}{\stepcounter{ind}}
\newcommand{\nocorrection}{\stepcounter{cor}}

\newcommand{\fiche}[1]{} \newcommand{\finfiche}{}
\newcommand{\titre}[1]{\centerline{\large \bf #1}}
\newcommand{\addcommand}[1]{}
\newcommand{\video}[1]{}

% Marge
\newcommand{\mymargin}[1]{\marginpar{{\small #1}}}



%----- Presentation ------
\setlength{\parindent}{0cm}

%\newcommand{\ExoSept}{\href{http://exo7.emath.fr}{\textbf{\textsf{Exo7}}}}

\definecolor{myred}{rgb}{0.93,0.26,0}
\definecolor{myorange}{rgb}{0.97,0.58,0}
\definecolor{myyellow}{rgb}{1,0.86,0}

\newcommand{\LogoExoSept}[1]{  % input : echelle
{\usefont{U}{cmss}{bx}{n}
\begin{tikzpicture}[scale=0.1*#1,transform shape]
  \fill[color=myorange] (0,0)--(4,0)--(4,-4)--(0,-4)--cycle;
  \fill[color=myred] (0,0)--(0,3)--(-3,3)--(-3,0)--cycle;
  \fill[color=myyellow] (4,0)--(7,4)--(3,7)--(0,3)--cycle;
  \node[scale=5] at (3.5,3.5) {Exo7};
\end{tikzpicture}}
}



\theoremstyle{definition}
%\newtheorem{proposition}{Proposition}
%\newtheorem{exemple}{Exemple}
%\newtheorem{theoreme}{Théorème}
\newtheorem{lemme}{Lemme}
\newtheorem{corollaire}{Corollaire}
%\newtheorem*{remarque*}{Remarque}
%\newtheorem*{miniexercice}{Mini-exercices}
%\newtheorem{definition}{Définition}




%definition d'un terme
\newcommand{\defi}[1]{{\color{myorange}\textbf{\emph{#1}}}}
\newcommand{\evidence}[1]{{\color{blue}\textbf{\emph{#1}}}}



 %----- Commandes divers ------

\newcommand{\codeinline}[1]{\texttt{#1}}

%%%%%%%%%%%%%%%%%%%%%%%%%%%%%%%%%%%%%%%%%%%%%%%%%%%%%%%%%%%%%
%%%%%%%%%%%%%%%%%%%%%%%%%%%%%%%%%%%%%%%%%%%%%%%%%%%%%%%%%%%%%



\begin{document}

\debuttexte


%%%%%%%%%%%%%%%%%%%%%%%%%%%%%%%%%%%%%%%%%%%%%%%%%%%%%%%%%%%
\diapo

Dans cette leçon on continue l'étude de la définition d'espace vectoriel.

\change

\change

On reprend et on détaille les axiomes de la définition

\change

On regarde des exemples un peu plus sophistiqués

\change

On termine avec les règles de calcul dans les espaces vectoriels.

%%%%%%%%%%%%%%%%%%%%%%%%%%%%%%%%%%%%%%%%%%%%%%%%%%%%%%%%%%
\diapo

Revenons en détail sur la définition d'un espace vectoriel.

Soit $E$ un $\Kk$-espace vectoriel.


Les éléments de $E$ seront appelés des \defi{vecteurs}


Les éléments de $\Kk$ seront appelés des \defi{scalaires}.

Rappelons que $\Kk$ est un corps, et le plus souvent $\Kk$ sera le corps des nombres réels.

\change

Rappelons aussi qu'il y a deux lois : une loi interne et une loi externe.


La loi de composition interne dans $E$, qui est l'addition de deux vecteurs, 
est une application de $E \times E$ dans $E$ :

à partir de deux vecteurs $u$ et $v$ de $E$, 
on nous en fournit un troisième, qui sera noté $u+v$.

\change


La loi de composition externe, appelé multiplication par un scalaire,
 c'est une application de $\Kk \times E$ dans $E$ : 

à partir d'un scalaire $\lambda \in \Kk$ et d'un vecteur $u \in E$, on nous
fournit un autre vecteur, qui sera noté $\lambda\cdot u$.




%%%%%%%%%%%%%%%%%%%%%%%%%%%%%%%%%%%%%%%%%%%%%%%%%%%%%%%%%%%
\diapo

Les deux lois respectent une liste de 8 axiomes.

On commence par revoir les axiomes liés à l'addition.

\change


L'addition est commutative. Pour tous $u,v \in E$, $u + v = v + u$.
 On peut donc additionner des vecteurs dans l'ordre que l'on souhaite.
  
\change

L'addition est associative.  Pour tous $u,v,w \in E$, on a $u + (v+w) = (u+v) +w$.
 Conséquence : on peut \og oublier \fg\ les parenthèses et noter sans ambiguïté $u+v+w$.

\change

 Il existe un \evidence{élément neutre}, c'est-à-dire qu'il existe un élément de $E$,  
notée $0_{E}$, vérifiant : pour tout $u \in E$, $u+0_{E}=u$ 

\change

et on a aussi $0_E+u=u$ par commutativité.

\change

Cet élément $0_E$ s'appelle souvent le \defi{vecteur nul}.

\change

Tout élément $u$ de $E$ admet un \evidence{symétrique}
 (ou \defi{opposé}), 
 c'est-à-dire qu'il existe un élément $u'$ de $E$ tel que 
$u+u'=0_E$ 

\change

et on a aussi $u'+u=0_E$ par commutativité.

\change

Cet élément $u'$ de $E$ est noté $-u$.


Si vous connaissez la théorie des groupes vous reconnaissez, dans 
ces quatre premières propriétés, les axiomes caractérisant 
un groupe commutatif.

%%%%%%%%%%%%%%%%%%%%%%%%%%%%%%%%%%%%%%%%%%%%%%%%%%%%%%%%%%%
\diapo

Avant de passer aux axiomes suivants, montrons deux propriétés.

Premièrement 
un élément neutre $0_{E}$ vérifiant l'axiome (3) précédent, est unique.

\change

Deuxièmement 

Un symétrique d'un vecteur $u$ élément de $E$, vérifiant l'axiome 
(4) précédent, est unique.

On dira *le* symétrique, ou *l'opposé* de $u$.


\change

Prouvons ces deux propriétés, tout d'abord l'unicité de l'élément neutre.

Soient donc $0_{E}$ et $0'_{E}$ deux éléments vérifiant 
  la définition de l'élément neutre. 

  \change
  
  Comme $0_{E}$ est un élément neutre alors, pour tout vecteur $u$ :
$u + 0_{E}=0_{E}+u=u $

et comme $0'_{E}$ est aussi élément neutre alors $u + 0'_{E}=0'_{E}+u=u$.

\change

Cette première égalité  avec $u=0'_{E}$ donne $0'_{E}+0_{E}=0_{E}+0'_{E}=0'_{E}$.

\change

La deuxième égalité appliquée cette fois avec $u=0_{E}$ donne $0_{E}+0'_{E}=0'_{E}+0_{E}=0_{E}$.

\change

En comparant ces deux résultats, il vient $0_{E}=0'_{E}$.

\change

Passons à l'unicité du symétrique.

Fixons un vecteur $u$ et supposons qu'il existe deux symétriques de $u$ notés $u'$ et $u''$. 

\change

On a les deux égalités :
$$u+u'=u'+u=0_{E}  \qquad \text{ et } \qquad u+u''=u''+u=0_{E}.$$

\change

Calculons $u'+(u+u'')$ de deux façons différentes. 

Tout d'abord $u'+(u+u'')= u'+ 0_{E}= u'$ (en utilisant la second égalité)

\change

Et par l'associativité $u'+(u+u'')=(u'+u)+u''$ ce qui par la première égalité vaut $0_{E}+u''=u''$.

\change

On en déduit $u'=u''$. 


%%%%%%%%%%%%%%%%%%%%%%%%%%%%%%%%%%%%%%%%%%%%%%%%%%%%%%%%%%%
\diapo

On continue le détails des axiomes de la définition d'espace vectoriel. 

\change

On passe aux axiomes impliquant la multiplication par un scalaire.


 $1$ étant l'élément neutre de la multiplication de $\Kk$ alors
 pour tout vecteur $u$, $1 \cdot u=u.$
 
\change

Si maintenant on a deux scalaires $\lambda$ et $\mu$, c-à-d deux éléments  de $\Kk$ 
et si on a un vecteur $u$ alors 
 $$ \lambda \cdot (\mu \cdot u) = (\lambda \times \mu )\cdot u.$$
 
Ici $\times$ est la multiplication dans le corps $\Kk$ alors que $\cdot$ désigne la multiplication par un scalaire.

Avec l'habitude, on omet ces deux signes de multiplications, et selon le contexte vous devriez savoir
qu'elle est le type de multiplication à effectuer.

\change

Les deux derniers axiomes concerne le lien entre la loi interne et la loi externe.


\change

Plus précisément il s'agit de propriétés de distributivité.




 $$\lambda \cdot (u+v) =\lambda \cdot u + \lambda \cdot v$$
 
ceci pour tout scalaire $\lambda$ et tous vecteurs $u$ et $v$.

\change


Enfin  
$$(\lambda + \mu ) \cdot u=\lambda \cdot u + \mu \cdot u$$
Pour tous scalaires $\lambda$ et $\mu$ et tout vecteur $u$



 La loi interne et la loi externe doivent donc satisfaire ces huit axiomes pour que $E$ 
 soit un espace vectoriel sur $\Kk$.



%%%%%%%%%%%%%%%%%%%%%%%%%%%%%%%%%%%%%%%%%%%%%%%%%%%%%%%%%%
\diapo

Nous allons voir que l'ensemble des fonctions est un espace vectoriel.

Notons $\mathcal{F}(\Rr, \Rr)$ l'ensemble des fonctions $f : \Rr \longrightarrow \Rr$.

et le corps $\Kk$ est le corps des nombres réels.

Nous munissons l'ensemble des fonctions d'une structure d'espace vectoriel de la manière  suivante. 

\change

Tout d'abord l'addition :

Soient $f$ et $g$ deux fonctions. La fonction $f+g$ est définie par  :
$$(f+g)(x)=f(x)+g(x)$$


(le signe $+$ désigne la loi interne de $\mathcal{F}(\Rr , \Rr)$ dans le membre de gauche 
et l'addition dans $\Rr$ dans le membre de droite). 
 
\change

Puis la multiplication par un scalaire :


Si $\lambda$ est un nombre réel et $f$ une fonction, la fonction 
$\lambda \cdot f$ est définie par  :
$$(\lambda \cdot f) (x)=\lambda \times f (x).$$


(Ici $\cdot$ est la loi externe de $\mathcal{F}(\Rr, \Rr)$ et $\times $ est la multiplication dans 
$\Rr$.)
 
\change

L'élément neutre pour l'addition est la fonction nulle définie :
$$\forall x \in \Rr \quad f(x)=0.$$

\change

Enfin le symétrique d'une fonction $f$ est la fonction $g$ définie par : 
$$g(x)=-f(x).$$

Le symétrique de $f$ sera noté $-f$.    


Pour cet exemple un "vecteur" est donc une fonction,
[montrer] on peut additionner deux vecteurs,
multiplier un vecteur par un scalaire, le vecteur nul est la fonction nulle, 
le vecteur opposé est la fonction $-f$,...

Je vous laisse le soin de vérifier les 8 axiomes en détails.


%%%%%%%%%%%%%%%%%%%%%%%%%%%%%%%%%%%%%%%%%%%%%%%%%%%%%%%%%%%
\diapo

Un exemple similaire au précédent est l'espace vectoriel des suites.



On note $\mathcal{S}$ l'ensemble des suites réelles $(u_n)_{n\in \Nn}$.
Cet ensemble peut être vu comme l'ensemble des applications de $\N$ dans $\Rr$.

Un vecteur est donc ici une suite.

\change


Soient $u$ la suite des termes $(u_n)$ et $v$ la suite des termes $(v_n)$
alors la suite $u+v$ est la suite  dont le terme général est 
$u_n+v_n$ 

(ici $u_n+v_n$ désigne la somme de deux réels $u_n$ et $v_n$). 


\change


Si $\lambda$ est un réel et $u$ une suite de terme $(u_n)$ 
alors $\lambda \cdot u$ est la suite 
de terme général $\lambda \times u_n$

(ici $\times$ désigne la multiplication dans $\Rr$.)

\change

L'élément neutre est la suite nulle : c'est la suite dont tous les termes sont nuls.
 
 
\change

Enfin le symétrique de la suite $u=(u_n)_{n \in \Nn}$ est la suite de terme général $-u_n.$



%%%%%%%%%%%%%%%%%%%%%%%%%%%%%%%%%%%%%%%%%%%%%%%%%%%%%%%%%%%
\diapo

L'ensemble des matrices à $n$ lignes et $p$ colonnes à coefficients 
réels peut être vu comme un $\Rr$-espace vectoriel.


\change

La loi interne est l'addition de deux matrices.

La loi externe est la multiplication d'une matrice par un scalaire (c-à-d on multiplie tous les coefficients par le même facteur $\lambda$).

L'élément neutre est la matrice nulle (tous les coefficients sont nuls).

Le symétrique de la matrice $(a_{i,j})$ est la matrice 
dont les coefficients sont les $-a_{i,j}$. 
 

%%%%%%%%%%%%%%%%%%%%%%%%%%%%%%%%%%%%%%%%%%%%%%%%%%%%%%%%%%
\diapo

La définition d'espaces vectoriels est suffisamment générale pour englober
beaucoup de situations.

\change

Voyons encore quelques exemples :

L'ensemble $\Rr[X]$ des polynômes à coefficients réels
est un $\Rr$-espace vectoriel.

\change

L'addition de deux vecteurs est l'addition de deux polynômes $P(X)+Q(X)$, 

la multiplication par un scalaire 
  $\lambda$ est $\lambda \cdot P(X)$. 
  
  L'élément neutre est le polynôme nul. 
  
  L'opposé de $P(X)$ est $-P(X)$.
  
\change

L'ensemble des fonctions continues de $\Rr$ dans $\Rr$ est un $\Rr$-espace vectoriel ; 

\change

il en est de même pour l'ensemble des fonctions dérivables, les fonctions indéfiniment dérivables,...

\change

Il y aurait beaucoup d'autres exemples à étudier !

%%%%%%%%%%%%%%%%%%%%%%%%%%%%%%%%%%%%%%%%%%%%%%%%%%%%%%%%%%%
\diapo

On termine par les règles de calculs pour l'opération externe :

On revient à la situation générale : $E$ est un espace vectoriel quelconque sur un corps $\Kk$.

Soient $u$ un vecteur de $E$ et $\lambda$ un scalaire dans $\Kk$. 


\change

Tout d'abord $0 \cdot u = 0_E$, multiplier un vecteur par le scalaire $0$ donne le vecteur nul,

\change

$\lambda \cdot 0_E = 0_E$, 
multiplier le vecteur nul par n'importe quel scalaire $\lambda$ reste le vecteur nul.

\change

$(-1)\cdot u = -u$, multiplier un vecteur par le scalaire $-1$ donne le symétrique.

Cela justifie la notation $-u$ pour le symétrique.

\change

Enfin une propriété importante 

$\lambda \cdot u = 0_E \iff \lambda = 0$ \ ou \ $u = 0_E$

C'est une propriété similaire à celle bien connu pour les réels : 
"un produit est nul si et seulement si l'un des facteurs est nul"


\change

Terminons avec la définition de la soustraction
$u-v$ est l'addition $u+(-v)$

\change

La soustraction se comporte comme il convient :
par exemple  $\lambda (u-v)=\lambda u -\lambda v$
 et $(\lambda -\mu)u=\lambda u-\mu u$.

%%%%%%%%%%%%%%%%%%%%%%%%%%%%%%%%%%%%%%%%%%%%%%%%%%%%%%%%%%%
\diapo

Encore une fois il n'y a pas de secret pour comprendre les espaces vectoriels, il faut
revoir et apprendre le cours et faire beaucoup d'exercices. En voilà quelques uns.



\end{document}

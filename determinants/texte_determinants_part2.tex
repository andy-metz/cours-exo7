
%%%%%%%%%%%%%%%%%% PREAMBULE %%%%%%%%%%%%%%%%%%


\documentclass[12pt]{article}

\usepackage{amsfonts,amsmath,amssymb,amsthm}
\usepackage[utf8]{inputenc}
\usepackage[T1]{fontenc}
\usepackage[francais]{babel}


% packages
\usepackage{amsfonts,amsmath,amssymb,amsthm}
\usepackage[utf8]{inputenc}
\usepackage[T1]{fontenc}
%\usepackage{lmodern}

\usepackage[francais]{babel}
\usepackage{fancybox}
\usepackage{graphicx}

\usepackage{float}

%\usepackage[usenames, x11names]{xcolor}
\usepackage{tikz}
\usepackage{datetime}

\usepackage{mathptmx}
%\usepackage{fouriernc}
%\usepackage{newcent}
\usepackage[mathcal,mathbf]{euler}

%\usepackage{palatino}
%\usepackage{newcent}


% Commande spéciale prompteur

%\usepackage{mathptmx}
%\usepackage[mathcal,mathbf]{euler}
%\usepackage{mathpple,multido}

\usepackage[a4paper]{geometry}
\geometry{top=2cm, bottom=2cm, left=1cm, right=1cm, marginparsep=1cm}

\newcommand{\change}{{\color{red}\rule{\textwidth}{1mm}\\}}

\newcounter{mydiapo}

\newcommand{\diapo}{\newpage
\hfill {\normalsize  Diapo \themydiapo \quad \texttt{[\jobname]}} \\
\stepcounter{mydiapo}}


%%%%%%% COULEURS %%%%%%%%%%

% Pour blanc sur noir :
%\pagecolor[rgb]{0.5,0.5,0.5}
% \pagecolor[rgb]{0,0,0}
% \color[rgb]{1,1,1}



%\DeclareFixedFont{\myfont}{U}{cmss}{bx}{n}{18pt}
\newcommand{\debuttexte}{
%%%%%%%%%%%%% FONTES %%%%%%%%%%%%%
\renewcommand{\baselinestretch}{1.5}
\usefont{U}{cmss}{bx}{n}
\bfseries

% Taille normale : commenter le reste !
%Taille Arnaud
%\fontsize{19}{19}\selectfont

% Taille Barbara
%\fontsize{21}{22}\selectfont

%Taille François
\fontsize{25}{30}\selectfont

%Taille Pascal
%\fontsize{25}{30}\selectfont

%Taille Laura
%\fontsize{30}{35}\selectfont


%\myfont
%\usefont{U}{cmss}{bx}{n}

%\Huge
%\addtolength{\parskip}{\baselineskip}
}


% \usepackage{hyperref}
% \hypersetup{colorlinks=true, linkcolor=blue, urlcolor=blue,
% pdftitle={Exo7 - Exercices de mathématiques}, pdfauthor={Exo7}}


%section
% \usepackage{sectsty}
% \allsectionsfont{\bf}
%\sectionfont{\color{Tomato3}\upshape\selectfont}
%\subsectionfont{\color{Tomato4}\upshape\selectfont}

%----- Ensembles : entiers, reels, complexes -----
\newcommand{\Nn}{\mathbb{N}} \newcommand{\N}{\mathbb{N}}
\newcommand{\Zz}{\mathbb{Z}} \newcommand{\Z}{\mathbb{Z}}
\newcommand{\Qq}{\mathbb{Q}} \newcommand{\Q}{\mathbb{Q}}
\newcommand{\Rr}{\mathbb{R}} \newcommand{\R}{\mathbb{R}}
\newcommand{\Cc}{\mathbb{C}} 
\newcommand{\Kk}{\mathbb{K}} \newcommand{\K}{\mathbb{K}}

%----- Modifications de symboles -----
\renewcommand{\epsilon}{\varepsilon}
\renewcommand{\Re}{\mathop{\text{Re}}\nolimits}
\renewcommand{\Im}{\mathop{\text{Im}}\nolimits}
%\newcommand{\llbracket}{\left[\kern-0.15em\left[}
%\newcommand{\rrbracket}{\right]\kern-0.15em\right]}

\renewcommand{\ge}{\geqslant}
\renewcommand{\geq}{\geqslant}
\renewcommand{\le}{\leqslant}
\renewcommand{\leq}{\leqslant}

%----- Fonctions usuelles -----
\newcommand{\ch}{\mathop{\mathrm{ch}}\nolimits}
\newcommand{\sh}{\mathop{\mathrm{sh}}\nolimits}
\renewcommand{\tanh}{\mathop{\mathrm{th}}\nolimits}
\newcommand{\cotan}{\mathop{\mathrm{cotan}}\nolimits}
\newcommand{\Arcsin}{\mathop{\mathrm{Arcsin}}\nolimits}
\newcommand{\Arccos}{\mathop{\mathrm{Arccos}}\nolimits}
\newcommand{\Arctan}{\mathop{\mathrm{Arctan}}\nolimits}
\newcommand{\Argsh}{\mathop{\mathrm{Argsh}}\nolimits}
\newcommand{\Argch}{\mathop{\mathrm{Argch}}\nolimits}
\newcommand{\Argth}{\mathop{\mathrm{Argth}}\nolimits}
\newcommand{\pgcd}{\mathop{\mathrm{pgcd}}\nolimits} 

\newcommand{\Card}{\mathop{\text{Card}}\nolimits}
\newcommand{\Ker}{\mathop{\text{Ker}}\nolimits}
\newcommand{\id}{\mathop{\text{id}}\nolimits}
\newcommand{\ii}{\mathrm{i}}
\newcommand{\dd}{\mathrm{d}}
\newcommand{\Vect}{\mathop{\text{Vect}}\nolimits}
\newcommand{\Mat}{\mathop{\mathrm{Mat}}\nolimits}
\newcommand{\rg}{\mathop{\text{rg}}\nolimits}
\newcommand{\tr}{\mathop{\text{tr}}\nolimits}
\newcommand{\ppcm}{\mathop{\text{ppcm}}\nolimits}

%----- Structure des exercices ------

\newtheoremstyle{styleexo}% name
{2ex}% Space above
{3ex}% Space below
{}% Body font
{}% Indent amount 1
{\bfseries} % Theorem head font
{}% Punctuation after theorem head
{\newline}% Space after theorem head 2
{}% Theorem head spec (can be left empty, meaning ‘normal’)

%\theoremstyle{styleexo}
\newtheorem{exo}{Exercice}
\newtheorem{ind}{Indications}
\newtheorem{cor}{Correction}


\newcommand{\exercice}[1]{} \newcommand{\finexercice}{}
%\newcommand{\exercice}[1]{{\tiny\texttt{#1}}\vspace{-2ex}} % pour afficher le numero absolu, l'auteur...
\newcommand{\enonce}{\begin{exo}} \newcommand{\finenonce}{\end{exo}}
\newcommand{\indication}{\begin{ind}} \newcommand{\finindication}{\end{ind}}
\newcommand{\correction}{\begin{cor}} \newcommand{\fincorrection}{\end{cor}}

\newcommand{\noindication}{\stepcounter{ind}}
\newcommand{\nocorrection}{\stepcounter{cor}}

\newcommand{\fiche}[1]{} \newcommand{\finfiche}{}
\newcommand{\titre}[1]{\centerline{\large \bf #1}}
\newcommand{\addcommand}[1]{}
\newcommand{\video}[1]{}

% Marge
\newcommand{\mymargin}[1]{\marginpar{{\small #1}}}



%----- Presentation ------
\setlength{\parindent}{0cm}

%\newcommand{\ExoSept}{\href{http://exo7.emath.fr}{\textbf{\textsf{Exo7}}}}

\definecolor{myred}{rgb}{0.93,0.26,0}
\definecolor{myorange}{rgb}{0.97,0.58,0}
\definecolor{myyellow}{rgb}{1,0.86,0}

\newcommand{\LogoExoSept}[1]{  % input : echelle
{\usefont{U}{cmss}{bx}{n}
\begin{tikzpicture}[scale=0.1*#1,transform shape]
  \fill[color=myorange] (0,0)--(4,0)--(4,-4)--(0,-4)--cycle;
  \fill[color=myred] (0,0)--(0,3)--(-3,3)--(-3,0)--cycle;
  \fill[color=myyellow] (4,0)--(7,4)--(3,7)--(0,3)--cycle;
  \node[scale=5] at (3.5,3.5) {Exo7};
\end{tikzpicture}}
}



\theoremstyle{definition}
%\newtheorem{proposition}{Proposition}
%\newtheorem{exemple}{Exemple}
%\newtheorem{theoreme}{Théorème}
\newtheorem{lemme}{Lemme}
\newtheorem{corollaire}{Corollaire}
%\newtheorem*{remarque*}{Remarque}
%\newtheorem*{miniexercice}{Mini-exercices}
%\newtheorem{definition}{Définition}




%definition d'un terme
\newcommand{\defi}[1]{{\color{myorange}\textbf{\emph{#1}}}}
\newcommand{\evidence}[1]{{\color{blue}\textbf{\emph{#1}}}}



 %----- Commandes divers ------

\newcommand{\codeinline}[1]{\texttt{#1}}

%%%%%%%%%%%%%%%%%%%%%%%%%%%%%%%%%%%%%%%%%%%%%%%%%%%%%%%%%%%%%
%%%%%%%%%%%%%%%%%%%%%%%%%%%%%%%%%%%%%%%%%%%%%%%%%%%%%%%%%%%%%

\newcommand{\deter}{déter\-mi\-nant\ }
\newcommand{\deters}{déter\-mi\-nants\ }

\begin{document}

\debuttexte


%%%%%%%%%%%%%%%%%%%%%%%%%%%%%%%%%%%%%%%%%%%%%%%%%%%%%%%%%%%
\diapo

\change

Cette leçon est consacrée à la définition générale du \deter.
La définition du \deter est assez abstraite et il faudra attendre encore un peu pour pouvoir vraiment calculer des \deters.

\change
Nous commencerons tout de suite par donner la définition du \deter

\change
et ses premières propriétés.

\change
Nous verrons ensuite comment appliquer cette définition en calculant explicitement le \deter
de matrices particulières.

\change
Enfin, nous donnerons la démonstration de l'existence du \deter.

%%%%%%%%%%%%%%%%%%%%%%%%%%%%%%%%%%%%%%%%%%%%%%%%%%%%%%%%%%%
\diapo
Nous allons caractériser le \deter comme une application, 
qui à une matrice carrée $M \in M_n(\Kk)$ associe un scalaire 
$\det(M) \in \Kk$.

\change
Théorème : il existe une unique application de $M_n(\Kk)$ dans $\Kk$, 
appelée \defi{\deter}, telle que:

\change
Premièrement, le \deter est linéaire par rapport à chaque
vecteur colonne, les autres étant fixés

\change
Deuxièmement, si une matrice $A$ a deux colonnes identiques, 
alors son \deter est nul

\change
$3^\text{èmement}$, le \deter de la matrice identité $I_n$ vaut $1$.

\change
Quelques remarques. Une application de $M_n(\Kk)$ dans $\Kk$ qui satisfait la propriété
(i) est appelée \defi{forme multilinéaire}. 

\change
Si elle satisfait (ii), on dit qu'elle est \defi{alternée}.

Le \deter est donc la seule forme multilinéaire alternée qui prend 
comme valeur $1$ sur la matrice identité. 
%Les autres formes multilinéaires alternées sont les multiples scalaires du déterminant.  \\

On verra plus loin comment on
peut calculer en pratique les \deters.

%%%%%%%%%%%%%%%%%%%%%%%%%%%%%%%%%%%%%%%%%%%%%%%%%%%%%%%%%%%
\diapo

On note le  \deter d'une matrice $A = (a_{ij})$ par $\det A$ ou bien comme une matrice 
avec ses coefficients, mais en remplaçant
%ainsi : les coefficients sont ceux de la matrice $A$, mais on a remplacé 
les parenthèses par des barres verticales.

\change
Si on note $C_{i}$ la $i$-ème colonne de $A$, alors on écrit le \deter de $A$ 
comme le \deter des vecteurs colonnes $C_1,C_2,\ldots,C_n$.


%%%%%%%%%%%%%%%%%%%%%%%%%%%%%%%%%%%%%%%%%%%%%%%%%%%%%%%%%%%
\diapo

Avec cette notation, la propriété (i) de linéarité par rapport à la $j$-ème colonne s'écrit:

\change
pour tout $\lambda,\mu \in \Kk$,  $\det (C_1,\ldots,\lambda C_j + \mu C'_j,\ldots, C_n) $

\change
$= \lambda  \det (C_1,\ldots,C_j,\ldots, C_n)+ \mu \det (C_1,\ldots,C_j',\ldots, C_n)$,

\change
soit, si on utilise la notation avec les coefficients, ce \deter...
%$$\begin{vmatrix}
%a_{11} & \cdots & \lambda a_{1j}+\mu a_{1j}' & \cdots & a_{1n} \\
%\vdots &        &       \vdots                &        & \vdots\\
%a_{i1} & \cdots & \lambda a_{ij}+\mu a_{ij}' & \cdots & a_{in} \\
%\vdots &        &       \vdots                &        & \vdots\\
%a_{n1} & \cdots & \lambda a_{nj}+\mu a_{nj}' & \cdots & a_{nn} \\
%\end{vmatrix} 
%$$

\change
... vaut $\lambda$ fois ce \deter, plus $\mu$ fois celui-ci.

%$$
%=
%\lambda \begin{vmatrix}
%a_{11} & \cdots &  a_{1j} & \cdots & a_{1n} \\
%\vdots &        &       \vdots                &        & \vdots\\
%a_{i1} & \cdots & a_{ij} & \cdots & a_{in} \\
%\vdots &        &       \vdots                &        & \vdots\\
%a_{n1} & \cdots & a_{nj} & \cdots & a_{nn} \\
%\end{vmatrix}
% +
%\mu \begin{vmatrix}
%a_{11} & \cdots & a_{1j}' & \cdots & a_{1n} \\
%\vdots &        &       \vdots                &        & \vdots\\
%a_{i1} & \cdots & a_{ij}' & \cdots & a_{in} \\
%\vdots &        &       \vdots                &        & \vdots\\
%a_{n1} & \cdots & a_{nj}' & \cdots & a_{nn} \\
%\end{vmatrix}.
%$$


%%%%%%%%%%%%%%%%%%%%%%%%%%%%%%%%%%%%%%%%%%%%%%%%%%%%%%%%%%%
\diapo

Considérons ce \deter.  Comme la seconde colonne est un multiple de $5$,

\change
il est à égal à $5$ fois cet autre \deter
%$$
%5\times\begin{vmatrix}
%6&1&4\\7&-2&-3\\12&5&-1    
%  \end{vmatrix} $$
dans lequel la première et la dernière colonnes n'ont pas changé, 
et la deuxième colonne a été divisée par $5$. \\
On a donc factorisé par $5$ sur la deuxième colonne.

\change
Dans cet autre exemple, on peut écrire la troisième colonne comme différence de deux colonnes.

\change
Par linéarité (sur la troisième colonne) on peut écrire
le \deter comme différence de deux \deters.
%$$\begin{vmatrix}
%3&2&4-3\\7&-5&3-2\\9&2&10-4\\   
%  \end{vmatrix}
%$$
%
%\change
%$$=
%\begin{vmatrix}
%3&2&4\\7&-5&3\\9&2&10\\   
%  \end{vmatrix}
%  - \begin{vmatrix}
%3&2&3\\7&-5&2\\9&2&4\\   
%  \end{vmatrix}
%$$

%%%%%%%%%%%%%%%%%%%%%%%%%%%%%%%%%%%%%%%%%%%%%%%%%%%%%%%%%%%
\diapo
Nous connaissons déjà le \deter de deux matrices: 

\change
le \deter de la matrice nulle $0_n$ vaut $0$ (par la propriété (ii)),

\change
le \deter de la matrice identité $I_n$ vaut $1$ (par la propriété (iii)).

Donnons maintenant quelques propriétés importantes du \deter: comment se comporte le \deter face aux opérations élémentaires sur les colonnes ?

\change
Proposition : Soit $A \in M_n(\Kk)$ une matrice ayant les colonnes $C_1, C_2, \ldots, C_n$.

\change
On note $A'$ la matrice obtenue par une des opérations élémentaires sur les colonnes, qui sont :

\change
(1) $A'$ est obtenue en multipliant une colonne de $A$ par un scalaire $\lambda$ non nul \\
ce qui se note $C_i \leftarrow \lambda C_i$. 
  
\change
Alors $\det A' = \lambda \det A$.
  
\change
(2) $A'$ est obtenue en ajoutant à une colonne de $A$ un multiple d'une autre colonne de $A$,
ce qui se note $C_i \leftarrow C_i+\lambda C_j$ avec $\lambda \in \Kk$ (et $j\neq i$).

\change
Alors $\det A' = \det A$.
  
\change
(3) $A'$ est obtenue en échangeant deux colonnes distinctes de $A$, ce qui se note 
$C_i$ échangé avec $C_j$.

\change
Alors $\det A' = **-** \det A$.

\change
Plus généralement pour (2): l'opération d'ajouter à une colonne une combinaison linéaire 
des autres colonnes con\-serve le \deter.\\

% Attention ! [montrer (3)] \'Echanger deux colonnes change le signe du déterminant.

%%%%%%%%%%%%%%%%%%%%%%%%%%%%%%%%%%%%%%%%%%%%%%%%%%%%%%%%%%%
\diapo
Démontrons par exemple le point (2) de la proposition précédente.

\change
Soit $A=\begin{pmatrix}
C_1&\cdots&C_i&\cdots&C_j&\cdots&C_n
\end{pmatrix}$ une matrice représentée par ses vecteurs colonnes $C_k$. 

\change
L'opération $C_i \leftarrow C_i+\lambda C_j$ transforme la matrice $A$ en 
%la matrice 
$A'=\begin{pmatrix}
C_1&\cdots&\displaystyle C_i+\lambda C_j \text{ (en ième position) }
&\cdots&C_j&\cdots&C_n\end{pmatrix}$

\change
Par linéarité par rapport à la colonne $i$, on sait que 

$\det A'=\det A+\lambda \det \begin{pmatrix} C_1&\cdots&C_j&\cdots&C_j&\cdots&C_n
\end{pmatrix}.$

\change
Or les colonnes $i$ et $j$ de cette matrice 

sont identiques.

\change
Son \deter est donc nul. 

\change
On a bien alors que les \deters des matrices $A$ et $A'$ sont égaux, ce qui conclut la démonstration.

\change
On déduit de la proposition le corollaire suivant: 
si une colonne $C_i$ de la matrice $A$ est combinaison linéaire des autres colonnes, alors $\det A=0$.

%%%%%%%%%%%%%%%%%%%%%%%%%%%%%%%%%%%%%%%%%%%%%%%%%%%%%%%%%%%
\diapo

Calculer des \deters n'est pas toujours facile. Mais c'est facile si
la matrice est triangulaire.


Proposition: Le \deter d'une matrice triangulaire supérieure (ou inférieure)
est égal au produit des termes diagonaux.

\change
Autrement dit, pour une matrice triangulaire $A = (a_{ij})$  on a que $\det A$ est égal

\change
au produit $ a_{11}\cdot a_{22} \ \cdots\  a_{nn}.$

\change
Comme cas particulier important, on obtient le corollaire suivant: 
le \deter d'une matrice diagonale est égal au produit des termes diagonaux. 



%%%%%%%%%%%%%%%%%%%%%%%%%%%%%%%%%%%%%%%%%%%%%%%%%%%%%%%%%%%
\diapo

Démontrons que pour une matrice triangulaire supérieure
le \deter est égal au produit des termes diagonaux.
\\
Soit donc $A$ une matrice triangulaire supérieure.

\change
La façon de procéder utilise l'algorithme du pivot de Gauss sur les colonnes.

% (alors qu'il est en général défini sur les lignes).

\change
Par linéarité par rapport à la première colonne, on peut factoriser par $a_{11}$ 
et obtenir le \deter suivant
%$$\det A=a_{11}\left|\begin{matrix}
%1&a_{12}&a_{13}&\cdots&a_{1n}\\
%0&{a_{22}}&a_{23}&\cdots&a_{2n}\\
%0&0&{a_{33}}&\cdots&a_{3n}\\
%\vdots&\vdots&\vdots&\ddots&\vdots\\
%0&0&0&\cdots&{a_{nn}}
%\end{matrix}\right|$$
\\
(et si $a_{11}$ est nul, le résultat est encore vrai, de façon évidente).

\change
On soustrait maintenant de chaque colonne $C_j$, pour $j\ge 2$, la nouvelle colonne $C_1$ multipliée 
par $-a_{1j}$.
%C'est l'opération élémentaire $C_j \leftarrow C_j - a_{1j}C_1$. 
Ceci ne modifie pas le \deter d'après la section précédente,

\change
et a pour effet de placer des $0$ sur tout le reste de la première ligne.

%Il vient donc
%$$\det A=a_{11}\left|\begin{matrix}
%1&0&0&\cdots&0\\
%0&{a_{22}}&a_{23}&\cdots&a_{2n}\\
%0&0&{a_{33}}&\cdots&a_{3n}\\
%\vdots&\vdots&\vdots&\ddots&\vdots\\
%0&0&0&\cdots&{a_{nn}}
%\end{matrix}\right|.$$



%%%%%%%%%%%%%%%%%%%%%%%%%%%%%%%%%%%%%%%%%%%%%%%%%%%%%%%%%%%
\diapo
Par linéarité par rapport à la deuxième colonne, on factorise par $a_{22}$

et on obtient $det A = a_{11} \times a_{22} \times$ ce nouveau \deter, qui est plus simple.
%$$\det A=a_{11} \cdot a_{22}\left|\begin{matrix}
%1&0&0&\cdots&0\\
%0&1&a_{23}&\cdots&a_{2n}\\
%0&0&{a_{33}}&\cdots&a_{3n}\\
%\vdots&\vdots&\vdots&\ddots&\vdots\\
%0&0&0&\cdots&{a_{nn}}
%\end{matrix}\right|,$$


\change
et l'on continue ainsi jusqu'à avoir parcouru toutes les colonnes de la matrice.  

Au bout de $n$ étapes, on a obtenu ceci

%$$\det A=a_{11} \cdot a_{22} \cdot a_{33}\cdots{a_{nn}}\left|\begin{matrix}
%1&0&0&\cdots&0\\
%0&1&0&\cdots&0\\
%0&0&1&\cdots&0\\
%\vdots&\vdots&\vdots&\ddots&\vdots\\
%0&0&0&\cdots&1
%\end{matrix}\right|
%$$

\change
Ce qui donne finalement que 
\\
 $\det A = a_{11} \times a_{22} \times a_{33}\times \cdots \times {a_{nn}} \times \det I_n$

\change
d'où le résultat, car le \deter de la matrice identité vaut $1$ par définition.


%%%%%%%%%%%%%%%%%%%%%%%%%%%%%%%%%%%%%%%%%%%%%%%%%%%%%%%%%%%
\diapo
La démonstration du théorème d'existence du \deter est ardue. Nous allons voir ici, 
sans le démontrer, le résultat sur lequel est basée cette démonstration. 
Par ailleurs, l'unicité du \deter, plus difficile, est admise.

Nous avons besoin d'une notation. \\
Soit $A \in M_n(\Kk)$ une matrice carrée de taille $n \times n$. 
Il est évident que si l'on supprime une ligne et une colonne dans $A$, 
la matrice obtenue a $n-1$ lignes et $n-1$ colonnes. 

\change
On note $A_{i,j}$ la matrice obtenue en supprimant la $i$-ème ligne et la $j$-ème
colonne de $A$. 

%\change
%Il s'agit donc de démontrer l'existence d'une application satisfaisant aux conditions (i), (ii), (iii) caractérisant le \deter. 

\change
Le théorème d'existence peut s'énoncer ainsi.

Les formules suivantes définissent par récurrence, pour $n\ge 1$, 
une application de $M_n(\Kk)$ dans $\Kk$ qui satisfait aux propriétés
(i), (ii), (iii) caractérisant le \deter :

\change
Pour une matrice $1\times 1$, composée d'un seul élément, égal à petit $a$, $\det A = \text{(petit)} a$.
  
\change
(...)
\newpage
Et la formule de récurrence :

Si $A=(a_{i,j})$ est une matrice carrée de taille $n \times n$, alors pour tout $i$ fixé
$$\det A = (-1)^{i+1}a_{i,1}\det A_{i,1} +\dots + (-1)^{i+n}a_{i,n}\det A_{i,n}.$$  

On retrouvera plus loin dans le chapitre cette formule, 
dite de développement par rapport à une ligne (ici la ligne numéro $i$). 
Elle nous offre une méthode de calcul par récurrence du \deter d'une matrice
en fonction de \deters de matrices de taille plus petite.
\\

La formule donnée ci-dessus semble dépendre du choix d'un indice $i$ de ligne. 
% On peut se demander ce qui se passerait si l'on prenait une autre valeur de $i$. \\
Mais grâce à l'unicité du \deter d'une matrice, le résultat est le même, quel que soit l'indice.   


%%%%%%%%%%%%%%%%%%%%%%%%%%%%%%%%%%%%%%%%%%%%%%%%%%%%%%%%%%%
\diapo

Voici enfin quelques petits exercices, pour tester votre compréhension. 

\end{document}

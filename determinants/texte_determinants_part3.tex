
%%%%%%%%%%%%%%%%%% PREAMBULE %%%%%%%%%%%%%%%%%%


\documentclass[12pt]{article}

\usepackage{amsfonts,amsmath,amssymb,amsthm}
\usepackage[utf8]{inputenc}
\usepackage[T1]{fontenc}
\usepackage[francais]{babel}


% packages
\usepackage{amsfonts,amsmath,amssymb,amsthm}
\usepackage[utf8]{inputenc}
\usepackage[T1]{fontenc}
%\usepackage{lmodern}

\usepackage[francais]{babel}
\usepackage{fancybox}
\usepackage{graphicx}

\usepackage{float}

%\usepackage[usenames, x11names]{xcolor}
\usepackage{tikz}
\usepackage{datetime}

\usepackage{mathptmx}
%\usepackage{fouriernc}
%\usepackage{newcent}
\usepackage[mathcal,mathbf]{euler}

%\usepackage{palatino}
%\usepackage{newcent}


% Commande spéciale prompteur

%\usepackage{mathptmx}
%\usepackage[mathcal,mathbf]{euler}
%\usepackage{mathpple,multido}

\usepackage[a4paper]{geometry}
\geometry{top=2cm, bottom=2cm, left=1cm, right=1cm, marginparsep=1cm}

\newcommand{\change}{{\color{red}\rule{\textwidth}{1mm}\\}}

\newcounter{mydiapo}

\newcommand{\diapo}{\newpage
\hfill {\normalsize  Diapo \themydiapo \quad \texttt{[\jobname]}} \\
\stepcounter{mydiapo}}


%%%%%%% COULEURS %%%%%%%%%%

% Pour blanc sur noir :
%\pagecolor[rgb]{0.5,0.5,0.5}
% \pagecolor[rgb]{0,0,0}
% \color[rgb]{1,1,1}



%\DeclareFixedFont{\myfont}{U}{cmss}{bx}{n}{18pt}
\newcommand{\debuttexte}{
%%%%%%%%%%%%% FONTES %%%%%%%%%%%%%
\renewcommand{\baselinestretch}{1.5}
\usefont{U}{cmss}{bx}{n}
\bfseries

% Taille normale : commenter le reste !
%Taille Arnaud
%\fontsize{19}{19}\selectfont

% Taille Barbara
%\fontsize{21}{22}\selectfont

%Taille François
\fontsize{25}{30}\selectfont

%Taille Pascal
%\fontsize{25}{30}\selectfont

%Taille Laura
%\fontsize{30}{35}\selectfont


%\myfont
%\usefont{U}{cmss}{bx}{n}

%\Huge
%\addtolength{\parskip}{\baselineskip}
}


% \usepackage{hyperref}
% \hypersetup{colorlinks=true, linkcolor=blue, urlcolor=blue,
% pdftitle={Exo7 - Exercices de mathématiques}, pdfauthor={Exo7}}


%section
% \usepackage{sectsty}
% \allsectionsfont{\bf}
%\sectionfont{\color{Tomato3}\upshape\selectfont}
%\subsectionfont{\color{Tomato4}\upshape\selectfont}

%----- Ensembles : entiers, reels, complexes -----
\newcommand{\Nn}{\mathbb{N}} \newcommand{\N}{\mathbb{N}}
\newcommand{\Zz}{\mathbb{Z}} \newcommand{\Z}{\mathbb{Z}}
\newcommand{\Qq}{\mathbb{Q}} \newcommand{\Q}{\mathbb{Q}}
\newcommand{\Rr}{\mathbb{R}} \newcommand{\R}{\mathbb{R}}
\newcommand{\Cc}{\mathbb{C}} 
\newcommand{\Kk}{\mathbb{K}} \newcommand{\K}{\mathbb{K}}

%----- Modifications de symboles -----
\renewcommand{\epsilon}{\varepsilon}
\renewcommand{\Re}{\mathop{\text{Re}}\nolimits}
\renewcommand{\Im}{\mathop{\text{Im}}\nolimits}
%\newcommand{\llbracket}{\left[\kern-0.15em\left[}
%\newcommand{\rrbracket}{\right]\kern-0.15em\right]}

\renewcommand{\ge}{\geqslant}
\renewcommand{\geq}{\geqslant}
\renewcommand{\le}{\leqslant}
\renewcommand{\leq}{\leqslant}

%----- Fonctions usuelles -----
\newcommand{\ch}{\mathop{\mathrm{ch}}\nolimits}
\newcommand{\sh}{\mathop{\mathrm{sh}}\nolimits}
\renewcommand{\tanh}{\mathop{\mathrm{th}}\nolimits}
\newcommand{\cotan}{\mathop{\mathrm{cotan}}\nolimits}
\newcommand{\Arcsin}{\mathop{\mathrm{Arcsin}}\nolimits}
\newcommand{\Arccos}{\mathop{\mathrm{Arccos}}\nolimits}
\newcommand{\Arctan}{\mathop{\mathrm{Arctan}}\nolimits}
\newcommand{\Argsh}{\mathop{\mathrm{Argsh}}\nolimits}
\newcommand{\Argch}{\mathop{\mathrm{Argch}}\nolimits}
\newcommand{\Argth}{\mathop{\mathrm{Argth}}\nolimits}
\newcommand{\pgcd}{\mathop{\mathrm{pgcd}}\nolimits} 

\newcommand{\Card}{\mathop{\text{Card}}\nolimits}
\newcommand{\Ker}{\mathop{\text{Ker}}\nolimits}
\newcommand{\id}{\mathop{\text{id}}\nolimits}
\newcommand{\ii}{\mathrm{i}}
\newcommand{\dd}{\mathrm{d}}
\newcommand{\Vect}{\mathop{\text{Vect}}\nolimits}
\newcommand{\Mat}{\mathop{\mathrm{Mat}}\nolimits}
\newcommand{\rg}{\mathop{\text{rg}}\nolimits}
\newcommand{\tr}{\mathop{\text{tr}}\nolimits}
\newcommand{\ppcm}{\mathop{\text{ppcm}}\nolimits}

%----- Structure des exercices ------

\newtheoremstyle{styleexo}% name
{2ex}% Space above
{3ex}% Space below
{}% Body font
{}% Indent amount 1
{\bfseries} % Theorem head font
{}% Punctuation after theorem head
{\newline}% Space after theorem head 2
{}% Theorem head spec (can be left empty, meaning ‘normal’)

%\theoremstyle{styleexo}
\newtheorem{exo}{Exercice}
\newtheorem{ind}{Indications}
\newtheorem{cor}{Correction}


\newcommand{\exercice}[1]{} \newcommand{\finexercice}{}
%\newcommand{\exercice}[1]{{\tiny\texttt{#1}}\vspace{-2ex}} % pour afficher le numero absolu, l'auteur...
\newcommand{\enonce}{\begin{exo}} \newcommand{\finenonce}{\end{exo}}
\newcommand{\indication}{\begin{ind}} \newcommand{\finindication}{\end{ind}}
\newcommand{\correction}{\begin{cor}} \newcommand{\fincorrection}{\end{cor}}

\newcommand{\noindication}{\stepcounter{ind}}
\newcommand{\nocorrection}{\stepcounter{cor}}

\newcommand{\fiche}[1]{} \newcommand{\finfiche}{}
\newcommand{\titre}[1]{\centerline{\large \bf #1}}
\newcommand{\addcommand}[1]{}
\newcommand{\video}[1]{}

% Marge
\newcommand{\mymargin}[1]{\marginpar{{\small #1}}}



%----- Presentation ------
\setlength{\parindent}{0cm}

%\newcommand{\ExoSept}{\href{http://exo7.emath.fr}{\textbf{\textsf{Exo7}}}}

\definecolor{myred}{rgb}{0.93,0.26,0}
\definecolor{myorange}{rgb}{0.97,0.58,0}
\definecolor{myyellow}{rgb}{1,0.86,0}

\newcommand{\LogoExoSept}[1]{  % input : echelle
{\usefont{U}{cmss}{bx}{n}
\begin{tikzpicture}[scale=0.1*#1,transform shape]
  \fill[color=myorange] (0,0)--(4,0)--(4,-4)--(0,-4)--cycle;
  \fill[color=myred] (0,0)--(0,3)--(-3,3)--(-3,0)--cycle;
  \fill[color=myyellow] (4,0)--(7,4)--(3,7)--(0,3)--cycle;
  \node[scale=5] at (3.5,3.5) {Exo7};
\end{tikzpicture}}
}



\theoremstyle{definition}
%\newtheorem{proposition}{Proposition}
%\newtheorem{exemple}{Exemple}
%\newtheorem{theoreme}{Théorème}
\newtheorem{lemme}{Lemme}
\newtheorem{corollaire}{Corollaire}
%\newtheorem*{remarque*}{Remarque}
%\newtheorem*{miniexercice}{Mini-exercices}
%\newtheorem{definition}{Définition}




%definition d'un terme
\newcommand{\defi}[1]{{\color{myorange}\textbf{\emph{#1}}}}
\newcommand{\evidence}[1]{{\color{blue}\textbf{\emph{#1}}}}



 %----- Commandes divers ------

\newcommand{\codeinline}[1]{\texttt{#1}}

%%%%%%%%%%%%%%%%%%%%%%%%%%%%%%%%%%%%%%%%%%%%%%%%%%%%%%%%%%%%%
%%%%%%%%%%%%%%%%%%%%%%%%%%%%%%%%%%%%%%%%%%%%%%%%%%%%%%%%%%%%%



\begin{document}

\debuttexte


%%%%%%%%%%%%%%%%%%%%%%%%%%%%%%%%%%%%%%%%%%%%%%%%%%%%%%%%%%%
\diapo

\change
Nous poursuivons ce chapitre consacré aux déterminants par une leçon 
consacrée à certaines propriétés essentielles.

\change
Mais avant tout, pour établir ces propriétés, nous aurons besoin des matrices élémentaires.

\change
Nous montrerons ensuite trois propriétés importantes : le déterminant d'un produit de matrices,

\change
 le déterminant de l'inverse d'une matrice, 

\change
et enfin le déterminant de la transposée d'une matrice. 

%%%%%%%%%%%%%%%%%%%%%%%%%%%%%%%%%%%%%%%%%%%%%%%%%%%%%%%%%%%
\diapo
Pour chacune des opérations élémentaires sur les colonnes d'une matrice $A$, 
on associe une matrice élémentaire $E$, de sorte que la matrice obtenue par l'opération élémentaire sur $A$ soit $A'=A\times E$.

\change
Pour l'opération $C_i \leftarrow \lambda C_i$, la matrice élémentaire $E$ est la matrice diagonale ne comportant que des $1$, sauf en position $(i,i)$.

\change
Pour l'opération $C_i \leftarrow C_i+\lambda C_j$, la matrice $E$ est comme la matrice identité, sauf en position $(j,i)$ où son coefficient vaut $\lambda$.
 
\change
Pour l'opération $C_i \leftrightarrow C_j$, la matrice $E$ est comme la matrice identité, sauf que ses coefficients $(i,i)$ et $(j,j)$ s'annulent, tandis que les coefficients  $(i,j)$ et $(j,i)$ valent $1$.

%%%%%%%%%%%%%%%%%%%%%%%%%%%%%%%%%%%%%%%%%%%%%%%%%%%%%%%%%%%
\diapo
Nous allons détailler le cas de chaque opération et son effet sur le déterminant. 
\\

Proposition.

\change
Le déterminant de la matrice élémentaire $ E_{C_i \leftarrow \lambda C_i}$ vaut $ \lambda$.

\change
$\det E_{C_i \leftarrow C_i+\lambda C_j} = +1$

\change
$\det E_{C_i \leftrightarrow C_j} = -1$

\change
Enfin, si $E$ est une des matrices élémentaires ci-dessus, alors $\det \left( A \times E \right) = \det A \times \det E$.

\change
Cette proposition nous permet de calculer le déterminant d'une matrice $A$ de façon 
relativement simple, en utilisant l'algorithme de Gauss. 

\change
En effet, en multipliant successivement $A$ par des matrices élémentaires 
$E_1,\ldots,E_r$ on sait se ramener à une matrice $T$ échelonnée, donc triangulaire.

\change
(...)
\newpage
En prenant le déterminant on obtient $\det T=\det (A \cdot E_1 \cdots E_r) $

\change
Alors, en appliquant la proposition précédente

\change
 $r$-fois on obtient: $\det T = \det A \cdot \det E_1\cdot \det E_2 \cdots \det E_r$ .
 
Comme on sait calculer le déterminant de la matrice triangulaire $T$ et les déterminants des matrices élémentaires $E_i$, on en déduit le déterminant de $A$.

%%%%%%%%%%%%%%%%%%%%%%%%%%%%%%%%%%%%%%%%%%%%%%%%%%%%%%%%%%%
\diapo
En pratique cela ce passe comme sur l'exemple suivant. On souhaite calculer  le déterminant de la matrice $A$ suivante.

\change
On commence le pivot de Gauss par l'opération $C_1 \leftrightarrow C_2$ de manière à avoir un pivot en haut à gauche.

\change
On obtient que le déterminant de $A$ est égal au produit du déterminant de la matrice de cette opération élémentaire, c'est-à-dire $-1$, par le déterminant de la matrice transformée.

\change
On poursuit par l'opération élémentaire $C_1 \leftarrow \frac{1}{3} C_1$ 
ce qui revient à factoriser par $3$ la première colonne,

\change
ce qui donne 
$(-1)\times 3 \times \det \begin{pmatrix}
        1 & 0 & 2\\
        -2& 1 & 6\\
        3 & 5 & 1  
       \end{pmatrix} 
$

\change
Pour faire apparaître un $0$ ici, on fait l'opération \\
$C_3 \leftarrow C_3-2C_1 $

\change
(...)
\newpage
et on obtient $(-1)\times 3 \times \det \begin{pmatrix}
        1 & 0 & 0\\
        -2& 1 & 10\\
        3 & 5 & -5  
       \end{pmatrix}$.

\change
Enfin, l'opération $C_3 \leftarrow C_3-10C_2 $

\change
permet d'obtenir une matrice triangulaire, ce qui conclut l'algorithme.

\change
Le déterminant de cette matrice triangulaire étant égal au produit de ces éléments diagonaux, c'est-à-dire $-55$, on trouve donc que le déterminant de $A$ vaut $ (-1)\times 3 \times (-55)$,

\change
ce qui donne $165$.

%%%%%%%%%%%%%%%%%%%%%%%%%%%%%%%%%%%%%%%%%%%%%%%%%%%%%%%%%%%
\diapo

Voici un théorème fondamental.

Théorème :  $\det (AB)=\det A \cdot \det B$.

C'est-à-dire, le déterminant d'un produit est le produit des déterminants.

La démonstration de ce théorème va utiliser les matrices élémentaires et sera vu en fin de séquence.

%%%%%%%%%%%%%%%%%%%%%%%%%%%%%%%%%%%%%%%%%%%%%%%%%%%%%%%%%%%
\diapo
Comment savoir si une matrice est inversible ? Il suffit de calculer son déterminant !

En effet, on a le résultat suivant. Une matrice carrée $A$ est inversible si et seulement si son
déterminant est non nul. 

\change
De plus lorsque $A$ est inversible,  
alors:  $\displaystyle \det \big(A^{-1}\big)=\frac1{\det A}$.\\

«le déterminant de l'inverse est l'inverse du déterminant.»

\change
Comme conséquence de ce résultat, montrons que deux  matrices semblables ont même déterminant. 

\change
En effet: soit $B=P^{-1}AP$ une matrice semblable à $A$, avec $P$ une matrice inversible. 

\change
Par multiplicativité du déterminant, on en déduit que:
$\det B=\det (P^{-1}AP)=\det P^{-1}\det A\det P$

\change
ce qui est égal à $\det A$ puisque $\det P^{-1}=\frac{1}{\det P}$.

%%%%%%%%%%%%%%%%%%%%%%%%%%%%%%%%%%%%%%%%%%%%%%%%%%%%%%%%%%%
\diapo

Dernier théorème : $\det \big(A^T\big)=\det A$

\change
Une conséquence de ce dernier résultat, est que par transposition, tout ce que l'on a 
dit des déterminants à propos des colonnes est 
vrai également pour les lignes. 

% Ainsi, le déterminant est multilinéaire 
% par rapport aux lignes, si une matrice a deux lignes égales, 
% son déterminant est nul, on ne modifie pas un déterminant en 
% ajoutant à une ligne une combinaison linéaire des autres lignes, etc.
% 
% \change

Voici le détail pour les opérations élémentaires sur les lignes :

\change
$L_i \leftarrow \lambda L_i$ : le déterminant est 
  multiplié par $\lambda$.
  
\change
$L_i \leftarrow L_i+\lambda L_j$ : le déterminant ne change pas.
  
\change
$L_i \leftrightarrow L_j$ : le déterminant change de signe.


%%%%%%%%%%%%%%%%%%%%%%%%%%%%%%%%%%%%%%%%%%%%%%%%%%%%%%%%%%%
\diapo

Passons maintenant à la démonstration du théorème : 
"le déterminant d'un produit égal le produit des déterminants"

\change
La démonstration de ce théorème va utiliser les matrices élémentaires.
En effet, on a vu que lorsque $E$ est une matrice correspondant à une opération élémentaire,
on a déjà que $\det (A \times E) = \det A \times \det E.$

\change

Revenons à un cas plus général, mais supposons pour l'instant que la matrice $B$ soit inversible. 


\change
On a vu dans le chapitre «Matrices» 
qu'une matrice $B$ est inversible si et seulement si sa forme échelonnée réduite 
par le pivot de Gauss est égale à $I_n$, c-à-d s'il existe des matrices élémentaires 
$E_i$ telles que $B E_1\cdots E_r = I_{n}$.

\change
En prenant le déterminant, et d'après le rappel précédent appliqué $r$ fois,

\change
(...)
\newpage
on a $\det (B  \cdot E_1 E_2 \cdots E_r)
=\det B  \cdot \det E_1  \cdot \det E_2 \cdots \det E_r $

\change
qui vaut donc $= \det I_n$ c-à-d $1$.

\change
On en déduit $\det B=1 / (\det E_1 \times \det E_2 \cdots \times \det E_r)$.

\change
Pour la matrice $AB$, il vient $(AB) \cdot (E_1\cdots E_r) = A \cdot I_n =A$.

\change
Ainsi $\det(A B E_1 \cdots E_r) = \det(A B) \cdot \det E_1  \cdots \det E_r$

\change
qui est aussi égal à $ \det A $.

\change
Donc :
$\det(AB) = \det A \times \frac1{\det E_1 \cdots \det E_r} $

\change
c-à-d $= \det A \times \det B.$
D'où le résultat dans ce cas.

\change
(...)
\newpage
Considérons à présent le cas où la matrice $B$ n'est pas inver\-sible, c-à-d lorsque son rang est strictement inférieur à $n$

\change
Alors il existe une relation de dépendance linéaire entre les colonnes de $B$, 
ce qui revient à dire qu'il existe un vecteur colonne $X$ tel que $BX=0$.

\change
 Donc  $\det B=0$.
 
\change
Or $BX=0$ implique $(AB)X=0$. 

\change
Ainsi $AB$ n'est pas inversible non plus

\change
d'où 
$\det (AB)=0$

\change
qui est alors égal à $\det A\det B$ dans ce cas également.


%%%%%%%%%%%%%%%%%%%%%%%%%%%%%%%%%%%%%%%%%%%%%%%%%%%%%%%%%%%
\diapo
Et bien sûr, voici quelques exercices d'application qui vous permettrons de mieux vous approprier le cours.

\end{document}

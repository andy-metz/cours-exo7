
%%%%%%%%%%%%%%%%%% PREAMBULE %%%%%%%%%%%%%%%%%%


\documentclass[12pt]{article}

\usepackage{amsfonts,amsmath,amssymb,amsthm}
\usepackage[utf8]{inputenc}
\usepackage[T1]{fontenc}
\usepackage[francais]{babel}


% packages
\usepackage{amsfonts,amsmath,amssymb,amsthm}
\usepackage[utf8]{inputenc}
\usepackage[T1]{fontenc}
%\usepackage{lmodern}

\usepackage[francais]{babel}
\usepackage{fancybox}
\usepackage{graphicx}

\usepackage{float}

%\usepackage[usenames, x11names]{xcolor}
\usepackage{tikz}
\usepackage{datetime}

\usepackage{mathptmx}
%\usepackage{fouriernc}
%\usepackage{newcent}
\usepackage[mathcal,mathbf]{euler}

%\usepackage{palatino}
%\usepackage{newcent}


% Commande spéciale prompteur

%\usepackage{mathptmx}
%\usepackage[mathcal,mathbf]{euler}
%\usepackage{mathpple,multido}

\usepackage[a4paper]{geometry}
\geometry{top=2cm, bottom=2cm, left=1cm, right=1cm, marginparsep=1cm}

\newcommand{\change}{{\color{red}\rule{\textwidth}{1mm}\\}}

\newcounter{mydiapo}

\newcommand{\diapo}{\newpage
\hfill {\normalsize  Diapo \themydiapo \quad \texttt{[\jobname]}} \\
\stepcounter{mydiapo}}


%%%%%%% COULEURS %%%%%%%%%%

% Pour blanc sur noir :
%\pagecolor[rgb]{0.5,0.5,0.5}
% \pagecolor[rgb]{0,0,0}
% \color[rgb]{1,1,1}



%\DeclareFixedFont{\myfont}{U}{cmss}{bx}{n}{18pt}
\newcommand{\debuttexte}{
%%%%%%%%%%%%% FONTES %%%%%%%%%%%%%
\renewcommand{\baselinestretch}{1.5}
\usefont{U}{cmss}{bx}{n}
\bfseries

% Taille normale : commenter le reste !
%Taille Arnaud
%\fontsize{19}{19}\selectfont

% Taille Barbara
%\fontsize{21}{22}\selectfont

%Taille François
\fontsize{25}{30}\selectfont

%Taille Pascal
%\fontsize{25}{30}\selectfont

%Taille Laura
%\fontsize{30}{35}\selectfont


%\myfont
%\usefont{U}{cmss}{bx}{n}

%\Huge
%\addtolength{\parskip}{\baselineskip}
}


% \usepackage{hyperref}
% \hypersetup{colorlinks=true, linkcolor=blue, urlcolor=blue,
% pdftitle={Exo7 - Exercices de mathématiques}, pdfauthor={Exo7}}


%section
% \usepackage{sectsty}
% \allsectionsfont{\bf}
%\sectionfont{\color{Tomato3}\upshape\selectfont}
%\subsectionfont{\color{Tomato4}\upshape\selectfont}

%----- Ensembles : entiers, reels, complexes -----
\newcommand{\Nn}{\mathbb{N}} \newcommand{\N}{\mathbb{N}}
\newcommand{\Zz}{\mathbb{Z}} \newcommand{\Z}{\mathbb{Z}}
\newcommand{\Qq}{\mathbb{Q}} \newcommand{\Q}{\mathbb{Q}}
\newcommand{\Rr}{\mathbb{R}} \newcommand{\R}{\mathbb{R}}
\newcommand{\Cc}{\mathbb{C}} 
\newcommand{\Kk}{\mathbb{K}} \newcommand{\K}{\mathbb{K}}

%----- Modifications de symboles -----
\renewcommand{\epsilon}{\varepsilon}
\renewcommand{\Re}{\mathop{\text{Re}}\nolimits}
\renewcommand{\Im}{\mathop{\text{Im}}\nolimits}
%\newcommand{\llbracket}{\left[\kern-0.15em\left[}
%\newcommand{\rrbracket}{\right]\kern-0.15em\right]}

\renewcommand{\ge}{\geqslant}
\renewcommand{\geq}{\geqslant}
\renewcommand{\le}{\leqslant}
\renewcommand{\leq}{\leqslant}

%----- Fonctions usuelles -----
\newcommand{\ch}{\mathop{\mathrm{ch}}\nolimits}
\newcommand{\sh}{\mathop{\mathrm{sh}}\nolimits}
\renewcommand{\tanh}{\mathop{\mathrm{th}}\nolimits}
\newcommand{\cotan}{\mathop{\mathrm{cotan}}\nolimits}
\newcommand{\Arcsin}{\mathop{\mathrm{Arcsin}}\nolimits}
\newcommand{\Arccos}{\mathop{\mathrm{Arccos}}\nolimits}
\newcommand{\Arctan}{\mathop{\mathrm{Arctan}}\nolimits}
\newcommand{\Argsh}{\mathop{\mathrm{Argsh}}\nolimits}
\newcommand{\Argch}{\mathop{\mathrm{Argch}}\nolimits}
\newcommand{\Argth}{\mathop{\mathrm{Argth}}\nolimits}
\newcommand{\pgcd}{\mathop{\mathrm{pgcd}}\nolimits} 

\newcommand{\Card}{\mathop{\text{Card}}\nolimits}
\newcommand{\Ker}{\mathop{\text{Ker}}\nolimits}
\newcommand{\id}{\mathop{\text{id}}\nolimits}
\newcommand{\ii}{\mathrm{i}}
\newcommand{\dd}{\mathrm{d}}
\newcommand{\Vect}{\mathop{\text{Vect}}\nolimits}
\newcommand{\Mat}{\mathop{\mathrm{Mat}}\nolimits}
\newcommand{\rg}{\mathop{\text{rg}}\nolimits}
\newcommand{\tr}{\mathop{\text{tr}}\nolimits}
\newcommand{\ppcm}{\mathop{\text{ppcm}}\nolimits}

%----- Structure des exercices ------

\newtheoremstyle{styleexo}% name
{2ex}% Space above
{3ex}% Space below
{}% Body font
{}% Indent amount 1
{\bfseries} % Theorem head font
{}% Punctuation after theorem head
{\newline}% Space after theorem head 2
{}% Theorem head spec (can be left empty, meaning ‘normal’)

%\theoremstyle{styleexo}
\newtheorem{exo}{Exercice}
\newtheorem{ind}{Indications}
\newtheorem{cor}{Correction}


\newcommand{\exercice}[1]{} \newcommand{\finexercice}{}
%\newcommand{\exercice}[1]{{\tiny\texttt{#1}}\vspace{-2ex}} % pour afficher le numero absolu, l'auteur...
\newcommand{\enonce}{\begin{exo}} \newcommand{\finenonce}{\end{exo}}
\newcommand{\indication}{\begin{ind}} \newcommand{\finindication}{\end{ind}}
\newcommand{\correction}{\begin{cor}} \newcommand{\fincorrection}{\end{cor}}

\newcommand{\noindication}{\stepcounter{ind}}
\newcommand{\nocorrection}{\stepcounter{cor}}

\newcommand{\fiche}[1]{} \newcommand{\finfiche}{}
\newcommand{\titre}[1]{\centerline{\large \bf #1}}
\newcommand{\addcommand}[1]{}
\newcommand{\video}[1]{}

% Marge
\newcommand{\mymargin}[1]{\marginpar{{\small #1}}}



%----- Presentation ------
\setlength{\parindent}{0cm}

%\newcommand{\ExoSept}{\href{http://exo7.emath.fr}{\textbf{\textsf{Exo7}}}}

\definecolor{myred}{rgb}{0.93,0.26,0}
\definecolor{myorange}{rgb}{0.97,0.58,0}
\definecolor{myyellow}{rgb}{1,0.86,0}

\newcommand{\LogoExoSept}[1]{  % input : echelle
{\usefont{U}{cmss}{bx}{n}
\begin{tikzpicture}[scale=0.1*#1,transform shape]
  \fill[color=myorange] (0,0)--(4,0)--(4,-4)--(0,-4)--cycle;
  \fill[color=myred] (0,0)--(0,3)--(-3,3)--(-3,0)--cycle;
  \fill[color=myyellow] (4,0)--(7,4)--(3,7)--(0,3)--cycle;
  \node[scale=5] at (3.5,3.5) {Exo7};
\end{tikzpicture}}
}



\theoremstyle{definition}
%\newtheorem{proposition}{Proposition}
%\newtheorem{exemple}{Exemple}
%\newtheorem{theoreme}{Théorème}
\newtheorem{lemme}{Lemme}
\newtheorem{corollaire}{Corollaire}
%\newtheorem*{remarque*}{Remarque}
%\newtheorem*{miniexercice}{Mini-exercices}
%\newtheorem{definition}{Définition}




%definition d'un terme
\newcommand{\defi}[1]{{\color{myorange}\textbf{\emph{#1}}}}
\newcommand{\evidence}[1]{{\color{blue}\textbf{\emph{#1}}}}



 %----- Commandes divers ------

\newcommand{\codeinline}[1]{\texttt{#1}}

%%%%%%%%%%%%%%%%%%%%%%%%%%%%%%%%%%%%%%%%%%%%%%%%%%%%%%%%%%%%%
%%%%%%%%%%%%%%%%%%%%%%%%%%%%%%%%%%%%%%%%%%%%%%%%%%%%%%%%%%%%%



\begin{document}

\debuttexte

%%%%%%%%%%%%%%%%%%%%%%%%%%%%%%%%%%%%%%%%%%%%%%%%%%%%%%%%%%%
\diapo

\change

Pour trouver une primitive d'une fonction on peut avoir la chance de reconnaître 
que c'est la dérivée d'une fonction bien connue. C'est malheureusement très rarement le cas : 
on ne connaît pas les primitives de la plupart des fonctions. 

\change

Nous allons 
voir deux techniques qui permettent des calculer des intégrales et des primitives : 
l'intégration par parties 

\change

et le changement de variable.



%%%%%%%%%%%%%%%%%%%%%%%%%%%%%%%%%%%%%%%%%%%%%%%%%%%%%%%%%%%
\diapo

Voici la formule d'intégration par parties :

pour $u$ et $v$ deux fonctions de classe $\mathcal{C}^1$ 
c'est-à-dire dérivable et de dérivée continue alors on a l'égalité :

$\int_a^b u(x) \, v'(x)\;dx= \big[uv\big]_a^b - \int_a^b u'(x) \, v(x)\;dx$

\change
 Le crochet $\big[F\big]_a^b$ est la notation pour $F(b)-F(a)$.

En particulier $\big[uv\big]_a^b = u(b)v(b)-u(a)v(a)$.

\change

La formule d'intégration par parties pour les primitives est la même mais sans les bornes :
$$\int u(x)v'(x)\;dx= \big[uv\big] - \int u'(x)v(x)\;dx.$$

\change

Ici le crochet sans les bornes $\big[F\big]$ désigne la fonction $F+c$ où $c$ est une constante quelconque.


\change


La preuve est très simple, elle est basée sur la formule de la dérivée du produit :

Partons de $(uv)'=u'v+uv'$.

\change

Lorsque l'on intègre cette égalité de part et d'autre alors on obtient l'égalité 
$\int_a^b (u'v+uv')=\int_a^b (uv)'$

\change

Mais intégrer une dérivée c'est évaluer la fonction entre $a$ et $b$ :

c-à-d $\big[uv\big]_a^b$

\change

D'où la formule $\int_a^b uv'= \big[uv\big]_a^b - \int_a^b u'v$.



%%%%%%%%%%%%%%%%%%%%%%%%%%%%%%%%%%%%%%%%%%%%%%%%%%%%%%%%%%%
\diapo

% Lorsque l'on ne sait pas calculer directement l'intégrale d'une fonction $f$ s'écrivant 
% 
% on peut essayer d'écrire la fonction sous la forme d'un produit $f(x)=u(x)v'(x)$ 
% 
% et utiliser la formule d'intégration par parties pour calculer l'intégrale de $f$
% 
% il faut bien sûr pour cela calculer cette fois l'intégrale $u'(x)v(x)$ que l'on espère plus simple !

Nous allons voir plusieurs exemples d'intégration par parties.

\change

Commençons par le calcul de $\int_0^1 x e^x \; dx$.

\change

On doit l'écrire sous la forme $uv'$

\change

On va donc poser ici $u=x$ et $v'=e^x$.

\change


Nous aurons besoin de savoir que $u'(x)=1$ 

\change

Nous aurons besoin d'une primitive de $v'$ 
et on prend simplement $v(x)=e^x$.

\change


La formule d'intégration par parties donne :
$\int_0^1 u(x)v'(x)\;dx  =  \big[u(x)v(x)\big]_0^1 - \int_0^1 u'(x)v(x)\;dx$

\change

Donc en remplaçant par nos fonctions on obtient :

$\big[x e^x\big]_0^1 \ \  - \  \int_0^1 1\cdot e^x \;dx$

\change

Le crochet c'est  $x e^x$ évaluer en $1$ moins $xe^x$ évaluer en $0$ c'est donc 
$1\cdot e^1-0\cdot e^0$

\change

Maintenant le point crucial c'est que la nouvelle intégrale que l'on doit calculer est beaucoup plus simple.
Il s'agit d'intégrer la fonction $e^x$ dont une primitive est $e^x$.

Donc l'intégrale vaut $\big[e^x\big]_0^1$

\change

Ce second crochet vaut $e^1-e^0$

\change

Donc l'intégrale vaut 
$e-(e-1)$ donc $1$.


%%%%%%%%%%%%%%%%%%%%%%%%%%%%%%%%%%%%%%%%%%%%%%%%%%%%%%%%%%%
\diapo

On prend un nouvel exemple : il s'agit de calculer  $\int_1^e x\ln x \; dx$.


Le problème de l'intégration par partie est de choisir quelle fonction va être $u$ et quelle fonction
va être $v'$. 

\change

\change

Il n'y a pas de méthode miracle : c'est la pratique et les différents essais
qui conduisent à la solutions 

\change

Cependant on doit se ramener à une intégrale plus simple 


Ici on va poser $u=\ln x$ et $v'=x$.
C'est-à-dire on dérive $\ln x$ et on intègre $x$.


\change

Ainsi 
$u'=\frac1x$ 

\change

et $v=\frac{x^2}{2}$.

\change

Par la formule d'IPP 

\change

l'intégrale vaut 
$\big[\ln x \cdot\tfrac{x^2}{2}\big]_1^e - \int_1^e \tfrac 1x \tfrac{x^2}{2} \; dx$

\change

Le crochet vaut $\ln e \tfrac{e^2}{2} - \ln 1 \tfrac{1^2}{2}$ 

\change

Et la deuxième intégrale est $\tfrac12 \int_1^e x  \; dx $

\change

Comme $\ln 1=0$ alors le crochet vaut $\tfrac{e^2}{2}$


et pour l'intégrale une primitive de $x$ est $\frac{x^2}{2}$ qu'il faut évaluer entre $1$ et $e$.

\change

Notre intégrale vaut donc $ \tfrac{e^2}{2} - \tfrac{e^2}{4} + \tfrac{1}{4}$

ou encore $\tfrac{e^2+1}{4}$.


%%%%%%%%%%%%%%%%%%%%%%%%%%%%%%%%%%%%%%%%%%%%%%%%%%%%%%%%%%%
\diapo

On continue avec la primitive de  $\arcsin x$. 

Problème : il n'y a qu'une fonction, comment peut-on trouver un produit $u \times v'$.

\change

L'astuce est toute simple on considère que la fonction
c'est $1 \cdot \arcsin x$

\change


On prend donc  $u=\arcsin x$, $v'=1$ 

\change 

la dérivée de $\arcsin$ est $u'=\frac{1}{\sqrt{1-x^2}}$ et une primitive de $1$ est $v=x$.

\change 

la formule d'intégration par parties donne donc 

$\int 1\cdot \arcsin x \; dx = \big[x\arcsin x\big] - \int \frac{x}{\sqrt{1-x^2}} \; dx$

cette fonction est presque de la forme $u'/\sqrt{u}$ dont une primitive est $2\sqrt{u}$.

\change

Donc ici une primitive de $\frac{x}{\sqrt{1-x^2}}$ est $-\sqrt {1-x^2}$

\change

Le crochet c'est la fonction $x\arcsin x$ plus une constante,
le second crochet c'est $-\sqrt {1-x^2}$ plus une constante 

ainsi les primitives de $\arcsin$ sont les 
 $x\arcsin x+ \sqrt {1-x^2}+c$


\change

Autre exemple à calculer : 

\change

les primitives de $\int x^2e^x \; dx$.

\change

On va dériver $x^2$ et intégrer $e^x$ c'est à dire on pose 
On pose $u=x^2$ et $v'=e^x$ 

\change

$\int x^2e^x \; dx = \big[ x^2e^x \big] - \int 2x e^x \; dx$
[lire $-\int$ dérivée de $x^2$ fois primitive de $e^x$.]

\change

A-t-on vraiment progresser ? car on a encore une intégrale qui ne s'intègre pas
au premier coup d’oeil

\change

On ne se décourage pas et on fait une deuxième intégration par partie

On dérive $x$ et on intègre $e^x$.
On obtient  
$\int x e^x \; dx = \big[x e^x\big] - \int e^x \; dx$ 

qui cette fois se calcule immédiatement et vaut $xe^x - e^x$ autrement dit $(x-1)e^x+c$.

\change

On injecte cette formule dans notre première IPP et on 
conclue :
$\int x^2e^x \; dx = (x^2-2x+2) e^x + c.$


%%%%%%%%%%%%%%%%%%%%%%%%%%%%%%%%%%%%%%%%%%%%%%%%%%%%%%%%%%%
\diapo

Voici le théorème de changement de variable :


Soit $f$ une fonction définie sur un intervalle $I$ et $\varphi : J \to I$ une bijection de classe $\mathcal{C}^1$. 

Alors pour tout $a,b\in J$ 

$\int_{\varphi(a)}^{\varphi(b)} f(x) \; dx = \int_a^b f\big(\varphi(t)\big)\cdot\varphi'(t) \; dt$

\change


Voici un moyen simple de s'en souvenir. 
Si l'on note $x=\varphi(t)$ 

\change

alors par dérivation on obtient $\frac{dx}{dt} = \varphi'(t)$

\change

ce qui conduit à l'écriture $dx = \varphi'(t) \; dt$. 

\change 

si l'on remplace $x$ par $\varphi(t)$ alors 

$f(x) \; dx$ devient $f(\varphi(t)) \; \varphi'(t) \; dt$

\change

Que l'on intègre en faisant attention aux bornes

$\int_{\varphi(a)}^{\varphi(b)} f(x) \; dx = \int_a^b f(\varphi(t)) \; \varphi'(t) \; dt$.

Une façon plus rigoureuse de faire cette preuve repose
sur la formule de la dérivée d'une composition.


%%%%%%%%%%%%%%%%%%%%%%%%%%%%%%%%%%%%%%%%%%%%%%%%%%%%%%%%%%%
\diapo

Commençons par un exemple : on souhaite calculer la primitive $\tan t$.

\change

Lorsque l'on écrit $\tan t$ sous la forme  $\frac{\sin t}{\cos t}$ 


\change

on reconnaît une forme du type $\frac{u'}{u}$ (au signe près)

avec ici $u=\cos t$ et $u'=-\sin t$.

\change

Une primitive de $\frac{u'}{u}$est $\ln|u|$.

\change

Donc notre primitive $F$ vaut ici $-\ln|u|$ plus une une constante.

Mais comme $u=\cos t$ alors $F = -\ln|\cos t|+c$.

\change

Ce que nous avons fait, sans le dire c'est un changement de variable.

Si on note $\varphi$ la fonction $\cos t$

\change 

alors $\varphi'(t) = -\sin t$

Donc $dx = -\sin t \; dt$.

Notre primitive s'écrit donc 

\change

$F = \int \tan t \; dt$

\change

$= \int -\frac{\varphi'(t)}{\varphi(t)} \; dt$

\change

Si on appelle petit $f$ la fonction $f(x)=\frac1x$

\change

alors $F = - \int \varphi'(t) f(\varphi(t))\; dt$

\change

 on reconnaît exactement la formule du changement de variable,

par conséquent 
$F = -\int f(x) \; dx$


\change

On s'est ramené à une fonction beaucoup plus simple :
il s'agit d'intégrer $-1/x$

\change

dont une primitive est bien sûr $- \ln|x|+c $

\change

et on remplace $x$ par $\cos t$.

Pour retrouver que $F(t) = \ln|\cos t| + c$

C'est primitive sont définies sur des intervalles du type 
$]-\frac{\pi}{2}, \frac{\pi}{2}[$ translaté de $k\pi$.
Mais attention la constante $c$ peut être différente sur des intervalles différents.


%%%%%%%%%%%%%%%%%%%%%%%%%%%%%%%%%%%%%%%%%%%%%%%%%%%%%%%%%%%
\diapo


Calculons $\int_0^{1/2}\frac{x}{(1-x^2)^{3/2}} \;dx$ par changement de variable.


\change


Nous allons poser $u=\varphi(x) = 1-x^2$

\change

Alors $du = -2x \; dx$. 

\change

Que devient notre intégrale ?

\change

Le dénominateur $(1-x^2)^{3/2}$ devient $u^{3/2}$

et le numérateur $x dx$ devient simplement $-\frac12 du$.

On a bien une intégrale beaucoup plus facile à calculer.


Occupons nous-maintenant des bornes !

\change

L'intégrale de départ varie entre $x=0$ et $x=1/2$.


Pour $x=0$, alors que vaut $u$ ? Pour $x=0$ on a $u=\varphi(0)=1-0^2=1$.

\change

Notre intégrale en $x$ démarre en $0$ 

\change

donc notre intégrale en $u$ démarre en $1$.

\change

Pour $x=1/2$ alors $u=\varphi(\frac{1}{2})=1-(\frac12)^2=\frac34$.

\change

Notre intégrale en $x$ finit en $1/2$

\change

donc notre intégrale en $u$ finit en $\frac34$.

Donc l'intégrale de départ vaut  maintenant $\int_1^{3/4}\frac{-\tfrac12 \; du}{u^{3/2}} $

L'ordre des bornes est important.

\change

On peut sortir le $-1/2$ pour obtenir l'intégrale de $u^{-3/2}$.

\change

Une primitive de $u^{-3/2}$ est $-2u^{-1/2}$

\change

qu'il faut évaluer entre $1$ et $3/4$.

\change

Donc notre intégrale vaut$\frac1{\sqrt{\frac34}}-1$

\change

ou encore $\frac{2}{\sqrt3}-1$



%%%%%%%%%%%%%%%%%%%%%%%%%%%%%%%%%%%%%%%%%%%%%%%%%%%%%%%%%%%
\diapo


On continue avec un dernier exemple.
Il s'agit de calculer $\int_0^{1/2}\frac{1}{(1-x^2)^{3/2}} \;dx$.


Attention par rapport à l'exemple précédent le dénominateur est le même mais ici le numérateur vaut $1$.

Vous allez voir que le changement de variable est complètement différent !

Il n'y a pas de méthode générale pour savoir quel changement 
de variable effectuer : il faut tester plusieurs possibilités et avec pas mal de pratique
on arrive à trouver celui qui convient.


Ici il faut essayer de simplifier le dénominateur : est-ce que l'on ne connaît pas une fonction
qui simplifierait l'équation $1-x^2$ ?

\change

Essayons le changement de variable $x=\varphi(t) = \sin t$.
L'intérêt est que alors $1-x^2$ devient $1-\sin^2 t$ c'est-à-dire simplement $\cos^2 t$.

\change

On a alors $dx = \cos t \; dt$.

\change

Sur l'intervalle $[0,1/2]$ sinus est un bijection, sa réciproque est arcsin donc
on a $t=\arcsin x$.  

Pour $x=0$ on a $t=\arcsin(0)=0$


\change

et pour $x=\frac12$ on a $t=\arcsin(\frac12)=\frac\pi6$



\change


Allons-y pour le calcul de l'intégrale :

\change

on remplace $x$ par $sin t$,

$dx$ par $\cos t dt$

et on fait attention de changer aussi le borne 

$t$ varie de $0$ à $\pi/6$.

\change

On a déjà dit que le point clé c'est que le dénominateur se simplifie 
en $(\cos^2 t)^{3/2}$.

\change

ce qui donne $\cos^3 t$ en bas

\change

qui va simplifie avec le $\cos t$ du haut.

On trouve donc l'intégrale de $1/\cos^2 t$.

\change

Mais $1/\cos^2 t$ est la dérivée de $\tan t$.

Donc nous évaluons la primitive $\tan t$ entre $0$ et $\pi/6$.


\change

Donc notre intégrale vaut $1/\sqrt3$.


%%%%%%%%%%%%%%%%%%%%%%%%%%%%%%%%%%%%%%%%%%%%%%%%%%%%%%%%%%%
\diapo

L'intégration par parties et le changement de variable sont deux outils incontournables
pour le calcul d'intégrales. Nous avons vu quelques exemples
C'est maintenant à vous de vous entraîner d'abord sur ces exemples simples
avant de passer à des exercices plus compliqués.


\end{document}
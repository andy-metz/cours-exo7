
%%%%%%%%%%%%%%%%%% PREAMBULE %%%%%%%%%%%%%%%%%%

\documentclass[aspectratio=169,utf8]{beamer}
%\documentclass[aspectratio=169,handout]{beamer}

\usetheme{Boadilla}
%\usecolortheme{seahorse}
%\usecolortheme[RGB={245,66,24}]{structure}
\useoutertheme{infolines}

% packages
\usepackage{amsfonts,amsmath,amssymb,amsthm}
\usepackage[utf8]{inputenc}
\usepackage[T1]{fontenc}
\usepackage{lmodern}

\usepackage[francais]{babel}
\usepackage{fancybox}
\usepackage{graphicx}

\usepackage{float}
\usepackage{xfrac}

%\usepackage[usenames, x11names]{xcolor}
\usepackage{pgfplots}
\usepackage{datetime}


% ----------------------------------------------------------------------
% Pour les images
\usepackage{tikz}
\usetikzlibrary{calc,shadows,arrows.meta,patterns,matrix}

\newcommand{\tikzinput}[1]{\input{figures/#1.tikz}}
% --- les figures avec échelle éventuel
\newcommand{\myfigure}[2]{% entrée : échelle, fichier(s) figure à inclure
\begin{center}\small%
\tikzstyle{every picture}=[scale=1.0*#1]% mise en échelle + 0% (automatiquement annulé à la fin du groupe)
#2%
\end{center}}



%-----  Package unités -----
\usepackage{siunitx}
\sisetup{locale = FR,detect-all,per-mode = symbol}

%\usepackage{mathptmx}
%\usepackage{fouriernc}
%\usepackage{newcent}
%\usepackage[mathcal,mathbf]{euler}

%\usepackage{palatino}
%\usepackage{newcent}
% \usepackage[mathcal,mathbf]{euler}



% \usepackage{hyperref}
% \hypersetup{colorlinks=true, linkcolor=blue, urlcolor=blue,
% pdftitle={Exo7 - Exercices de mathématiques}, pdfauthor={Exo7}}


%section
% \usepackage{sectsty}
% \allsectionsfont{\bf}
%\sectionfont{\color{Tomato3}\upshape\selectfont}
%\subsectionfont{\color{Tomato4}\upshape\selectfont}

%----- Ensembles : entiers, reels, complexes -----
\newcommand{\Nn}{\mathbb{N}} \newcommand{\N}{\mathbb{N}}
\newcommand{\Zz}{\mathbb{Z}} \newcommand{\Z}{\mathbb{Z}}
\newcommand{\Qq}{\mathbb{Q}} \newcommand{\Q}{\mathbb{Q}}
\newcommand{\Rr}{\mathbb{R}} \newcommand{\R}{\mathbb{R}}
\newcommand{\Cc}{\mathbb{C}} 
\newcommand{\Kk}{\mathbb{K}} \newcommand{\K}{\mathbb{K}}

%----- Modifications de symboles -----
\renewcommand{\epsilon}{\varepsilon}
\renewcommand{\Re}{\mathop{\text{Re}}\nolimits}
\renewcommand{\Im}{\mathop{\text{Im}}\nolimits}
%\newcommand{\llbracket}{\left[\kern-0.15em\left[}
%\newcommand{\rrbracket}{\right]\kern-0.15em\right]}

\renewcommand{\ge}{\geqslant}
\renewcommand{\geq}{\geqslant}
\renewcommand{\le}{\leqslant}
\renewcommand{\leq}{\leqslant}
\renewcommand{\epsilon}{\varepsilon}

%----- Fonctions usuelles -----
\newcommand{\ch}{\mathop{\text{ch}}\nolimits}
\newcommand{\sh}{\mathop{\text{sh}}\nolimits}
\renewcommand{\tanh}{\mathop{\text{th}}\nolimits}
\newcommand{\cotan}{\mathop{\text{cotan}}\nolimits}
\newcommand{\Arcsin}{\mathop{\text{arcsin}}\nolimits}
\newcommand{\Arccos}{\mathop{\text{arccos}}\nolimits}
\newcommand{\Arctan}{\mathop{\text{arctan}}\nolimits}
\newcommand{\Argsh}{\mathop{\text{argsh}}\nolimits}
\newcommand{\Argch}{\mathop{\text{argch}}\nolimits}
\newcommand{\Argth}{\mathop{\text{argth}}\nolimits}
\newcommand{\pgcd}{\mathop{\text{pgcd}}\nolimits} 


%----- Commandes divers ------
\newcommand{\ii}{\mathrm{i}}
\newcommand{\dd}{\text{d}}
\newcommand{\id}{\mathop{\text{id}}\nolimits}
\newcommand{\Ker}{\mathop{\text{Ker}}\nolimits}
\newcommand{\Card}{\mathop{\text{Card}}\nolimits}
\newcommand{\Vect}{\mathop{\text{Vect}}\nolimits}
\newcommand{\Mat}{\mathop{\text{Mat}}\nolimits}
\newcommand{\rg}{\mathop{\text{rg}}\nolimits}
\newcommand{\tr}{\mathop{\text{tr}}\nolimits}


%----- Structure des exercices ------

\newtheoremstyle{styleexo}% name
{2ex}% Space above
{3ex}% Space below
{}% Body font
{}% Indent amount 1
{\bfseries} % Theorem head font
{}% Punctuation after theorem head
{\newline}% Space after theorem head 2
{}% Theorem head spec (can be left empty, meaning ‘normal’)

%\theoremstyle{styleexo}
\newtheorem{exo}{Exercice}
\newtheorem{ind}{Indications}
\newtheorem{cor}{Correction}


\newcommand{\exercice}[1]{} \newcommand{\finexercice}{}
%\newcommand{\exercice}[1]{{\tiny\texttt{#1}}\vspace{-2ex}} % pour afficher le numero absolu, l'auteur...
\newcommand{\enonce}{\begin{exo}} \newcommand{\finenonce}{\end{exo}}
\newcommand{\indication}{\begin{ind}} \newcommand{\finindication}{\end{ind}}
\newcommand{\correction}{\begin{cor}} \newcommand{\fincorrection}{\end{cor}}

\newcommand{\noindication}{\stepcounter{ind}}
\newcommand{\nocorrection}{\stepcounter{cor}}

\newcommand{\fiche}[1]{} \newcommand{\finfiche}{}
\newcommand{\titre}[1]{\centerline{\large \bf #1}}
\newcommand{\addcommand}[1]{}
\newcommand{\video}[1]{}

% Marge
\newcommand{\mymargin}[1]{\marginpar{{\small #1}}}

\def\noqed{\renewcommand{\qedsymbol}{}}


%----- Presentation ------
\setlength{\parindent}{0cm}

%\newcommand{\ExoSept}{\href{http://exo7.emath.fr}{\textbf{\textsf{Exo7}}}}

\definecolor{myred}{rgb}{0.93,0.26,0}
\definecolor{myorange}{rgb}{0.97,0.58,0}
\definecolor{myyellow}{rgb}{1,0.86,0}

\newcommand{\LogoExoSept}[1]{  % input : echelle
{\usefont{U}{cmss}{bx}{n}
\begin{tikzpicture}[scale=0.1*#1,transform shape]
  \fill[color=myorange] (0,0)--(4,0)--(4,-4)--(0,-4)--cycle;
  \fill[color=myred] (0,0)--(0,3)--(-3,3)--(-3,0)--cycle;
  \fill[color=myyellow] (4,0)--(7,4)--(3,7)--(0,3)--cycle;
  \node[scale=5] at (3.5,3.5) {Exo7};
\end{tikzpicture}}
}


\newcommand{\debutmontitre}{
  \author{} \date{} 
  \thispagestyle{empty}
  \hspace*{-10ex}
  \begin{minipage}{\textwidth}
    \titlepage  
  \vspace*{-2.5cm}
  \begin{center}
    \LogoExoSept{2.5}
  \end{center}
  \end{minipage}

  \vspace*{-0cm}
  
  % Astuce pour que le background ne soit pas discrétisé lors de la conversion pdf -> png
\begin{tikzpicture}
        \fill[opacity=0,green!60!black] (0,0)--++(0,0)--++(0,0)--++(0,0)--cycle; 
\end{tikzpicture}

% toc S'affiche trop tot :
% \tableofcontents[hideallsubsections, pausesections]
}

\newcommand{\finmontitre}{
  \end{frame}
  \setcounter{framenumber}{0}
} % ne marche pas pour une raison obscure

%----- Commandes supplementaires ------

% \usepackage[landscape]{geometry}
% \geometry{top=1cm, bottom=3cm, left=2cm, right=10cm, marginparsep=1cm
% }
% \usepackage[a4paper]{geometry}
% \geometry{top=2cm, bottom=2cm, left=2cm, right=2cm, marginparsep=1cm
% }

%\usepackage{standalone}


% New command Arnaud -- november 2011
\setbeamersize{text margin left=24ex}
% si vous modifier cette valeur il faut aussi
% modifier le decalage du titre pour compenser
% (ex : ici =+10ex, titre =-5ex

\theoremstyle{definition}
%\newtheorem{proposition}{Proposition}
%\newtheorem{exemple}{Exemple}
%\newtheorem{theoreme}{Théorème}
%\newtheorem{lemme}{Lemme}
%\newtheorem{corollaire}{Corollaire}
%\newtheorem*{remarque*}{Remarque}
%\newtheorem*{miniexercice}{Mini-exercices}
%\newtheorem{definition}{Définition}

% Commande tikz
\usetikzlibrary{calc}
\usetikzlibrary{patterns,arrows}
\usetikzlibrary{matrix}
\usetikzlibrary{fadings} 

%definition d'un terme
\newcommand{\defi}[1]{{\color{myorange}\textbf{\emph{#1}}}}
\newcommand{\evidence}[1]{{\color{blue}\textbf{\emph{#1}}}}
\newcommand{\assertion}[1]{\emph{\og#1\fg}}  % pour chapitre logique
%\renewcommand{\contentsname}{Sommaire}
\renewcommand{\contentsname}{}
\setcounter{tocdepth}{2}



%------ Encadrement ------

\usepackage{fancybox}


\newcommand{\mybox}[1]{
\setlength{\fboxsep}{7pt}
\begin{center}
\shadowbox{#1}
\end{center}}

\newcommand{\myboxinline}[1]{
\setlength{\fboxsep}{5pt}
\raisebox{-10pt}{
\shadowbox{#1}
}
}

%--------------- Commande beamer---------------
\newcommand{\beameronly}[1]{#1} % permet de mettre des pause dans beamer pas dans poly


\setbeamertemplate{navigation symbols}{}
\setbeamertemplate{footline}  % tiré du fichier beamerouterinfolines.sty
{
  \leavevmode%
  \hbox{%
  \begin{beamercolorbox}[wd=.333333\paperwidth,ht=2.25ex,dp=1ex,center]{author in head/foot}%
    % \usebeamerfont{author in head/foot}\insertshortauthor%~~(\insertshortinstitute)
    \usebeamerfont{section in head/foot}{\bf\insertshorttitle}
  \end{beamercolorbox}%
  \begin{beamercolorbox}[wd=.333333\paperwidth,ht=2.25ex,dp=1ex,center]{title in head/foot}%
    \usebeamerfont{section in head/foot}{\bf\insertsectionhead}
  \end{beamercolorbox}%
  \begin{beamercolorbox}[wd=.333333\paperwidth,ht=2.25ex,dp=1ex,right]{date in head/foot}%
    % \usebeamerfont{date in head/foot}\insertshortdate{}\hspace*{2em}
    \insertframenumber{} / \inserttotalframenumber\hspace*{2ex} 
  \end{beamercolorbox}}%
  \vskip0pt%
}


\definecolor{mygrey}{rgb}{0.5,0.5,0.5}
\setlength{\parindent}{0cm}
%\DeclareTextFontCommand{\helvetica}{\fontfamily{phv}\selectfont}

% background beamer
\definecolor{couleurhaut}{rgb}{0.85,0.9,1}  % creme
\definecolor{couleurmilieu}{rgb}{1,1,1}  % vert pale
\definecolor{couleurbas}{rgb}{0.85,0.9,1}  % blanc
\setbeamertemplate{background canvas}[vertical shading]%
[top=couleurhaut,middle=couleurmilieu,midpoint=0.4,bottom=couleurbas] 
%[top=fondtitre!05,bottom=fondtitre!60]



\makeatletter
\setbeamertemplate{theorem begin}
{%
  \begin{\inserttheoremblockenv}
  {%
    \inserttheoremheadfont
    \inserttheoremname
    \inserttheoremnumber
    \ifx\inserttheoremaddition\@empty\else\ (\inserttheoremaddition)\fi%
    \inserttheorempunctuation
  }%
}
\setbeamertemplate{theorem end}{\end{\inserttheoremblockenv}}

\newenvironment{theoreme}[1][]{%
   \setbeamercolor{block title}{fg=structure,bg=structure!40}
   \setbeamercolor{block body}{fg=black,bg=structure!10}
   \begin{block}{{\bf Th\'eor\`eme }#1}
}{%
   \end{block}%
}


\newenvironment{proposition}[1][]{%
   \setbeamercolor{block title}{fg=structure,bg=structure!40}
   \setbeamercolor{block body}{fg=black,bg=structure!10}
   \begin{block}{{\bf Proposition }#1}
}{%
   \end{block}%
}

\newenvironment{corollaire}[1][]{%
   \setbeamercolor{block title}{fg=structure,bg=structure!40}
   \setbeamercolor{block body}{fg=black,bg=structure!10}
   \begin{block}{{\bf Corollaire }#1}
}{%
   \end{block}%
}

\newenvironment{mydefinition}[1][]{%
   \setbeamercolor{block title}{fg=structure,bg=structure!40}
   \setbeamercolor{block body}{fg=black,bg=structure!10}
   \begin{block}{{\bf Définition} #1}
}{%
   \end{block}%
}

\newenvironment{lemme}[0]{%
   \setbeamercolor{block title}{fg=structure,bg=structure!40}
   \setbeamercolor{block body}{fg=black,bg=structure!10}
   \begin{block}{\bf Lemme}
}{%
   \end{block}%
}

\newenvironment{remarque}[1][]{%
   \setbeamercolor{block title}{fg=black,bg=structure!20}
   \setbeamercolor{block body}{fg=black,bg=structure!5}
   \begin{block}{Remarque #1}
}{%
   \end{block}%
}


\newenvironment{exemple}[1][]{%
   \setbeamercolor{block title}{fg=black,bg=structure!20}
   \setbeamercolor{block body}{fg=black,bg=structure!5}
   \begin{block}{{\bf Exemple }#1}
}{%
   \end{block}%
}


\newenvironment{miniexercice}[0]{%
   \setbeamercolor{block title}{fg=structure,bg=structure!20}
   \setbeamercolor{block body}{fg=black,bg=structure!5}
   \begin{block}{Mini-exercices}
}{%
   \end{block}%
}


\newenvironment{tp}[0]{%
   \setbeamercolor{block title}{fg=structure,bg=structure!40}
   \setbeamercolor{block body}{fg=black,bg=structure!10}
   \begin{block}{\bf Travaux pratiques}
}{%
   \end{block}%
}
\newenvironment{exercicecours}[1][]{%
   \setbeamercolor{block title}{fg=structure,bg=structure!40}
   \setbeamercolor{block body}{fg=black,bg=structure!10}
   \begin{block}{{\bf Exercice }#1}
}{%
   \end{block}%
}
\newenvironment{algo}[1][]{%
   \setbeamercolor{block title}{fg=structure,bg=structure!40}
   \setbeamercolor{block body}{fg=black,bg=structure!10}
   \begin{block}{{\bf Algorithme}\hfill{\color{gray}\texttt{#1}}}
}{%
   \end{block}%
}


\setbeamertemplate{proof begin}{
   \setbeamercolor{block title}{fg=black,bg=structure!20}
   \setbeamercolor{block body}{fg=black,bg=structure!5}
   \begin{block}{{\footnotesize Démonstration}}
   \footnotesize
   \smallskip}
\setbeamertemplate{proof end}{%
   \end{block}}
\setbeamertemplate{qed symbol}{\openbox}


\makeatother
\usecolortheme[RGB={205,100,0}]{structure}

%%%%%%%%%%%%%%%%%%%%%%%%%%%%%%%%%%%%%%%%%%%%%%%%%%%%%%%%%%%%%
%%%%%%%%%%%%%%%%%%%%%%%%%%%%%%%%%%%%%%%%%%%%%%%%%%%%%%%%%%%%%

\begin{document}


\title{{\bf Systèmes linéaires}}
\subtitle{Introduction aux systèmes d'équations linéaires}

\begin{frame}
  
  \debutmontitre

  \pause

{\footnotesize
\hfill
\setbeamercovered{transparent=50}
\begin{minipage}{0.6\textwidth}
  \begin{itemize}
    \item<3-> Exemple : deux droites dans le plan
    \item<4-> Résolution par substitution
    \item<5-> Exemple : deux plans dans l'espace
    \item<6-> Résolution par la méthode de Cramer
    \item<7-> Résolution par inversion de matrice  
  \end{itemize}
\end{minipage}
}

\end{frame}

\setcounter{framenumber}{0}


%%%%%%%%%%%%%%%%%%%%%%%%%%%%%%%%%%%%%%%%%%%%%%%%%%%%%%%%%%%%%%%%
\section{Exemple : deux droites dans le plan}

\begin{frame}
\begin{itemize}
  \item L'équation d'une droite dans le plan $(Oxy)$ s'écrit
$$a x + b y = e$$
où $a, b$ et $e$ sont des constantes réelles
\pause  

  \item Cette équation s'appelle 
\defi{équation linéaire} dans les variables 
(ou inconnues) $x$ et $y$
%. Par exemple, $2x + 3y = 6$ est une équation linéaire.


\pause   
  \item Intersection de deux droites $D_1$ et $D_2$

\pause
  
  \item Un point $(x,y)$ est dans l'intersection
$D_1 \cap D_2$ s'il est solution du système : 
\begin{equation}
\left\{\begin{array}{rcl} 
a x + b y & = & e\\
c x + d y & = & f 
\end{array}\right.  
\tag{$S$}  
\label{eq:syslin1}  
\end{equation}
\end{itemize}










\end{frame}

\begin{frame}

\myfigure{0.55}{
\tikzinput{fig_syslin01} 
\qquad
\uncover<2->{\tikzinput{fig_syslin02}}
\qquad
\uncover<3->{\tikzinput{fig_syslin03}}
}



\begin{enumerate}
  \item Les droites $D_1$ et $D_2$ se coupent en un seul point

 Le système (\ref{eq:syslin1}) a une seule solution

\pause

  \item Les droites $D_1$ et $D_2$ sont parallèles
  
  Le système  (\ref{eq:syslin1}) n'a pas de solution
 
 \pause

  \item Les droites $D_1$ et $D_2$ sont confondues
  
  Le système  (\ref{eq:syslin1}) a une infinité de solutions 

\end{enumerate}

\end{frame}


%%%%%%%%%%%%%%%%%%%%%%%%%%%%%%%%%%%%%%%%%%%%%%%%%%%%%%%%%%%%%%%%
\section{Résolution par substitution}

\begin{frame}
\hfill\evidence{Méthode de la substitution}

$$
\left\{\begin{array}{rcl} 
3 x + 2 y & = & 1\\
2 x - 7 y & = & -2 
\end{array}\right.  
$$
\pause
%nous isolons $y$ dans la première équation, puis nous le remplaçons par sa valeur dans la seconde, ce qui donne le système équivalent :
$$
\iff \left\{\begin{array}{rcl} 
y & = & \frac12 - \frac32x\\
2 x - 7 (\frac12 - \frac32x) & = & -2 
\end{array}\right.
$$
\pause
$$
\iff
\left\{\begin{array}{rcl} 
y & = & \frac12 - \frac32x\\
(2+7\times\frac32)x  & = & -2 +\frac72
\end{array}\right.$$
\pause
$$
% \iff \left\{\begin{array}{rcl} 
% y & = & \frac12 - \frac32x\\
% x & = & \frac{3}{25} 
% \end{array}\right.
% \pause
\iff \left\{\begin{array}{rcl} 
\uncover<5->{y & = & \frac{8}{25}}\\
x & = & \frac{3}{25}
\end{array}\right.$$
\pause\pause

L'ensemble des solutions du système est :
$$\mathcal{S} = \left\lbrace \left(\frac{3}{25},\frac{8}{25}\right) \right\rbrace$$
\end{frame}




%%%%%%%%%%%%%%%%%%%%%%%%%%%%%%%%%%%%%%%%%%%%%%%%%%%%%%%%%%%%%%%%
\section{Exemple : deux plans dans l'espace}

\begin{frame}

\begin{itemize}
  \item Dans l'espace $(Oxyz)$, une équation linéaire 
$$a x + b y  + c z = d$$
où $(a,b,c)\neq (0,0,0)$ est l'équation d'un plan
\pause
  
  \item Les triplets $(x,y,z)$  solutions du système
à $2$ équations et à $3$ inconnues suivant : 
$$\left\{\begin{array}{rcl} 
a x + b y + c z    & = & d \\
a' x + b' y + c' z & = & d'
\end{array}\right.$$
sont les coordonnées des points dans l'intersection de deux plans dans l'espace
\end{itemize}

\pause

\begin{enumerate}
\item les plans sont parallèles (et distincts) et il n'y a alors aucune solution au système
\item les plans sont confondus et il y a une infinité de solutions
\item les plans se coupent en une droite et il y a une infinité de solutions
\end{enumerate}
\end{frame}


\begin{frame}
\begin{exemple}
\begin{enumerate}\setlength{\itemsep}{12pt}
  \item $\left\{\begin{array}{rcl} 
2 x + 3 y - 4 z  & = & 7 \\
4 x + 6 y - 8 z  & = & -1
\end{array}\right.$ \hfill deux équations incompatibles

\centerline{$\mathcal{S}= \varnothing$}

 \pause

  \item $\left\{\begin{array}{rcl} 
2 x + 3 y - 4 z  & = & 7 \\
4 x + 6 y - 8 z  & = & 14
 \end{array}\right.$  \hfill deux équations équivalentes
 
\centerline{$\mathcal{S}= \big\lbrace (x,y,\frac12 x + \frac34 y - \frac74) \mid x,y\in \Rr \big\rbrace$}
  \pause

 \item $\left\{\begin{array}{rcl} 
7 x + 2 y - 2 z  & = & 1 \\
2 x + 3 y + 2 z  & = & 1
 \end{array}\right.$ \hfill on choisit $x$ comme paramètre
 
\centerline{$\mathcal{S}= \left\lbrace \left(x,-\frac{9}{5} x +  \frac25,\frac{17}{10}x-\frac{1}{10}\right) \mid x\in \Rr \right\rbrace$}

\end{enumerate}
  
  
\end{exemple}

\end{frame}


%%%%%%%%%%%%%%%%%%%%%%%%%%%%%%%%%%%%%%%%%%%%%%%%%%%%%%%%%%%%%%%%
\section{Résolution par la méthode de Cramer}

\begin{frame}

\hfill\evidence{Méthode de Cramer}

Système de $2$ équations à $2$ inconnues :
$$
\left\{\begin{array}{rcl} 
a x + b y & = & e\\
c x + d y & = & f 
\end{array}\right.  
$$

\pause

Le \defi{déterminant} du système :
$$ \begin{vmatrix} a & b \\ c & d \end{vmatrix}=ad-bc$$

\pause

Si $ad-bc\neq 0$, il existe \evidence{une unique solution} $(x,y)$ :
$$x = \frac{\begin{vmatrix} e & b \\ f & d \end{vmatrix}}{\begin{vmatrix} a & b \\ c & d \end{vmatrix}} \qquad 
y = \frac{\begin{vmatrix} a & e \\ c & f \end{vmatrix}}{\begin{vmatrix} a & b \\ c & d \end{vmatrix}}$$

\end{frame}


\begin{frame}
\begin{exemple}
\begin{itemize}\setlength{\itemsep}{8pt}
  \item $
\left\{\begin{array}{rcl} 
t x - 2y   & = & 1\\
3x + t y & = & 1 
\end{array}\right.  
$ \qquad $t\in \Rr$

\pause
  
  \item $\left| \begin{smallmatrix} t & -2 \\ 3 & t \end{smallmatrix}\right|= t^2+6$
et ne s'annule jamais

\pause
  
  \item Il existe donc une unique solution $(x,y)$ :
$$x = \frac{\begin{vmatrix} 1 & -2 \\ 1 & t \end{vmatrix}}{t^2+6} = \frac{t+2}{t^2+6}, \qquad
y = \frac{\begin{vmatrix} t & 1 \\ 3 & 1 \end{vmatrix}}{t^2+6} = \frac{t-3}{t^2+6}$$
\pause
  
  \item Pour chaque $t$, l'ensemble des solutions est 
$\mathcal{S}= \left\lbrace \left(\frac{t+2}{t^2+6},\frac{t-3}{t^2+6}\right) \right\rbrace$
\end{itemize}

\end{exemple}

\end{frame}

%%%%%%%%%%%%%%%%%%%%%%%%%%%%%%%%%%%%%%%%%%%%%%%%%%%%%%%%%%%%%%%%
\section{Résolution par inversion de matrice}

\begin{frame}

\hfill\evidence{Inversion de matrice}

\begin{itemize}
  \item Le système linéaire
$$
\left\{\begin{array}{rcl} 
a x + b y & = & e\\
c x + d y & = & f 
\end{array}\right.  
$$

\pause
est équivalent à
$$AX = Y \quad \text{ où } \quad A = \begin{pmatrix} a & b \\ c & d \end{pmatrix} 
\quad  X = \begin{pmatrix} x \\ y \end{pmatrix} \quad Y = \begin{pmatrix} e \\ f \end{pmatrix}$$

\pause

  \item Si le déterminant de la matrice $A$ est non nul, c'est-à-dire si $ad-bc \neq 0$, 
alors la matrice $A$ est inversible et
$$A^{-1} = \frac{1}{ad-bc} \begin{pmatrix} d & -b \\ -c & a \end{pmatrix}$$

\pause


et l'unique solution $X=\left( \begin{smallmatrix} x \\ y \end{smallmatrix}\right)$
du système est donnée par 
$$X = A^{-1} Y$$
\end{itemize}

\end{frame}


\begin{frame}
\begin{exemple}

\begin{itemize}
  \item $
\left\{\begin{array}{rcl} 
 x + y  & = & 1\\
 x + t^2 y & = & t 
\end{array}\right.  
$ 
\qquad  $t\in \Rr$

\pause
  
  \item Le déterminant est 
  $\left| \begin{smallmatrix} 1 & 1 \\ 1 & t^2 \end{smallmatrix}\right|= t^2-1$

\pause  

  \item \textbf{Premier cas : $t\neq+1$ et $t\neq-1$}

  \begin{itemize}
    \item Alors $t^2-1\neq 0$
\pause

    \item La matrice $A=\left( \begin{smallmatrix} 1 & 1 \\ 1 & t^2 \end{smallmatrix}\right)$
est inversible d'inverse $A^{-1} = \frac{1}{t^2-1}\left( \begin{smallmatrix} t^2 & -1 \\ -1 & 1 \end{smallmatrix}\right)$
\pause

    \item La solution $X=\left( \begin{smallmatrix} x \\ y \end{smallmatrix}\right)$ est
$$\hspace*{-2em}X = A^{-1} Y = \frac{1}{t^2-1}\begin{pmatrix} t^2 & -1 \\ -1 & 1 \end{pmatrix} \begin{pmatrix} 1 \\ t \end{pmatrix} 
= \frac{1}{t^2-1}\begin{pmatrix} t^2-t\\ t-1 \end{pmatrix}
= \begin{pmatrix} \frac{t}{t+1} \\ \frac{1}{t+1} \end{pmatrix}$$
\pause

    \item Pour chaque $t\neq \pm1$, l'ensemble des solutions est
    
\centerline{$\mathcal{S}= \left\lbrace \left(\frac{t}{t+1},\frac{1}{t+1}\right) \right\rbrace$}
  \end{itemize}
  
\end{itemize}

\end{exemple}

\end{frame}


\begin{frame}
\begin{exemple}
\begin{itemize}
  \item $
\left\{\begin{array}{rcl} 
 x + y  & = & 1\\
 x + t^2 y & = & t 
\end{array}\right.  
$ 
\qquad  $t\in \Rr$

  \item \textbf{Deuxième cas : $t=+1$}
\pause  
  \begin{itemize}
    \item Le système s'écrit alors :$
\left\{\begin{array}{rcl} 
 x + y  & = & 1\\
 x + y & = & 1 
\end{array}\right.  
$ 

\pause

    \item Les deux équations sont identiques
\pause  
    \item Il y a une infinité de solutions 
  
  \centerline{$\mathcal{S}= \big\lbrace (x,1-x) \mid x\in \Rr \big\rbrace$}
  \end{itemize}
\pause

  \item \textbf{Troisième cas : $t=-1$}
  
\pause  
  \begin{itemize}
    \item Le système s'écrit alors :
$
\left\{\begin{array}{rcl} 
 x + y  & = & 1\\
 x + y & = & -1 
\end{array}\right.$

  \pause
  
    \item Les deux équations sont incompatibles 
  \pause
    \item Donc $\mathcal{S}= \varnothing$
  \end{itemize}
\end{itemize}
\end{exemple}

\end{frame}

%%%%%%%%%%%%%%%%%%%%%%%%%%%%%%%%%%%%%%%%%%%%%%%%%%%%%%%%%%%%%%%%
\section{Mini-exercices}

\begin{frame}

\begin{miniexercice}
\begin{enumerate}
  \item Tracer les droites et résoudre le système linéaire   
$\left\{\begin{array}{rcl} 
x-2y  & = & -1\\
-x+3y & = & 3 
\end{array}\right.$ de trois façons différentes : 
substitution, méthode de Cramer, inverse d'une matrice.
Idem avec $\left\{\begin{array}{rcl} 
2x-y  & = & 4\\
3x+3y & = & -5 
\end{array}\right..$

  \item Résoudre suivant la valeur du paramètre $t\in \Rr$ : 
$\left\{\begin{array}{rcl} 
4x-3y & = & t\\
2x-y  & = & t^2 
\end{array}\right..$

  \item  Discuter et résoudre suivant la valeur du paramètre $t\in \Rr$ :  
$\left\{\begin{array}{rcl} 
tx-y     & = & 1\\
x+(t-2)y & = & -1 
\end{array}\right..$
Idem avec $\left\{\begin{array}{rcl} 
(t-1)x+y & = & 1\\
2x+ty    & = & -1 
\end{array}\right..$

 % \item ???
\end{enumerate}

\end{miniexercice}

\end{frame}

\end{document}

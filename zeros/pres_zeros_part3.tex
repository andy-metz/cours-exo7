
%%%%%%%%%%%%%%%%%% PREAMBULE %%%%%%%%%%%%%%%%%%

\documentclass[aspectratio=169,utf8]{beamer}
%\documentclass[aspectratio=169,handout]{beamer}

\usetheme{Boadilla}
%\usecolortheme{seahorse}
%\usecolortheme[RGB={245,66,24}]{structure}
\useoutertheme{infolines}

% packages
\usepackage{amsfonts,amsmath,amssymb,amsthm}
\usepackage[utf8]{inputenc}
\usepackage[T1]{fontenc}
\usepackage{lmodern}

\usepackage[francais]{babel}
\usepackage{fancybox}
\usepackage{graphicx}

\usepackage{float}
\usepackage{xfrac}

%\usepackage[usenames, x11names]{xcolor}
\usepackage{pgfplots}
\usepackage{datetime}


% ----------------------------------------------------------------------
% Pour les images
\usepackage{tikz}
\usetikzlibrary{calc,shadows,arrows.meta,patterns,matrix}

\newcommand{\tikzinput}[1]{\input{figures/#1.tikz}}
% --- les figures avec échelle éventuel
\newcommand{\myfigure}[2]{% entrée : échelle, fichier(s) figure à inclure
\begin{center}\small%
\tikzstyle{every picture}=[scale=1.0*#1]% mise en échelle + 0% (automatiquement annulé à la fin du groupe)
#2%
\end{center}}



%-----  Package unités -----
\usepackage{siunitx}
\sisetup{locale = FR,detect-all,per-mode = symbol}

%\usepackage{mathptmx}
%\usepackage{fouriernc}
%\usepackage{newcent}
%\usepackage[mathcal,mathbf]{euler}

%\usepackage{palatino}
%\usepackage{newcent}
% \usepackage[mathcal,mathbf]{euler}



% \usepackage{hyperref}
% \hypersetup{colorlinks=true, linkcolor=blue, urlcolor=blue,
% pdftitle={Exo7 - Exercices de mathématiques}, pdfauthor={Exo7}}


%section
% \usepackage{sectsty}
% \allsectionsfont{\bf}
%\sectionfont{\color{Tomato3}\upshape\selectfont}
%\subsectionfont{\color{Tomato4}\upshape\selectfont}

%----- Ensembles : entiers, reels, complexes -----
\newcommand{\Nn}{\mathbb{N}} \newcommand{\N}{\mathbb{N}}
\newcommand{\Zz}{\mathbb{Z}} \newcommand{\Z}{\mathbb{Z}}
\newcommand{\Qq}{\mathbb{Q}} \newcommand{\Q}{\mathbb{Q}}
\newcommand{\Rr}{\mathbb{R}} \newcommand{\R}{\mathbb{R}}
\newcommand{\Cc}{\mathbb{C}} 
\newcommand{\Kk}{\mathbb{K}} \newcommand{\K}{\mathbb{K}}

%----- Modifications de symboles -----
\renewcommand{\epsilon}{\varepsilon}
\renewcommand{\Re}{\mathop{\text{Re}}\nolimits}
\renewcommand{\Im}{\mathop{\text{Im}}\nolimits}
%\newcommand{\llbracket}{\left[\kern-0.15em\left[}
%\newcommand{\rrbracket}{\right]\kern-0.15em\right]}

\renewcommand{\ge}{\geqslant}
\renewcommand{\geq}{\geqslant}
\renewcommand{\le}{\leqslant}
\renewcommand{\leq}{\leqslant}
\renewcommand{\epsilon}{\varepsilon}

%----- Fonctions usuelles -----
\newcommand{\ch}{\mathop{\text{ch}}\nolimits}
\newcommand{\sh}{\mathop{\text{sh}}\nolimits}
\renewcommand{\tanh}{\mathop{\text{th}}\nolimits}
\newcommand{\cotan}{\mathop{\text{cotan}}\nolimits}
\newcommand{\Arcsin}{\mathop{\text{arcsin}}\nolimits}
\newcommand{\Arccos}{\mathop{\text{arccos}}\nolimits}
\newcommand{\Arctan}{\mathop{\text{arctan}}\nolimits}
\newcommand{\Argsh}{\mathop{\text{argsh}}\nolimits}
\newcommand{\Argch}{\mathop{\text{argch}}\nolimits}
\newcommand{\Argth}{\mathop{\text{argth}}\nolimits}
\newcommand{\pgcd}{\mathop{\text{pgcd}}\nolimits} 


%----- Commandes divers ------
\newcommand{\ii}{\mathrm{i}}
\newcommand{\dd}{\text{d}}
\newcommand{\id}{\mathop{\text{id}}\nolimits}
\newcommand{\Ker}{\mathop{\text{Ker}}\nolimits}
\newcommand{\Card}{\mathop{\text{Card}}\nolimits}
\newcommand{\Vect}{\mathop{\text{Vect}}\nolimits}
\newcommand{\Mat}{\mathop{\text{Mat}}\nolimits}
\newcommand{\rg}{\mathop{\text{rg}}\nolimits}
\newcommand{\tr}{\mathop{\text{tr}}\nolimits}


%----- Structure des exercices ------

\newtheoremstyle{styleexo}% name
{2ex}% Space above
{3ex}% Space below
{}% Body font
{}% Indent amount 1
{\bfseries} % Theorem head font
{}% Punctuation after theorem head
{\newline}% Space after theorem head 2
{}% Theorem head spec (can be left empty, meaning ‘normal’)

%\theoremstyle{styleexo}
\newtheorem{exo}{Exercice}
\newtheorem{ind}{Indications}
\newtheorem{cor}{Correction}


\newcommand{\exercice}[1]{} \newcommand{\finexercice}{}
%\newcommand{\exercice}[1]{{\tiny\texttt{#1}}\vspace{-2ex}} % pour afficher le numero absolu, l'auteur...
\newcommand{\enonce}{\begin{exo}} \newcommand{\finenonce}{\end{exo}}
\newcommand{\indication}{\begin{ind}} \newcommand{\finindication}{\end{ind}}
\newcommand{\correction}{\begin{cor}} \newcommand{\fincorrection}{\end{cor}}

\newcommand{\noindication}{\stepcounter{ind}}
\newcommand{\nocorrection}{\stepcounter{cor}}

\newcommand{\fiche}[1]{} \newcommand{\finfiche}{}
\newcommand{\titre}[1]{\centerline{\large \bf #1}}
\newcommand{\addcommand}[1]{}
\newcommand{\video}[1]{}

% Marge
\newcommand{\mymargin}[1]{\marginpar{{\small #1}}}

\def\noqed{\renewcommand{\qedsymbol}{}}


%----- Presentation ------
\setlength{\parindent}{0cm}

%\newcommand{\ExoSept}{\href{http://exo7.emath.fr}{\textbf{\textsf{Exo7}}}}

\definecolor{myred}{rgb}{0.93,0.26,0}
\definecolor{myorange}{rgb}{0.97,0.58,0}
\definecolor{myyellow}{rgb}{1,0.86,0}

\newcommand{\LogoExoSept}[1]{  % input : echelle
{\usefont{U}{cmss}{bx}{n}
\begin{tikzpicture}[scale=0.1*#1,transform shape]
  \fill[color=myorange] (0,0)--(4,0)--(4,-4)--(0,-4)--cycle;
  \fill[color=myred] (0,0)--(0,3)--(-3,3)--(-3,0)--cycle;
  \fill[color=myyellow] (4,0)--(7,4)--(3,7)--(0,3)--cycle;
  \node[scale=5] at (3.5,3.5) {Exo7};
\end{tikzpicture}}
}


\newcommand{\debutmontitre}{
  \author{} \date{} 
  \thispagestyle{empty}
  \hspace*{-10ex}
  \begin{minipage}{\textwidth}
    \titlepage  
  \vspace*{-2.5cm}
  \begin{center}
    \LogoExoSept{2.5}
  \end{center}
  \end{minipage}

  \vspace*{-0cm}
  
  % Astuce pour que le background ne soit pas discrétisé lors de la conversion pdf -> png
\begin{tikzpicture}
        \fill[opacity=0,green!60!black] (0,0)--++(0,0)--++(0,0)--++(0,0)--cycle; 
\end{tikzpicture}

% toc S'affiche trop tot :
% \tableofcontents[hideallsubsections, pausesections]
}

\newcommand{\finmontitre}{
  \end{frame}
  \setcounter{framenumber}{0}
} % ne marche pas pour une raison obscure

%----- Commandes supplementaires ------

% \usepackage[landscape]{geometry}
% \geometry{top=1cm, bottom=3cm, left=2cm, right=10cm, marginparsep=1cm
% }
% \usepackage[a4paper]{geometry}
% \geometry{top=2cm, bottom=2cm, left=2cm, right=2cm, marginparsep=1cm
% }

%\usepackage{standalone}


% New command Arnaud -- november 2011
\setbeamersize{text margin left=24ex}
% si vous modifier cette valeur il faut aussi
% modifier le decalage du titre pour compenser
% (ex : ici =+10ex, titre =-5ex

\theoremstyle{definition}
%\newtheorem{proposition}{Proposition}
%\newtheorem{exemple}{Exemple}
%\newtheorem{theoreme}{Théorème}
%\newtheorem{lemme}{Lemme}
%\newtheorem{corollaire}{Corollaire}
%\newtheorem*{remarque*}{Remarque}
%\newtheorem*{miniexercice}{Mini-exercices}
%\newtheorem{definition}{Définition}

% Commande tikz
\usetikzlibrary{calc}
\usetikzlibrary{patterns,arrows}
\usetikzlibrary{matrix}
\usetikzlibrary{fadings} 

%definition d'un terme
\newcommand{\defi}[1]{{\color{myorange}\textbf{\emph{#1}}}}
\newcommand{\evidence}[1]{{\color{blue}\textbf{\emph{#1}}}}
\newcommand{\assertion}[1]{\emph{\og#1\fg}}  % pour chapitre logique
%\renewcommand{\contentsname}{Sommaire}
\renewcommand{\contentsname}{}
\setcounter{tocdepth}{2}



%------ Encadrement ------

\usepackage{fancybox}


\newcommand{\mybox}[1]{
\setlength{\fboxsep}{7pt}
\begin{center}
\shadowbox{#1}
\end{center}}

\newcommand{\myboxinline}[1]{
\setlength{\fboxsep}{5pt}
\raisebox{-10pt}{
\shadowbox{#1}
}
}

%--------------- Commande beamer---------------
\newcommand{\beameronly}[1]{#1} % permet de mettre des pause dans beamer pas dans poly


\setbeamertemplate{navigation symbols}{}
\setbeamertemplate{footline}  % tiré du fichier beamerouterinfolines.sty
{
  \leavevmode%
  \hbox{%
  \begin{beamercolorbox}[wd=.333333\paperwidth,ht=2.25ex,dp=1ex,center]{author in head/foot}%
    % \usebeamerfont{author in head/foot}\insertshortauthor%~~(\insertshortinstitute)
    \usebeamerfont{section in head/foot}{\bf\insertshorttitle}
  \end{beamercolorbox}%
  \begin{beamercolorbox}[wd=.333333\paperwidth,ht=2.25ex,dp=1ex,center]{title in head/foot}%
    \usebeamerfont{section in head/foot}{\bf\insertsectionhead}
  \end{beamercolorbox}%
  \begin{beamercolorbox}[wd=.333333\paperwidth,ht=2.25ex,dp=1ex,right]{date in head/foot}%
    % \usebeamerfont{date in head/foot}\insertshortdate{}\hspace*{2em}
    \insertframenumber{} / \inserttotalframenumber\hspace*{2ex} 
  \end{beamercolorbox}}%
  \vskip0pt%
}


\definecolor{mygrey}{rgb}{0.5,0.5,0.5}
\setlength{\parindent}{0cm}
%\DeclareTextFontCommand{\helvetica}{\fontfamily{phv}\selectfont}

% background beamer
\definecolor{couleurhaut}{rgb}{0.85,0.9,1}  % creme
\definecolor{couleurmilieu}{rgb}{1,1,1}  % vert pale
\definecolor{couleurbas}{rgb}{0.85,0.9,1}  % blanc
\setbeamertemplate{background canvas}[vertical shading]%
[top=couleurhaut,middle=couleurmilieu,midpoint=0.4,bottom=couleurbas] 
%[top=fondtitre!05,bottom=fondtitre!60]



\makeatletter
\setbeamertemplate{theorem begin}
{%
  \begin{\inserttheoremblockenv}
  {%
    \inserttheoremheadfont
    \inserttheoremname
    \inserttheoremnumber
    \ifx\inserttheoremaddition\@empty\else\ (\inserttheoremaddition)\fi%
    \inserttheorempunctuation
  }%
}
\setbeamertemplate{theorem end}{\end{\inserttheoremblockenv}}

\newenvironment{theoreme}[1][]{%
   \setbeamercolor{block title}{fg=structure,bg=structure!40}
   \setbeamercolor{block body}{fg=black,bg=structure!10}
   \begin{block}{{\bf Th\'eor\`eme }#1}
}{%
   \end{block}%
}


\newenvironment{proposition}[1][]{%
   \setbeamercolor{block title}{fg=structure,bg=structure!40}
   \setbeamercolor{block body}{fg=black,bg=structure!10}
   \begin{block}{{\bf Proposition }#1}
}{%
   \end{block}%
}

\newenvironment{corollaire}[1][]{%
   \setbeamercolor{block title}{fg=structure,bg=structure!40}
   \setbeamercolor{block body}{fg=black,bg=structure!10}
   \begin{block}{{\bf Corollaire }#1}
}{%
   \end{block}%
}

\newenvironment{mydefinition}[1][]{%
   \setbeamercolor{block title}{fg=structure,bg=structure!40}
   \setbeamercolor{block body}{fg=black,bg=structure!10}
   \begin{block}{{\bf Définition} #1}
}{%
   \end{block}%
}

\newenvironment{lemme}[0]{%
   \setbeamercolor{block title}{fg=structure,bg=structure!40}
   \setbeamercolor{block body}{fg=black,bg=structure!10}
   \begin{block}{\bf Lemme}
}{%
   \end{block}%
}

\newenvironment{remarque}[1][]{%
   \setbeamercolor{block title}{fg=black,bg=structure!20}
   \setbeamercolor{block body}{fg=black,bg=structure!5}
   \begin{block}{Remarque #1}
}{%
   \end{block}%
}


\newenvironment{exemple}[1][]{%
   \setbeamercolor{block title}{fg=black,bg=structure!20}
   \setbeamercolor{block body}{fg=black,bg=structure!5}
   \begin{block}{{\bf Exemple }#1}
}{%
   \end{block}%
}


\newenvironment{miniexercice}[0]{%
   \setbeamercolor{block title}{fg=structure,bg=structure!20}
   \setbeamercolor{block body}{fg=black,bg=structure!5}
   \begin{block}{Mini-exercices}
}{%
   \end{block}%
}


\newenvironment{tp}[0]{%
   \setbeamercolor{block title}{fg=structure,bg=structure!40}
   \setbeamercolor{block body}{fg=black,bg=structure!10}
   \begin{block}{\bf Travaux pratiques}
}{%
   \end{block}%
}
\newenvironment{exercicecours}[1][]{%
   \setbeamercolor{block title}{fg=structure,bg=structure!40}
   \setbeamercolor{block body}{fg=black,bg=structure!10}
   \begin{block}{{\bf Exercice }#1}
}{%
   \end{block}%
}
\newenvironment{algo}[1][]{%
   \setbeamercolor{block title}{fg=structure,bg=structure!40}
   \setbeamercolor{block body}{fg=black,bg=structure!10}
   \begin{block}{{\bf Algorithme}\hfill{\color{gray}\texttt{#1}}}
}{%
   \end{block}%
}


\setbeamertemplate{proof begin}{
   \setbeamercolor{block title}{fg=black,bg=structure!20}
   \setbeamercolor{block body}{fg=black,bg=structure!5}
   \begin{block}{{\footnotesize Démonstration}}
   \footnotesize
   \smallskip}
\setbeamertemplate{proof end}{%
   \end{block}}
\setbeamertemplate{qed symbol}{\openbox}


\makeatother
\usecolortheme[RGB={0,127,0}]{structure}

% Commande spécifique à ce chapitre


\makeatletter
\pgfdeclareverticalshading[black,bg]{bmb@shadow}{200cm}{%
  color(0bp)=(blue!25); color(4bp)=(black!50!bg); color(8bp)=(black!50!bg)}
\pgfdeclareradialshading[black,bg]{bmb@shadowball}{\pgfpointorigin}{%
  color(0bp)=(black!50!bg); color(4bp)=(blue!25)}
\pgfdeclareradialshading[black,bg]{bmb@shadowballlarge}{\pgfpointorigin}{%
  color(0bp)=(black!50!bg); color(4bp)=(black!50!bg); color(8bp)=(blue!25)}
  %
\makeatother
   
\newcommand{\Python}{\texttt{Python}}
\renewcommand{\evidence}[1]{{\color{blue}\textbf{#1}}}

\usepackage{textcomp}

\usepackage{listings}
\lstset{
  upquote=true,
  columns=flexible,
  keepspaces=true,
  basicstyle=\ttfamily,
  commentstyle=\color{gray},
  language=Python,
  showstringspaces=false,
  aboveskip=0em,  
  belowskip=0em,
  escapeinside=||
}

\lstset{
  literate={é}{{\'e}}1
           {è}{{\`e}}1
           {à}{{\`a}}1
}


\newcommand{\codeinline}[1]{\lstinline!#1!}


%%%%%%%%%%%%%%%%%%%%%%%%%%%%%%%%%%%%%%%%%%%%%%%%%%%%%%%%%%%%%
%%%%%%%%%%%%%%%%%%%%%%%%%%%%%%%%%%%%%%%%%%%%%%%%%%%%%%%%%%%%%


\begin{document}


\title{{\bf Zéros des fonctions}}
\subtitle{La méthode de Newton}

\begin{frame}
  
  \debutmontitre

  \pause

{\footnotesize
\hfill
\setbeamercovered{transparent=50}
\begin{minipage}{0.6\textwidth}
  \begin{itemize}
    \item<3-> Méthode de Newton 
    \item<4-> Résultats pour $\sqrt{10}$
    \item<5-> Résultats numériques pour $(1,10)^{1/12}$
    \item<6-> Calcul de l'erreur pour $\sqrt{10}$
    \item<7-> Algorithme    
  \end{itemize}
\end{minipage}
}

\end{frame}

\setcounter{framenumber}{0}

%%%%%%%%%%%%%%%%%%%%%%%%%%%%%%%%%%%%%%%%%%%%%%%%%%%%%%%%%%%%%%%%
\section{Méthode de Newton}

\begin{frame}

\begin{itemize}
  \item $f:[a,b] \to \Rr$ dérivable et d'un point $u_0 \in[a,b]$
 
\uncover<5->{  \item Tangente au graphe de $f$ en $(u_0,f(u_0))$}
 
\uncover<7->{  \item $(u_1,0)$ l'intersection de la tangente avec l'axe des abscisses }
  
\uncover<9->{  \item Si $u_1 \in[a,b]$ alors on recommence le processus }
\end{itemize}

\uncover<12->{
\mybox{$\displaystyle u_0 \in [a,b] \quad \text{ et } \quad u_{n+1} = u_n - \frac{f(u_n)}{f'(u_n)}$}
}

\uncover<2->{
\myfigure{1}{
\tikzinput{fig_zeros08-pres}
}
}

\end{frame}


\begin{frame}
\begin{minipage}{0.62\textwidth}
\mybox{$\displaystyle u_0 \in [a,b] \quad \text{ et } \quad u_{n+1} = u_n - \frac{f(u_n)}{f'(u_n)}$}  
\end{minipage}
\pause
\begin{minipage}{0.32\textwidth}
\myfigure{0.7}{
\tikzinput{fig_zeros09-pres}
}  
\end{minipage}
\pause\pause\pause
\vspace*{-3ex}
\begin{proof}
\begin{itemize}[<+->]
\pause

  \item La tangente au point d'abscisse $u_n$ a pour équation :
$y = f'(u_n)(x-u_n)+f(u_n)$
  \item Donc le point $(x,0)$ appartenant à la tangente 
(et à l'axe des abscisses)
vérifie $0=f'(u_n)(x-u_n)+f(u_n)$
  \item D'où $x=u_n - \frac{f(u_n)}{f'(u_n)}$ \qedhere
\end{itemize}

\end{proof}
\end{frame}


%%%%%%%%%%%%%%%%%%%%%%%%%%%%%%%%%%%%%%%%%%%%%%%%%%%%%%%%%%%%%%%%
\section{Résultats pour $\sqrt{10}$}

\begin{frame}
\medskip
\begin{minipage}{0.47\textwidth}
\begin{itemize}
  \item Calcul de $\sqrt{a}$
  \item $f(x)=x^2-a$ 
  \item $f'(x)=2x$
  \item $u_0>0$ et $u_{n+1} = u_n - \frac{u_n^2-a}{2u_n}$
\end{itemize}  
\end{minipage}
\pause 
\begin{minipage}{0.49\textwidth}
\hspace*{-2em}
\myboxinline{$\displaystyle u_0>0 \quad \text{ et } 
\quad u_{n+1} = \frac12 \left(u_n+\frac{a}{u_n}\right)$}

\   
\end{minipage}
\medskip
\pause 
\begin{proposition}
\label{prop:heron}
Cette suite $(u_n)$ converge vers $\sqrt{a}$
\end{proposition}
\pause 
$$
\begin{array}{l}
  u_0 = 4     \\
\pause 
  u_1 = \frac12 \left(u_0+\frac{10}{u_0}\right) = \frac12\left(4+\frac{10}{4}\right) = \frac{13}{4} = 3,25 \\
\pause 
  u_2 = \frac12 \left(u_1+\frac{10}{u_1}\right) = \frac12\left(\tfrac{13}{4}+\frac{10}{\tfrac{13}{4}}\right) 
  = \frac{329}{104} = 3,1634\ldots \\
\pause 
  u_3 = \frac12 \left(u_2+\frac{10}{u_2}\right) = \frac{216\,401}{68\,432} = 3,16227788 \ldots \\
\pause 
  u_4 = 3,1622776601683\ldots  
\end{array}
$$

\end{frame}


\begin{frame}
\vspace*{-1ex}
\mybox{$\displaystyle u_0>0 \quad \text{ et } \quad u_{n+1} = \frac12 \left(u_n+\frac{a}{u_n}\right)$}
\vspace*{-2ex}
\begin{proof} \vspace*{-1ex}
\pause
\begin{enumerate}
  \item $u_n \ge \sqrt{a}$ pour $n\ge1$
 \vspace*{-2ex}   
  $$u_{n+1}^2-a = \frac14 \left(\frac{u_n^2 + a}{u_n}\right)^2 - a 
  = \frac{1}{4u_n^2}(u_n^4-2au_n^2+a^2)=\frac14 \frac{(u_n^2-a)^2}{u_n^2}$$
\vspace*{-1ex}  
  Donc $u_{n+1}^2 - a \ge 0$, ainsi pour tout $n\ge 0$, $u_{n+1} \ge \sqrt{a}$
  
\pause  
  \item $(u_n)_{n\ge1}$ est décroissante et converge
  \begin{itemize}
    \item $\frac{u_{n+1}}{u_n} = \frac12 \left(1+\frac{a}{u_n^2}\right)$
    \item $u_n^2 \ge a$ (donc $\frac{a}{u_n^2}\le 1$) alors $\frac{u_{n+1}}{u_n} \le 1$
    \item $(u_n)_{n\ge1}$ est décroissante et minorée par $0$ donc elle converge
  \end{itemize}

\pause 
  \item $(u_n)$ converge vers $\sqrt{a}$
  \begin{itemize}
    \item $u_n \to \ell$ et $u_{n+1} \to \ell$
    \item relation $u_{n+1} = \frac12 \left(u_n+\frac{a}{u_n}\right)$ implique  $\ell = \frac12 \left(\ell+\frac{a}{\ell}\right)$
    \item d'où $\ell^2=a$ et ainsi $\ell = \sqrt{a}$ \qedhere
  \end{itemize}\vspace*{-1ex}   
\end{enumerate}

\end{proof}
\end{frame}


%%%%%%%%%%%%%%%%%%%%%%%%%%%%%%%%%%%%%%%%%%%%%%%%%%%%%%%%%%%%%%%%
\section{Résultats numériques pour $(1,10)^{1/12}$}

\begin{frame}

\begin{itemize}[<+->]\setlength{\itemsep}{6pt}
  \item Calcul de $(1,10)^{1/12}$
  \item On pose $f(x)=x^{12}-a$, $a=1,10$
  \item $f'(x)=12x^{11}$
  \item $u_0>0 \quad \text{ et } \quad u_{n+1}= u_n - \frac{u_n^{12}-a}{12u_n^{11}} = \frac1{12} \left(11u_n+\frac{a}{u_n^{11}}\right)$
  \item Résultats numériques pour $(1,10)^{1/12}$ en partant de $u_0=1$
  $$
\begin{array}{l}
  u_0 = 1     \\
  u_1 = 1,0083333333333333 \ldots \\
  u_2 = 1,0079748433368980\ldots \\
  u_3 = 1,0079741404315996 \ldots \\
  u_4 = 1,0079741404289038 \ldots \\
\end{array}
$$  
\end{itemize}


\end{frame}


%%%%%%%%%%%%%%%%%%%%%%%%%%%%%%%%%%%%%%%%%%%%%%%%%%%%%%%%%%%%%%%%
\section{Calcul de l'erreur pour $\sqrt{10}$}

\begin{frame}
\begin{proposition}
\begin{enumerate}
  \item Soit $k$ tel que $u_1-\sqrt a\le k$ alors pour $n\ge 1$ :
  $$u_n - \sqrt{a} \le 2\sqrt{a} \left( \frac{k}{2\sqrt{a}} \right)^{2^{n-1}}$$

  \pause
  
  \item Pour $a=10$, $u_0=4$, on a $u_1=3,25$, $k = \frac14$ et 
  $$u_n - \sqrt{10} \le 8 \left(\frac{1}{24} \right)^{2^{n-1}}$$
%  Donc si $\ell \ge 1+\frac{\log\left(\frac{\ell-\log8}{\log24}\right)}{\log2}$ alors $u_n - \sqrt{10} \le 10^{-\ell}$.
\end{enumerate}
\end{proposition}

  \pause

  \medskip
  
\begin{center}
\begin{tabular}{ll}
  $10^{-10}$ ($\sim 10$ décimales) &  $4$ itérations \\
  $10^{-100}$ ($\sim 100$ décimales) &  $8$ itérations \\ 
  $10^{-1000}$ ($\sim 1000$ décimales) &  $11$ itérations \\ 
\end{tabular}  
\end{center}


\end{frame}



%%%%%%%%%%%%%%%%%%%%%%%%%%%%%%%%%%%%%%%%%%%%%%%%%%%%%%%%%%%%%%%%
\section{Algorithme}

\begin{frame}[fragile]
\begin{algo}[secante.py]
\begin{lstlisting}
def racine_carree(a,n):
    u=4                # N'importe qu'elle valeur > 0
    for i in range(n):
        u = 0.5*(u+a/u)
    return u

\end{lstlisting}  
\end{algo} 
\end{frame}


\begin{frame}

\centerline{Pour $n=11$, $u_{11}$ donne $1000$ décimales de $\sqrt{10}$}

\begin{center}
{\scriptsize
\qquad\qquad\qquad 3, \hfill\hfill\   \\
16227766016837933199889354443271853371955513932521 \ 68268575048527925944386392382213442481083793002951 \\
87347284152840055148548856030453880014690519596700 \ 15390334492165717925994065915015347411333948412408 \\
53169295770904715764610443692578790620378086099418 \ 28371711548406328552999118596824564203326961604691 \\
31433612894979189026652954361267617878135006138818 \ 62785804636831349524780311437693346719738195131856 \\
78403231241795402218308045872844614600253577579702 \ 82864402902440797789603454398916334922265261206779 \\
26516760310484366977937569261557205003698949094694 \ 21850007358348844643882731109289109042348054235653 \\
40390727401978654372593964172600130699000095578446 \ 31096267906944183361301813028945417033158077316263 \\
86395193793704654765220632063686587197822049312426 \ 05345411160935697982813245229700079888352375958532 \\
85792513629646865114976752171234595592380393756251 \ 25369855194955325099947038843990336466165470647234 \\
99979613234340302185705218783667634578951073298287 \ 51579452157716521396263244383990184845609357626020 \\
  }  
\end{center}


\end{frame}


%%%%%%%%%%%%%%%%%%%%%%%%%%%%%%%%%%%%%%%%%%%%%%%%%%%%%%%%%%%%%%%%
\section{Mini-exercices}

\begin{frame}

\begin{miniexercice}
\begin{enumerate}
  \item \`A la calculette, calculer les trois premières étapes pour une approximation de 
  $\sqrt{3}$, sous forme de nombres rationnels.   Idem avec $\sqrt[3]{2}$. 
   
  \item Implémenter la méthode de Newton, étant données une fonction $f$ et sa dérivée $f'$.
  
  \item Calculer une approximation des solutions de l'équation $x^3+1=3x$.
  
  \item Soit $a>0$. Comment calculer $\frac{1}{a}$ par une méthode de Newton ?
  
  \item Calculer $n$ de sorte que $u_n-\sqrt{10} \le 10^{-\ell}$ (avec $u_0=4$, 
  $u_{n+1} =\frac{1}{2} \left(u_n+\frac{a}{u_n}\right)$, $a=10$).
  
\end{enumerate}
\end{miniexercice}

\end{frame}

\end{document}
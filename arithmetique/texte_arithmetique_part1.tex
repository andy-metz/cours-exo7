
%%%%%%%%%%%%%%%%%% PREAMBULE %%%%%%%%%%%%%%%%%%


\documentclass[12pt]{article}

\usepackage{amsfonts,amsmath,amssymb,amsthm}
\usepackage[utf8]{inputenc}
\usepackage[T1]{fontenc}
\usepackage[francais]{babel}


% packages
\usepackage{amsfonts,amsmath,amssymb,amsthm}
\usepackage[utf8]{inputenc}
\usepackage[T1]{fontenc}
%\usepackage{lmodern}

\usepackage[francais]{babel}
\usepackage{fancybox}
\usepackage{graphicx}

\usepackage{float}

%\usepackage[usenames, x11names]{xcolor}
\usepackage{tikz}
\usepackage{datetime}

\usepackage{mathptmx}
%\usepackage{fouriernc}
%\usepackage{newcent}
\usepackage[mathcal,mathbf]{euler}

%\usepackage{palatino}
%\usepackage{newcent}


% Commande spéciale prompteur

%\usepackage{mathptmx}
%\usepackage[mathcal,mathbf]{euler}
%\usepackage{mathpple,multido}

\usepackage[a4paper]{geometry}
\geometry{top=2cm, bottom=2cm, left=1cm, right=1cm, marginparsep=1cm}

\newcommand{\change}{{\color{red}\rule{\textwidth}{1mm}\\}}

\newcounter{mydiapo}

\newcommand{\diapo}{\newpage
\hfill {\normalsize  Diapo \themydiapo \quad \texttt{[\jobname]}} \\
\stepcounter{mydiapo}}


%%%%%%% COULEURS %%%%%%%%%%

% Pour blanc sur noir :
%\pagecolor[rgb]{0.5,0.5,0.5}
% \pagecolor[rgb]{0,0,0}
% \color[rgb]{1,1,1}



%\DeclareFixedFont{\myfont}{U}{cmss}{bx}{n}{18pt}
\newcommand{\debuttexte}{
%%%%%%%%%%%%% FONTES %%%%%%%%%%%%%
\renewcommand{\baselinestretch}{1.5}
\usefont{U}{cmss}{bx}{n}
\bfseries

% Taille normale : commenter le reste !
%Taille Arnaud
%\fontsize{19}{19}\selectfont

% Taille Barbara
%\fontsize{21}{22}\selectfont

%Taille François
\fontsize{25}{30}\selectfont

%Taille Pascal
%\fontsize{25}{30}\selectfont

%Taille Laura
%\fontsize{30}{35}\selectfont


%\myfont
%\usefont{U}{cmss}{bx}{n}

%\Huge
%\addtolength{\parskip}{\baselineskip}
}


% \usepackage{hyperref}
% \hypersetup{colorlinks=true, linkcolor=blue, urlcolor=blue,
% pdftitle={Exo7 - Exercices de mathématiques}, pdfauthor={Exo7}}


%section
% \usepackage{sectsty}
% \allsectionsfont{\bf}
%\sectionfont{\color{Tomato3}\upshape\selectfont}
%\subsectionfont{\color{Tomato4}\upshape\selectfont}

%----- Ensembles : entiers, reels, complexes -----
\newcommand{\Nn}{\mathbb{N}} \newcommand{\N}{\mathbb{N}}
\newcommand{\Zz}{\mathbb{Z}} \newcommand{\Z}{\mathbb{Z}}
\newcommand{\Qq}{\mathbb{Q}} \newcommand{\Q}{\mathbb{Q}}
\newcommand{\Rr}{\mathbb{R}} \newcommand{\R}{\mathbb{R}}
\newcommand{\Cc}{\mathbb{C}} 
\newcommand{\Kk}{\mathbb{K}} \newcommand{\K}{\mathbb{K}}

%----- Modifications de symboles -----
\renewcommand{\epsilon}{\varepsilon}
\renewcommand{\Re}{\mathop{\text{Re}}\nolimits}
\renewcommand{\Im}{\mathop{\text{Im}}\nolimits}
%\newcommand{\llbracket}{\left[\kern-0.15em\left[}
%\newcommand{\rrbracket}{\right]\kern-0.15em\right]}

\renewcommand{\ge}{\geqslant}
\renewcommand{\geq}{\geqslant}
\renewcommand{\le}{\leqslant}
\renewcommand{\leq}{\leqslant}

%----- Fonctions usuelles -----
\newcommand{\ch}{\mathop{\mathrm{ch}}\nolimits}
\newcommand{\sh}{\mathop{\mathrm{sh}}\nolimits}
\renewcommand{\tanh}{\mathop{\mathrm{th}}\nolimits}
\newcommand{\cotan}{\mathop{\mathrm{cotan}}\nolimits}
\newcommand{\Arcsin}{\mathop{\mathrm{Arcsin}}\nolimits}
\newcommand{\Arccos}{\mathop{\mathrm{Arccos}}\nolimits}
\newcommand{\Arctan}{\mathop{\mathrm{Arctan}}\nolimits}
\newcommand{\Argsh}{\mathop{\mathrm{Argsh}}\nolimits}
\newcommand{\Argch}{\mathop{\mathrm{Argch}}\nolimits}
\newcommand{\Argth}{\mathop{\mathrm{Argth}}\nolimits}
\newcommand{\pgcd}{\mathop{\mathrm{pgcd}}\nolimits} 

\newcommand{\Card}{\mathop{\text{Card}}\nolimits}
\newcommand{\Ker}{\mathop{\text{Ker}}\nolimits}
\newcommand{\id}{\mathop{\text{id}}\nolimits}
\newcommand{\ii}{\mathrm{i}}
\newcommand{\dd}{\mathrm{d}}
\newcommand{\Vect}{\mathop{\text{Vect}}\nolimits}
\newcommand{\Mat}{\mathop{\mathrm{Mat}}\nolimits}
\newcommand{\rg}{\mathop{\text{rg}}\nolimits}
\newcommand{\tr}{\mathop{\text{tr}}\nolimits}
\newcommand{\ppcm}{\mathop{\text{ppcm}}\nolimits}

%----- Structure des exercices ------

\newtheoremstyle{styleexo}% name
{2ex}% Space above
{3ex}% Space below
{}% Body font
{}% Indent amount 1
{\bfseries} % Theorem head font
{}% Punctuation after theorem head
{\newline}% Space after theorem head 2
{}% Theorem head spec (can be left empty, meaning ‘normal’)

%\theoremstyle{styleexo}
\newtheorem{exo}{Exercice}
\newtheorem{ind}{Indications}
\newtheorem{cor}{Correction}


\newcommand{\exercice}[1]{} \newcommand{\finexercice}{}
%\newcommand{\exercice}[1]{{\tiny\texttt{#1}}\vspace{-2ex}} % pour afficher le numero absolu, l'auteur...
\newcommand{\enonce}{\begin{exo}} \newcommand{\finenonce}{\end{exo}}
\newcommand{\indication}{\begin{ind}} \newcommand{\finindication}{\end{ind}}
\newcommand{\correction}{\begin{cor}} \newcommand{\fincorrection}{\end{cor}}

\newcommand{\noindication}{\stepcounter{ind}}
\newcommand{\nocorrection}{\stepcounter{cor}}

\newcommand{\fiche}[1]{} \newcommand{\finfiche}{}
\newcommand{\titre}[1]{\centerline{\large \bf #1}}
\newcommand{\addcommand}[1]{}
\newcommand{\video}[1]{}

% Marge
\newcommand{\mymargin}[1]{\marginpar{{\small #1}}}



%----- Presentation ------
\setlength{\parindent}{0cm}

%\newcommand{\ExoSept}{\href{http://exo7.emath.fr}{\textbf{\textsf{Exo7}}}}

\definecolor{myred}{rgb}{0.93,0.26,0}
\definecolor{myorange}{rgb}{0.97,0.58,0}
\definecolor{myyellow}{rgb}{1,0.86,0}

\newcommand{\LogoExoSept}[1]{  % input : echelle
{\usefont{U}{cmss}{bx}{n}
\begin{tikzpicture}[scale=0.1*#1,transform shape]
  \fill[color=myorange] (0,0)--(4,0)--(4,-4)--(0,-4)--cycle;
  \fill[color=myred] (0,0)--(0,3)--(-3,3)--(-3,0)--cycle;
  \fill[color=myyellow] (4,0)--(7,4)--(3,7)--(0,3)--cycle;
  \node[scale=5] at (3.5,3.5) {Exo7};
\end{tikzpicture}}
}



\theoremstyle{definition}
%\newtheorem{proposition}{Proposition}
%\newtheorem{exemple}{Exemple}
%\newtheorem{theoreme}{Théorème}
\newtheorem{lemme}{Lemme}
\newtheorem{corollaire}{Corollaire}
%\newtheorem*{remarque*}{Remarque}
%\newtheorem*{miniexercice}{Mini-exercices}
%\newtheorem{definition}{Définition}




%definition d'un terme
\newcommand{\defi}[1]{{\color{myorange}\textbf{\emph{#1}}}}
\newcommand{\evidence}[1]{{\color{blue}\textbf{\emph{#1}}}}



 %----- Commandes divers ------

\newcommand{\codeinline}[1]{\texttt{#1}}

%%%%%%%%%%%%%%%%%%%%%%%%%%%%%%%%%%%%%%%%%%%%%%%%%%%%%%%%%%%%%
%%%%%%%%%%%%%%%%%%%%%%%%%%%%%%%%%%%%%%%%%%%%%%%%%%%%%%%%%%%%%

\begin{document}

\debuttexte

%%%%%%%%%%%%%%%%%%%%%%%%%%%%%%%%%%%%%%%%%%%%%%%%%%%%%%%%%%%
\diapo

\change

 Nous aborderons quatre thèmes dans cette leçon.

\change

 1. Nous définirons ce que ``diviser'' veut dire pour des entiers, 
et énoncerons le théorème de la division euclidienne.

\change

 2. Nous définirons le pgcd de deux entiers.

\change

 3. Nous verrons comment calculer efficacement ce pgcd avec l'algorithme d'Euclide.

\change 

 4. Nous terminerons pas la notion de nombres premiers entres eux.


%%%%%%%%%%%%%%%%%%%%%%%%%%%%%%%%%%%%%%%%%%%%%%%%%%%%%%%%%%%
\diapo

Commençons par une motivation. [[rythme rapide]]

L'arithmétique est le coeur du système de cryptage le plus utilisé pour
les communications, en particulier sur internet.

Ce principe repose uniquement sur des connaissances que vous acquerrez dans ce chapitre.

Voyons quelles sont ces notions.

\change

Nous avons besoin de choisir deux nombres premiers (qui vous devez garder secrets).
Ce sont généralement des entiers ayant des centaines de chiffres. 
On calcule alors le produit $n=p\times q$. 

\change

Alors même si $n$ est rendu public, il est très difficile de retrouver les facteurs $p$ et $q$.


\change

Vous produisez ensuite deux nombres : un clé secrète et une clé publique à l'aide de l'algorithme d'Euclide et des
coefficients de Bézout.

\change

Pour crypter un message, votre interlocuteur transforme son message en nombre et fait des calculs 
avec la clé publique, à l'aide de congruences modulo $n$.

\change

Vous seul pouvez décrypter le message avec la clé secrète et grâce au petit théorème de Fermat.



%%%%%%%%%%%%%%%%%%%%%%%%%%%%%%%%%%%%%%%%%%%%%%%%%%%%%%%%%%%
\diapo


Prenons deux entiers $a$ et $b$.

On dit que $b$ divise $a$ s'il existe un entier $q$

tel que $a = b \times q$

\change

On note alors $b$ divise $a$ ainsi.

\change

Par exemple  $7 | 21$ 

car en prenant $q=3$ on a bien $21 = 7 \times 3$.

\change

$6 | 48$  

\change

ou encore $a$ est pair si et seulement si $2|a$

\change

Il y a deux cas spéciaux. Tout entier  divise $0$ (prenez $q=0$)

\change 

et $1$ divise tout entier.

\change

Les seuls entiers qui divisent $1$ sont $1$ lui-même et $-1$.

\change

Terminons par quelques propriétés faciles à démontrer :
si $(a|b \text{ et } b|a)$ alors $b= +a$ ou $b=-a$.

\change

 si $a|b$  et  $b|c$ alors $a$ divise aussi $c$ 

\change

si $a$ divise à la fois $b$ et $c$ alors $a$ divise la somme $b+c$
(et bien sûr aussi le produit $b\times c$).


%%%%%%%%%%%%%%%%%%%%%%%%%%%%%%%%%%%%%%%%%%%%%%%%%%%%%%%%%%%
\diapo

Bien sûr pour des entiers $a,b$ quelconques il n'est pas toujours vrai que $b$ divise $a$.
Cependant on peut trouver une écriture qui s'en rapproche : c'est la division euclidienne.

Prenons donc $a,b$ deux entiers quelconques alors il existe un entier $q$ et entier $r$
tel que $a=bq+r$. 

\change

De plus on peut choisir $q$ de sorte que le reste $r$ soit compris entre 
$0$ (au sens large) et $b$ (au sens stricte).

\change

Nous seulement on peut trouver $q$ et $r$ remplissant ces deux conditions
mais en plus les $q$ et $r$ que l'on trouve sont uniques.

\change


Nous avons l'existence et l'unicité.


Bien sûr $b$ divise $a$ si et seulement si le reste $r$ est nul.

\change

La division euclidienne est la division que vous avez appris à poser
à l'école primaire.

Pour $a=6789$ et $b=34$ la division euclidienne s'écrit
 $6789= 34 \times 199 + 23$

Ainsi $q=199$ et $r=23$, le reste $23$ est bien strictement plus petit que $34$.

\change

Et je vous laisse poser la division et vérifier les calculs.

\change

Voici les termes utilisés :

 - $a$ s'appelle le dividende,

 - $b$ s'appelle le diviseur,

 - $q$ est le quotient,

 - $r$ est le reste.

Le reste doit être strictement plus petit que le diviseur, 

sinon c'est que vous n'avez pas été assez
loin dans la division !



%%%%%%%%%%%%%%%%%%%%%%%%%%%%%%%%%%%%%%%%%%%%%%%%%%%%%%%%%%%
\diapo

Etant donnés deux entiers $a$ et $b$.

Le plus grand diviseur commun de $a$ et $b$ est le plus grand entier qui divise à la fois $a$ et
$b$.

\change

Il se note $\pgcd(a,b)$

\change

Voyons quelques exemples.

le $\pgcd(21,14)$ est $7$, en effet $7$ divise à la fois $21$ et $14$,
et aucun entier plus grand ne divise $21$ et $14$.

\change

 $\pgcd(12,32)=4$ 

\change

 $\pgcd(21,26)=1$

C'est-à-dire qu'aucun entier plus grand que $1$ ne divise $21$ et $26$.

\change

le pgcd de $a$ et d'un multiple de $a$ est $a$ lui-même.

\change

Enfin deux situations spéciales ou l'on connaît le pgcd :

le  $\pgcd(a,0)=a$ 

\change

et le $\pgcd(a,1)=1$


%%%%%%%%%%%%%%%%%%%%%%%%%%%%%%%%%%%%%%%%%%%%%%%%%%%%%%%%%%%
\diapo

Nous allons mettre en place une méthode pour calculer le pgcd : l'algorithme d'Euclide.

Pour cela nous avons besoin d'un résultat préliminaire.

Soient $a$ et $b$ deux entiers. 
\'Ecrivons la division euclidienne $a=bq+r$

Alors $\pgcd(a,b)=\pgcd(b,r)$

Le pgcd de $a$ et de $b$ est donc égal au pgcd 
de $b$ avec le reste de la division euclidienne de $a$ par $b$.

\change


En fait on a même $\pgcd(a,b) = \pgcd(b,a-qb)$ pour tout $q\in \Zz$.


Mais pour optimiser l'algorithme d'Euclide on applique le lemme avec $q$ le quotient.

\change

Effectuons la preuve.

Si $d$ est un diviseur de $a$ et de $b$. Alors $d$ divise $bq$ et $a$ donc aussi $bq-a$ donc divise $r$.

\change

Si $d$ un diviseur de $b$ et de $r$. Alors $d$ divise aussi $bq+r=a$.

\change

Les diviseurs communs à $a$ et $b$ sont exactement les mêmes que les diviseurs communs à $b$ et $r$.

En particulier les plus grand diviseurs communs sont égaux.



%%%%%%%%%%%%%%%%%%%%%%%%%%%%%%%%%%%%%%%%%%%%%%%%%%%%%%%%%%%
\diapo

Nous sommes prêts pour calculer le pgcd de deux entiers avec l'algorithme d'Euclide.

\change

Nous allons calculer successivement des division euclidienne.

\change

En utilisant le lemme précédent le pgcd sera le dernier 
reste non nul.


\change

Voyons comment fonctionne l'algorithme en calculant le pgcd de $600$ et $124$.

\change

On écrit la division euclidienne de $600$ par $124$ :

$600=124 \times 4 + 104$.


\change

L'étape suivante est la division euclidienne du diviseur par le reste de la ligne du dessus.



$124$ était le diviseur, $104$ le reste, on écrit donc la division euclidienne de $124$ par $104$ :

\change

$124$ = $104 \times 1 + 20$.

\change

On continue en écrivant la division euclidienne de $104$ (notre diviseur précédent)
par $20$ (notre reste précédent)

\change

$104= 20 \times 5 + 4$.

\change

Enfin la division euclidienne du diviseur précédent $20$ par le reste précédent $4$
est

$20 = 4 \times 5 + 0$.

\change

Nous obtenons un reste nul, et le pgcd est le dernier reste non nul, c'est donc $4$.

\change

(pause)
Justifions pourquoi le dernier reste non nul est le pgcd,

par le lemme vu précédemment
le pgcd de $600$ et $124$ égale le pgcd de $124$, $104$

Mais par le même lemme appliqué à la deuxième division euclidienne le pgcd de $124$ et $104$ 
égal au pgcd de $104$, $20$,

qui à son tour égale le pgcd de $20$ et $4$

qui égale au pgcd de $4$ et $0$ et donc vaut $4$ !


%%%%%%%%%%%%%%%%%%%%%%%%%%%%%%%%%%%%%%%%%%%%%%%%%%%%%%%%%%%
\diapo

C'est un algorithme donc vous devez arrivez au bon résultat en un nombre fini d'étape.

Voyons un exemple un chouïa plus compliqué
avec le pgcd de $9945$ et de $3003$.

\change

On calcule la division euclidienne de $9945$ par $3003$.

\change

Puis celle du diviseur précédent par le reste précédent, 
c'est-à-dire la division euclidienne de
 $3003$ par $936$.

\change

Puis la division de $936$ par $195$.

\change


Puis $195$ diviser par $156$

\change

Et enfin pour $156$ diviser par $39$ on obtient un reste nul.

\change

Le dernier reste non nul est donc $39$...

\change

et donc  le pgcd de $9945$ et  $3003$ vaut $39$.


%%%%%%%%%%%%%%%%%%%%%%%%%%%%%%%%%%%%%%%%%%%%%%%%%%%%%%%%%%%
\diapo

Nous dirons que $a$ et $b$ sont premiers entre eux si leur pgcd vaut $1$.

\change

Voici un petit exercice le pgcd de deux nombres consécutif $a$ et $a+1$ est toujours
égal à $1$.

\change

En effet soit $d$ un diviseur commun à $a$ et à $a+1$

\change

alors 

$d|a \text{ et } d|a+1$

\change

donc $d$ divise la différence $a+1-a$

\change

qui vaut $1$, donc $d$ divise $1$.

\change

Mais les seuls diviseurs de $1$ sont $+1$ ou $-1$.


\change

Et donc le plus grand entier qui divise à la fois $a$ et $a+1$ est $+1$.

Ainsi $a$ et $a+1$ sont premier entre eux.

Par exemple $14$ et $15$ sont premiers entre eux, $15$ et $16$ aussi, etc.

\change


Si deux entiers ne sont pas premiers entre eux, on peut s'y ramener en divisant par leur pgcd.

Plus précisément, étant donnés deux entiers $a,b \in \Zz$ quelconques, notons $d$ leur pgcd.


Alors la décomposition suivante est souvent utile :

$d$ divise $a$ donc on peut écrire $a=a'd$,

$d$ divise aussi $b$ donc on peut écrire $b=b'd$.

Bien sûr $a'$ et $b'$ sont des entiers mais en plus comme on a diviser par le pgcd

alors $a'$ et $b'$ sont premiers entre eux.


%%%%%%%%%%%%%%%%%%%%%%%%%%%%%%%%%%%%%%%%%%%%%%%%%%%%%%%%%%%
\diapo

Mettez en pratique les concepts de cette leçon pour les assimiler.



\end{document}
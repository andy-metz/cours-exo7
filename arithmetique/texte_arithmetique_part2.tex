
%%%%%%%%%%%%%%%%%% PREAMBULE %%%%%%%%%%%%%%%%%%


\documentclass[12pt]{article}

\usepackage{amsfonts,amsmath,amssymb,amsthm}
\usepackage[utf8]{inputenc}
\usepackage[T1]{fontenc}
\usepackage[francais]{babel}


% packages
\usepackage{amsfonts,amsmath,amssymb,amsthm}
\usepackage[utf8]{inputenc}
\usepackage[T1]{fontenc}
%\usepackage{lmodern}

\usepackage[francais]{babel}
\usepackage{fancybox}
\usepackage{graphicx}

\usepackage{float}

%\usepackage[usenames, x11names]{xcolor}
\usepackage{tikz}
\usepackage{datetime}

\usepackage{mathptmx}
%\usepackage{fouriernc}
%\usepackage{newcent}
\usepackage[mathcal,mathbf]{euler}

%\usepackage{palatino}
%\usepackage{newcent}


% Commande spéciale prompteur

%\usepackage{mathptmx}
%\usepackage[mathcal,mathbf]{euler}
%\usepackage{mathpple,multido}

\usepackage[a4paper]{geometry}
\geometry{top=2cm, bottom=2cm, left=1cm, right=1cm, marginparsep=1cm}

\newcommand{\change}{{\color{red}\rule{\textwidth}{1mm}\\}}

\newcounter{mydiapo}

\newcommand{\diapo}{\newpage
\hfill {\normalsize  Diapo \themydiapo \quad \texttt{[\jobname]}} \\
\stepcounter{mydiapo}}


%%%%%%% COULEURS %%%%%%%%%%

% Pour blanc sur noir :
%\pagecolor[rgb]{0.5,0.5,0.5}
% \pagecolor[rgb]{0,0,0}
% \color[rgb]{1,1,1}



%\DeclareFixedFont{\myfont}{U}{cmss}{bx}{n}{18pt}
\newcommand{\debuttexte}{
%%%%%%%%%%%%% FONTES %%%%%%%%%%%%%
\renewcommand{\baselinestretch}{1.5}
\usefont{U}{cmss}{bx}{n}
\bfseries

% Taille normale : commenter le reste !
%Taille Arnaud
%\fontsize{19}{19}\selectfont

% Taille Barbara
%\fontsize{21}{22}\selectfont

%Taille François
\fontsize{25}{30}\selectfont

%Taille Pascal
%\fontsize{25}{30}\selectfont

%Taille Laura
%\fontsize{30}{35}\selectfont


%\myfont
%\usefont{U}{cmss}{bx}{n}

%\Huge
%\addtolength{\parskip}{\baselineskip}
}


% \usepackage{hyperref}
% \hypersetup{colorlinks=true, linkcolor=blue, urlcolor=blue,
% pdftitle={Exo7 - Exercices de mathématiques}, pdfauthor={Exo7}}


%section
% \usepackage{sectsty}
% \allsectionsfont{\bf}
%\sectionfont{\color{Tomato3}\upshape\selectfont}
%\subsectionfont{\color{Tomato4}\upshape\selectfont}

%----- Ensembles : entiers, reels, complexes -----
\newcommand{\Nn}{\mathbb{N}} \newcommand{\N}{\mathbb{N}}
\newcommand{\Zz}{\mathbb{Z}} \newcommand{\Z}{\mathbb{Z}}
\newcommand{\Qq}{\mathbb{Q}} \newcommand{\Q}{\mathbb{Q}}
\newcommand{\Rr}{\mathbb{R}} \newcommand{\R}{\mathbb{R}}
\newcommand{\Cc}{\mathbb{C}} 
\newcommand{\Kk}{\mathbb{K}} \newcommand{\K}{\mathbb{K}}

%----- Modifications de symboles -----
\renewcommand{\epsilon}{\varepsilon}
\renewcommand{\Re}{\mathop{\text{Re}}\nolimits}
\renewcommand{\Im}{\mathop{\text{Im}}\nolimits}
%\newcommand{\llbracket}{\left[\kern-0.15em\left[}
%\newcommand{\rrbracket}{\right]\kern-0.15em\right]}

\renewcommand{\ge}{\geqslant}
\renewcommand{\geq}{\geqslant}
\renewcommand{\le}{\leqslant}
\renewcommand{\leq}{\leqslant}

%----- Fonctions usuelles -----
\newcommand{\ch}{\mathop{\mathrm{ch}}\nolimits}
\newcommand{\sh}{\mathop{\mathrm{sh}}\nolimits}
\renewcommand{\tanh}{\mathop{\mathrm{th}}\nolimits}
\newcommand{\cotan}{\mathop{\mathrm{cotan}}\nolimits}
\newcommand{\Arcsin}{\mathop{\mathrm{Arcsin}}\nolimits}
\newcommand{\Arccos}{\mathop{\mathrm{Arccos}}\nolimits}
\newcommand{\Arctan}{\mathop{\mathrm{Arctan}}\nolimits}
\newcommand{\Argsh}{\mathop{\mathrm{Argsh}}\nolimits}
\newcommand{\Argch}{\mathop{\mathrm{Argch}}\nolimits}
\newcommand{\Argth}{\mathop{\mathrm{Argth}}\nolimits}
\newcommand{\pgcd}{\mathop{\mathrm{pgcd}}\nolimits} 

\newcommand{\Card}{\mathop{\text{Card}}\nolimits}
\newcommand{\Ker}{\mathop{\text{Ker}}\nolimits}
\newcommand{\id}{\mathop{\text{id}}\nolimits}
\newcommand{\ii}{\mathrm{i}}
\newcommand{\dd}{\mathrm{d}}
\newcommand{\Vect}{\mathop{\text{Vect}}\nolimits}
\newcommand{\Mat}{\mathop{\mathrm{Mat}}\nolimits}
\newcommand{\rg}{\mathop{\text{rg}}\nolimits}
\newcommand{\tr}{\mathop{\text{tr}}\nolimits}
\newcommand{\ppcm}{\mathop{\text{ppcm}}\nolimits}

%----- Structure des exercices ------

\newtheoremstyle{styleexo}% name
{2ex}% Space above
{3ex}% Space below
{}% Body font
{}% Indent amount 1
{\bfseries} % Theorem head font
{}% Punctuation after theorem head
{\newline}% Space after theorem head 2
{}% Theorem head spec (can be left empty, meaning ‘normal’)

%\theoremstyle{styleexo}
\newtheorem{exo}{Exercice}
\newtheorem{ind}{Indications}
\newtheorem{cor}{Correction}


\newcommand{\exercice}[1]{} \newcommand{\finexercice}{}
%\newcommand{\exercice}[1]{{\tiny\texttt{#1}}\vspace{-2ex}} % pour afficher le numero absolu, l'auteur...
\newcommand{\enonce}{\begin{exo}} \newcommand{\finenonce}{\end{exo}}
\newcommand{\indication}{\begin{ind}} \newcommand{\finindication}{\end{ind}}
\newcommand{\correction}{\begin{cor}} \newcommand{\fincorrection}{\end{cor}}

\newcommand{\noindication}{\stepcounter{ind}}
\newcommand{\nocorrection}{\stepcounter{cor}}

\newcommand{\fiche}[1]{} \newcommand{\finfiche}{}
\newcommand{\titre}[1]{\centerline{\large \bf #1}}
\newcommand{\addcommand}[1]{}
\newcommand{\video}[1]{}

% Marge
\newcommand{\mymargin}[1]{\marginpar{{\small #1}}}



%----- Presentation ------
\setlength{\parindent}{0cm}

%\newcommand{\ExoSept}{\href{http://exo7.emath.fr}{\textbf{\textsf{Exo7}}}}

\definecolor{myred}{rgb}{0.93,0.26,0}
\definecolor{myorange}{rgb}{0.97,0.58,0}
\definecolor{myyellow}{rgb}{1,0.86,0}

\newcommand{\LogoExoSept}[1]{  % input : echelle
{\usefont{U}{cmss}{bx}{n}
\begin{tikzpicture}[scale=0.1*#1,transform shape]
  \fill[color=myorange] (0,0)--(4,0)--(4,-4)--(0,-4)--cycle;
  \fill[color=myred] (0,0)--(0,3)--(-3,3)--(-3,0)--cycle;
  \fill[color=myyellow] (4,0)--(7,4)--(3,7)--(0,3)--cycle;
  \node[scale=5] at (3.5,3.5) {Exo7};
\end{tikzpicture}}
}



\theoremstyle{definition}
%\newtheorem{proposition}{Proposition}
%\newtheorem{exemple}{Exemple}
%\newtheorem{theoreme}{Théorème}
\newtheorem{lemme}{Lemme}
\newtheorem{corollaire}{Corollaire}
%\newtheorem*{remarque*}{Remarque}
%\newtheorem*{miniexercice}{Mini-exercices}
%\newtheorem{definition}{Définition}




%definition d'un terme
\newcommand{\defi}[1]{{\color{myorange}\textbf{\emph{#1}}}}
\newcommand{\evidence}[1]{{\color{blue}\textbf{\emph{#1}}}}



 %----- Commandes divers ------

\newcommand{\codeinline}[1]{\texttt{#1}}

%%%%%%%%%%%%%%%%%%%%%%%%%%%%%%%%%%%%%%%%%%%%%%%%%%%%%%%%%%%%%
%%%%%%%%%%%%%%%%%%%%%%%%%%%%%%%%%%%%%%%%%%%%%%%%%%%%%%%%%%%%%

\begin{document}

\debuttexte



%%%%%%%%%%%%%%%%%%%%%%%%%%%%%%%%%%%%%%%%%%%%%%%%%%%%%%%%%%%
\diapo

\change

Nous continuons l'arithmétique avec une leçon sur le théorème de Bézout et
ses conséquences.

\change

Nous énonçons d'abord le théorème de Bézout

\change

nous continuons avec plusieurs applications importantes

\change 

nous résolvons ensuite les équations d'entiers du type
$ax+by=c$

\change

Nous terminons par la notion de ppcm.

%%%%%%%%%%%%%%%%%%%%%%%%%%%%%%%%%%%%%%%%%%%%%%%%%%%%%%%%%%%
\diapo

Voici le théorème de Bézout. Fixons deux entiers $a$ et $b$ quelconques.

Alors il existe deux entiers $u,v$  tels que 
$au+bv=\pgcd(a,b)$

\change

Ces entiers $u$ et $v$ s'appellent *des* coefficients de Bézout.

La preuve de ce théorème est une conséquence de l'algorithme d'Euclide.

En effet $u$ et $v$ s'obtiennent en ``remontant'' les équations obtenues
lors de l'algorithme d'Euclide.

Enfin, nous le verrons plus tard, il existe une infinité de couples $(u,v)$ qui conviennent.


%%%%%%%%%%%%%%%%%%%%%%%%%%%%%%%%%%%%%%%%%%%%%%%%%%%%%%%%%%%
\diapo

Voyons comment trouver une solution $(u,v)$ en nous aidant de l'algorithme d'Euclide.

Prenons $a=600$ et $b=124$.

\change

Nous avons déjà vu comment l'algorithme d'Euclide nous permettait de calculer le pgcd.

Reprenons très rapidement. 

On écrit la division euclidienne de $600$ par $124$, le reste est $104$.


Puis on effectue la division euclidienne de $124$ par le reste $104$,
le nouveau reste est $20$.

La division de $104$ par $20$ donne un reste de $4$.

Et enfin la division de $20$ par $4$ a un reste nul.

Le pgcd est le dernier reste non nul c'est donc $4$.

\change

Nous cherchons donc $u$ et $v$ tel que $au+bv=\pgcd(a,b)$
donc ici on cherche des entiers $u,v$ tels que $600u+124v=4$.


Pour cela nous allons partir de la ligne ou figure le pgcd et
substituer des coefficients en remontant ligne par ligne.

\change

Partons de la ligne ou figure le pgcd
$104 = 20 \times 5 + 4$

cela permet d'écrire le pgcd $4$ sous la forme
$4 = 104 - 20\times 5$.

\change

Le $20$ qui apparaît ici était le reste de la ligne au-dessus
on peut donc le remplacer par $124-104\times 1$.

\change

On récrit proprement $4$ en fonction $124$ et $104$.
$4=124\times (-5) + 104\times 6$.

\change

Ce n'est pas fini : $104$ était le reste encore au-dessus, ce qui permet
de remplacer $104$ par $600-124 \times 4$.



\change

Pour voir que $4$ s'écrit maintenant en fonction de $600$ et $124$, 
on regroupe les termes et on obtient :

$4=600 \times 6 + 124 \times (-29)$.


\change


On a bien trouver les coefficients de Bézout $u=6$ et $v=-29$ qui vérifient  
$600 u+124v=4$.


Pour assurer vos calculs prenez quelques secondes pour vérifier 
que  $600 \times 6 + 124 \times (-29)$ vaut bien $4$.

%%%%%%%%%%%%%%%%%%%%%%%%%%%%%%%%%%%%%%%%%%%%%%%%%%%%%%%%%%%
\diapo

Voyons un deuxième exemple.

Dans la leçons précédente nous avons vue comment calculer
le pgcd de $9945$ et $3003$ avec l'algorithme d'Euclide et nous avons calculé que ce pgcd valait $39$.

L'équation $9945u+3003v=39$ a donc des solutions.

\change

Pour trouver $u$ et $v$ on reprend les calculs effectués avec l'algorithme d'Euclide
pour calculer le pgcd. 

Le pgcd est le dernier reste non nul, c'est bien $39$.


\change

Maintenant cherchons $u$ et $v$ en remontant ces calculs.

\change

On part de l'égalité $39   =  195- 156$ donnée par l'avant-dernière ligne.

\change

A l'aide de la ligne du dessus on remplace $156$ par $936 - 4\times 195$ 

\change 


Jusqu'à écrire $39$ à l'aide de nos deux nombres $9945$ et $3003$.

Complétez les calculs manquants.

\change

Vous devez obtenir 

$9945 \times (-16) +  3003\times 53 = 39$

Et prenez quelques instants pour vérifier qu'effectivement ceci
est bien égal à $39$.

%%%%%%%%%%%%%%%%%%%%%%%%%%%%%%%%%%%%%%%%%%%%%%%%%%%%%%%%%%%
\diapo

Voici une première conséquence du théorème.

$a,b$ sont premiers entre eux 
si et seulement si il existe $u,v$ tels que
$au+bv=1$

On appelle souvent aussi ce résultat le théorème de Bézout.

C'est cette forme qui est la plus utile dans la pratique.

Une grande différence par rapport à l'énoncé du théorème de Bézout précédent,
c'est qu'ici nous avons l'équivalence.

Si $pgcd(a,b)=1$ alors on peut trouver des coefficients $u$, $v$ tels que $au+bv$ égal le pgcd donc $1$.

Réciproquement s'il existe des entiers $u,v$ tel que $au+bv=1$ alors le pgcd de $a$, $b$ vaut $1$.


\change

Passons à la preuve : le sens direct est le théorème de Bézout car
$a$ et $b$ premiers entre eux signifie $\pgcd(a,b)=1$.

\change

Montrons la réciproque. On suppose qu'il existe $u,v$ tels que $au+bv=1$

Mais le  $\pgcd(a,b)|a$ donc $\pgcd(a,b)$ divise aussi $au$

De même $\pgcd(a,b)|bv$

Donc $\pgcd(a,b)$ divise la somme $au+bv$ qui vaut $1$. Mais si  $\pgcd(a,b)|1$
c'est que $\pgcd(a,b)=1$. Comme nous le souhaitions.

\change

Dans ce corollaire nous avons une équivalence mais pas dans le théorème de Bézout 
vu auparavant. 

Si on trouve $u',v'$ tels que $au'+bv'=d$, 
alors on ne peut pas dire que le $\pgcd(a,b)$ égal $d$ 

on peut juste affirmer $\pgcd(a,b)$ divise $d$.

\change 

Retenez cet exemple si : $a=12$, $b=8$
on peut écrire par exemple  $12 \times 1 + 8 \times 3 = 36$ 

mais bien sûr ce n'est pas $36$ le $\pgcd(12,8)$.

Par contre le pgcd qui vaut $4$ divise bien $36$.


%%%%%%%%%%%%%%%%%%%%%%%%%%%%%%%%%%%%%%%%%%%%%%%%%%%%%%%%%%%
\diapo

Voici encore deux corollaires importants,
je vous encourage à chercher leur démonstration.


Tout d'abord si un entier  $d$ divise $a$ et divise aussi $b$ alors $d$ divise le pgcd de $a$ et $b$.

\change

Par exemple $4|16$ et $4|24$ donc $4$ doit diviser $pgcd(16,24)$ qui effectivement vaut $8$.


\change

Deuxième corollaire, qui s'appelle le lemme de Gauss.

Si $a$ divise le produit $b\times c$ et que $a$ et $b$ sont premiers entre eux alors
nécessairement $a$ divise $c$.

\change

Exemple : si $4 | 7\times c$, alors comme $4$ et $7$ sont premiers entre eux on a que $4|c$.


%%%%%%%%%%%%%%%%%%%%%%%%%%%%%%%%%%%%%%%%%%%%%%%%%%%%%%%%%%%
\diapo

Nous allons maintenant pouvoir résoudre toutes les équations diophantiennes
du type $ax+by=c$. 

Les nombres $a,b, c$ sont donnés ce sont trois entiers.

Les inconnues sont $x,y$, et elles aussi sont des entiers.

\change

Tout d'abord le premier point nous fournit un critère simple pour savoir s'il y a des solutions
 ou s'il n'y en a pas.

Cette équation possède des solutions $x,y$ entières si et seulement si
le $\pgcd(a,b)$ divise $c$.

\change

Deuxième point : si $\pgcd(a,b)$ divise $c$ alors il y a des solutions et il y en même une infinité
que l'on sait calculer. 

Les $x$ sont de la forme $x_0+\alpha k$,

les  $y$ de la forme $y_0+\beta k$ 

avec $k$ parcourant l'ensemble des entiers 

et  $x_0,y_0,\alpha,\beta$ sont des entiers que l'on va calculer en fonction de $a,b,c$.


Nous n'allons pas démontrer cette proposition ici, mais plutôt détailler
comment résoudre complètement chaque équation particulière.

%%%%%%%%%%%%%%%%%%%%%%%%%%%%%%%%%%%%%%%%%%%%%%%%%%%%%%%%%%%
\diapo


Voyons sur un exemple comment résoudre les équation du type 
$ax+by=c$.

Nous cherchons tous les couple $(x,y)$ solution de l'équation $161 x + 368 y=115$

Bien sûr nous cherchons $x$ et $y$ entiers. 

Nous allons résoudre ce problème en trois étapes :

  -1- y-a-t'il des solutions ?

  -2- si oui alors on trouve une solution particulière.

  -3- on en déduit toutes les solutions.
  

\change

Commençons par déterminer s'il existe des solutions.

\change

C'est le théorème de Bézout qui apporte la réponse :
si le pgcd de $a$ et $b$ divise $c$ alors il existe des solutions.

Nous allons donc calculer le pgcd de $161$ et $368$ avec l'algorithme d'Euclide.

\change

On écrit les divisions euclidiennes :
 
$368  =  161  \times  2  +  46$

\change

$161  =  46  \times  3  +  23$

\change

$46   =  23    \times  2$

\change

Le pgcd de $161$ et $368$ est donc $23$.

\change

Est-ce que le pgcd  divise $c$. Oui car $115=5 \times 23$.

\change 

Donc par le théorème de Bézout, il existe des solutions entières à
l'équation $161 x + 368 y=115$.

%%%%%%%%%%%%%%%%%%%%%%%%%%%%%%%%%%%%%%%%%%%%%%%%%%%%%%%%%%%
\diapo

Jusqu'ici nous savons qu'il existe des solutions de
notre équation, 

nous allons en expliciter une : 

 cette solution particulière correspond aux coefficients de Bézout 
que l'on obtient en remontant l'algorithme d'Euclide.

\change

Reprenons les calculs qui ont conduit au pgcd.

\change

L'égalité
$161  =  46  \times  3  +  23$

permet d'écrire le pgcd $23$ sous la forme

$23=161+46\times(-3)$.

\change

Ensuite on remplace $46$ 
par $368 - 2\times 161$.

\change 

On simplifie

\change

et nous obtenons $161\times7 + 368\times(-3)=23$.

Mais ceci n'est pas tout à fait une solution de notre équation,

Car nous voulons trouver $x$ et $y$ tels que 
$161 x + 368 y=115$ et pas $=23$.

\change

Mais $115 = 5\times 23$ donc en multipliant par $5$ nous obtenons

\change

$$161\times35 + 368\times(-15)=115$$

\change

Voilà une solution particulière de notre équation  :
le couple $(x_0,y_0) = (35,-15)$.


%%%%%%%%%%%%%%%%%%%%%%%%%%%%%%%%%%%%%%%%%%%%%%%%%%%%%%%%%%%
\diapo

Nous savons qu'il existe des solutions,
nous avons même trouver une solution particulière $(x_0,y_0)$, nous allons pouvoir en
déduire toutes les solutions.

\change

Supposons que le couple  $(x,y)$ soit une solutions de notre équation.

\change

Nous avons donc l'égalité $161 x + 368 y=115$

Nous allons déterminer la forme de $x$ et $y$.

\change

Nous avons aussi déterminer une solution particulière $(x_0,y_0)$.

\change

Donc nous avons l'égalité  $161x_0+368y_0=115$

Même si nous savons que $x_0=35$ et $y_0=-15$ nous 
n'avons aucun intérêt à substituer ces valeurs dès maintenant.

\change 

Soustrayons ces deux égalités,
cela nous conduit à 

 $161 \times (x-x_0) + 368 \times (y-y_0) = 0$ (la constante $115$ a disparu)

\change

Rappelons nous que $161$ et $368$ sont des multiples de $23$ 
(qui est leur pgcd).

\change

Donc en divisant par $23$ on obtient une égalité plus simple

 $7(x-x_0) = -16 (y-y_0)$

\change

Premièrement cela implique que $7$ divise $16(y-y_0)$

\change

mais $7$ et $16$ sont premier entre eux donc 
par le lemme de Gauss $7|y-y_0$.

\change

En d'autre terme il existe un entier $k$ tel que 
$y-y_0 = 7 \times k$.

\change

Voilà pour $y$. Cherchons la forme de $x$.

Nous repartons de l'égalité $7(x-x_0) = -16 (y-y_0)$

\change

où l'en remplace $y-y_0$ par $7\times k$.

\change

Et ainsi on trouve $x-x_0 = -16k$, 

\change

C'est bien le même entier $k$ pour $x$ et pour $y$,

donc notre couple solution est de la forme
$x=x_0-16k$ et $y=y_0+7k$.

Il n'est pas dur de voir que tout couple de cette forme est solution de notre l'équation.

\change

  On se rappelle maintenant des valeurs de $x_0$ et $y_0$.

\change

Nous obtenons toutes les solutions de l'équation $161 x + 368 y=115$
ce sont les couples d'entiers $(x,y)$
tel que 
$x=35-16k$ et $y=-15+7k$, avec $k$ parcourant l'ensemble des entiers $\Zz$.

Attention c'est bien le même $k$ pour $x$ et $y$.


%%%%%%%%%%%%%%%%%%%%%%%%%%%%%%%%%%%%%%%%%%%%%%%%%%%%%%%%%%%
\diapo

Terminons avec la notion de ppcm.

Le ppcm de deux entiers $a$, $b$ est le plus petit entier divisible à la fois par $a$ et
par $b$.

\change

Par exemple le ppcm de $12$ et $9$ est $36$ car $36$ est divisible par $12$ et par $9$
et tout autre entier divisible par $12$ et $9$ est plus grand que $36$.

\change

Le pgcd et le ppcm sont liés par la formule suivante :

$\pgcd(a,b) \times \text{ppcm}(a,b) = |ab|$

Sur notre exemple précédent le pgcd de $12$ et $9$ est $3$.
Et l'on bien $3\times 36=12\times9$

\change

Une autre propriété est que 
si $a|c$ et $b|c$ alors $\text{ppcm}(a,b)|c$

\change 

Par exemple : $6|36$, $9|36$ et  $\text{ppcm}(6,9)=18$ divise bien $36$

\change 

Attention ne commettez pas l'erreur suivante 
il n'est pas vrai que $ab|c$ : 

cela se voit avec notre exemple : $6\times 9$ ne divise pas $36$





%%%%%%%%%%%%%%%%%%%%%%%%%%%%%%%%%%%%%%%%%%%%%%%%%%%%%%%%%%%
\diapo

Ces mini-exercices sont indispensables pour s'entraîner et progresser.


\end{document}
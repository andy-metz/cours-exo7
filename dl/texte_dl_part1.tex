
%%%%%%%%%%%%%%%%%% PREAMBULE %%%%%%%%%%%%%%%%%%


\documentclass[12pt]{article}

\usepackage{amsfonts,amsmath,amssymb,amsthm}
\usepackage[utf8]{inputenc}
\usepackage[T1]{fontenc}
\usepackage[francais]{babel}


% packages
\usepackage{amsfonts,amsmath,amssymb,amsthm}
\usepackage[utf8]{inputenc}
\usepackage[T1]{fontenc}
%\usepackage{lmodern}

\usepackage[francais]{babel}
\usepackage{fancybox}
\usepackage{graphicx}

\usepackage{float}

%\usepackage[usenames, x11names]{xcolor}
\usepackage{tikz}
\usepackage{datetime}

\usepackage{mathptmx}
%\usepackage{fouriernc}
%\usepackage{newcent}
\usepackage[mathcal,mathbf]{euler}

%\usepackage{palatino}
%\usepackage{newcent}


% Commande spéciale prompteur

%\usepackage{mathptmx}
%\usepackage[mathcal,mathbf]{euler}
%\usepackage{mathpple,multido}

\usepackage[a4paper]{geometry}
\geometry{top=2cm, bottom=2cm, left=1cm, right=1cm, marginparsep=1cm}

\newcommand{\change}{{\color{red}\rule{\textwidth}{1mm}\\}}

\newcounter{mydiapo}

\newcommand{\diapo}{\newpage
\hfill {\normalsize  Diapo \themydiapo \quad \texttt{[\jobname]}} \\
\stepcounter{mydiapo}}


%%%%%%% COULEURS %%%%%%%%%%

% Pour blanc sur noir :
%\pagecolor[rgb]{0.5,0.5,0.5}
% \pagecolor[rgb]{0,0,0}
% \color[rgb]{1,1,1}



%\DeclareFixedFont{\myfont}{U}{cmss}{bx}{n}{18pt}
\newcommand{\debuttexte}{
%%%%%%%%%%%%% FONTES %%%%%%%%%%%%%
\renewcommand{\baselinestretch}{1.5}
\usefont{U}{cmss}{bx}{n}
\bfseries

% Taille normale : commenter le reste !
%Taille Arnaud
%\fontsize{19}{19}\selectfont

% Taille Barbara
%\fontsize{21}{22}\selectfont

%Taille François
\fontsize{25}{30}\selectfont

%Taille Pascal
%\fontsize{25}{30}\selectfont

%Taille Laura
%\fontsize{30}{35}\selectfont


%\myfont
%\usefont{U}{cmss}{bx}{n}

%\Huge
%\addtolength{\parskip}{\baselineskip}
}


% \usepackage{hyperref}
% \hypersetup{colorlinks=true, linkcolor=blue, urlcolor=blue,
% pdftitle={Exo7 - Exercices de mathématiques}, pdfauthor={Exo7}}


%section
% \usepackage{sectsty}
% \allsectionsfont{\bf}
%\sectionfont{\color{Tomato3}\upshape\selectfont}
%\subsectionfont{\color{Tomato4}\upshape\selectfont}

%----- Ensembles : entiers, reels, complexes -----
\newcommand{\Nn}{\mathbb{N}} \newcommand{\N}{\mathbb{N}}
\newcommand{\Zz}{\mathbb{Z}} \newcommand{\Z}{\mathbb{Z}}
\newcommand{\Qq}{\mathbb{Q}} \newcommand{\Q}{\mathbb{Q}}
\newcommand{\Rr}{\mathbb{R}} \newcommand{\R}{\mathbb{R}}
\newcommand{\Cc}{\mathbb{C}} 
\newcommand{\Kk}{\mathbb{K}} \newcommand{\K}{\mathbb{K}}

%----- Modifications de symboles -----
\renewcommand{\epsilon}{\varepsilon}
\renewcommand{\Re}{\mathop{\text{Re}}\nolimits}
\renewcommand{\Im}{\mathop{\text{Im}}\nolimits}
%\newcommand{\llbracket}{\left[\kern-0.15em\left[}
%\newcommand{\rrbracket}{\right]\kern-0.15em\right]}

\renewcommand{\ge}{\geqslant}
\renewcommand{\geq}{\geqslant}
\renewcommand{\le}{\leqslant}
\renewcommand{\leq}{\leqslant}

%----- Fonctions usuelles -----
\newcommand{\ch}{\mathop{\mathrm{ch}}\nolimits}
\newcommand{\sh}{\mathop{\mathrm{sh}}\nolimits}
\renewcommand{\tanh}{\mathop{\mathrm{th}}\nolimits}
\newcommand{\cotan}{\mathop{\mathrm{cotan}}\nolimits}
\newcommand{\Arcsin}{\mathop{\mathrm{Arcsin}}\nolimits}
\newcommand{\Arccos}{\mathop{\mathrm{Arccos}}\nolimits}
\newcommand{\Arctan}{\mathop{\mathrm{Arctan}}\nolimits}
\newcommand{\Argsh}{\mathop{\mathrm{Argsh}}\nolimits}
\newcommand{\Argch}{\mathop{\mathrm{Argch}}\nolimits}
\newcommand{\Argth}{\mathop{\mathrm{Argth}}\nolimits}
\newcommand{\pgcd}{\mathop{\mathrm{pgcd}}\nolimits} 

\newcommand{\Card}{\mathop{\text{Card}}\nolimits}
\newcommand{\Ker}{\mathop{\text{Ker}}\nolimits}
\newcommand{\id}{\mathop{\text{id}}\nolimits}
\newcommand{\ii}{\mathrm{i}}
\newcommand{\dd}{\mathrm{d}}
\newcommand{\Vect}{\mathop{\text{Vect}}\nolimits}
\newcommand{\Mat}{\mathop{\mathrm{Mat}}\nolimits}
\newcommand{\rg}{\mathop{\text{rg}}\nolimits}
\newcommand{\tr}{\mathop{\text{tr}}\nolimits}
\newcommand{\ppcm}{\mathop{\text{ppcm}}\nolimits}

%----- Structure des exercices ------

\newtheoremstyle{styleexo}% name
{2ex}% Space above
{3ex}% Space below
{}% Body font
{}% Indent amount 1
{\bfseries} % Theorem head font
{}% Punctuation after theorem head
{\newline}% Space after theorem head 2
{}% Theorem head spec (can be left empty, meaning ‘normal’)

%\theoremstyle{styleexo}
\newtheorem{exo}{Exercice}
\newtheorem{ind}{Indications}
\newtheorem{cor}{Correction}


\newcommand{\exercice}[1]{} \newcommand{\finexercice}{}
%\newcommand{\exercice}[1]{{\tiny\texttt{#1}}\vspace{-2ex}} % pour afficher le numero absolu, l'auteur...
\newcommand{\enonce}{\begin{exo}} \newcommand{\finenonce}{\end{exo}}
\newcommand{\indication}{\begin{ind}} \newcommand{\finindication}{\end{ind}}
\newcommand{\correction}{\begin{cor}} \newcommand{\fincorrection}{\end{cor}}

\newcommand{\noindication}{\stepcounter{ind}}
\newcommand{\nocorrection}{\stepcounter{cor}}

\newcommand{\fiche}[1]{} \newcommand{\finfiche}{}
\newcommand{\titre}[1]{\centerline{\large \bf #1}}
\newcommand{\addcommand}[1]{}
\newcommand{\video}[1]{}

% Marge
\newcommand{\mymargin}[1]{\marginpar{{\small #1}}}



%----- Presentation ------
\setlength{\parindent}{0cm}

%\newcommand{\ExoSept}{\href{http://exo7.emath.fr}{\textbf{\textsf{Exo7}}}}

\definecolor{myred}{rgb}{0.93,0.26,0}
\definecolor{myorange}{rgb}{0.97,0.58,0}
\definecolor{myyellow}{rgb}{1,0.86,0}

\newcommand{\LogoExoSept}[1]{  % input : echelle
{\usefont{U}{cmss}{bx}{n}
\begin{tikzpicture}[scale=0.1*#1,transform shape]
  \fill[color=myorange] (0,0)--(4,0)--(4,-4)--(0,-4)--cycle;
  \fill[color=myred] (0,0)--(0,3)--(-3,3)--(-3,0)--cycle;
  \fill[color=myyellow] (4,0)--(7,4)--(3,7)--(0,3)--cycle;
  \node[scale=5] at (3.5,3.5) {Exo7};
\end{tikzpicture}}
}



\theoremstyle{definition}
%\newtheorem{proposition}{Proposition}
%\newtheorem{exemple}{Exemple}
%\newtheorem{theoreme}{Théorème}
\newtheorem{lemme}{Lemme}
\newtheorem{corollaire}{Corollaire}
%\newtheorem*{remarque*}{Remarque}
%\newtheorem*{miniexercice}{Mini-exercices}
%\newtheorem{definition}{Définition}




%definition d'un terme
\newcommand{\defi}[1]{{\color{myorange}\textbf{\emph{#1}}}}
\newcommand{\evidence}[1]{{\color{blue}\textbf{\emph{#1}}}}



 %----- Commandes divers ------

\newcommand{\codeinline}[1]{\texttt{#1}}

%%%%%%%%%%%%%%%%%%%%%%%%%%%%%%%%%%%%%%%%%%%%%%%%%%%%%%%%%%%%%
%%%%%%%%%%%%%%%%%%%%%%%%%%%%%%%%%%%%%%%%%%%%%%%%%%%%%%%%%%%%%



\begin{document}

\debuttexte

%%%%%%%%%%%%%%%%%%%%%%%%%%%%%%%%%%%%%%%%%%%%%%%%%%%%%%%%%%%
\diapo

\change

Ce chapitre sur les développements limités commence par une partie théorique : les formules de Taylor.
Nous allons énoncer trois formule de Taylor :

\change

la formule de Taylor avec reste intégral

\change

la formule de Taylor avec reste $f^{(n+1)}(c)$

\change
   
et enfin la formule de Taylor-Young


%%%%%%%%%%%%%%%%%%%%%%%%%%%%%%%%%%%%%%%%%%%%%%%%%%%%%%%%%%%
\diapo

Avant les formules de Taylor motivons sur un exemple les développement limités.
Prenons l'exemple de la fonction exponentielle.

On souhaite avoir une idée du comportement de $\exp x$ autour de $x=0$ 


\change

La meilleure droite qui approxime le graphe de la fonction en $x=0$
est la tangente.

\change

Son équation  est $y=1+x$.

\change

Si l'on souhaite faire mieux, quelle parabole d'équation $y = c_0 + c_1x + c_2 x^2$ approche le mieux 
le graphe de $f$ autour de $x=0$ ? 

\change


Voici le dessin

\change


c'est la parabole d'équation $y=1+x+\frac12 x^2$. 


\change

Si on note $g(x)=\exp x - \big(1+x+\frac12 x^2\big)$ alors
cette équation à la propriété remarquable qu'alors
$g(0)=0$, $g'(0)=0$ et $g''(0)=0$. 


\change

Trouver l'équation de cette parabole c'est faire un développement limité à l'ordre $2$
de la fonction $f$.

\change

Bien sûr si l'on veut être plus précis, on continuerait avec une courbe du troisième degré 
qui serait en fait celle-ci.

\change

C'est exactement cela la notion de développement limité : trouver le polynôme qui approche au mieux une fonction
autour d'un point.


%%%%%%%%%%%%%%%%%%%%%%%%%%%%%%%%%%%%%%%%%%%%%%%%%%%%%%%%%%%
\diapo


Sans plus attendre, voici une  formule, dite formule de Taylor-Young :
$$f(x)= f(0)+f'(0)x+f''(0)\frac{x^2}{2!}+\cdots
+f^{(n)}(0)\frac{x^n}{n!} + x^n\epsilon(x).$$
Dans ce chapitre, pour n'importe quelle fonction, nous allons trouver le polynôme de degré $n$
qui approche le mieux la fonction. 

\change

La partie polynomiale $f(0)+f'(0)x+\cdots+f^{(n)}(0)\frac{x^n}{n!}$
est le polynôme de degré $n$ qui approche le mieux $f(x)$ autour de $x=0$.

\change

Les résultats ne sont valables que pour $x$ autour d'une valeur
fixée (ici autour de $x=0$). 


\change

Ce polynôme sera calculé à partir des dérivées successives au point
considéré. 

\change

La partie $x^n\epsilon(x)$ est le <<reste>>  dans lequel $\epsilon(x)$ est une fonction 
qui tend vers $0$ (quand $x$ tend vers $0$) et qui est négligeable 
devant la partie polynomiale.

\change

Nous allons voir trois formules de Taylor, elles auront toutes la même partie polynomiale
mais donnent plus ou moins d'informations sur le reste. 

\change

Nous commencerons par la formule de
Taylor avec reste intégral qui donne une expression exacte du reste.

\change

 Puis la formule de Taylor avec
reste $f^{(n+1)}(c)$ qui permet d'obtenir un encadrement du reste 


\change

nous terminons avec la formule de Taylor-Young
très pratique si l'on n'a pas besoin d'information sur le reste.

\change

On rappelle enfin qu'une fonction $f$ est dites de \defi{classe $\mathcal{C}^n$}
si $f$ est $n$ fois dérivable sur l'intervalle $I$ et la dérivée $n$-ième est continue.



%%%%%%%%%%%%%%%%%%%%%%%%%%%%%%%%%%%%%%%%%%%%%%%%%%%%%%%%%%%
\diapo

Soit $I$ un intervalle ouvert et fixons $n$ un entier.
Soit $f : I\to\Rr$ une fonction de classe $\mathcal{C}^{n+1}$
et soit $a,x$ deux points de l'intervalle $I$.

La formule de Taylor avec reste intégral s'écrit :
$
f(x)=f(a)+f'(a)(x-a)+\frac{f''(a)}{2!}(x-a)^2+\cdots
+\frac{f^{(n)}(a)}{n!}(x-a)^n$

\change

avec comme reste 
$\int_a^x \frac{f^{(n+1)}(t)}{n!}(x-t)^ndt.$

\change

\change

Par exemple la fonction $f(x)=\exp x$ est de indéfiniment dérivable donc de classe $\mathcal{C}^{n+1}$ pour tout $n$, 

ceci sur l'intervalle $I=\Rr$. 

\change

Comme $f'(x)=\exp x$, $f''(x)=\exp x$,\ldots alors pour $a$ fixé et pour tout $x$ on a

\change

$$\exp x=\exp a+\exp a \cdot (x-a)+\cdots+\frac{\exp a}{n!}(x-a)^n+\int_a^x\frac{\exp t}{n!}(x-t)^ndt.$$

\change

Bien sûr si l'on se place en $a=0$ alors on retrouve le début de notre approximation de la fonction
exponentielle en $x=0$ : $\exp x=1+x+\frac{x^2}{2!}+\frac{x^3}{3!}+\cdots$

(pause)

La preuve du théorème sera vue à la fin de cette leçon.



%%%%%%%%%%%%%%%%%%%%%%%%%%%%%%%%%%%%%%%%%%%%%%%%%%%%%%%%%%%
\diapo

Repartons d'une fonction $f : I\to\Rr$ de classe $\mathcal{C}^{n+1}$
et soit $a,x \in I$.

Alors :
$f(x)=f(a)+f'(a)(x-a)+\frac{f''(a)}{2!}(x-a)^2+\cdots
+\frac{f^{(n)}(a)}{n!}(x-a)^n +...$

\change

La partie polynomiale est la même qu'auparavant.
Mais cette fois le reste est 
$\frac{f^{(n+1)}(c)}{(n+1)!}(x-a)^{n+1}$
pour un certain réel $c$ entre $a$ et $x$.

\change

\change

Reprenons l'exemple de l'exponentielle :
 il existe $c$ entre $a$ et $x$ tel que 
$\exp x=\exp a+\exp a \cdot (x-a)+\cdots+\frac{\exp a}{n!}(x-a)^n+\frac{\exp c}{(n+1)!}(x-a)^{n+1}.$  


\change

Ce théorème de Taylor est une généralisation du théorème des accroissements finis.

\change

En effet pour $n=0$  l'énoncé du théorème entre $a$ et $b$ devient
il existe $c$ entre $a$ et $b$ tel que $f(b)=f(a)+f'(c)(b-a)$.


C'est exactement l'énoncé du théorème des accroissements finis.

(pause) [[désigner la formule du haut]]

Notez au passage la phrase 
 <<le réel $c$ est entre $a$ et $x$>> signifie <<$c$ appartient à l'intervalle $]a,x[$ ou $c$ appartient 
à l'intervalle $]x,a[$>> selon que  $a$ est plus petit (ou plus grand) que $x$.




%%%%%%%%%%%%%%%%%%%%%%%%%%%%%%%%%%%%%%%%%%%%%%%%%%%%%%%%%%%
\diapo

Dans la plupart des cas on ne connaîtra pas le $c$ du théorème précédent. Mais le théorème permet d'encadrer le reste.
Ceci s'exprime par le corollaire suivant :


Si en plus d'être de classe $\mathcal{C}^{n+1}$
la fonction $f^{(n+1)}$ est majorée sur l'intervalle $I$ par un réel $M$, 
alors pour tout $a, x$ de l'intervalle $I$, on a : 
\[
\big|f(x)-T_n(x)\big|\le M\frac{|x-a|^{n+1}}{(n+1)! \ } \cdotp
\]

où $T_n(x)$ désigne la partie polynomiale de nos formules de Taylor.

\change

Nous allons illustrer ceci en calculant un approximation de $\sin(0,01)$.

Notons $f(x)$ la fonction $\sin x$. 

\change

Alors $f'(x)=\cos x$, $f''(x)=-\sin x$, $f^{(3)}(x)=-\cos x$, $f^{(4)}(x)=\sin x$.

\change

On obtient donc $f(0)=0$, $f'(0)=1$, $f''(0)=0$, $f^{(3)}(0)=-1$.

\change

La formule de Taylor avec reste $f^{(n+1)}(c)$ à l'ordre $3$ en $0$ 
donne 

$f(x)=0+1\cdot x +0\cdot \frac{x^2}{2!}-1\frac{x^3}{3!} + f^{(4)}(c)\frac{x^4}{4!}$

\change

c'est-à-dire 
$x -\frac{x^3}{6} + f^{(4)}(c)\frac{x^4}{24}$

\change

Appliquons ceci à la valeur  $0,01$ 

en substituant $x$ par $0,01$ on trouve que

 $\sin(0,01)$ vaut environ $0,01 - \frac{(0,01)^3}{6}$ donc environ $0,00999983333\ldots$

\change

Mais quelle est la précision de cette valeur approchée ?

La formule de Taylor, ou plus précisément le corollaire,
permet d'estimer l'erreur commise.

Comme $f^{(4)}(x)=\sin x$ alors 
$|f^{(4)}(c)|\le 1$ alors 

$\big|f(x) - \big(x -\frac{x^3}{6} \big) \big| \le \frac{x^4}{24}$

\change

Pour notre calcul cela signifie que l'écart entre $\sin(0,01)$
et notre approximation $0,01 - \frac{(0,01)^3}{6}$
est moins de $\frac{(0,01)^4}{24}$.

\change

Et donc notre approximation donne la certitude d'avoir au moins $8$ chiffres exacts après la virgule.



%%%%%%%%%%%%%%%%%%%%%%%%%%%%%%%%%%%%%%%%%%%%%%%%%%%%%%%%%%%
\diapo

Passons à notre dernière formule : la formule de Taylor-Young.

On part cette fois d'une fonction $f$ de classe $\mathcal{C}^n$.

$f(x)$ s'écrit encore fois comme la somme de la partie polynomiale (toujours la même)
et d'un reste
qui est $(x-a)^n\epsilon(x)$
où $\epsilon$ est une fonction qui tend vers $0$ lorsque $x$ tend vers $a$.

La seule information que donne cette fonction $\epsilon$ est que le reste
est négligeable devant la partie polynomiale, c'est-à-dire le reste tend plus vite vers
$0$ que chacun des monômes du polynôme de Taylor.



%%%%%%%%%%%%%%%%%%%%%%%%%%%%%%%%%%%%%%%%%%%%%%%%%%%%%%%%%%%
\diapo


Attardons sur un exemple que nous traitons en détails.

Soit $f$ la fonction définie par $f(x)= \ln(1+x)$

\change 


Voici le graphe de $f$.

\change


Cette fonction est définie sur l'intervalle $]-1,+\infty[$ et  $f$ y est infiniment dérivable.

Nous allons calculer les formules de Taylor en $0$ pour les premiers ordres.

\change

Tous d'abord $f(0)=0$. 

\change

Donc le polynôme de Taylor d'ordre $0$ est $T_0(x)=0$

\change

Approcher notre fonction par une droite horizontale n'est pas très probant.

\change

On continue avec $f'(x)=\frac{1}{1+x}$ donc $f'(0)=1$.

\change

$T_1(x) = f(0) + f'(0)x = x$

\change

Bien sûr c'est l'équation de la tangente en $0$.

\change

On continue avec $f''(x) = -\frac{1}{(1+x)^2}$ donc $f''(0)=-1$.

\change

Le polynôme de Taylor de degré $2$ est donc $T_2(x) = f(0) + f'(0)x+f''(0)\frac{x^2}{2!} =x -\frac{x^2}{2}$

\change

Nous avons maintenant un bonne approximation de notre courbe par une parabole.

\change

Enfin on calcule la dérivée troisième.

\change

Le monôme de degré $3$ est $\frac{x^3}{3!}f'''(0)$
et donc le polynôme de Taylor de degré $3$ est $T_3(x) = x -\frac{x^2}{2} + \frac{x^3}{3}$

\change

Notre approximation est excellente autour de $0$.

\change

On remarque sur cet exemple que
plus $n$ est grand meilleur est notre approximation.
Cela s'explique car le reste est de plus en plus en petit.

Sur le dessin les graphes des polynômes $T_0, T_1, T_2, T_3$ s'approchent de plus en plus du graphe de $f$.

Mais il faut bien être conscient que cette approximation n'est valable que autour 
de $0$, c'est-à-dire pour des $x$ assez petits.
Cela se voit tout de suite en regardant le comportement pour des grandes valeurs de $x$,
le log croit très doucement devant les fonctions polynomiales. 



%%%%%%%%%%%%%%%%%%%%%%%%%%%%%%%%%%%%%%%%%%%%%%%%%%%%%%%%%%%
\diapo


Continuons avec notre exemple de la fonction $\ln(1+x)$

et essayons d'obtenir les termes suivants.

\change

Nous avons vu que sa dérivée était $\frac{1}{1+x}$

\change

Par récurrence on montre que $f^{(n)}(x) = (-1)^{n-1} (n-1)!\frac{1}{(1+x)^n}$
et donc $f^{(n)}(0)= (-1)^{n-1} (n-1)!$.

\change

Maintenant pour obtenir le coefficient du polynôme de Taylor
on divise par $n!$ et donc $(n-1)!$ divisé par $n!$ donne $1/n$.
et donc le monôme de degré $n$ est $(-1)^{n-1}\frac{x^n}{n}$.

\change

Voici donc le polynôme de Taylor en $0$ de $\ln(1+x)$ à l'ordre $n$.

$T_n(x)=\sum_{k=1}^n (-1)^{n-1}\frac{x^n}{n}$
autrement dit c'est

$x-\frac{x^2}{2} + \frac{x^3}{3}-\frac{x^4}{4}+ \cdots + (-1)^{n-1}\frac{x^n}{n}$

Pour les formule de Taylor n'oubliez pas d'ajouter le reste.

%%%%%%%%%%%%%%%%%%%%%%%%%%%%%%%%%%%%%%%%%%%%%%%%%%%%%%%%%%%
\diapo

Il y a donc trois formules de Taylor qui s'écrivent toutes sous la forme

$f(x) = T_n(x) + R_n(x)$

\change

où $T_n(x)$ est toujours le même polynôme de Taylor :

$$T_n(x) =f(a)+f'(a)(x-a)+\cdots
+\frac{f^{(n)}(a)}{n!}(x-a)^n.$$

C'est l'expression du reste $R_n(x)$ qui change (attention ! le reste n'a aucune raison d'être un polynôme). 

\change


\begin{align*}
R_n(x) & = \int_a^x \frac{f^{(n+1)}(t)}{n!}(x-t)^ndt 
 & \quad   \text{Taylor avec reste intégral} \\
R_n(x) & =\frac{f^{(n+1)}(c)}{(n+1)!}(x-a)^{n+1}   
 & \quad   \text{Taylor avec reste } f^{(n+1)}(c),\   c \text{ entre } a \text{ et } x \\
R_n(x) & = (x-a)^n\epsilon(x) 
 & \quad \text{Taylor-Young avec } \epsilon(x) \xrightarrow[x\to a]{} 0 \\
\end{align*}

\change

\change



Selon les situations l'une des formulations est plus adaptée que les autres.


Notons que les trois formules ne requièrent pas exactement les mêmes hypothèses
 on manipule le plus souvent des fonctions de classe $\mathcal{C}^\infty$
et les trois formules sont alors valides quelque soit $n$.

\change

Pour terminer, à l'aide du changement de variable $x=a+h$ 
on réécrit les formules de Taylor sous une forme parfois plus pratique.

Par exemple la formule de Taylor-Young devient 

$$f(a+h)=f(a)+f'(a)h+\cdots
+\frac{f^{(n)}(a)}{n!}h^n+h^n \epsilon(h)$$

où $\epsilon(h) \xrightarrow[h\to 0]{} 0$.


%%%%%%%%%%%%%%%%%%%%%%%%%%%%%%%%%%%%%%%%%%%%%%%%%%%%%%%%%%%
\diapo

Bien souvent nous n'avons pas besoin de beaucoup d'information sur le reste et 
c'est donc la formule de Taylor-Young qui sera la plus utile.

En plus on se ramène souvent à étudier la fonction autour de $x=0$.

Donc si vous ne deviez retenir qu'une seule formule ce serait celle-là 

$f(x)= f(0)+f'(0)x+f''(0)\frac{x^2}{2!}+\cdots
+f^{(n)}(0)\frac{x^n}{n!} + x^n\epsilon(x)$
où \ $\lim_{x\to0}\epsilon (x)=0$


\change

Le terme $x^n\epsilon(x)$ où $\epsilon(x) \xrightarrow[x\to 0]{} 0$
est souvent abrégé en <<\defi{petit o}>> de $x^n$ .

Il faut s'habituer à cette notation qui simplifie les écritures, 
mais il faut toujours garder à l'esprit ce qu'elle signifie.


\change

Avec la notation <<petit o>> 
la formule de Taylor-Young en zéro s'écrit
la partie polynomiale + un <<petit o>> de $x^n$.


%%%%%%%%%%%%%%%%%%%%%%%%%%%%%%%%%%%%%%%%%%%%%%%%%%%%%%%%%%%
\diapo

Prouvons une des formules de Taylor.

\change

Nous prouvons la formule de Taylor avec reste intégral.
Les deux autres formules se démontrent à partir de celle-ci.

\change

Nous allons prouver cette formule par récurrence sur $n$.

Pour éviter les confusions entre ce qui varie et ce qui est fixe dans cette
preuve on a remplacé $x$ par $b$.

Nous supposons que $f$ est une fonction de classe $\mathcal{C}^{n+1}$.


\change

Pour $n=0$, une primitive de $f'(t)$ est $f(t)$ donc 
$\int_a^b f'(t) \, dt=f(b)-f(a)$, 

donc $f(b)=f(a)+\int_a^b f'(t) \, dt$. 


On rappelle que par convention $(b-t)^0=1$ et $0!=1$.

\change

Passons à l'hérédité et supposons la formule vraie au rang $n-1$ 

on a donc cette formule là.

\change

Nous allons effectuer une intégration par parties dans l'intégrale du reste

en posant 

$u =f^{(n)}(t)$ et $v' = \frac{(b-t)^{n-1}}{(n-1)!}$

\change

Donc on a $u'= f^{(n+1)}(t)$ et $v = - \frac{(b-t)^{n}}{n!}$

\change 

On applique maintenant la formule d'intégration par parties :

intégrale de $uv'$... 

\change

...égale $uv$ moins intégrale de $u'v$.

\change

Lorsque l'on évalue le crochet en $b$ cela donne $0$,
et évaluer le crochet en $a$ donne ceci.

On reconnaît exactement le monôme de degré $n$ du polynôme de Taylor
plus le reste intégral au rang d'après.


\change

On vient donc de prouver la formule au rang $n$

Et ainsi par le principe de récurrence la formule de Taylor est vraie 
pour tous les entiers $n$ pour lesquels $f$ est classe $\mathcal{C}^{n+1}$.  



%%%%%%%%%%%%%%%%%%%%%%%%%%%%%%%%%%%%%%%%%%%%%%%%%%%%%%%%%%%
\diapo

Après avoir appris le cours passez à la pratique 
avec ces exercices d'applications !


\end{document}

%%%%%%%%%%%%%%%%%% PREAMBULE %%%%%%%%%%%%%%%%%%

\documentclass[aspectratio=169,utf8]{beamer}
%\documentclass[aspectratio=169,handout]{beamer}

\usetheme{Boadilla}
%\usecolortheme{seahorse}
%\usecolortheme[RGB={245,66,24}]{structure}
\useoutertheme{infolines}

% packages
\usepackage{amsfonts,amsmath,amssymb,amsthm}
\usepackage[utf8]{inputenc}
\usepackage[T1]{fontenc}
\usepackage{lmodern}

\usepackage[francais]{babel}
\usepackage{fancybox}
\usepackage{graphicx}

\usepackage{float}
\usepackage{xfrac}

%\usepackage[usenames, x11names]{xcolor}
\usepackage{pgfplots}
\usepackage{datetime}


% ----------------------------------------------------------------------
% Pour les images
\usepackage{tikz}
\usetikzlibrary{calc,shadows,arrows.meta,patterns,matrix}

\newcommand{\tikzinput}[1]{\input{figures/#1.tikz}}
% --- les figures avec échelle éventuel
\newcommand{\myfigure}[2]{% entrée : échelle, fichier(s) figure à inclure
\begin{center}\small%
\tikzstyle{every picture}=[scale=1.0*#1]% mise en échelle + 0% (automatiquement annulé à la fin du groupe)
#2%
\end{center}}



%-----  Package unités -----
\usepackage{siunitx}
\sisetup{locale = FR,detect-all,per-mode = symbol}

%\usepackage{mathptmx}
%\usepackage{fouriernc}
%\usepackage{newcent}
%\usepackage[mathcal,mathbf]{euler}

%\usepackage{palatino}
%\usepackage{newcent}
% \usepackage[mathcal,mathbf]{euler}



% \usepackage{hyperref}
% \hypersetup{colorlinks=true, linkcolor=blue, urlcolor=blue,
% pdftitle={Exo7 - Exercices de mathématiques}, pdfauthor={Exo7}}


%section
% \usepackage{sectsty}
% \allsectionsfont{\bf}
%\sectionfont{\color{Tomato3}\upshape\selectfont}
%\subsectionfont{\color{Tomato4}\upshape\selectfont}

%----- Ensembles : entiers, reels, complexes -----
\newcommand{\Nn}{\mathbb{N}} \newcommand{\N}{\mathbb{N}}
\newcommand{\Zz}{\mathbb{Z}} \newcommand{\Z}{\mathbb{Z}}
\newcommand{\Qq}{\mathbb{Q}} \newcommand{\Q}{\mathbb{Q}}
\newcommand{\Rr}{\mathbb{R}} \newcommand{\R}{\mathbb{R}}
\newcommand{\Cc}{\mathbb{C}} 
\newcommand{\Kk}{\mathbb{K}} \newcommand{\K}{\mathbb{K}}

%----- Modifications de symboles -----
\renewcommand{\epsilon}{\varepsilon}
\renewcommand{\Re}{\mathop{\text{Re}}\nolimits}
\renewcommand{\Im}{\mathop{\text{Im}}\nolimits}
%\newcommand{\llbracket}{\left[\kern-0.15em\left[}
%\newcommand{\rrbracket}{\right]\kern-0.15em\right]}

\renewcommand{\ge}{\geqslant}
\renewcommand{\geq}{\geqslant}
\renewcommand{\le}{\leqslant}
\renewcommand{\leq}{\leqslant}
\renewcommand{\epsilon}{\varepsilon}

%----- Fonctions usuelles -----
\newcommand{\ch}{\mathop{\text{ch}}\nolimits}
\newcommand{\sh}{\mathop{\text{sh}}\nolimits}
\renewcommand{\tanh}{\mathop{\text{th}}\nolimits}
\newcommand{\cotan}{\mathop{\text{cotan}}\nolimits}
\newcommand{\Arcsin}{\mathop{\text{arcsin}}\nolimits}
\newcommand{\Arccos}{\mathop{\text{arccos}}\nolimits}
\newcommand{\Arctan}{\mathop{\text{arctan}}\nolimits}
\newcommand{\Argsh}{\mathop{\text{argsh}}\nolimits}
\newcommand{\Argch}{\mathop{\text{argch}}\nolimits}
\newcommand{\Argth}{\mathop{\text{argth}}\nolimits}
\newcommand{\pgcd}{\mathop{\text{pgcd}}\nolimits} 


%----- Commandes divers ------
\newcommand{\ii}{\mathrm{i}}
\newcommand{\dd}{\text{d}}
\newcommand{\id}{\mathop{\text{id}}\nolimits}
\newcommand{\Ker}{\mathop{\text{Ker}}\nolimits}
\newcommand{\Card}{\mathop{\text{Card}}\nolimits}
\newcommand{\Vect}{\mathop{\text{Vect}}\nolimits}
\newcommand{\Mat}{\mathop{\text{Mat}}\nolimits}
\newcommand{\rg}{\mathop{\text{rg}}\nolimits}
\newcommand{\tr}{\mathop{\text{tr}}\nolimits}


%----- Structure des exercices ------

\newtheoremstyle{styleexo}% name
{2ex}% Space above
{3ex}% Space below
{}% Body font
{}% Indent amount 1
{\bfseries} % Theorem head font
{}% Punctuation after theorem head
{\newline}% Space after theorem head 2
{}% Theorem head spec (can be left empty, meaning ‘normal’)

%\theoremstyle{styleexo}
\newtheorem{exo}{Exercice}
\newtheorem{ind}{Indications}
\newtheorem{cor}{Correction}


\newcommand{\exercice}[1]{} \newcommand{\finexercice}{}
%\newcommand{\exercice}[1]{{\tiny\texttt{#1}}\vspace{-2ex}} % pour afficher le numero absolu, l'auteur...
\newcommand{\enonce}{\begin{exo}} \newcommand{\finenonce}{\end{exo}}
\newcommand{\indication}{\begin{ind}} \newcommand{\finindication}{\end{ind}}
\newcommand{\correction}{\begin{cor}} \newcommand{\fincorrection}{\end{cor}}

\newcommand{\noindication}{\stepcounter{ind}}
\newcommand{\nocorrection}{\stepcounter{cor}}

\newcommand{\fiche}[1]{} \newcommand{\finfiche}{}
\newcommand{\titre}[1]{\centerline{\large \bf #1}}
\newcommand{\addcommand}[1]{}
\newcommand{\video}[1]{}

% Marge
\newcommand{\mymargin}[1]{\marginpar{{\small #1}}}

\def\noqed{\renewcommand{\qedsymbol}{}}


%----- Presentation ------
\setlength{\parindent}{0cm}

%\newcommand{\ExoSept}{\href{http://exo7.emath.fr}{\textbf{\textsf{Exo7}}}}

\definecolor{myred}{rgb}{0.93,0.26,0}
\definecolor{myorange}{rgb}{0.97,0.58,0}
\definecolor{myyellow}{rgb}{1,0.86,0}

\newcommand{\LogoExoSept}[1]{  % input : echelle
{\usefont{U}{cmss}{bx}{n}
\begin{tikzpicture}[scale=0.1*#1,transform shape]
  \fill[color=myorange] (0,0)--(4,0)--(4,-4)--(0,-4)--cycle;
  \fill[color=myred] (0,0)--(0,3)--(-3,3)--(-3,0)--cycle;
  \fill[color=myyellow] (4,0)--(7,4)--(3,7)--(0,3)--cycle;
  \node[scale=5] at (3.5,3.5) {Exo7};
\end{tikzpicture}}
}


\newcommand{\debutmontitre}{
  \author{} \date{} 
  \thispagestyle{empty}
  \hspace*{-10ex}
  \begin{minipage}{\textwidth}
    \titlepage  
  \vspace*{-2.5cm}
  \begin{center}
    \LogoExoSept{2.5}
  \end{center}
  \end{minipage}

  \vspace*{-0cm}
  
  % Astuce pour que le background ne soit pas discrétisé lors de la conversion pdf -> png
\begin{tikzpicture}
        \fill[opacity=0,green!60!black] (0,0)--++(0,0)--++(0,0)--++(0,0)--cycle; 
\end{tikzpicture}

% toc S'affiche trop tot :
% \tableofcontents[hideallsubsections, pausesections]
}

\newcommand{\finmontitre}{
  \end{frame}
  \setcounter{framenumber}{0}
} % ne marche pas pour une raison obscure

%----- Commandes supplementaires ------

% \usepackage[landscape]{geometry}
% \geometry{top=1cm, bottom=3cm, left=2cm, right=10cm, marginparsep=1cm
% }
% \usepackage[a4paper]{geometry}
% \geometry{top=2cm, bottom=2cm, left=2cm, right=2cm, marginparsep=1cm
% }

%\usepackage{standalone}


% New command Arnaud -- november 2011
\setbeamersize{text margin left=24ex}
% si vous modifier cette valeur il faut aussi
% modifier le decalage du titre pour compenser
% (ex : ici =+10ex, titre =-5ex

\theoremstyle{definition}
%\newtheorem{proposition}{Proposition}
%\newtheorem{exemple}{Exemple}
%\newtheorem{theoreme}{Théorème}
%\newtheorem{lemme}{Lemme}
%\newtheorem{corollaire}{Corollaire}
%\newtheorem*{remarque*}{Remarque}
%\newtheorem*{miniexercice}{Mini-exercices}
%\newtheorem{definition}{Définition}

% Commande tikz
\usetikzlibrary{calc}
\usetikzlibrary{patterns,arrows}
\usetikzlibrary{matrix}
\usetikzlibrary{fadings} 

%definition d'un terme
\newcommand{\defi}[1]{{\color{myorange}\textbf{\emph{#1}}}}
\newcommand{\evidence}[1]{{\color{blue}\textbf{\emph{#1}}}}
\newcommand{\assertion}[1]{\emph{\og#1\fg}}  % pour chapitre logique
%\renewcommand{\contentsname}{Sommaire}
\renewcommand{\contentsname}{}
\setcounter{tocdepth}{2}



%------ Encadrement ------

\usepackage{fancybox}


\newcommand{\mybox}[1]{
\setlength{\fboxsep}{7pt}
\begin{center}
\shadowbox{#1}
\end{center}}

\newcommand{\myboxinline}[1]{
\setlength{\fboxsep}{5pt}
\raisebox{-10pt}{
\shadowbox{#1}
}
}

%--------------- Commande beamer---------------
\newcommand{\beameronly}[1]{#1} % permet de mettre des pause dans beamer pas dans poly


\setbeamertemplate{navigation symbols}{}
\setbeamertemplate{footline}  % tiré du fichier beamerouterinfolines.sty
{
  \leavevmode%
  \hbox{%
  \begin{beamercolorbox}[wd=.333333\paperwidth,ht=2.25ex,dp=1ex,center]{author in head/foot}%
    % \usebeamerfont{author in head/foot}\insertshortauthor%~~(\insertshortinstitute)
    \usebeamerfont{section in head/foot}{\bf\insertshorttitle}
  \end{beamercolorbox}%
  \begin{beamercolorbox}[wd=.333333\paperwidth,ht=2.25ex,dp=1ex,center]{title in head/foot}%
    \usebeamerfont{section in head/foot}{\bf\insertsectionhead}
  \end{beamercolorbox}%
  \begin{beamercolorbox}[wd=.333333\paperwidth,ht=2.25ex,dp=1ex,right]{date in head/foot}%
    % \usebeamerfont{date in head/foot}\insertshortdate{}\hspace*{2em}
    \insertframenumber{} / \inserttotalframenumber\hspace*{2ex} 
  \end{beamercolorbox}}%
  \vskip0pt%
}


\definecolor{mygrey}{rgb}{0.5,0.5,0.5}
\setlength{\parindent}{0cm}
%\DeclareTextFontCommand{\helvetica}{\fontfamily{phv}\selectfont}

% background beamer
\definecolor{couleurhaut}{rgb}{0.85,0.9,1}  % creme
\definecolor{couleurmilieu}{rgb}{1,1,1}  % vert pale
\definecolor{couleurbas}{rgb}{0.85,0.9,1}  % blanc
\setbeamertemplate{background canvas}[vertical shading]%
[top=couleurhaut,middle=couleurmilieu,midpoint=0.4,bottom=couleurbas] 
%[top=fondtitre!05,bottom=fondtitre!60]



\makeatletter
\setbeamertemplate{theorem begin}
{%
  \begin{\inserttheoremblockenv}
  {%
    \inserttheoremheadfont
    \inserttheoremname
    \inserttheoremnumber
    \ifx\inserttheoremaddition\@empty\else\ (\inserttheoremaddition)\fi%
    \inserttheorempunctuation
  }%
}
\setbeamertemplate{theorem end}{\end{\inserttheoremblockenv}}

\newenvironment{theoreme}[1][]{%
   \setbeamercolor{block title}{fg=structure,bg=structure!40}
   \setbeamercolor{block body}{fg=black,bg=structure!10}
   \begin{block}{{\bf Th\'eor\`eme }#1}
}{%
   \end{block}%
}


\newenvironment{proposition}[1][]{%
   \setbeamercolor{block title}{fg=structure,bg=structure!40}
   \setbeamercolor{block body}{fg=black,bg=structure!10}
   \begin{block}{{\bf Proposition }#1}
}{%
   \end{block}%
}

\newenvironment{corollaire}[1][]{%
   \setbeamercolor{block title}{fg=structure,bg=structure!40}
   \setbeamercolor{block body}{fg=black,bg=structure!10}
   \begin{block}{{\bf Corollaire }#1}
}{%
   \end{block}%
}

\newenvironment{mydefinition}[1][]{%
   \setbeamercolor{block title}{fg=structure,bg=structure!40}
   \setbeamercolor{block body}{fg=black,bg=structure!10}
   \begin{block}{{\bf Définition} #1}
}{%
   \end{block}%
}

\newenvironment{lemme}[0]{%
   \setbeamercolor{block title}{fg=structure,bg=structure!40}
   \setbeamercolor{block body}{fg=black,bg=structure!10}
   \begin{block}{\bf Lemme}
}{%
   \end{block}%
}

\newenvironment{remarque}[1][]{%
   \setbeamercolor{block title}{fg=black,bg=structure!20}
   \setbeamercolor{block body}{fg=black,bg=structure!5}
   \begin{block}{Remarque #1}
}{%
   \end{block}%
}


\newenvironment{exemple}[1][]{%
   \setbeamercolor{block title}{fg=black,bg=structure!20}
   \setbeamercolor{block body}{fg=black,bg=structure!5}
   \begin{block}{{\bf Exemple }#1}
}{%
   \end{block}%
}


\newenvironment{miniexercice}[0]{%
   \setbeamercolor{block title}{fg=structure,bg=structure!20}
   \setbeamercolor{block body}{fg=black,bg=structure!5}
   \begin{block}{Mini-exercices}
}{%
   \end{block}%
}


\newenvironment{tp}[0]{%
   \setbeamercolor{block title}{fg=structure,bg=structure!40}
   \setbeamercolor{block body}{fg=black,bg=structure!10}
   \begin{block}{\bf Travaux pratiques}
}{%
   \end{block}%
}
\newenvironment{exercicecours}[1][]{%
   \setbeamercolor{block title}{fg=structure,bg=structure!40}
   \setbeamercolor{block body}{fg=black,bg=structure!10}
   \begin{block}{{\bf Exercice }#1}
}{%
   \end{block}%
}
\newenvironment{algo}[1][]{%
   \setbeamercolor{block title}{fg=structure,bg=structure!40}
   \setbeamercolor{block body}{fg=black,bg=structure!10}
   \begin{block}{{\bf Algorithme}\hfill{\color{gray}\texttt{#1}}}
}{%
   \end{block}%
}


\setbeamertemplate{proof begin}{
   \setbeamercolor{block title}{fg=black,bg=structure!20}
   \setbeamercolor{block body}{fg=black,bg=structure!5}
   \begin{block}{{\footnotesize Démonstration}}
   \footnotesize
   \smallskip}
\setbeamertemplate{proof end}{%
   \end{block}}
\setbeamertemplate{qed symbol}{\openbox}


\makeatother
\usecolortheme[RGB={179,179,12}]{structure}

%%%%%%%%%%%%%%%%%%%%%%%%%%%%%%%%%%%%%%%%%%%%%%%%%%%%%%%%%%%%%
%%%%%%%%%%%%%%%%%%%%%%%%%%%%%%%%%%%%%%%%%%%%%%%%%%%%%%%%%%%%%


\begin{document}


\title{{\bf Développements limités}}
\subtitle{Opérations sur les développements limités}

\begin{frame}
  
  \debutmontitre

  \pause

{\footnotesize
\hfill
\setbeamercovered{transparent=50}
\begin{minipage}{0.6\textwidth}
  \begin{itemize}
    \item<3-> Somme et produit
    \item<4-> Composition
    \item<5-> Division
    \item<6-> Intégration
  \end{itemize}
\end{minipage}
}

\end{frame}

\setcounter{framenumber}{0}


%%%%%%%%%%%%%%%%%%%%%%%%%%%%%%%%%%%%%%%%%%%%%%%%%%%%%%%%%%%%%%%%


%---------------------------------------------------------------
\section{Somme et produit}

\begin{frame}
\center 
$f(x)=c_0+c_1x + \cdots +c_nx^n + x^n\epsilon_1(x)$ 

\smallskip 

$g(x)=d_0+d_1x + \cdots +d_nx^n + x^n\epsilon_2(x)$

\pause

\begin{proposition}
\begin{itemize}
  \item $f+g$  admet un DL en $0$ l'ordre $n$ 
\pause
\vspace*{-1ex}
$$f(x)+g(x)=(c_0+d_0)+(c_1+d_1)x+\cdots+(c_n+d_n)x^n +x^n\epsilon(x)$$
\pause
\vspace*{-2ex}
  \item $f\times g$ admet un DL en $0$ l'ordre $n$ 
\pause
\vspace*{-1ex}
$$f(x) \times g(x)= T_n(x)+x^n\epsilon (x)$$
où $T_n(x)$ est le  polynôme 
$(c_0+c_1x + \cdots +c_nx^n)\times(d_0+d_1x + \cdots +d_nx^n)$ 
tronqué à l'ordre $n$
\end{itemize}
\end{proposition}

\pause

\defi{Tronquer} c'est conserver seulement les monômes de degré $\le n$
  
\end{frame}




\begin{frame}

\begin{exemple}
\vspace*{-2ex}
$$\begin{array}{l@{\vrule depth 1.2ex height 3ex width 0mm }}
 {\color<2,3>{red} \cos x} \times {\color<3>{blue} \sqrt{1+x}}
\pause
   =  \left({\color<2,3>{red}1-\frac{1}{2}x^2+x^2\epsilon_1(x)}\right)\times \uncover<3->{\left({\color<3>{blue}1+\frac{1}{2}x-\frac{1}{8}x^2+x^2\epsilon_2(x)}\right)}\\
\pause
\pause
   =  1+\frac{1}{2}x-\frac{1}{8}x^2+x^2\epsilon_2(x)\\\pause
    \qquad -\frac{1}{2}x^2\left(1+\frac{1}{2}x-\frac{1}{8}x^2+x^2\epsilon_2(x)\right) \\\pause
    \qquad \qquad +x^2\epsilon_1(x)\left(1+\frac{1}{2}x-\frac{1}{8}x^2+x^2\epsilon_2(x)\right) \\\pause
   = 1+\frac{1}{2}x-\frac{1}{8}x^2+x^2\epsilon_2(x) \\\pause
    \qquad -\frac{1}{2}x^2-\frac{1}{4}x^3+\frac{1}{16}x^4-\frac12x^4\epsilon_2(x)\\\pause
    \qquad \qquad +x^2\epsilon_1(x) +\frac{1}{2}x^3\epsilon_1(x)-\frac{1}{8}x^4\epsilon_1(x)+x^4\epsilon_1(x)\epsilon_2(x) \\\pause
   = 1+\frac{1}{2}x + \left(-\frac{1}{8}x^2-\frac{1}{2}x^2\right)\pause
   \quad  + \quad  x^2\epsilon_2(x) -\frac{1}{4}x^3+\frac{1}{16}x^4+\cdots \\ \pause
   =  1+\frac{1}{2}x-\frac{5}{8}x^2 \pause \ + \ x^2\epsilon(x) \\
\end{array}$$
\end{exemple}
\end{frame}

\begin{frame}
\begin{exemple}
$$\begin{array}{rcl@{\vrule depth 1.2ex height 3ex width 0mm }}
\cos x \times \sqrt{1+x} 
  &\pause  = & \left( 1-\frac{1}{2}x^2+ o(x^2)\right)\times \left(1+\frac{1}{2}x-\frac{1}{8}x^2+o(x^2)\right) \\ \pause
  & = & 1+\frac{1}{2}x-\frac{1}{8}x^2+o(x^2)\\ \pause
  &&  \qquad -\frac{1}{2}x^2 + o(x^2) \\ \pause
  &&  \qquad \qquad + o(x^2) \\
\pause
  & = & 1+\frac{1}{2}x-\frac{5}{8}x^2+o(x^2) \\
\end{array}$$
\end{exemple}

\end{frame}


%---------------------------------------------------------------
\section{Composition}

\begin{frame}

$$f(x) =c_0+c_1x + \cdots +c_nx^n + x^n\epsilon_1(x) \uncover<2->{= C(x) + x^n\epsilon_1(x)}$$
$$g(x) =d_0+d_1x + \cdots +d_nx^n + x^n\epsilon_2(x) \uncover<3->{= D(x) + x^n\epsilon_2(x)}$$

\pause
\pause
\pause


\begin{proposition} 
\begin{itemize}
  \item Si $g(0)=0$ (c'est-à-dire $d_0=0$) alors la fonction $f\circ g$ admet un DL en $0$ à l'ordre $n$
\pause
  \item Sa partie polynomiale est le polynôme tronqué à l'ordre $n$ de la composition $C(D(x))$
\end{itemize}
\end{proposition}  

\end{frame}

\begin{frame}

\begin{exemple}
DL de $\sin\big(\ln(1+x)\big)$ en $0$ à l'ordre $3$
\pause
\begin{enumerate}
  \item On pose $f(u)=\sin u$ et $g(x)=\ln(1+x)$ \pause
$f\circ g(x) = \sin\big(\ln(1+x)\big)$ \pause et $g(0)=0$
\pause
  \item $f(u)=\sin u = u-\frac{u^3}{3!}+u^3\epsilon_1(u)$ 
\pause
  \item $u=g(x)=\ln(1+x)=x-\frac{x^2}{2}+\frac{x^3}{3}+x^3\epsilon_2(x)$
\pause
  \item $u^2 = \big(x-\frac{x^2}{2}+\frac{x^3}{3}+x^3\epsilon_2(x)\big)^2 \pause = x^2-x^3+x^3\epsilon_3(x)$
\pause
  \item $u^3 = u \times u^2 \pause = x^3+x^3\epsilon_4(x)$
\pause
  \item \ \\ \vspace*{-3ex}
$\begin{array}{rcl@{\vrule depth 1.2ex height 3ex width 0mm }}
f\circ g(x) \pause & = & f(u) \pause = u-\frac{u^3}{3!}+u^3\epsilon_1(u) \\
\pause
 & = & \big(x-\frac{1}{2}x^{2}+\frac{1}{3}x^3\big) -\frac16 x^3  +x^3\epsilon(x) \\
\pause
 & = & x-\frac{1}{2}x^{2}+\frac16 x^3  +x^3\epsilon(x) \\
\end{array}$
\end{enumerate}
\end{exemple}
\end{frame}


\begin{frame}

\begin{exemple}
DL de $h(x)=\sqrt{\cos x}$ en $0$ à l'ordre $4$ 

\pause
\begin{itemize}
  \item $f(u)=\sqrt{1+u}\pause =1+\frac{1}{2}u-\frac{1}{8}u^2 + o(u^2)$ \pause

  \item On pose $u(x)=\cos x-1$ \pause 

  \begin{itemize}
     \item $h(x)=\sqrt{\cos x}= f\big(u(x)\big)$ et $u(0)=0$ \pause 
     \item $u=\cos x-1=-\frac{1}{2}x^2+\frac{1}{24}x^4+o(x^4)$ \pause 
     \item $u^2 = \frac{1}{4}x^4 + o(x^4)$ \pause 
  \end{itemize}

  \item  \ \\ \vspace*{-3ex}
$\begin{array}{rcl@{\vrule depth 1.2ex height 3ex width 0mm }}
h(x)\pause  & = & f\big(u\big) \pause=  1+\frac{1}{2}u-\frac{1}{8}u^2 + o(u^2) \\ \pause
     & = & 1 + \frac{1}{2}\big(-\frac{1}{2}x^2+\frac{1}{24}x^4\big)-\frac{1}{8}\big(\frac{1}{4}x^4\big) + o(x^4) \\ \pause 
     & = & 1-\frac{1}{4}x^2+\frac{1}{48}x^4 -\frac{1}{32}x^4+o(x^4) \\ \pause 
     & = & 1-\frac{1}{4}x^2-\frac{1}{96}x^4+o(x^4) \\ 
\end{array}$
\end{itemize}

\end{exemple}


\end{frame}

%---------------------------------------------------------------
\section{Division}

\begin{frame}


$$f(x)=c_0+c_1x + \cdots +c_nx^n + x^n\epsilon_1(x)$$ 
$$g(x)=d_0+d_1x + \cdots +d_nx^n + x^n\epsilon_2(x)$$ \pause
$$\frac{1}{1+u} = 1-u+u^2-u^3+\cdots$$
\pause
\begin{enumerate}
  \item Si $d_0= 1$ on pose $u = d_1x + \cdots +d_nx^n + x^n\epsilon_2(x) $ 
\pause
$$f/g = f \times \frac{1}{1+u}$$
\pause

  \item Si $d_0\neq 0$ on se ramène au cas précédent en écrivant 
\[
\frac{1}{g(x)} =\frac{1}{d_0} \frac{1}{1+\frac{d_1}{d_0}x+\cdots+\frac{d_n}{d_0}x^n
+\frac{x^n\epsilon_2(x)}{d_0}}
\]
 \pause

  \item Si $d_0=0$ on factorise par $x^k$ afin de se ramener aux cas précédents
\end{enumerate}
  
\end{frame}

\begin{frame}

\begin{exemple}
DL de $\tan x$ en $0$ à l'ordre $5$
 
 \pause
\begin{itemize}
  \item $\sin x=x-\frac{x^3}{6}+\frac{x^5}{120}+x^5\epsilon(x)$ 
\pause
  \item $\cos  x=1-\frac{x^2}{2}+\frac{x^4}{24} +x^5\epsilon(x)$
\pause
  \item on pose $u= -\frac{x^2}{2}+\frac{x^4}{24} +x^5\epsilon(x)$ de sorte que $\cos x = 1+u$
\pause
  \item $u^2 = \big(-\frac{x^2}{2}+\frac{x^4}{24} +x^5\epsilon(x)\big)^2 \pause
= \frac{x^4}{4}+x^5\epsilon(x)$
\pause
  \item $u^3 = x^5\epsilon(x)$
\pause

  \item   \ \\ \vspace*{-3ex}
$\begin{array}{l@{\vrule depth 1.2ex height 3ex width 0mm }}
\frac{1}{\cos x} = \frac{1}{1+u} \pause =1-u+u^2-u^3+u^3\epsilon(u) \\ \pause
=1+\frac{x^2}{2}-\frac{x^4}{24}+\frac{x^4}{4}+x^5\epsilon(x) \pause
=1+\frac{x^2}{2}+\frac{5}{24}x^4+x^5\epsilon(x)           
\end{array}$

\pause
  \item   \ \\ \vspace*{-3ex}
$\begin{array}{l@{\vrule depth 1.2ex height 3ex width 0mm }} 
\tan x= \sin x \times \frac1{\cos x} \\ \pause 
=  \big(x-\frac{x^3}{6}+\frac{x^5}{120}+x^5\epsilon(x)\big)\times\big(1+\frac{x^2}{2}+\frac{5}{24}x^4+x^5\epsilon(x)\big) \\ \pause
= \cdots  = x +\frac{x^3}{3}+\frac{2}{15}x^5+x^5\epsilon(x)
\end{array}$
\end{itemize}

\end{exemple}
\end{frame}



\begin{frame}

\begin{exemple}
DL de $\frac{1+x}{2+x}$ en $0$ à l'ordre $4$
\vspace*{-2ex}
\pause
$$\begin{array}{rcl@{\vrule depth 1.2ex height 3ex width 0mm }}
\frac{1+x}{2+x}  
  & = & \displaystyle (1+x)\tfrac12\frac{1}{1+\frac{x}{2}} \\ \pause
  & = & \frac12(1+x) \left( 1-\frac{x}{2}+\left(\frac{x}{2}\right)^2-\left(\frac{x}{2}\right)^3
+\left( \frac{x}{2} \right)^4 + o(x^4) \right) \\ \pause
  & = & \frac12+\frac{x}{4}-\frac{x^2}{8}+\frac{x^3}{16}-\frac{x^4}{32} + o(x^4) \\
\end{array}$$
\end{exemple}


\pause

\begin{exemple}
DL de $\frac{\sin x}{\sh x}$ en $0$ à l'ordre $4$
\vspace*{-3ex}
\pause
$$\begin{array}{rcl@{\vrule depth 1.2ex height 4ex width 0mm }}
\frac{\sin x}{\sh x} 
  & = & \frac{x-\frac{x^3}{3!} + \frac{x^5}{5!}+ o(x^5)}{x+\frac{x^3}{3!} + \frac{x^5}{5!} + o(x^5)}  \pause
   =  \frac{x\big(1-\frac{x^2}{3!} + \frac{x^4}{5!}+ o(x^4)\big)}{x\big(1+\frac{x^2}{3!}+ \frac{x^4}{5!} + o(x^4)\big)} \\ \pause
  & = & \big(1-\frac{x^2}{3!}  + \frac{x^4}{5!}+ o(x^4)\big) \times \frac{1}{1+\frac{x^2}{3!}+ \frac{x^4}{5!} + o(x^4)} \\ \pause
  & = & \quad \cdots \quad \pause =  1-\frac{x^2}{3}+\frac{x^4}{18} + o(x^4) \\
\end{array}
$$
\end{exemple}

\end{frame}

%---------------------------------------------------------------
\section{Intégration}

\begin{frame}

$$f(x)=c_0+c_1x+c_2x^2+\cdots+c_nx^n+x^n\epsilon(x)$$

\pause

\begin{theoreme}
Une primitive $F$ de $f$ admet un DL en $0$ à l'ordre $n+1$
\pause
$$F(x)= \uncover<4->{\alert<4>{F(0)}+}c_0 x+c_1\frac{x^2}{2}+ c_2\frac{x^3}{3}+\cdots+c_n\frac{x^{n+1}}{n+1} 
+x^{n+1}\eta(x)$$

où $\displaystyle\lim_{x\to 0}\eta(x)=0$
\end{theoreme}
\pause  
\pause

On intègre la partie polynomiale terme à terme 
puis on ajoute la constante $F(0)$

\end{frame}

\begin{frame}

\begin{exemple}
DL de $\arctan x$
{\small
\pause
\vspace*{-2ex}
$$\arctan' x=\frac{1}{1+x^2} \pause =\sum_{k=0}^{n}(-1)^kx^{2k}+x^{2n}\epsilon(x)$$
\pause
\vspace*{-1ex}
$$\arctan x= \uncover<5->{\alert<5>{0}+} \sum_{k=0}^{n}\frac{(-1)^k}{2k+1}x^{2k+1}+x^{2n+1}\epsilon(x)$$
\pause
\pause
\vspace*{-1ex}
$$\arctan x=x-\frac{x^3}{3}+\frac{x^5}{5}-\frac{x^7}{7} +\cdots$$
\vspace*{-2ex}
}
\end{exemple}

\pause

\begin{exemple}
DL de $\arcsin x$ en $0$ à l'ordre $5$
{\small
\pause

\vspace*{1ex}
\qquad 
$\begin{array}{rcl}
\arcsin' x 
 & = & (1-x^2)^{-\frac{1}{2}} \pause
=1-\frac{1}{2}(-x^2)
+\frac{-\frac{1}{2}(-\frac{1}{2}-1)}{2}(-x^2)^2+x^4\epsilon(x) \\
\pause
 & = & \vphantom{\Big(} 1+\frac{1}{2}x^2 + \frac{3}{8}x^4+x^4\epsilon(x)  \\
 \end{array}
$
\pause
\vspace*{-1ex}
$$\arcsin x =x+\tfrac{1}{6}x^3+\tfrac{3}{40}x^5+x^5\epsilon(x)$$
\vspace*{-2ex}
}
\end{exemple}

  
\end{frame}


%%%%%%%%%%%%%%%%%%%%%%%%%%%%%%%%%%%%%%%%%%%%%%%%%%%%%%%%%%%%%%%%
\section{Mini-exercices}

\begin{frame}

\begin{miniexercice}
\begin{enumerate}
  \item Calculer le DL en $0$ à l'ordre $3$ de $\exp(x) -\frac{1}{1+x}$, puis de $x\cos(2x)$ et $\cos(x)\times \sin(2x)$.
  \item Calculer le DL en $0$ à l'ordre $2$ de $\sqrt{1+2\cos x}$, puis de $\exp\big(\sqrt{1+2\cos x}\big)$.
  \item Calculer le DL en $0$ à l'ordre $3$ de $\ln(1+\sin x)$. Idem à l'ordre $6$ pour $\big(\ln(1+x^2)\big)^2$.
  \item Calculer le DL en $0$ à l'ordre $n$ de $\frac{\ln(1+x^3)}{x^3}$. Idem à l'ordre $3$ avec $\frac{e^x}{1+x}$.
  \item Par intégration retrouver la formule du DL de $\ln(1+x)$. Idem à l'ordre $3$ pour $\arccos x$.
\end{enumerate}
\end{miniexercice}
\end{frame}

\end{document}

%%%%%%%%%%%%%%%%%% PREAMBULE %%%%%%%%%%%%%%%%%%

\documentclass[aspectratio=169,utf8]{beamer}
%\documentclass[aspectratio=169,handout]{beamer}

\usetheme{Boadilla}
%\usecolortheme{seahorse}
%\usecolortheme[RGB={245,66,24}]{structure}
\useoutertheme{infolines}

% packages
\usepackage{amsfonts,amsmath,amssymb,amsthm}
\usepackage[utf8]{inputenc}
\usepackage[T1]{fontenc}
\usepackage{lmodern}

\usepackage[francais]{babel}
\usepackage{fancybox}
\usepackage{graphicx}

\usepackage{float}
\usepackage{xfrac}

%\usepackage[usenames, x11names]{xcolor}
\usepackage{pgfplots}
\usepackage{datetime}


% ----------------------------------------------------------------------
% Pour les images
\usepackage{tikz}
\usetikzlibrary{calc,shadows,arrows.meta,patterns,matrix}

\newcommand{\tikzinput}[1]{\input{figures/#1.tikz}}
% --- les figures avec échelle éventuel
\newcommand{\myfigure}[2]{% entrée : échelle, fichier(s) figure à inclure
\begin{center}\small%
\tikzstyle{every picture}=[scale=1.0*#1]% mise en échelle + 0% (automatiquement annulé à la fin du groupe)
#2%
\end{center}}



%-----  Package unités -----
\usepackage{siunitx}
\sisetup{locale = FR,detect-all,per-mode = symbol}

%\usepackage{mathptmx}
%\usepackage{fouriernc}
%\usepackage{newcent}
%\usepackage[mathcal,mathbf]{euler}

%\usepackage{palatino}
%\usepackage{newcent}
% \usepackage[mathcal,mathbf]{euler}



% \usepackage{hyperref}
% \hypersetup{colorlinks=true, linkcolor=blue, urlcolor=blue,
% pdftitle={Exo7 - Exercices de mathématiques}, pdfauthor={Exo7}}


%section
% \usepackage{sectsty}
% \allsectionsfont{\bf}
%\sectionfont{\color{Tomato3}\upshape\selectfont}
%\subsectionfont{\color{Tomato4}\upshape\selectfont}

%----- Ensembles : entiers, reels, complexes -----
\newcommand{\Nn}{\mathbb{N}} \newcommand{\N}{\mathbb{N}}
\newcommand{\Zz}{\mathbb{Z}} \newcommand{\Z}{\mathbb{Z}}
\newcommand{\Qq}{\mathbb{Q}} \newcommand{\Q}{\mathbb{Q}}
\newcommand{\Rr}{\mathbb{R}} \newcommand{\R}{\mathbb{R}}
\newcommand{\Cc}{\mathbb{C}} 
\newcommand{\Kk}{\mathbb{K}} \newcommand{\K}{\mathbb{K}}

%----- Modifications de symboles -----
\renewcommand{\epsilon}{\varepsilon}
\renewcommand{\Re}{\mathop{\text{Re}}\nolimits}
\renewcommand{\Im}{\mathop{\text{Im}}\nolimits}
%\newcommand{\llbracket}{\left[\kern-0.15em\left[}
%\newcommand{\rrbracket}{\right]\kern-0.15em\right]}

\renewcommand{\ge}{\geqslant}
\renewcommand{\geq}{\geqslant}
\renewcommand{\le}{\leqslant}
\renewcommand{\leq}{\leqslant}
\renewcommand{\epsilon}{\varepsilon}

%----- Fonctions usuelles -----
\newcommand{\ch}{\mathop{\text{ch}}\nolimits}
\newcommand{\sh}{\mathop{\text{sh}}\nolimits}
\renewcommand{\tanh}{\mathop{\text{th}}\nolimits}
\newcommand{\cotan}{\mathop{\text{cotan}}\nolimits}
\newcommand{\Arcsin}{\mathop{\text{arcsin}}\nolimits}
\newcommand{\Arccos}{\mathop{\text{arccos}}\nolimits}
\newcommand{\Arctan}{\mathop{\text{arctan}}\nolimits}
\newcommand{\Argsh}{\mathop{\text{argsh}}\nolimits}
\newcommand{\Argch}{\mathop{\text{argch}}\nolimits}
\newcommand{\Argth}{\mathop{\text{argth}}\nolimits}
\newcommand{\pgcd}{\mathop{\text{pgcd}}\nolimits} 


%----- Commandes divers ------
\newcommand{\ii}{\mathrm{i}}
\newcommand{\dd}{\text{d}}
\newcommand{\id}{\mathop{\text{id}}\nolimits}
\newcommand{\Ker}{\mathop{\text{Ker}}\nolimits}
\newcommand{\Card}{\mathop{\text{Card}}\nolimits}
\newcommand{\Vect}{\mathop{\text{Vect}}\nolimits}
\newcommand{\Mat}{\mathop{\text{Mat}}\nolimits}
\newcommand{\rg}{\mathop{\text{rg}}\nolimits}
\newcommand{\tr}{\mathop{\text{tr}}\nolimits}


%----- Structure des exercices ------

\newtheoremstyle{styleexo}% name
{2ex}% Space above
{3ex}% Space below
{}% Body font
{}% Indent amount 1
{\bfseries} % Theorem head font
{}% Punctuation after theorem head
{\newline}% Space after theorem head 2
{}% Theorem head spec (can be left empty, meaning ‘normal’)

%\theoremstyle{styleexo}
\newtheorem{exo}{Exercice}
\newtheorem{ind}{Indications}
\newtheorem{cor}{Correction}


\newcommand{\exercice}[1]{} \newcommand{\finexercice}{}
%\newcommand{\exercice}[1]{{\tiny\texttt{#1}}\vspace{-2ex}} % pour afficher le numero absolu, l'auteur...
\newcommand{\enonce}{\begin{exo}} \newcommand{\finenonce}{\end{exo}}
\newcommand{\indication}{\begin{ind}} \newcommand{\finindication}{\end{ind}}
\newcommand{\correction}{\begin{cor}} \newcommand{\fincorrection}{\end{cor}}

\newcommand{\noindication}{\stepcounter{ind}}
\newcommand{\nocorrection}{\stepcounter{cor}}

\newcommand{\fiche}[1]{} \newcommand{\finfiche}{}
\newcommand{\titre}[1]{\centerline{\large \bf #1}}
\newcommand{\addcommand}[1]{}
\newcommand{\video}[1]{}

% Marge
\newcommand{\mymargin}[1]{\marginpar{{\small #1}}}

\def\noqed{\renewcommand{\qedsymbol}{}}


%----- Presentation ------
\setlength{\parindent}{0cm}

%\newcommand{\ExoSept}{\href{http://exo7.emath.fr}{\textbf{\textsf{Exo7}}}}

\definecolor{myred}{rgb}{0.93,0.26,0}
\definecolor{myorange}{rgb}{0.97,0.58,0}
\definecolor{myyellow}{rgb}{1,0.86,0}

\newcommand{\LogoExoSept}[1]{  % input : echelle
{\usefont{U}{cmss}{bx}{n}
\begin{tikzpicture}[scale=0.1*#1,transform shape]
  \fill[color=myorange] (0,0)--(4,0)--(4,-4)--(0,-4)--cycle;
  \fill[color=myred] (0,0)--(0,3)--(-3,3)--(-3,0)--cycle;
  \fill[color=myyellow] (4,0)--(7,4)--(3,7)--(0,3)--cycle;
  \node[scale=5] at (3.5,3.5) {Exo7};
\end{tikzpicture}}
}


\newcommand{\debutmontitre}{
  \author{} \date{} 
  \thispagestyle{empty}
  \hspace*{-10ex}
  \begin{minipage}{\textwidth}
    \titlepage  
  \vspace*{-2.5cm}
  \begin{center}
    \LogoExoSept{2.5}
  \end{center}
  \end{minipage}

  \vspace*{-0cm}
  
  % Astuce pour que le background ne soit pas discrétisé lors de la conversion pdf -> png
\begin{tikzpicture}
        \fill[opacity=0,green!60!black] (0,0)--++(0,0)--++(0,0)--++(0,0)--cycle; 
\end{tikzpicture}

% toc S'affiche trop tot :
% \tableofcontents[hideallsubsections, pausesections]
}

\newcommand{\finmontitre}{
  \end{frame}
  \setcounter{framenumber}{0}
} % ne marche pas pour une raison obscure

%----- Commandes supplementaires ------

% \usepackage[landscape]{geometry}
% \geometry{top=1cm, bottom=3cm, left=2cm, right=10cm, marginparsep=1cm
% }
% \usepackage[a4paper]{geometry}
% \geometry{top=2cm, bottom=2cm, left=2cm, right=2cm, marginparsep=1cm
% }

%\usepackage{standalone}


% New command Arnaud -- november 2011
\setbeamersize{text margin left=24ex}
% si vous modifier cette valeur il faut aussi
% modifier le decalage du titre pour compenser
% (ex : ici =+10ex, titre =-5ex

\theoremstyle{definition}
%\newtheorem{proposition}{Proposition}
%\newtheorem{exemple}{Exemple}
%\newtheorem{theoreme}{Théorème}
%\newtheorem{lemme}{Lemme}
%\newtheorem{corollaire}{Corollaire}
%\newtheorem*{remarque*}{Remarque}
%\newtheorem*{miniexercice}{Mini-exercices}
%\newtheorem{definition}{Définition}

% Commande tikz
\usetikzlibrary{calc}
\usetikzlibrary{patterns,arrows}
\usetikzlibrary{matrix}
\usetikzlibrary{fadings} 

%definition d'un terme
\newcommand{\defi}[1]{{\color{myorange}\textbf{\emph{#1}}}}
\newcommand{\evidence}[1]{{\color{blue}\textbf{\emph{#1}}}}
\newcommand{\assertion}[1]{\emph{\og#1\fg}}  % pour chapitre logique
%\renewcommand{\contentsname}{Sommaire}
\renewcommand{\contentsname}{}
\setcounter{tocdepth}{2}



%------ Encadrement ------

\usepackage{fancybox}


\newcommand{\mybox}[1]{
\setlength{\fboxsep}{7pt}
\begin{center}
\shadowbox{#1}
\end{center}}

\newcommand{\myboxinline}[1]{
\setlength{\fboxsep}{5pt}
\raisebox{-10pt}{
\shadowbox{#1}
}
}

%--------------- Commande beamer---------------
\newcommand{\beameronly}[1]{#1} % permet de mettre des pause dans beamer pas dans poly


\setbeamertemplate{navigation symbols}{}
\setbeamertemplate{footline}  % tiré du fichier beamerouterinfolines.sty
{
  \leavevmode%
  \hbox{%
  \begin{beamercolorbox}[wd=.333333\paperwidth,ht=2.25ex,dp=1ex,center]{author in head/foot}%
    % \usebeamerfont{author in head/foot}\insertshortauthor%~~(\insertshortinstitute)
    \usebeamerfont{section in head/foot}{\bf\insertshorttitle}
  \end{beamercolorbox}%
  \begin{beamercolorbox}[wd=.333333\paperwidth,ht=2.25ex,dp=1ex,center]{title in head/foot}%
    \usebeamerfont{section in head/foot}{\bf\insertsectionhead}
  \end{beamercolorbox}%
  \begin{beamercolorbox}[wd=.333333\paperwidth,ht=2.25ex,dp=1ex,right]{date in head/foot}%
    % \usebeamerfont{date in head/foot}\insertshortdate{}\hspace*{2em}
    \insertframenumber{} / \inserttotalframenumber\hspace*{2ex} 
  \end{beamercolorbox}}%
  \vskip0pt%
}


\definecolor{mygrey}{rgb}{0.5,0.5,0.5}
\setlength{\parindent}{0cm}
%\DeclareTextFontCommand{\helvetica}{\fontfamily{phv}\selectfont}

% background beamer
\definecolor{couleurhaut}{rgb}{0.85,0.9,1}  % creme
\definecolor{couleurmilieu}{rgb}{1,1,1}  % vert pale
\definecolor{couleurbas}{rgb}{0.85,0.9,1}  % blanc
\setbeamertemplate{background canvas}[vertical shading]%
[top=couleurhaut,middle=couleurmilieu,midpoint=0.4,bottom=couleurbas] 
%[top=fondtitre!05,bottom=fondtitre!60]



\makeatletter
\setbeamertemplate{theorem begin}
{%
  \begin{\inserttheoremblockenv}
  {%
    \inserttheoremheadfont
    \inserttheoremname
    \inserttheoremnumber
    \ifx\inserttheoremaddition\@empty\else\ (\inserttheoremaddition)\fi%
    \inserttheorempunctuation
  }%
}
\setbeamertemplate{theorem end}{\end{\inserttheoremblockenv}}

\newenvironment{theoreme}[1][]{%
   \setbeamercolor{block title}{fg=structure,bg=structure!40}
   \setbeamercolor{block body}{fg=black,bg=structure!10}
   \begin{block}{{\bf Th\'eor\`eme }#1}
}{%
   \end{block}%
}


\newenvironment{proposition}[1][]{%
   \setbeamercolor{block title}{fg=structure,bg=structure!40}
   \setbeamercolor{block body}{fg=black,bg=structure!10}
   \begin{block}{{\bf Proposition }#1}
}{%
   \end{block}%
}

\newenvironment{corollaire}[1][]{%
   \setbeamercolor{block title}{fg=structure,bg=structure!40}
   \setbeamercolor{block body}{fg=black,bg=structure!10}
   \begin{block}{{\bf Corollaire }#1}
}{%
   \end{block}%
}

\newenvironment{mydefinition}[1][]{%
   \setbeamercolor{block title}{fg=structure,bg=structure!40}
   \setbeamercolor{block body}{fg=black,bg=structure!10}
   \begin{block}{{\bf Définition} #1}
}{%
   \end{block}%
}

\newenvironment{lemme}[0]{%
   \setbeamercolor{block title}{fg=structure,bg=structure!40}
   \setbeamercolor{block body}{fg=black,bg=structure!10}
   \begin{block}{\bf Lemme}
}{%
   \end{block}%
}

\newenvironment{remarque}[1][]{%
   \setbeamercolor{block title}{fg=black,bg=structure!20}
   \setbeamercolor{block body}{fg=black,bg=structure!5}
   \begin{block}{Remarque #1}
}{%
   \end{block}%
}


\newenvironment{exemple}[1][]{%
   \setbeamercolor{block title}{fg=black,bg=structure!20}
   \setbeamercolor{block body}{fg=black,bg=structure!5}
   \begin{block}{{\bf Exemple }#1}
}{%
   \end{block}%
}


\newenvironment{miniexercice}[0]{%
   \setbeamercolor{block title}{fg=structure,bg=structure!20}
   \setbeamercolor{block body}{fg=black,bg=structure!5}
   \begin{block}{Mini-exercices}
}{%
   \end{block}%
}


\newenvironment{tp}[0]{%
   \setbeamercolor{block title}{fg=structure,bg=structure!40}
   \setbeamercolor{block body}{fg=black,bg=structure!10}
   \begin{block}{\bf Travaux pratiques}
}{%
   \end{block}%
}
\newenvironment{exercicecours}[1][]{%
   \setbeamercolor{block title}{fg=structure,bg=structure!40}
   \setbeamercolor{block body}{fg=black,bg=structure!10}
   \begin{block}{{\bf Exercice }#1}
}{%
   \end{block}%
}
\newenvironment{algo}[1][]{%
   \setbeamercolor{block title}{fg=structure,bg=structure!40}
   \setbeamercolor{block body}{fg=black,bg=structure!10}
   \begin{block}{{\bf Algorithme}\hfill{\color{gray}\texttt{#1}}}
}{%
   \end{block}%
}


\setbeamertemplate{proof begin}{
   \setbeamercolor{block title}{fg=black,bg=structure!20}
   \setbeamercolor{block body}{fg=black,bg=structure!5}
   \begin{block}{{\footnotesize Démonstration}}
   \footnotesize
   \smallskip}
\setbeamertemplate{proof end}{%
   \end{block}}
\setbeamertemplate{qed symbol}{\openbox}


\makeatother
\usecolortheme[RGB={179,179,12}]{structure}

%%%%%%%%%%%%%%%%%%%%%%%%%%%%%%%%%%%%%%%%%%%%%%%%%%%%%%%%%%%%%
%%%%%%%%%%%%%%%%%%%%%%%%%%%%%%%%%%%%%%%%%%%%%%%%%%%%%%%%%%%%%


\begin{document}


\title{{\bf Développements limités}}
\subtitle{Développements limités au voisinage d'un point}

\begin{frame}
  
  \debutmontitre

  \pause

{\footnotesize
\hfill
\setbeamercovered{transparent=50}
\begin{minipage}{0.6\textwidth}
  \begin{itemize}
    \item<3-> Définition
    \item<4-> Unicité
    \item<5-> DL des fonctions usuelles
    \item<6-> DL en un point quelconque
  \end{itemize}
\end{minipage}
}

\end{frame}

\setcounter{framenumber}{0}


%%%%%%%%%%%%%%%%%%%%%%%%%%%%%%%%%%%%%%%%%%%%%%%%%%%%%%%%%%%%%%%%


%---------------------------------------------------------------
\section{Définition et existence}

\begin{frame}
\begin{itemize}
\item 
$f: I \to \Rr$ admet un \defi{développement limité} (\defi{DL}) au point $a$ et à l'ordre $n$, 
s'il existe des réels $c_0, c_1,\ldots,c_n$
et une fonction $\epsilon : I \to \Rr$ telle que $\lim_{x\to a} \epsilon(x)=0$ et
$$f(x)=c_0+c_1 (x-a)+\cdots+c_n(x-a)^n+(x-a)^n\epsilon(x)$$

\pause

\item $c_0+c_1(x-a)+\cdots+c_n(x-a)^n$ est la \defi{partie polynomiale}

\pause

\item $(x-a)^n\epsilon(x)$ est le \defi{reste}
\end{itemize}


\pause

\begin{proposition}
Si $f$ est de classe $\mathcal{C}^n$ alors $f$ admet un DL au point $a$ à l'ordre $n$

\hfil $f(x)= f(a)+\frac{f'(a)}{1!}(x-a)+ \frac{f''(a)}{2!}(x-a)^2+\cdots
+\frac{f^{(n)}(a)}{n!}(x-a)^n+(x-a)^n\epsilon(x)$
\end{proposition}

\pause

\begin{itemize}

  \item Preuve : formule de Taylor-Young : $c_k = \frac{f^{(k)}(a)}{k!}$

\pause

  \item $f(x)= f(0)+f'(0)x+f''(0)\frac{x^2}{2!}+\cdots+f^{(n)}(0)\frac{x^n}{n!} + x^n\epsilon(x)$
\end{itemize}


\end{frame}

%---------------------------------------------------------------
\section{Unicité}

\begin{frame}


\begin{proposition}
Si $f$ admet un DL alors ce DL est unique
\end{proposition}

\pause

\begin{proof}
{\center
$\begin{array}{rcl}
f(x) 
& = & c_0+c_1(x-a)+\cdots+c_n(x-a)^n+(x-a)^n\epsilon_1(x) \\
\pause
& = & d_0+d_1(x-a)+\cdots+d_n(x-a)^n+(x-a)^n\epsilon_2(x)   
\end{array}
$

\pause
\medskip

$(d_0-c_0)+(d_1-c_1)(x-a)+\cdots+(d_n-c_n)(x-a)^n+(x-a)^n(\epsilon_2(x)-\epsilon_1(x))=0$
}

\pause
\begin{itemize}
  \item $x=a$ \pause donne $d_0-c_0=0$ \pause et on divise l'égalité par $x-a$
\pause
  \item $(d_1-c_1)+(d_2-c_2)(x-a)+\cdots+(d_n-c_n)(x-a)^{n-1}+(x-a)^{n-1}(\epsilon_2-\epsilon_1)(x)=0$
\pause
  \item $d_1-c_1=0$,...
\pause
  \item $c_0=d_0$, $c_1=d_1$, \ldots, $c_n=d_n$ \pause et donc $\epsilon_1(x)=\epsilon_2(x)$
\end{itemize}

\end{proof}
\end{frame}



\begin{frame}

\begin{corollaire}
Si $f$ est paire (resp. impaire) alors la partie polynomiale de son DL en $0$
ne contient que des monômes de degrés pairs (resp. impairs) 
\end{corollaire}

\pause

Exemple : $\cos x=1-\frac{x^2}{2!}+\frac{x^4}{4!}-\frac{x^6}{6!}+\cdots$

\pause

\begin{proof}
Pour $f$ paire
\medskip

$\begin{array}{rccl}
  & f(x)  & = & c_0+c_1 x +c_2x^2+c_3x^3+\cdots+c_nx^n+x^n\epsilon(x)  \\
  & \pause
\uncover<5->{\alert<5>{\mathbf\parallel}} && \\
  & f(-x) & = & c_0-c_1 x+c_2x^2-c_3x^3 +\cdots+(-1)^nc_nx^n+x^n\epsilon(x)
\end{array}$

\pause

\pause
\medskip

Par l'unicité : $c_1=-c_1$, $c_3=-c_3$, \ldots \ 
\pause et donc $c_1=0$, $c_3=0$,\ldots
\end{proof}

\pause

\begin{itemize}
  \item $c_k = \frac{f^{(k)}(a)}{k!}$
\pause
  \item En particulier $c_0=f(a)$ et $c_1=f'(a)$ 
\pause
  \item $y=c_0+c_1(x-a)$ est la tangente 
\end{itemize}
 
\end{frame}

%---------------------------------------------------------------
\section{DL des fonctions usuelles à l'origine}

\begin{frame}

\begin{center}

\mybox{$\exp x=1+\frac{x}{1!}+\frac{x^2}{2!}+\frac{x^3}{3!}+\cdots+\frac{x^n}{n!}
+x^n\epsilon(x)$}

\pause
\smallskip

$\exp x=\sum_ {k=0}^n \frac{x^k}{k!} \ \ + o(x^n)$

\bigskip

\smallskip
\pause

\mybox{$\ln(1+x)=x-\frac{x^2}{2}+\frac{x^3}{3}-\cdots
+(-1)^{n-1}\frac{x^{n}}{n} +x^{n}\epsilon(x)$}

\pause

$\ln(1+x)=\sum_ {k=1}^n (-1)^{k-1}\frac{x^k}{k} \ \ + o(x^n)$

\bigskip
\pause

$\ch x=1+\frac{x^2}{2!}+\frac{x^4}{4!}+\cdots+\frac{x^{2n}}{(2n)!}
+x^{2n+1}\epsilon(x)$ 

\smallskip
\pause


$\sh x=\frac{x}{1!}+\frac{x^3}{3!}+\frac{x^5}{5!}+\cdots
+\frac{x^{2n+1}}{(2n+1)!} 
+x^{2n+2}\epsilon(x)$

\smallskip
\pause

$\cos x=1-\frac{x^2}{2!}+\frac{x^4}{4!}-\cdots+(-1)^n\frac{x^{2n}}{(2n)!}
+x^{2n+1}\epsilon(x)$ 

\smallskip
\pause

$\sin x=\frac{x}{1!}-\frac{x^3}{3!}+\frac{x^5}{5!}-\cdots
+(-1)^n\frac{x^{2n+1}}{(2n+1)!} 
+x^{2n+2}\epsilon(x)$
\end{center}
\end{frame}


\begin{frame}

\begin{center}


\mybox{$(1+x)^{\alpha}=1+\alpha x+\frac{\alpha(\alpha-1)}{2!}x^2+\cdots
+\frac{\alpha(\alpha-1)...(\alpha-n+1)}{n!}x^n+x^n\epsilon(x)$}

\bigskip
\pause

${\displaystyle \frac{1}{1+x}}=1-x+x^2-x^3+\cdots+(-1)^nx^n+x^n\epsilon(x)$

\smallskip
\pause

${\displaystyle \frac{1}{1-x}} = 1+x+x^2+\cdots+x^n+x^n\epsilon(x)$

\medskip
\pause

$\sqrt{1+x}  =\ \ 1 + \frac{x}{2} - \frac{1}{8}x^2+ \cdots +
(-1)^{n-1} \frac{1\cdot1\cdot3\cdot5\cdots(2n-3)}{2^n n!}x^n\ \  + x^n\epsilon(x)$
\end{center}


\end{frame}

%---------------------------------------------------------------
\section{DL des fonctions en un point quelconque}

\begin{frame}

\evidence{DL de $f$ en $a$}

\pause

On se ramène à un DL au voisinage de $0$ en posant $h=x-a$

\pause

\begin{exemple}
\begin{enumerate}
  \item DL de $\exp x$ en $1$ 
\pause \qquad 
On pose $h=x-1$ 

\pause
\medskip

\hspace*{-2em}
$\begin{array}{rcl}
  \exp x & = & \pause \exp( 1+ (x-1) ) \pause = \exp(1) \exp (x-1) \pause = e \exp h \\
\pause
& = & e \left(1+h+ \frac{h^2}{2!} + \cdots + \frac{h^n}{n!}+h^n\epsilon(h)\right) \\
\pause
& =  & e \left(1+(x-1)+\frac{(x-1)^2}{2!}+\cdots
+\frac{(x-1)^n}{n!}+(x-1)^n\epsilon(x-1)\right) \\
\end{array}$
\pause
avec  \quad $\lim_{x\to1}\epsilon(x-1)=0$
\pause

  \item DL de $\ln(1+3x)$ en $1$ à l'ordre $3$
\pause \qquad  
$h=x-1$

\pause
\medskip

\hspace*{-3em}
$\begin{array}{rl}
\ln(1+3x)\!\!\! &= \pause \ln\big(1+3(1+h)\big) \pause =  \ln(4 + 3h) \\
 \pause 
&= \ln\big(4 \cdot (1+\frac{3h}{4})\big) \pause = \ln 4 + \ln\big(1+\frac{3h}{4}\big) \\
\pause
&= \ln 4 + \frac{3h}{4} - \frac12 \big(\frac{3h}{4} \big)^2
+ \frac13 \big(\frac{3h}{4} \big)^3 + h^3 \epsilon(h) \\
\pause
&= \ln 4 + \frac{3(x-1)}{4} - \frac{9}{32}(x-1)^2
+ \frac{9}{64}(x-1)^3 + (x-1)^3 \epsilon(x\!-\!1)
\end{array}$
\pause
 où $\lim_{x\to1}\epsilon(x-1)=0$

\end{enumerate}
\end{exemple}
  
\end{frame}


%%%%%%%%%%%%%%%%%%%%%%%%%%%%%%%%%%%%%%%%%%%%%%%%%%%%%%%%%%%%%%%%
\section{Mini-exercices}

\begin{frame}

\begin{miniexercice}
\begin{enumerate}
  \item Calculer le DL en $0$ de $x\mapsto \ch x$ par la formule de Taylor-Young.
Retrouver ce DL en utilisant que $\ch x = \frac{e^x-e^{-x}}{2}$.
  \item Écrire le DL en $0$ à l'ordre $3$ de $\sqrt[3]{1+x}$. Idem avec $\frac{1}{\sqrt{1+x}}$.
  \item Écrire le DL en $2$ à l'ordre $2$ de $\sqrt{x}$.
  \item Justifier l'expression du DL de $\frac{1}{1-x}$ à l'aide de l'unicité des DL 
de la somme d'une suite géométrique.
\end{enumerate}
\end{miniexercice}
\end{frame}

\end{document}

%%%%%%%%%%%%%%%%%% PREAMBULE %%%%%%%%%%%%%%%%%%


\documentclass[12pt]{article}

\usepackage{amsfonts,amsmath,amssymb,amsthm}
\usepackage[utf8]{inputenc}
\usepackage[T1]{fontenc}
\usepackage[francais]{babel}


% packages
\usepackage{amsfonts,amsmath,amssymb,amsthm}
\usepackage[utf8]{inputenc}
\usepackage[T1]{fontenc}
%\usepackage{lmodern}

\usepackage[francais]{babel}
\usepackage{fancybox}
\usepackage{graphicx}

\usepackage{float}

%\usepackage[usenames, x11names]{xcolor}
\usepackage{tikz}
\usepackage{datetime}

\usepackage{mathptmx}
%\usepackage{fouriernc}
%\usepackage{newcent}
\usepackage[mathcal,mathbf]{euler}

%\usepackage{palatino}
%\usepackage{newcent}


% Commande spéciale prompteur

%\usepackage{mathptmx}
%\usepackage[mathcal,mathbf]{euler}
%\usepackage{mathpple,multido}

\usepackage[a4paper]{geometry}
\geometry{top=2cm, bottom=2cm, left=1cm, right=1cm, marginparsep=1cm}

\newcommand{\change}{{\color{red}\rule{\textwidth}{1mm}\\}}

\newcounter{mydiapo}

\newcommand{\diapo}{\newpage
\hfill {\normalsize  Diapo \themydiapo \quad \texttt{[\jobname]}} \\
\stepcounter{mydiapo}}


%%%%%%% COULEURS %%%%%%%%%%

% Pour blanc sur noir :
%\pagecolor[rgb]{0.5,0.5,0.5}
% \pagecolor[rgb]{0,0,0}
% \color[rgb]{1,1,1}



%\DeclareFixedFont{\myfont}{U}{cmss}{bx}{n}{18pt}
\newcommand{\debuttexte}{
%%%%%%%%%%%%% FONTES %%%%%%%%%%%%%
\renewcommand{\baselinestretch}{1.5}
\usefont{U}{cmss}{bx}{n}
\bfseries

% Taille normale : commenter le reste !
%Taille Arnaud
%\fontsize{19}{19}\selectfont

% Taille Barbara
%\fontsize{21}{22}\selectfont

%Taille François
\fontsize{25}{30}\selectfont

%Taille Pascal
%\fontsize{25}{30}\selectfont

%Taille Laura
%\fontsize{30}{35}\selectfont


%\myfont
%\usefont{U}{cmss}{bx}{n}

%\Huge
%\addtolength{\parskip}{\baselineskip}
}


% \usepackage{hyperref}
% \hypersetup{colorlinks=true, linkcolor=blue, urlcolor=blue,
% pdftitle={Exo7 - Exercices de mathématiques}, pdfauthor={Exo7}}


%section
% \usepackage{sectsty}
% \allsectionsfont{\bf}
%\sectionfont{\color{Tomato3}\upshape\selectfont}
%\subsectionfont{\color{Tomato4}\upshape\selectfont}

%----- Ensembles : entiers, reels, complexes -----
\newcommand{\Nn}{\mathbb{N}} \newcommand{\N}{\mathbb{N}}
\newcommand{\Zz}{\mathbb{Z}} \newcommand{\Z}{\mathbb{Z}}
\newcommand{\Qq}{\mathbb{Q}} \newcommand{\Q}{\mathbb{Q}}
\newcommand{\Rr}{\mathbb{R}} \newcommand{\R}{\mathbb{R}}
\newcommand{\Cc}{\mathbb{C}} 
\newcommand{\Kk}{\mathbb{K}} \newcommand{\K}{\mathbb{K}}

%----- Modifications de symboles -----
\renewcommand{\epsilon}{\varepsilon}
\renewcommand{\Re}{\mathop{\text{Re}}\nolimits}
\renewcommand{\Im}{\mathop{\text{Im}}\nolimits}
%\newcommand{\llbracket}{\left[\kern-0.15em\left[}
%\newcommand{\rrbracket}{\right]\kern-0.15em\right]}

\renewcommand{\ge}{\geqslant}
\renewcommand{\geq}{\geqslant}
\renewcommand{\le}{\leqslant}
\renewcommand{\leq}{\leqslant}

%----- Fonctions usuelles -----
\newcommand{\ch}{\mathop{\mathrm{ch}}\nolimits}
\newcommand{\sh}{\mathop{\mathrm{sh}}\nolimits}
\renewcommand{\tanh}{\mathop{\mathrm{th}}\nolimits}
\newcommand{\cotan}{\mathop{\mathrm{cotan}}\nolimits}
\newcommand{\Arcsin}{\mathop{\mathrm{Arcsin}}\nolimits}
\newcommand{\Arccos}{\mathop{\mathrm{Arccos}}\nolimits}
\newcommand{\Arctan}{\mathop{\mathrm{Arctan}}\nolimits}
\newcommand{\Argsh}{\mathop{\mathrm{Argsh}}\nolimits}
\newcommand{\Argch}{\mathop{\mathrm{Argch}}\nolimits}
\newcommand{\Argth}{\mathop{\mathrm{Argth}}\nolimits}
\newcommand{\pgcd}{\mathop{\mathrm{pgcd}}\nolimits} 

\newcommand{\Card}{\mathop{\text{Card}}\nolimits}
\newcommand{\Ker}{\mathop{\text{Ker}}\nolimits}
\newcommand{\id}{\mathop{\text{id}}\nolimits}
\newcommand{\ii}{\mathrm{i}}
\newcommand{\dd}{\mathrm{d}}
\newcommand{\Vect}{\mathop{\text{Vect}}\nolimits}
\newcommand{\Mat}{\mathop{\mathrm{Mat}}\nolimits}
\newcommand{\rg}{\mathop{\text{rg}}\nolimits}
\newcommand{\tr}{\mathop{\text{tr}}\nolimits}
\newcommand{\ppcm}{\mathop{\text{ppcm}}\nolimits}

%----- Structure des exercices ------

\newtheoremstyle{styleexo}% name
{2ex}% Space above
{3ex}% Space below
{}% Body font
{}% Indent amount 1
{\bfseries} % Theorem head font
{}% Punctuation after theorem head
{\newline}% Space after theorem head 2
{}% Theorem head spec (can be left empty, meaning ‘normal’)

%\theoremstyle{styleexo}
\newtheorem{exo}{Exercice}
\newtheorem{ind}{Indications}
\newtheorem{cor}{Correction}


\newcommand{\exercice}[1]{} \newcommand{\finexercice}{}
%\newcommand{\exercice}[1]{{\tiny\texttt{#1}}\vspace{-2ex}} % pour afficher le numero absolu, l'auteur...
\newcommand{\enonce}{\begin{exo}} \newcommand{\finenonce}{\end{exo}}
\newcommand{\indication}{\begin{ind}} \newcommand{\finindication}{\end{ind}}
\newcommand{\correction}{\begin{cor}} \newcommand{\fincorrection}{\end{cor}}

\newcommand{\noindication}{\stepcounter{ind}}
\newcommand{\nocorrection}{\stepcounter{cor}}

\newcommand{\fiche}[1]{} \newcommand{\finfiche}{}
\newcommand{\titre}[1]{\centerline{\large \bf #1}}
\newcommand{\addcommand}[1]{}
\newcommand{\video}[1]{}

% Marge
\newcommand{\mymargin}[1]{\marginpar{{\small #1}}}



%----- Presentation ------
\setlength{\parindent}{0cm}

%\newcommand{\ExoSept}{\href{http://exo7.emath.fr}{\textbf{\textsf{Exo7}}}}

\definecolor{myred}{rgb}{0.93,0.26,0}
\definecolor{myorange}{rgb}{0.97,0.58,0}
\definecolor{myyellow}{rgb}{1,0.86,0}

\newcommand{\LogoExoSept}[1]{  % input : echelle
{\usefont{U}{cmss}{bx}{n}
\begin{tikzpicture}[scale=0.1*#1,transform shape]
  \fill[color=myorange] (0,0)--(4,0)--(4,-4)--(0,-4)--cycle;
  \fill[color=myred] (0,0)--(0,3)--(-3,3)--(-3,0)--cycle;
  \fill[color=myyellow] (4,0)--(7,4)--(3,7)--(0,3)--cycle;
  \node[scale=5] at (3.5,3.5) {Exo7};
\end{tikzpicture}}
}



\theoremstyle{definition}
%\newtheorem{proposition}{Proposition}
%\newtheorem{exemple}{Exemple}
%\newtheorem{theoreme}{Théorème}
\newtheorem{lemme}{Lemme}
\newtheorem{corollaire}{Corollaire}
%\newtheorem*{remarque*}{Remarque}
%\newtheorem*{miniexercice}{Mini-exercices}
%\newtheorem{definition}{Définition}




%definition d'un terme
\newcommand{\defi}[1]{{\color{myorange}\textbf{\emph{#1}}}}
\newcommand{\evidence}[1]{{\color{blue}\textbf{\emph{#1}}}}



 %----- Commandes divers ------

\newcommand{\codeinline}[1]{\texttt{#1}}

%%%%%%%%%%%%%%%%%%%%%%%%%%%%%%%%%%%%%%%%%%%%%%%%%%%%%%%%%%%%%
%%%%%%%%%%%%%%%%%%%%%%%%%%%%%%%%%%%%%%%%%%%%%%%%%%%%%%%%%%%%%

\begin{document}

\debuttexte

%%%%%%%%%%%%%%%%%%%%%%%%%%%%%%%%%%%%%%%%%%%%%%%%%%%%%%%%%%%
\diapo

\change

Il existe une façon plus simple de montrer qu'un ensemble est un
groupe,

\change

 nous allons donc définir ce que sont des sous-groupes 
(qui sont eux-même des groupes).

\change

Nous regarderons quelques exemples


\change

Nous déterminerons tous les sous-groupes du groupe $(\Zz,+)$

\change

Nous terminons par la notion de sous-groupe engendré.

%%%%%%%%%%%%%%%%%%%%%%%%%%%%%%%%%%%%%%%%%%%%%%%%%%%%%%%%%%%
\diapo

On se donne un groupe $G$ avec une loi $\star$.

Un sous-ensemble $H$ de $G$ est un sous-groupe de $G$ si les trois points suivants
sont vérifiés :

\change

l'élément neutre $e$ de $G$ appartient à l'ensemble $H$

\change

pour chaque $x$ et chaque $y$ de $H$ alors le composé $x \star y$ est encore un élément de $H$

\change

pour chaque $x$ de $H$ alors l'inverse de $x$, $x^{-1}$ (qui a priori est dans $G$)
est en fait dans $H$.

\change

Notez qu'un sous-groupe $H$ devient lui-même un groupe
$(H,\star)$ avec la loi $\star$ qui est celle de $G$.


%%%%%%%%%%%%%%%%%%%%%%%%%%%%%%%%%%%%%%%%%%%%%%%%%%%%%%%%%%%
\diapo

Voici toute une série d'exemples :

L'ensemble $H=\Rr_+^*$ des réels strictement positifs est un sous-groupe
 du groupe $G=\Rr^*$ muni de la multiplication habituelle.

Vérifions les trois points :

\change

l'élément neutre $1$ du groupe $\Rr^*$ est bien un réel strictement positif donc appartient à $\Rr_+^*$

\change

si $x,y$ sont deux réels strictement positifs alors le produit $x \times y$ est aussi strictement positifs
(on reste donc dans  $\Rr_+^*$)

\change

idem pour l'inverse : si $x$ est strictement positif alors $1/x$ aussi


Donc $\Rr_+^*$ est un sous-groupe de $\Rr^*$.

\change

A vous de vérifier que les exemples suivants sont des sous-groupes.

l'ensemble $\mathbb{U}$ des nombres complexes de module $1$ est un sous-groupe de $(\Cc^*,\times)$

\change

$\Zz$ est un sous-groupe de $(\Rr,+)$

\change

N'importe quel groupe $G$ possède toujours au moins deux sous-groupes, appelé sous-groupes triviaux :

le sous-groupe formé du singleton élément neutre $\{e\}$ 

et le groupe $G$ tout entier

\change

 L'ensemble $\mathcal{R}$ des rotations du plan dont le centre est à l'origine 
est un sous-groupe du groupe des isométries $\mathcal{I}$

\change

 L'ensemble des matrices diagonales $\left(\begin{smallmatrix} a & 0 \\ 0 & d \\ \end{smallmatrix}\right)$
avec $a$ et $d$ non nuls est un sous-groupe de $(G\ell_2,\times)$



%%%%%%%%%%%%%%%%%%%%%%%%%%%%%%%%%%%%%%%%%%%%%%%%%%%%%%%%%%%
\diapo

[[ 2 ou 3 prises]]

Pour le groupe $(\Zz,+)$ nous allons déterminer tous ses sous-groupes.

En effet :

Les sous-groupes de $(\Zz,+)$ sont les $n\Zz$, pour $n\in \Zz$

\change

Je vous rappelle que $n\Zz$, pour $n$ fixé, est l'ensemble des multiples de $n$ 

c'est donc l'ensemble des $k\cdot n$ pour $k$ parcourant $\Zz$

\change

Par exemple $2\Zz$ est l'ensemble des multiples de $2$, ce sont les entiers pairs.

\change

Quant à $7\Zz$, c'est l'ensemble des multiples de $7$.

\change

Passons à la démonstration :

il y a une partie facile à vérifier, c'est que $n\Zz$ est bien un sous-groupe
de $\Zz$

En effet $0$ appartient à $n\Zz$, la somme de deux multiples de $n$ reste un multiple de $n$,
l'opposé d'un multiple de $n$ est aussi un multiple de $n$.

\change

Voyons l'autre sens : il s'agit de montrer que si $H$ est un sous-groupe de $\Zz$
alors il est de la forme $n\Zz$.

Tout d'abord si $H$ est réduit au singleton $\{0\}$ alors on a pour $n=0$, $H = 0\Zz$
et c'est fini.

Sinon, on note $n$ le minimum des petit $h$ strictement positifs appartenant à grand $H$.

Nous allons montrer que c'est ce $n$ qui donne $H=n\Zz$.

\change

Première inclusion : : par définition de $n$, $n\in H$ donc $2n=n+n$ aussi, $3n$ aussi
et $-n$, $-2n$ etc. également. Ainsi du fait que $H$ soit un sous-groupe alors $n\Zz$
est inclus dans $H$

\change

Pour l'inclusion inverse, nous devons montrer $H$ inclus dans $n\Zz$.

C'est la division euclidienne qui va être le point clé.

Pour chaque petit $h$ de grand $H$ on écrit la division euclidienne de petit $h$
par notre entier $n$ : 

$h = kn+r$

avec $k,r$ des entiers et le reste $r$ qui vérifie $0\le r < n$.

Mais $kn$ et petit $h$ sont des éléments du sous-groupe grand $H$

Ainsi $r$ qui s'écrit $h-kn$ est aussi un élément de grand $H$

Que peut valoir $r$ ?

 - $r$ est positif ou nul

 - $r$ est strictement plus petit que $n$

 - $r\in H$

Mais $n$ est le plus petit élément strictement positif de $H$.

La seule possibilité pour éviter une contradiction c'est que $r=0$.

Donc la division euclidienne devient $h=kn$ et $h$ est un multiple de $n$.

\change

Nous venons de montrer l'inclusion $H$ inclus dans $n\Zz$,
sachant que nous avions déjà montrer l'inclusion inverse
donc $H = n\Zz$.

Cela termine la preuve.


%%%%%%%%%%%%%%%%%%%%%%%%%%%%%%%%%%%%%%%%%%%%%%%%%%%%%%%%%%%
\diapo

Un point que vous pouvez esquiver en première lecture est la
notion de sous-groupe engendré.

Pour une partie $E$ quelconque d'un groupe $G$, le sous-groupe engendré par $E$ 
est le plus petit sous-groupe de $G$ contenant $E$.

\change

Voici un premier exemple, le groupe est $(\Rr^*,\times)$
et $E$ est le singleton  $\{2\}$ 

alors le sous-groupe $H$ engendré par $E$ doit contenir $2$,
et comme c'est un sous-groupe alors $H$ contient aussi 
$2\times 2= 2^2$, et aussi $2\times 2 \times 2 = 2^3$ et ainsi
de suite $H$ contient tous les $2^n$, 

mais $H$ doit contenir $\frac 12$ l'inverse de deux, puis $\frac 1{2^n}$.

On se convainc assez vite que le sous-groupe engendré par $2$ est 

$H = \{ 2^n \mid n \in \Zz\}$

\change

Pour le montrer rigoureusement il faut montrer que :

 $H$ est un sous-groupe, 

que $2 \in H$, 

et $H$ est bien le plus petit sous-groupe ayant ces propriétés 

c'est-à-dire que si $H'$ est un autre sous-groupe 
contenant $2$ alors $H \subset H'$


\change

Voici d'autres exemples avec le groupe $\Zz$ muni de l'addition.

le sous-groupe engendré par le singleton $\{2\}$ est $H_1 = 2\Zz$

\change

le sous-groupe engendré par $\{8,12\}$ est $H_2 = 4\Zz$ (qui contient bien $8$ et $12$)


\change
 plus généralement le sous-groupe de $\Zz$ engendré par deux entiers
$a$ et $b$ est $H = n\Zz$ où $n$ est le $\pgcd(a,b)$.


%%%%%%%%%%%%%%%%%%%%%%%%%%%%%%%%%%%%%%%%%%%%%%%%%%%%%%%%%%%
\diapo

Réviser les définitions à l'aide des exercices que voilà.



\end{document}
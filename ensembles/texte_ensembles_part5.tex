
%%%%%%%%%%%%%%%%%% PREAMBULE %%%%%%%%%%%%%%%%%%


\documentclass[12pt]{article}

\usepackage{amsfonts,amsmath,amssymb,amsthm}
\usepackage[utf8]{inputenc}
\usepackage[T1]{fontenc}
\usepackage[francais]{babel}


% packages
\usepackage{amsfonts,amsmath,amssymb,amsthm}
\usepackage[utf8]{inputenc}
\usepackage[T1]{fontenc}
%\usepackage{lmodern}

\usepackage[francais]{babel}
\usepackage{fancybox}
\usepackage{graphicx}

\usepackage{float}

%\usepackage[usenames, x11names]{xcolor}
\usepackage{tikz}
\usepackage{datetime}

\usepackage{mathptmx}
%\usepackage{fouriernc}
%\usepackage{newcent}
\usepackage[mathcal,mathbf]{euler}

%\usepackage{palatino}
%\usepackage{newcent}


% Commande spéciale prompteur

%\usepackage{mathptmx}
%\usepackage[mathcal,mathbf]{euler}
%\usepackage{mathpple,multido}

\usepackage[a4paper]{geometry}
\geometry{top=2cm, bottom=2cm, left=1cm, right=1cm, marginparsep=1cm}

\newcommand{\change}{{\color{red}\rule{\textwidth}{1mm}\\}}

\newcounter{mydiapo}

\newcommand{\diapo}{\newpage
\hfill {\normalsize  Diapo \themydiapo \quad \texttt{[\jobname]}} \\
\stepcounter{mydiapo}}


%%%%%%% COULEURS %%%%%%%%%%

% Pour blanc sur noir :
%\pagecolor[rgb]{0.5,0.5,0.5}
% \pagecolor[rgb]{0,0,0}
% \color[rgb]{1,1,1}



%\DeclareFixedFont{\myfont}{U}{cmss}{bx}{n}{18pt}
\newcommand{\debuttexte}{
%%%%%%%%%%%%% FONTES %%%%%%%%%%%%%
\renewcommand{\baselinestretch}{1.5}
\usefont{U}{cmss}{bx}{n}
\bfseries

% Taille normale : commenter le reste !
%Taille Arnaud
%\fontsize{19}{19}\selectfont

% Taille Barbara
%\fontsize{21}{22}\selectfont

%Taille François
\fontsize{25}{30}\selectfont

%Taille Pascal
%\fontsize{25}{30}\selectfont

%Taille Laura
%\fontsize{30}{35}\selectfont


%\myfont
%\usefont{U}{cmss}{bx}{n}

%\Huge
%\addtolength{\parskip}{\baselineskip}
}


% \usepackage{hyperref}
% \hypersetup{colorlinks=true, linkcolor=blue, urlcolor=blue,
% pdftitle={Exo7 - Exercices de mathématiques}, pdfauthor={Exo7}}


%section
% \usepackage{sectsty}
% \allsectionsfont{\bf}
%\sectionfont{\color{Tomato3}\upshape\selectfont}
%\subsectionfont{\color{Tomato4}\upshape\selectfont}

%----- Ensembles : entiers, reels, complexes -----
\newcommand{\Nn}{\mathbb{N}} \newcommand{\N}{\mathbb{N}}
\newcommand{\Zz}{\mathbb{Z}} \newcommand{\Z}{\mathbb{Z}}
\newcommand{\Qq}{\mathbb{Q}} \newcommand{\Q}{\mathbb{Q}}
\newcommand{\Rr}{\mathbb{R}} \newcommand{\R}{\mathbb{R}}
\newcommand{\Cc}{\mathbb{C}} 
\newcommand{\Kk}{\mathbb{K}} \newcommand{\K}{\mathbb{K}}

%----- Modifications de symboles -----
\renewcommand{\epsilon}{\varepsilon}
\renewcommand{\Re}{\mathop{\text{Re}}\nolimits}
\renewcommand{\Im}{\mathop{\text{Im}}\nolimits}
%\newcommand{\llbracket}{\left[\kern-0.15em\left[}
%\newcommand{\rrbracket}{\right]\kern-0.15em\right]}

\renewcommand{\ge}{\geqslant}
\renewcommand{\geq}{\geqslant}
\renewcommand{\le}{\leqslant}
\renewcommand{\leq}{\leqslant}

%----- Fonctions usuelles -----
\newcommand{\ch}{\mathop{\mathrm{ch}}\nolimits}
\newcommand{\sh}{\mathop{\mathrm{sh}}\nolimits}
\renewcommand{\tanh}{\mathop{\mathrm{th}}\nolimits}
\newcommand{\cotan}{\mathop{\mathrm{cotan}}\nolimits}
\newcommand{\Arcsin}{\mathop{\mathrm{Arcsin}}\nolimits}
\newcommand{\Arccos}{\mathop{\mathrm{Arccos}}\nolimits}
\newcommand{\Arctan}{\mathop{\mathrm{Arctan}}\nolimits}
\newcommand{\Argsh}{\mathop{\mathrm{Argsh}}\nolimits}
\newcommand{\Argch}{\mathop{\mathrm{Argch}}\nolimits}
\newcommand{\Argth}{\mathop{\mathrm{Argth}}\nolimits}
\newcommand{\pgcd}{\mathop{\mathrm{pgcd}}\nolimits} 

\newcommand{\Card}{\mathop{\text{Card}}\nolimits}
\newcommand{\Ker}{\mathop{\text{Ker}}\nolimits}
\newcommand{\id}{\mathop{\text{id}}\nolimits}
\newcommand{\ii}{\mathrm{i}}
\newcommand{\dd}{\mathrm{d}}
\newcommand{\Vect}{\mathop{\text{Vect}}\nolimits}
\newcommand{\Mat}{\mathop{\mathrm{Mat}}\nolimits}
\newcommand{\rg}{\mathop{\text{rg}}\nolimits}
\newcommand{\tr}{\mathop{\text{tr}}\nolimits}
\newcommand{\ppcm}{\mathop{\text{ppcm}}\nolimits}

%----- Structure des exercices ------

\newtheoremstyle{styleexo}% name
{2ex}% Space above
{3ex}% Space below
{}% Body font
{}% Indent amount 1
{\bfseries} % Theorem head font
{}% Punctuation after theorem head
{\newline}% Space after theorem head 2
{}% Theorem head spec (can be left empty, meaning ‘normal’)

%\theoremstyle{styleexo}
\newtheorem{exo}{Exercice}
\newtheorem{ind}{Indications}
\newtheorem{cor}{Correction}


\newcommand{\exercice}[1]{} \newcommand{\finexercice}{}
%\newcommand{\exercice}[1]{{\tiny\texttt{#1}}\vspace{-2ex}} % pour afficher le numero absolu, l'auteur...
\newcommand{\enonce}{\begin{exo}} \newcommand{\finenonce}{\end{exo}}
\newcommand{\indication}{\begin{ind}} \newcommand{\finindication}{\end{ind}}
\newcommand{\correction}{\begin{cor}} \newcommand{\fincorrection}{\end{cor}}

\newcommand{\noindication}{\stepcounter{ind}}
\newcommand{\nocorrection}{\stepcounter{cor}}

\newcommand{\fiche}[1]{} \newcommand{\finfiche}{}
\newcommand{\titre}[1]{\centerline{\large \bf #1}}
\newcommand{\addcommand}[1]{}
\newcommand{\video}[1]{}

% Marge
\newcommand{\mymargin}[1]{\marginpar{{\small #1}}}



%----- Presentation ------
\setlength{\parindent}{0cm}

%\newcommand{\ExoSept}{\href{http://exo7.emath.fr}{\textbf{\textsf{Exo7}}}}

\definecolor{myred}{rgb}{0.93,0.26,0}
\definecolor{myorange}{rgb}{0.97,0.58,0}
\definecolor{myyellow}{rgb}{1,0.86,0}

\newcommand{\LogoExoSept}[1]{  % input : echelle
{\usefont{U}{cmss}{bx}{n}
\begin{tikzpicture}[scale=0.1*#1,transform shape]
  \fill[color=myorange] (0,0)--(4,0)--(4,-4)--(0,-4)--cycle;
  \fill[color=myred] (0,0)--(0,3)--(-3,3)--(-3,0)--cycle;
  \fill[color=myyellow] (4,0)--(7,4)--(3,7)--(0,3)--cycle;
  \node[scale=5] at (3.5,3.5) {Exo7};
\end{tikzpicture}}
}



\theoremstyle{definition}
%\newtheorem{proposition}{Proposition}
%\newtheorem{exemple}{Exemple}
%\newtheorem{theoreme}{Théorème}
\newtheorem{lemme}{Lemme}
\newtheorem{corollaire}{Corollaire}
%\newtheorem*{remarque*}{Remarque}
%\newtheorem*{miniexercice}{Mini-exercices}
%\newtheorem{definition}{Définition}




%definition d'un terme
\newcommand{\defi}[1]{{\color{myorange}\textbf{\emph{#1}}}}
\newcommand{\evidence}[1]{{\color{blue}\textbf{\emph{#1}}}}



 %----- Commandes divers ------

\newcommand{\codeinline}[1]{\texttt{#1}}

%%%%%%%%%%%%%%%%%%%%%%%%%%%%%%%%%%%%%%%%%%%%%%%%%%%%%%%%%%%%%
%%%%%%%%%%%%%%%%%%%%%%%%%%%%%%%%%%%%%%%%%%%%%%%%%%%%%%%%%%%%%

\begin{document}

\debuttexte

%%%%%%%%%%%%%%%%%%%%%%%%%%%%%%%%%%%%%%%%%%%%%%%%%%%%%%%%%%%
\diapo

\change

\change

Nous commençons par définir ce qu'est une relation et une relation d'équivalence,

\change

nous verrons quelques exemples,

\change

nous continuons avec la notion de classes d'équivalence

\change

et terminons par un exemple très important l'ensemble $\Zz/n\Zz$.


%%%%%%%%%%%%%%%%%%%%%%%%%%%%%%%%%%%%%%%%%%%%%%%%%%%%%%%%%%%
\diapo


Définir une \defi{relation} sur un ensemble $E$, c'est associer à un couple 
$(x,y)$ d'éléments de $E$ la valeur <<Vrai>> (s'ils sont en relation) et la valeur <<Faux>> sinon

\change

On notera $x\mathcal{R}y$ si $x$ et $y$ sont en relation 

\change


Nous schématisons les éléments de $E$ par des points et si l'élément $x$ est en relation avec 
l'élément $y$ alors on dessine une flèche de $x$ vers $y$.

S'il n'y a pas de flèche de $x$ vers $y$ alors $x$ n'est pas en relation avec $y$.


%%%%%%%%%%%%%%%%%%%%%%%%%%%%%%%%%%%%%%%%%%%%%%%%%%%%%%%%%%%
\diapo

Une relation $\mathcal{R}$ est dite une relation d'équivalence si elle vérifie les trois points suivants.


1. la relation est réflexive : tout élément $x$ est en relation avec lui-même

\change

Symboliquement, de tout élément $x$ part une flèche qui arrive à $x$ lui-même

\change

2. la relation est symétrique : si $x$ est en relation avec $y$ alors $y$ doit aussi être en relation avec $x$

\change

Autrement dit s'il l'on a une flèche de $x$ vers $y$ alors il y a une flèche de $y$ vers $x$.

\change

3. la relation est transitive : si $x$ est en relation avec $y$, et que $y$ est lui-même en relation avec un
élément $z$, alors $x$ doit être en relation avec $z$

\change

Pour notre représentation cela signifie que si une première flèche va de $x$ vers $y$
et une deuxième va de $y$ vers $z$ alors il existe une flèche de $x$ vers $z$.

\change

Voici une représentation d'une ensemble $E$ constitué de $6$ points,
muni d'une relation $\mathcal{R}$ qui vérifie la réflexivité, la symétrie, la transitivité
et qui est donc une relation d'équivalence.

%%%%%%%%%%%%%%%%%%%%%%%%%%%%%%%%%%%%%%%%%%%%%%%%%%%%%%%%%%%
\diapo

Premier exemple :  <<\emph{être parallèle}>> est une relation d'équivalence 
sur l'ensemble des droites

\change

En effet la relation <<\emph{être parallèle}>> est réflexive : une droite est parallèle à elle-même

\change

La relation <<\emph{être parallèle}>> est symétrique :si une droite $D$ est parallèle à une droite $D'$ alors $D'$ est parallèle à $D$

\change

La relation <<\emph{être parallèle}>> est transitive : si $D$ parallèle à $D'$ et $D'$ parallèle à $D''$ 
alors $D$ est parallèle à $D''$

\change


Deuxième exemple : <<\emph{être du même âge}>> est une relation d'équivalence sur l'ensemble des personnes.
Vérifier les trois propriétés.

\change

Par contre la relation <<être perpendiculaire>> n'est pas une relation d'équivalence 
(ni la réflexivité, ni la transitivité ne sont vérifiées) 
 

\change

La relation $\le$ (sur $E=\Rr$ par exemple) n'est pas non plus une relation d'équivalence
(la symétrie n'est pas vérifiée).


%%%%%%%%%%%%%%%%%%%%%%%%%%%%%%%%%%%%%%%%%%%%%%%%%%%%%%%%%%%
\diapo

Revenons à un ensemble $E$ et soit $\mathcal{R}$ une relation d'équivalence.

Fixons $x\in E$. 

La \defi{classe d'équivalence} de $x$ est l'ensemble des éléments $y$ qui sont en relation avec $x$.

\change

Voyons un exemple.

Quelle est la classe d'équivalence du point $x$ ?

C'est l'ensemble des points qui sont en relations avec $x$, 
donc sur notre représentation ce sont tous les points reliés
à $x$ par une flèche.

\change

Il y a donc $4$ éléments dans la classe de $x$.

Même question avec $x'$. Quelle est sa classe d'équivalence ?

\change

La classe de $x'$ est constituée de deux éléments ! N'oubliez pas $x'$
qui est en relation avec lui-même donc appartient à $cl(x')$.


%%%%%%%%%%%%%%%%%%%%%%%%%%%%%%%%%%%%%%%%%%%%%%%%%%%%%%%%%%%
\diapo

Reprenons. La \defi{classe d'équivalence} de $x$ est l'ensemble des éléments $y$ qui sont en relation avec $x$.

C'est donc un sous-ensemble de $E$, et on le notera aussi parfois $\overline{x}$.

\change

Un élément $y$ de la classe d'équivalence de $x$ s'appelle un représentant de $cl(x)$.

\change

Voici quelques propriétés.

Tout d'abord si $x$ et $y$ sont en relation alors ils ont la même classe d'équivalence.
Réciproquement si deux classes d'équivalence sont des ensembles égaux alors $x$ et $y$ sont en relation.

\change

On vient de voir que si deux éléments sont en relation, les classes sont égales ;
si deux éléments ne sont pas en relation, les classes sont différentes.

Mais on a davantage si  $x$ et  $y$ ne sont pas en relation alors les classes de $x$ et $y$ sont disjointes :
l'intersection de $cl(x)$ avec $cl(y)$ est vide.

\change

Enfin si l'on prend un représentant de chacune des classes alors lorsque l'on considère
l'ensemble des classes de ces représentants cela forme une partition de $E$.


\change

Rappelons qu'une partition d'un ensemble $E$ est juste un découpage de l'ensemble 
$E$ est sous-ensemble $E_i$ de telle sorte que $E$ soit l'union de tous les $E_i$
et que deux sous-ensembles quelconque $E_i$ et $E_j$ ne s'intersectent pas.



%%%%%%%%%%%%%%%%%%%%%%%%%%%%%%%%%%%%%%%%%%%%%%%%%%%%%%%%%%%
\diapo

Pour notre exemple de relation <<\emph{être du même âge}>>, la classe d'équivalence d'une personne est simplement l'ensemble
des personnes ayant le même âge. 

Il y a donc une classe d'équivalence formée des personnes
de $18$ ans, une autre formée des personnes de $19$ ans,...

Pour cet exemple les trois assertions de la proposition précédente se lisent ainsi

\change

Deux personnes sont dans la même classe d'équivalence si et seulement si elles sont du même âge

\change

Deux personnes appartiennent soit à la même classe, soit à des classes disjointes

\change

Si on choisit une personne de chaque âge possible, cela forme un ensemble de représentants $C$

Une personne quelconque appartient à une unique classe d'un de ces représentants

\change

Pour la relation <<\emph{être parallèle}>>, la classe d'équivalence d'une droite $D$
est l'ensemble des droites parallèles à $D$.


Une classe d'équivalence correspond à une unique direction.



%%%%%%%%%%%%%%%%%%%%%%%%%%%%%%%%%%%%%%%%%%%%%%%%%%%%%%%%%%%
\diapo

Venons-en à l'exemple le plus important : la relation <<modulo $n$>>.

Considérons l'ensemble des entiers relatifs $\Zz$ et fixons un entier $n$.

Pour deux entiers $a$ et $b$ on dira que $a$ est congru à $b$ modulo $n$ si et seulement si
$a-b$ est un multiple de $n$.

\change

Prenons l'exemple de $n=7$.
Alors $10$ est congru à $3$ modulo $7$ car $10-3$ est un multiple de $7$.

\change

De même $19-5=14$ donc $19$ est congru à $5$ modulo $7$.

\change

$77$ est un multiple de $7$ dont $77\equiv 0 \pmod 7$

\change

Enfin $-1 \equiv 20 \pmod 7$.

\change

La relation être congru modulo $n$, définie une relation d'équivalence sur $\Zz$.

\change

$a$ est congru à lui-même car $a-a=0$ est $0 \times n$


\change

si $a$ est congru à $b$ alors $a-b$ est un multiple de $n$
et on déduit que $b-a$ est aussi un multiple de $n$ donc $b$ congru à $a$.

\change

Enfin $a$ congru à $b$ et $b$ congru à $c$
alors $a-b=kn$ et $b-c=k'n$. Donc $a-c = (k+k')n$ et ainsi 
$a \equiv c \pmod n$.


%%%%%%%%%%%%%%%%%%%%%%%%%%%%%%%%%%%%%%%%%%%%%%%%%%%%%%%%%%%
\diapo

La classe d'équivalence de $a$ pour la relation <<être congru modulo $n$>>
est donc l'ensemble des entiers $b$ tels que $b \equiv a \pmod n$.

On note ici cette classe $\overline a$.

\change

Les $b$ sont donc de la forme $a$ plus un multiple de $n$,

$\overline a$ est donc l'ensemble de $a+kn$ pour $k$ parcourant $\Zz$.

Cet ensemble se note aussi $a+n\Zz$.

\change

$n$ est congru à $0$ modulo $n$ donc $\overline n = \overline 0$

de même $n+1$ est congru à $1$ modulo $n$ donc $\overline {n+1} = \overline 1$,
$\overline {n+2} = \overline 2$, et ainsi de suite.

\change

Il n'y a donc $n$ classes d'équivalence

et on note $\Zz/n\Zz$ l'ensemble de toutes les classes d'équivalences 
$\big\{ \overline 0, \overline 1, \overline 2, \ldots, \overline{n-1} \big\}$

\change

Reprenons l'exemple de $n=7$.

Nous avons $\Zz/7\Zz = \big\{  \overline 0, \overline 1, \overline 2,\ldots, \overline{6} \big\}$


Nous allons expliciter ces $7$ classes d'équivalences.

\change

Tout d'abord la classe d'équivalence de $0$ est l'ensemble des multiples de $7$, c'est donc $7\Zz$.

\change

La classe d'équivalence de $1$ est l'ensemble des $1+k \times 7$ 

\change

etc
jusqu'à 

La classe d'équivalence de $6$ est l'ensemble des $6+k \times 7$ 

\change

Et c'est terminé, car ensuite $\overline 7$
égale $\overline{0}$ car $7$ est congru à $0$ modulo $7$.

Nous avons bien trouver les $7$ classes d'équivalence de $\Zz/7\Zz$.

Plus tard nous considérerons ces classes comme des nombres, nous additionnerons 
les classes $\overline a + \overline b$ et nous les multiplierons.



%%%%%%%%%%%%%%%%%%%%%%%%%%%%%%%%%%%%%%%%%%%%%%%%%%%%%%%%%%%
\diapo

Entraînez-vous avec ces différents exercices.





\end{document}
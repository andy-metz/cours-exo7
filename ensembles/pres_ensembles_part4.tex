
%%%%%%%%%%%%%%%%%% PREAMBULE %%%%%%%%%%%%%%%%%%

\documentclass[aspectratio=169,utf8]{beamer}
%\documentclass[aspectratio=169,handout]{beamer}

\usetheme{Boadilla}
%\usecolortheme{seahorse}
%\usecolortheme[RGB={245,66,24}]{structure}
\useoutertheme{infolines}

% packages
\usepackage{amsfonts,amsmath,amssymb,amsthm}
\usepackage[utf8]{inputenc}
\usepackage[T1]{fontenc}
\usepackage{lmodern}

\usepackage[francais]{babel}
\usepackage{fancybox}
\usepackage{graphicx}

\usepackage{float}
\usepackage{xfrac}

%\usepackage[usenames, x11names]{xcolor}
\usepackage{pgfplots}
\usepackage{datetime}


% ----------------------------------------------------------------------
% Pour les images
\usepackage{tikz}
\usetikzlibrary{calc,shadows,arrows.meta,patterns,matrix}

\newcommand{\tikzinput}[1]{\input{figures/#1.tikz}}
% --- les figures avec échelle éventuel
\newcommand{\myfigure}[2]{% entrée : échelle, fichier(s) figure à inclure
\begin{center}\small%
\tikzstyle{every picture}=[scale=1.0*#1]% mise en échelle + 0% (automatiquement annulé à la fin du groupe)
#2%
\end{center}}



%-----  Package unités -----
\usepackage{siunitx}
\sisetup{locale = FR,detect-all,per-mode = symbol}

%\usepackage{mathptmx}
%\usepackage{fouriernc}
%\usepackage{newcent}
%\usepackage[mathcal,mathbf]{euler}

%\usepackage{palatino}
%\usepackage{newcent}
% \usepackage[mathcal,mathbf]{euler}



% \usepackage{hyperref}
% \hypersetup{colorlinks=true, linkcolor=blue, urlcolor=blue,
% pdftitle={Exo7 - Exercices de mathématiques}, pdfauthor={Exo7}}


%section
% \usepackage{sectsty}
% \allsectionsfont{\bf}
%\sectionfont{\color{Tomato3}\upshape\selectfont}
%\subsectionfont{\color{Tomato4}\upshape\selectfont}

%----- Ensembles : entiers, reels, complexes -----
\newcommand{\Nn}{\mathbb{N}} \newcommand{\N}{\mathbb{N}}
\newcommand{\Zz}{\mathbb{Z}} \newcommand{\Z}{\mathbb{Z}}
\newcommand{\Qq}{\mathbb{Q}} \newcommand{\Q}{\mathbb{Q}}
\newcommand{\Rr}{\mathbb{R}} \newcommand{\R}{\mathbb{R}}
\newcommand{\Cc}{\mathbb{C}} 
\newcommand{\Kk}{\mathbb{K}} \newcommand{\K}{\mathbb{K}}

%----- Modifications de symboles -----
\renewcommand{\epsilon}{\varepsilon}
\renewcommand{\Re}{\mathop{\text{Re}}\nolimits}
\renewcommand{\Im}{\mathop{\text{Im}}\nolimits}
%\newcommand{\llbracket}{\left[\kern-0.15em\left[}
%\newcommand{\rrbracket}{\right]\kern-0.15em\right]}

\renewcommand{\ge}{\geqslant}
\renewcommand{\geq}{\geqslant}
\renewcommand{\le}{\leqslant}
\renewcommand{\leq}{\leqslant}
\renewcommand{\epsilon}{\varepsilon}

%----- Fonctions usuelles -----
\newcommand{\ch}{\mathop{\text{ch}}\nolimits}
\newcommand{\sh}{\mathop{\text{sh}}\nolimits}
\renewcommand{\tanh}{\mathop{\text{th}}\nolimits}
\newcommand{\cotan}{\mathop{\text{cotan}}\nolimits}
\newcommand{\Arcsin}{\mathop{\text{arcsin}}\nolimits}
\newcommand{\Arccos}{\mathop{\text{arccos}}\nolimits}
\newcommand{\Arctan}{\mathop{\text{arctan}}\nolimits}
\newcommand{\Argsh}{\mathop{\text{argsh}}\nolimits}
\newcommand{\Argch}{\mathop{\text{argch}}\nolimits}
\newcommand{\Argth}{\mathop{\text{argth}}\nolimits}
\newcommand{\pgcd}{\mathop{\text{pgcd}}\nolimits} 


%----- Commandes divers ------
\newcommand{\ii}{\mathrm{i}}
\newcommand{\dd}{\text{d}}
\newcommand{\id}{\mathop{\text{id}}\nolimits}
\newcommand{\Ker}{\mathop{\text{Ker}}\nolimits}
\newcommand{\Card}{\mathop{\text{Card}}\nolimits}
\newcommand{\Vect}{\mathop{\text{Vect}}\nolimits}
\newcommand{\Mat}{\mathop{\text{Mat}}\nolimits}
\newcommand{\rg}{\mathop{\text{rg}}\nolimits}
\newcommand{\tr}{\mathop{\text{tr}}\nolimits}


%----- Structure des exercices ------

\newtheoremstyle{styleexo}% name
{2ex}% Space above
{3ex}% Space below
{}% Body font
{}% Indent amount 1
{\bfseries} % Theorem head font
{}% Punctuation after theorem head
{\newline}% Space after theorem head 2
{}% Theorem head spec (can be left empty, meaning ‘normal’)

%\theoremstyle{styleexo}
\newtheorem{exo}{Exercice}
\newtheorem{ind}{Indications}
\newtheorem{cor}{Correction}


\newcommand{\exercice}[1]{} \newcommand{\finexercice}{}
%\newcommand{\exercice}[1]{{\tiny\texttt{#1}}\vspace{-2ex}} % pour afficher le numero absolu, l'auteur...
\newcommand{\enonce}{\begin{exo}} \newcommand{\finenonce}{\end{exo}}
\newcommand{\indication}{\begin{ind}} \newcommand{\finindication}{\end{ind}}
\newcommand{\correction}{\begin{cor}} \newcommand{\fincorrection}{\end{cor}}

\newcommand{\noindication}{\stepcounter{ind}}
\newcommand{\nocorrection}{\stepcounter{cor}}

\newcommand{\fiche}[1]{} \newcommand{\finfiche}{}
\newcommand{\titre}[1]{\centerline{\large \bf #1}}
\newcommand{\addcommand}[1]{}
\newcommand{\video}[1]{}

% Marge
\newcommand{\mymargin}[1]{\marginpar{{\small #1}}}

\def\noqed{\renewcommand{\qedsymbol}{}}


%----- Presentation ------
\setlength{\parindent}{0cm}

%\newcommand{\ExoSept}{\href{http://exo7.emath.fr}{\textbf{\textsf{Exo7}}}}

\definecolor{myred}{rgb}{0.93,0.26,0}
\definecolor{myorange}{rgb}{0.97,0.58,0}
\definecolor{myyellow}{rgb}{1,0.86,0}

\newcommand{\LogoExoSept}[1]{  % input : echelle
{\usefont{U}{cmss}{bx}{n}
\begin{tikzpicture}[scale=0.1*#1,transform shape]
  \fill[color=myorange] (0,0)--(4,0)--(4,-4)--(0,-4)--cycle;
  \fill[color=myred] (0,0)--(0,3)--(-3,3)--(-3,0)--cycle;
  \fill[color=myyellow] (4,0)--(7,4)--(3,7)--(0,3)--cycle;
  \node[scale=5] at (3.5,3.5) {Exo7};
\end{tikzpicture}}
}


\newcommand{\debutmontitre}{
  \author{} \date{} 
  \thispagestyle{empty}
  \hspace*{-10ex}
  \begin{minipage}{\textwidth}
    \titlepage  
  \vspace*{-2.5cm}
  \begin{center}
    \LogoExoSept{2.5}
  \end{center}
  \end{minipage}

  \vspace*{-0cm}
  
  % Astuce pour que le background ne soit pas discrétisé lors de la conversion pdf -> png
\begin{tikzpicture}
        \fill[opacity=0,green!60!black] (0,0)--++(0,0)--++(0,0)--++(0,0)--cycle; 
\end{tikzpicture}

% toc S'affiche trop tot :
% \tableofcontents[hideallsubsections, pausesections]
}

\newcommand{\finmontitre}{
  \end{frame}
  \setcounter{framenumber}{0}
} % ne marche pas pour une raison obscure

%----- Commandes supplementaires ------

% \usepackage[landscape]{geometry}
% \geometry{top=1cm, bottom=3cm, left=2cm, right=10cm, marginparsep=1cm
% }
% \usepackage[a4paper]{geometry}
% \geometry{top=2cm, bottom=2cm, left=2cm, right=2cm, marginparsep=1cm
% }

%\usepackage{standalone}


% New command Arnaud -- november 2011
\setbeamersize{text margin left=24ex}
% si vous modifier cette valeur il faut aussi
% modifier le decalage du titre pour compenser
% (ex : ici =+10ex, titre =-5ex

\theoremstyle{definition}
%\newtheorem{proposition}{Proposition}
%\newtheorem{exemple}{Exemple}
%\newtheorem{theoreme}{Théorème}
%\newtheorem{lemme}{Lemme}
%\newtheorem{corollaire}{Corollaire}
%\newtheorem*{remarque*}{Remarque}
%\newtheorem*{miniexercice}{Mini-exercices}
%\newtheorem{definition}{Définition}

% Commande tikz
\usetikzlibrary{calc}
\usetikzlibrary{patterns,arrows}
\usetikzlibrary{matrix}
\usetikzlibrary{fadings} 

%definition d'un terme
\newcommand{\defi}[1]{{\color{myorange}\textbf{\emph{#1}}}}
\newcommand{\evidence}[1]{{\color{blue}\textbf{\emph{#1}}}}
\newcommand{\assertion}[1]{\emph{\og#1\fg}}  % pour chapitre logique
%\renewcommand{\contentsname}{Sommaire}
\renewcommand{\contentsname}{}
\setcounter{tocdepth}{2}



%------ Encadrement ------

\usepackage{fancybox}


\newcommand{\mybox}[1]{
\setlength{\fboxsep}{7pt}
\begin{center}
\shadowbox{#1}
\end{center}}

\newcommand{\myboxinline}[1]{
\setlength{\fboxsep}{5pt}
\raisebox{-10pt}{
\shadowbox{#1}
}
}

%--------------- Commande beamer---------------
\newcommand{\beameronly}[1]{#1} % permet de mettre des pause dans beamer pas dans poly


\setbeamertemplate{navigation symbols}{}
\setbeamertemplate{footline}  % tiré du fichier beamerouterinfolines.sty
{
  \leavevmode%
  \hbox{%
  \begin{beamercolorbox}[wd=.333333\paperwidth,ht=2.25ex,dp=1ex,center]{author in head/foot}%
    % \usebeamerfont{author in head/foot}\insertshortauthor%~~(\insertshortinstitute)
    \usebeamerfont{section in head/foot}{\bf\insertshorttitle}
  \end{beamercolorbox}%
  \begin{beamercolorbox}[wd=.333333\paperwidth,ht=2.25ex,dp=1ex,center]{title in head/foot}%
    \usebeamerfont{section in head/foot}{\bf\insertsectionhead}
  \end{beamercolorbox}%
  \begin{beamercolorbox}[wd=.333333\paperwidth,ht=2.25ex,dp=1ex,right]{date in head/foot}%
    % \usebeamerfont{date in head/foot}\insertshortdate{}\hspace*{2em}
    \insertframenumber{} / \inserttotalframenumber\hspace*{2ex} 
  \end{beamercolorbox}}%
  \vskip0pt%
}


\definecolor{mygrey}{rgb}{0.5,0.5,0.5}
\setlength{\parindent}{0cm}
%\DeclareTextFontCommand{\helvetica}{\fontfamily{phv}\selectfont}

% background beamer
\definecolor{couleurhaut}{rgb}{0.85,0.9,1}  % creme
\definecolor{couleurmilieu}{rgb}{1,1,1}  % vert pale
\definecolor{couleurbas}{rgb}{0.85,0.9,1}  % blanc
\setbeamertemplate{background canvas}[vertical shading]%
[top=couleurhaut,middle=couleurmilieu,midpoint=0.4,bottom=couleurbas] 
%[top=fondtitre!05,bottom=fondtitre!60]



\makeatletter
\setbeamertemplate{theorem begin}
{%
  \begin{\inserttheoremblockenv}
  {%
    \inserttheoremheadfont
    \inserttheoremname
    \inserttheoremnumber
    \ifx\inserttheoremaddition\@empty\else\ (\inserttheoremaddition)\fi%
    \inserttheorempunctuation
  }%
}
\setbeamertemplate{theorem end}{\end{\inserttheoremblockenv}}

\newenvironment{theoreme}[1][]{%
   \setbeamercolor{block title}{fg=structure,bg=structure!40}
   \setbeamercolor{block body}{fg=black,bg=structure!10}
   \begin{block}{{\bf Th\'eor\`eme }#1}
}{%
   \end{block}%
}


\newenvironment{proposition}[1][]{%
   \setbeamercolor{block title}{fg=structure,bg=structure!40}
   \setbeamercolor{block body}{fg=black,bg=structure!10}
   \begin{block}{{\bf Proposition }#1}
}{%
   \end{block}%
}

\newenvironment{corollaire}[1][]{%
   \setbeamercolor{block title}{fg=structure,bg=structure!40}
   \setbeamercolor{block body}{fg=black,bg=structure!10}
   \begin{block}{{\bf Corollaire }#1}
}{%
   \end{block}%
}

\newenvironment{mydefinition}[1][]{%
   \setbeamercolor{block title}{fg=structure,bg=structure!40}
   \setbeamercolor{block body}{fg=black,bg=structure!10}
   \begin{block}{{\bf Définition} #1}
}{%
   \end{block}%
}

\newenvironment{lemme}[0]{%
   \setbeamercolor{block title}{fg=structure,bg=structure!40}
   \setbeamercolor{block body}{fg=black,bg=structure!10}
   \begin{block}{\bf Lemme}
}{%
   \end{block}%
}

\newenvironment{remarque}[1][]{%
   \setbeamercolor{block title}{fg=black,bg=structure!20}
   \setbeamercolor{block body}{fg=black,bg=structure!5}
   \begin{block}{Remarque #1}
}{%
   \end{block}%
}


\newenvironment{exemple}[1][]{%
   \setbeamercolor{block title}{fg=black,bg=structure!20}
   \setbeamercolor{block body}{fg=black,bg=structure!5}
   \begin{block}{{\bf Exemple }#1}
}{%
   \end{block}%
}


\newenvironment{miniexercice}[0]{%
   \setbeamercolor{block title}{fg=structure,bg=structure!20}
   \setbeamercolor{block body}{fg=black,bg=structure!5}
   \begin{block}{Mini-exercices}
}{%
   \end{block}%
}


\newenvironment{tp}[0]{%
   \setbeamercolor{block title}{fg=structure,bg=structure!40}
   \setbeamercolor{block body}{fg=black,bg=structure!10}
   \begin{block}{\bf Travaux pratiques}
}{%
   \end{block}%
}
\newenvironment{exercicecours}[1][]{%
   \setbeamercolor{block title}{fg=structure,bg=structure!40}
   \setbeamercolor{block body}{fg=black,bg=structure!10}
   \begin{block}{{\bf Exercice }#1}
}{%
   \end{block}%
}
\newenvironment{algo}[1][]{%
   \setbeamercolor{block title}{fg=structure,bg=structure!40}
   \setbeamercolor{block body}{fg=black,bg=structure!10}
   \begin{block}{{\bf Algorithme}\hfill{\color{gray}\texttt{#1}}}
}{%
   \end{block}%
}


\setbeamertemplate{proof begin}{
   \setbeamercolor{block title}{fg=black,bg=structure!20}
   \setbeamercolor{block body}{fg=black,bg=structure!5}
   \begin{block}{{\footnotesize Démonstration}}
   \footnotesize
   \smallskip}
\setbeamertemplate{proof end}{%
   \end{block}}
\setbeamertemplate{qed symbol}{\openbox}


\makeatother
\usecolortheme[RGB={153,0,0}]{structure}


%%%%%%%%%%%%%%%%%%%%%%%%%%%%%%%%%%%%%%%%%%%%%%%%%%%%%%%%%%%%%
%%%%%%%%%%%%%%%%%%%%%%%%%%%%%%%%%%%%%%%%%%%%%%%%%%%%%%%%%%%%%

\begin{document}

\title{{\bf Ensembles et applications}}
\subtitle{Ensembles finis}

\begin{frame}
  
  \debutmontitre

  \pause

{\footnotesize
\hfill
\setbeamercovered{transparent=50}
\begin{minipage}{0.6\textwidth}
  \begin{itemize}
    \item<3-> Cardinal
    \item<4-> Injection, surjection, bijection et ensembles finis
    \item<5-> Nombres d'applications
    \item<6-> Nombres de sous-ensembles
    \item<7-> Coefficients du binôme de Newton
    \item<8-> Formule du binôme de Newton
  \end{itemize}
\end{minipage}
}

\end{frame}

\setcounter{framenumber}{0}




%---------------------------------------------------------------
\section{Ensembles finis}

\begin{frame}


\begin{mydefinition}
Un ensemble $E$ est \defi{fini} s'il existe $n\in \Nn$ et
une bijection de $E$ vers l'ensemble $\{1,2,\ldots,n\}$

\pause

Ce $n$ est unique, s'appelle le \defi{cardinal} de $E$ (ou le \defi{nombre d'éléments})
et est noté $\Card E$
\end{mydefinition}

\pause
\bigskip


\begin{itemize}
  \item $E=\{\text{rouge},\text{noir}\}$ est en bijection avec $\{1,2\}$ et donc $\Card E =2$
\pause
  \item $\Nn$ n'est pas un ensemble fini
\pause
  \item par définition le cardinal de l'ensemble vide est $0$
\end{itemize}
\end{frame}


\begin{frame}



Si $A$ est un ensemble fini et $B \subset A$
\begin{enumerate}
  \item<1-> $B$ est aussi fini et $\Card B \le \Card A$
  \item<2-> $\Card (A \setminus B) = \Card A - \Card B$
\end{enumerate}

\pause
\pause 
\bigskip

Si $A, B$ sont deux ensembles finis 
%\setcounter{theenumi}{2}
\begin{enumerate}

  \item<3-> Si $A \cap B = \varnothing$ alors $\Card (A \cup B) = \Card A + \Card B$

  \item<4-> Pour $A,B$ finis quelconques:
\mybox{$\Card (A \cup B) = \Card A + \Card B - \Card (A\cap B)$}
\pause\pause
\myfigure{1.3}{
\tikzinput{fig_ensembles14} 
}
\end{enumerate}

\end{frame}

%---------------------------------------------------------------

\section{Applications et ensembles finis}


\begin{frame}

Soit $E,F$ deux ensembles finis et $f : E \to F$
\begin{proposition}
\begin{enumerate}
  \item<1-> \label{it:bij1} Si $f$ est injective alors $\Card E \le \Card F$
  \item<2-> \label{it:bij2} Si $f$ est surjective alors $\Card E \ge \Card F$
  \item<3-> \label{it:bij3} Si $f$ est bijective alors $\Card E = \Card F$
\end{enumerate}
\end{proposition}
\end{frame}



\begin{frame}
Soient $E,F$ deux ensembles finis et $f : E \to F$
\begin{proposition}
Si $\Card E = \Card F$
alors les assertions suivantes sont équivalentes
\pause
\begin{enumerate}
  \item[i.] $f$ est injective
  \item[ii.] $f$ est surjective
  \item[iii.] $f$ est bijective
\end{enumerate}
\end{proposition}

\pause

Schéma de la preuve \qquad $ (i) \implies (ii) \implies (iii) \implies (i) $

\pause
\bigskip

Le \defi{principe des tiroirs}
\begin{proposition}
Si l'on range dans $k$ tiroirs, $n > k$ paires de chaussettes
alors il existe un tiroir contenant (au moins) deux paires
de chaussettes
\end{proposition}

  
\end{frame}

%---------------------------------------------------------------

\section{Nombres d'applications}

\begin{frame}

$\Card E=n$ \quad et \quad $\Card F=p$
\begin{proposition}
Le nombre d'applications différentes de $E$ dans $F$ est 
\pause
 \myboxinline{$p^n$}
\end{proposition}

\pause
\medskip

Autrement dit c'est \myboxinline{$(\Card F)^{\Card E}$}

\pause 
\bigskip

\begin{exemple}
En particulier le nombre  d'applications de $E$ dans lui-même est $n^n$

\pause 
\smallskip

Par exemple si $E=\{1,2,3,4,5\}$ alors ce nombre est $5^5 = 3125$
\end{exemple}
\end{frame}

%---------------------------------------------------------------

\begin{frame}

$\Card E = n$
\begin{proposition}
Le nombre de bijections de $E$ dans $E$ est 
\pause
\mybox{$n!=1\times 2 \times 3 \times \cdots \times n$}
\end{proposition}

\pause

\begin{exemple}
Parmi les $3125$ applications de $E=\{1,2,3,4,5\}$ dans lui-même,

il y en a $5! = 120$ qui sont bijectives
\end{exemple}

\pause

\begin{proof}[Preuve rapide]
Pour l'image du premier élément il y a $n$ choix

\pause

Pour l'image du deuxième : $n-1$ choix

\pause

\ldots

Pour l'image du $k$-ième : $n-k+1$ choix

\ldots

\pause

Pour l'image du $n$-ième : $1$ choix
\end{proof}
  
\end{frame}

%---------------------------------------------------------------
\section{Nombres de sous-ensembles}

\begin{frame}

$\Card E = n$
\begin{proposition}
Il y a $2^{\Card E}$ sous-ensembles de $E$ 
\pause
\mybox{$\Card \mathcal{P}(E) = 2^n$}
\end{proposition}

\pause

\begin{exemple}
Si $E= \{1,2,3,4,5\}$ alors $\mathcal{P}(E)$ a $2^5 = 32$ parties

\pause

\begin{itemize}
  \item l'ensemble vide : $\varnothing$
\pause
  \item $5$ singletons : $\{1\}, \{2\},\ldots$
\pause
  \item $10$ paires : $\{1,2\}, \{1,3\}, \ldots, \{2,3\}, \ldots$
\pause
  \item $10$ triplets : $\{1,2,3\},\{1,2,4\},\ldots$
\pause
  \item $5$ ensembles à $4$ éléments : $\{1,2,3,4\}, \{1,2,3,5\},\ldots$
\pause
  \item $E$ tout entier :  $\{1,2,3,4,5\}$
\end{itemize}

\end{exemple}

\end{frame}


%---------------------------------------------------------------

\section{Coefficients du binôme de Newton}

\begin{frame}

\begin{mydefinition}
Le nombre de parties à $k$ éléments d'un ensemble à $n$ éléments est
noté

\smallskip

\centerline{$\binom{n}{k}$ \quad ou \quad $C_n^k$}
\end{mydefinition}

\pause

{\footnotesize
\begin{exemple}
\begin{itemize}
\item Les parties à $k\!=\!2$ éléments de $\{1,2,3\}$ sont $\{1,2\}$, $\{1,3\}$, $\{2,3\}$
donc $\binom{3}{2} = 3$
\pause
\item Pour $n=5$ \qquad
$
\binom{5}{0} = 1 \;\;
\binom{5}{1} = 5 \;\;
\binom{5}{2} = 10 \;\;
\binom{5}{3} = 10 \;\;
\binom{5}{4} = 5 \;\;
\binom{5}{5} = 1
$
\end{itemize}
\end{exemple}
}
\pause

\begin{proposition}
\begin{itemize}
  \item $\binom{n}{0}=1$ \quad  $\binom{n}{1}=n$ \quad  $\binom{n}{n}=1$
\pause
  \item \myboxinline{$\binom{n}{n-k} = \binom{n}{k}$}
\pause
  \item \myboxinline{$\binom{n}{0}+\binom{n}{1}+\cdots+\binom{n}{k}+\cdots+\binom{n}{n} = 2^n$}
\end{itemize}
\end{proposition}

\end{frame}

%---------------------------------------------------------------

\begin{frame}


\begin{proposition}\ 
\mybox{$\displaystyle \binom n k = \binom{n-1}{k} + \binom{n-1}{k-1} \qquad 0<k<n$}
\end{proposition}

\pause

\begin{proof}
$\Card E = n$, \quad $a \in E$, \quad  $E' = E \setminus \{ a \}$

\pause

Il y a deux sortes de parties $A \subset E$ ayant $k$ éléments :
\begin{itemize}
\pause
  \item celles qui ne contiennent pas $a$ : 
\pause
ce sont donc des parties à $k$ éléments 
dans $E'$ qui possède $n-1$ éléments, il y a en a donc $\binom{n-1}{k}$
\pause
  \item celles qui contiennent $a$ : 
\pause
elles sont de la forme $A = \{a\} \cup A'$ 
avec $A'$ une partie à $k-1$ éléments dans $E'$, il y en a $\binom{n-1}{k-1}$
\end{itemize}
\pause
Bilan : $\binom n k = \binom{n-1}{k-1} + \binom{n-1}{k}$
\end{proof}
\end{frame}

%---------------------------------------------------------------

\begin{frame}
Le \evidence{triangle de Pascal} %la table des $\binom{n}{k}$ 
\qquad\qquad \uncover<2->{$k \le n$}
\uncover<3->{\quad  $\binom{n}{0}=1 \quad \binom{n}{n}=1$}


\[\hspace*{-2cm}
\begin{array}{ccc}
%\hspace*{3cm}
\uncover<4->{\binom{n-1}{k-1} & + & \binom{n-1}{k}
\\
& & \shortparallel
\\
& & \binom{n}{k}
}
\\ \\
\uncover<5-10>{
\binom{
\only<5>{{\color{red} 2}-1}
\only<6>{{\color{blue} 1}}
\only<7,9>{{\color{red} 3}-1}
\only<8,10>{{\color{blue} 2}}
}{
\only<5,7>{{\color{red} 1}-1}
\only<6,8>{{\color{blue} 0}}
\only<9>{{\color{red} 2} -1}
\only<10>{{\color{blue} 1}}
} & + & 
\binom{
\only<5>{{\color{red} 2}-1}
\only<6>{\color{blue} 1}
\only<7,9>{{\color{red} 3}-1}
\only<8,10>{\color{blue} 2}
}{
{\color{red} \only<5-8>{1} \only<9-10>{2}}
}
\\
& & \shortparallel
\\
& & \binom{
\only<5-6>{\color{red} 2}
\only<7-10>{\color{red} 3}
}{
{\color{red} \only<5-8>{1} \only<9-10>{2}}
}
}
\end{array}
\qquad
\begin{array}{cc|ccccccc}
&&&& k
\\
&	& 0 & {\color<5-8>{red} 1} & {\color<9-10>{red} 2} & 3 & 4 & 5 
\\ \hline 
& 0 & \temporal<2>{}{?}{1} & & \phantom{22} & \phantom{22} & \phantom{22} & \phantom{22} & \uncover<15>{\;}
\\
& 1 	& \temporal<2>{}{?}{\color<6>{blue} 1} 	& \temporal<2>{}{?}{\color<6>{blue} 1}
\\
& {\color<5,6>{red}2} 	& \temporal<2>{}{?}{\color<8>{blue} 1}	
& \temporal<2-5>{}{{\color<5>{red}?}}{\color<6>{red} \color<8,10>{blue} 2}		
& \temporal<2>{}{?}{\color<10>{blue} 1}
\\
n & {\color<7,8,9,10>{red} 3}
& \temporal<2>{}{?}{\color<11>{blue} 1}	
& \temporal<2-7>{}{\color<7>{red}?}{\color<8>{red} \color<11-12>{blue} 3}		
& \temporal<2-9>{}{\color<9>{red} ?}{\color<10>{red} \color<12-13>{blue} 3}		
& \temporal<2>{}{?}{\color<13>{blue} 1}
\\ 
& 4		& \temporal<2>{}{?}{1}	
& \temporal<2-10>{}{?}{\color<11>{red} 4}		
& \temporal<2-11>{}{?}{\color<12>{red} 6}
& \temporal<2-12>{}{?}{\color<13>{red} 4}		& \temporal<2>{}{?}{1}
\\
& 5 & \temporal<2>{}{?}{1} 
& \temporal<2-13>{ & & & }{? & ? & ? & ? }{5 & 10 & 10 & 5}
& \temporal<2>{}{?}{1}
\\
& 6 & \temporal<2>{}{?}{1} 
& \temporal<2-14>{ & & & & }{? & ? & ? & ? & ? }{6 & 15 & 20 & 15 & 6}
& \temporal<2>{}{?}{1}
\end{array}
\]  
\end{frame}

%---------------------------------------------------------------

\begin{frame}

\begin{proposition}
\mybox{$\displaystyle \binom n k = \frac{n!}{k!(n-k)!}$}
\end{proposition}

\pause

\begin{proof}
Récurrence sur $n$. Vrai pour $n=1$
\pause

Supposons vrai au rang $n-1$ 
$$\begin{array}{rcl}
\displaystyle \binom n k \pause & = & \displaystyle \binom{n-1}{k-1} + \binom{n-1}{k} \\ \\
\pause
 & = & \displaystyle \frac{(n-1)!}{(k-1)!(n-1-(k-1))!} + \frac{(n-1)!}{k!(n-1-k)!} \\ \\
\pause
& = & \displaystyle \frac{(n-1)!}{(k-1)!(n-k-1)!} \times \left( \frac{1}{n-k} + \frac{1}{k} \right) = \frac{n!}{k!(n-k)!}
\end{array}$$
\end{proof}
\end{frame}

%---------------------------------------------------------------
\section{Formule du binôme de Newton}

\begin{frame}
\begin{theoreme}
Soit $a,b \in\Rr$ et  $n \ge 1$
\mybox{$\displaystyle (a+b)^n = \pause \sum_{k=0}^n \binom{n}{k} \ a^{n-k} \cdot b^{k}$}
\end{theoreme}


\pause

{\small
$$(a+b)^n = \binom{n}{0}\ a^n + \binom{n}{1}\ a^{n-1}\cdot b
+ \cdots + \binom{n}{k} \ a^{n-k} \cdot b^{k}+\cdots + \binom{n}{n}\ b^n$$
}

\pause

\begin{exemple}
\begin{enumerate}
  \item $(a+b)^2= a^2 + 2ab + b^2$
\pause
  \item $(a+b)^3 = a^3 + 3a^2b + 3ab^2 + b^3$
\pause
  \item Si $a=1$ et $b=1$, on retrouve $\sum_{k=0}^n \binom{n}{k} = 2^n$
\end{enumerate}
\end{exemple}
\end{frame}


%---------------------------------------------------------------
\section{Mini-exercices}

\begin{frame}
\begin{miniexercice}
\begin{enumerate}
  \item Combien y a-t-il d'applications injectives d'un ensemble à $n$ éléments dans un ensemble à $n+1$ éléments ?
  \item Combien y a-t-il d'applications surjectives d'un ensemble à $n+1$ éléments dans un ensemble à $n$ éléments ?
  \item Calculer le nombre de façons de choisir $5$ cartes dans un jeux de $32$ cartes.
  \item Calculer le nombre de listes à $k$ éléments dans un ensemble à $n$ éléments ({\small les listes sont ordonnées :
par exemple $(1,2,3) \neq (1,3,2)$}).
  \item Développer $(a-b)^4$, $(a+b)^5$.
  \item Que donne la formule du binôme pour $a=-1$, $b=+1$ ? En déduire que dans un ensemble à $n$ éléments il y a
autant de parties de cardinal pair que de cardinal impair.
\end{enumerate}
\end{miniexercice}
\end{frame}

\end{document}
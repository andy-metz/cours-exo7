
%%%%%%%%%%%%%%%%%% PREAMBULE %%%%%%%%%%%%%%%%%%


\documentclass[12pt]{article}

\usepackage{amsfonts,amsmath,amssymb,amsthm}
\usepackage[utf8]{inputenc}
\usepackage[T1]{fontenc}
\usepackage[francais]{babel}


% packages
\usepackage{amsfonts,amsmath,amssymb,amsthm}
\usepackage[utf8]{inputenc}
\usepackage[T1]{fontenc}
%\usepackage{lmodern}

\usepackage[francais]{babel}
\usepackage{fancybox}
\usepackage{graphicx}

\usepackage{float}

%\usepackage[usenames, x11names]{xcolor}
\usepackage{tikz}
\usepackage{datetime}

\usepackage{mathptmx}
%\usepackage{fouriernc}
%\usepackage{newcent}
\usepackage[mathcal,mathbf]{euler}

%\usepackage{palatino}
%\usepackage{newcent}


% Commande spéciale prompteur

%\usepackage{mathptmx}
%\usepackage[mathcal,mathbf]{euler}
%\usepackage{mathpple,multido}

\usepackage[a4paper]{geometry}
\geometry{top=2cm, bottom=2cm, left=1cm, right=1cm, marginparsep=1cm}

\newcommand{\change}{{\color{red}\rule{\textwidth}{1mm}\\}}

\newcounter{mydiapo}

\newcommand{\diapo}{\newpage
\hfill {\normalsize  Diapo \themydiapo \quad \texttt{[\jobname]}} \\
\stepcounter{mydiapo}}


%%%%%%% COULEURS %%%%%%%%%%

% Pour blanc sur noir :
%\pagecolor[rgb]{0.5,0.5,0.5}
% \pagecolor[rgb]{0,0,0}
% \color[rgb]{1,1,1}



%\DeclareFixedFont{\myfont}{U}{cmss}{bx}{n}{18pt}
\newcommand{\debuttexte}{
%%%%%%%%%%%%% FONTES %%%%%%%%%%%%%
\renewcommand{\baselinestretch}{1.5}
\usefont{U}{cmss}{bx}{n}
\bfseries

% Taille normale : commenter le reste !
%Taille Arnaud
%\fontsize{19}{19}\selectfont

% Taille Barbara
%\fontsize{21}{22}\selectfont

%Taille François
\fontsize{25}{30}\selectfont

%Taille Pascal
%\fontsize{25}{30}\selectfont

%Taille Laura
%\fontsize{30}{35}\selectfont


%\myfont
%\usefont{U}{cmss}{bx}{n}

%\Huge
%\addtolength{\parskip}{\baselineskip}
}


% \usepackage{hyperref}
% \hypersetup{colorlinks=true, linkcolor=blue, urlcolor=blue,
% pdftitle={Exo7 - Exercices de mathématiques}, pdfauthor={Exo7}}


%section
% \usepackage{sectsty}
% \allsectionsfont{\bf}
%\sectionfont{\color{Tomato3}\upshape\selectfont}
%\subsectionfont{\color{Tomato4}\upshape\selectfont}

%----- Ensembles : entiers, reels, complexes -----
\newcommand{\Nn}{\mathbb{N}} \newcommand{\N}{\mathbb{N}}
\newcommand{\Zz}{\mathbb{Z}} \newcommand{\Z}{\mathbb{Z}}
\newcommand{\Qq}{\mathbb{Q}} \newcommand{\Q}{\mathbb{Q}}
\newcommand{\Rr}{\mathbb{R}} \newcommand{\R}{\mathbb{R}}
\newcommand{\Cc}{\mathbb{C}} 
\newcommand{\Kk}{\mathbb{K}} \newcommand{\K}{\mathbb{K}}

%----- Modifications de symboles -----
\renewcommand{\epsilon}{\varepsilon}
\renewcommand{\Re}{\mathop{\text{Re}}\nolimits}
\renewcommand{\Im}{\mathop{\text{Im}}\nolimits}
%\newcommand{\llbracket}{\left[\kern-0.15em\left[}
%\newcommand{\rrbracket}{\right]\kern-0.15em\right]}

\renewcommand{\ge}{\geqslant}
\renewcommand{\geq}{\geqslant}
\renewcommand{\le}{\leqslant}
\renewcommand{\leq}{\leqslant}

%----- Fonctions usuelles -----
\newcommand{\ch}{\mathop{\mathrm{ch}}\nolimits}
\newcommand{\sh}{\mathop{\mathrm{sh}}\nolimits}
\renewcommand{\tanh}{\mathop{\mathrm{th}}\nolimits}
\newcommand{\cotan}{\mathop{\mathrm{cotan}}\nolimits}
\newcommand{\Arcsin}{\mathop{\mathrm{Arcsin}}\nolimits}
\newcommand{\Arccos}{\mathop{\mathrm{Arccos}}\nolimits}
\newcommand{\Arctan}{\mathop{\mathrm{Arctan}}\nolimits}
\newcommand{\Argsh}{\mathop{\mathrm{Argsh}}\nolimits}
\newcommand{\Argch}{\mathop{\mathrm{Argch}}\nolimits}
\newcommand{\Argth}{\mathop{\mathrm{Argth}}\nolimits}
\newcommand{\pgcd}{\mathop{\mathrm{pgcd}}\nolimits} 

\newcommand{\Card}{\mathop{\text{Card}}\nolimits}
\newcommand{\Ker}{\mathop{\text{Ker}}\nolimits}
\newcommand{\id}{\mathop{\text{id}}\nolimits}
\newcommand{\ii}{\mathrm{i}}
\newcommand{\dd}{\mathrm{d}}
\newcommand{\Vect}{\mathop{\text{Vect}}\nolimits}
\newcommand{\Mat}{\mathop{\mathrm{Mat}}\nolimits}
\newcommand{\rg}{\mathop{\text{rg}}\nolimits}
\newcommand{\tr}{\mathop{\text{tr}}\nolimits}
\newcommand{\ppcm}{\mathop{\text{ppcm}}\nolimits}

%----- Structure des exercices ------

\newtheoremstyle{styleexo}% name
{2ex}% Space above
{3ex}% Space below
{}% Body font
{}% Indent amount 1
{\bfseries} % Theorem head font
{}% Punctuation after theorem head
{\newline}% Space after theorem head 2
{}% Theorem head spec (can be left empty, meaning ‘normal’)

%\theoremstyle{styleexo}
\newtheorem{exo}{Exercice}
\newtheorem{ind}{Indications}
\newtheorem{cor}{Correction}


\newcommand{\exercice}[1]{} \newcommand{\finexercice}{}
%\newcommand{\exercice}[1]{{\tiny\texttt{#1}}\vspace{-2ex}} % pour afficher le numero absolu, l'auteur...
\newcommand{\enonce}{\begin{exo}} \newcommand{\finenonce}{\end{exo}}
\newcommand{\indication}{\begin{ind}} \newcommand{\finindication}{\end{ind}}
\newcommand{\correction}{\begin{cor}} \newcommand{\fincorrection}{\end{cor}}

\newcommand{\noindication}{\stepcounter{ind}}
\newcommand{\nocorrection}{\stepcounter{cor}}

\newcommand{\fiche}[1]{} \newcommand{\finfiche}{}
\newcommand{\titre}[1]{\centerline{\large \bf #1}}
\newcommand{\addcommand}[1]{}
\newcommand{\video}[1]{}

% Marge
\newcommand{\mymargin}[1]{\marginpar{{\small #1}}}



%----- Presentation ------
\setlength{\parindent}{0cm}

%\newcommand{\ExoSept}{\href{http://exo7.emath.fr}{\textbf{\textsf{Exo7}}}}

\definecolor{myred}{rgb}{0.93,0.26,0}
\definecolor{myorange}{rgb}{0.97,0.58,0}
\definecolor{myyellow}{rgb}{1,0.86,0}

\newcommand{\LogoExoSept}[1]{  % input : echelle
{\usefont{U}{cmss}{bx}{n}
\begin{tikzpicture}[scale=0.1*#1,transform shape]
  \fill[color=myorange] (0,0)--(4,0)--(4,-4)--(0,-4)--cycle;
  \fill[color=myred] (0,0)--(0,3)--(-3,3)--(-3,0)--cycle;
  \fill[color=myyellow] (4,0)--(7,4)--(3,7)--(0,3)--cycle;
  \node[scale=5] at (3.5,3.5) {Exo7};
\end{tikzpicture}}
}



\theoremstyle{definition}
%\newtheorem{proposition}{Proposition}
%\newtheorem{exemple}{Exemple}
%\newtheorem{theoreme}{Théorème}
\newtheorem{lemme}{Lemme}
\newtheorem{corollaire}{Corollaire}
%\newtheorem*{remarque*}{Remarque}
%\newtheorem*{miniexercice}{Mini-exercices}
%\newtheorem{definition}{Définition}




%definition d'un terme
\newcommand{\defi}[1]{{\color{myorange}\textbf{\emph{#1}}}}
\newcommand{\evidence}[1]{{\color{blue}\textbf{\emph{#1}}}}



 %----- Commandes divers ------

\newcommand{\codeinline}[1]{\texttt{#1}}

%%%%%%%%%%%%%%%%%%%%%%%%%%%%%%%%%%%%%%%%%%%%%%%%%%%%%%%%%%%%%
%%%%%%%%%%%%%%%%%%%%%%%%%%%%%%%%%%%%%%%%%%%%%%%%%%%%%%%%%%%%%

\begin{document}

\debuttexte

%%%%%%%%%%%%%%%%%%%%%%%%%%%%%%%%%%%%%%%%%%%%%%%%%%%%%%%%%%%
\diapo

\change

\change

Pour passer d'un ensemble à un autre nous allons 
définir les applications

\change

puis nous verrons deux concepts : l'image directe et l'image réciproque.


%%%%%%%%%%%%%%%%%%%%%%%%%%%%%%%%%%%%%%%%%%%%%%%%%%%%%%%%%%%
\diapo

Une \defi{application} (ou une \defi{fonction}) $f : E \to F$,
c'est associer à chaque élément $x$ de $E$ un unique élément $f(x)$ appartenant à $F$

\change

Nous représenterons les applications par deux types d'illustrations :
tout d'abord à l'aide de <<patates>>, l'ensemble de départ (et celui d'arrivée) 
est un
schématisé par un ovale, ses éléments par des points. 
L'association $x \mapsto f(x)$ est représentée par une flèche.

Une flèche doit partir de chaque point de l'ensemble de départ, mais
il n'arrive pas nécessairement une flèche à chaque point de l'ensemble d'arrivée.


\change

L'autre représentation est celle des fonctions de $\Rr$ dans $\Rr$ (ou de sous-ensembles de $\Rr$).
L'ensemble de départ est représenté par l'axe des abscisses et celui d'arrivée par l'axe des ordonnées.
L'association $x \mapsto f(x)$ est représentée par le point $(x,f(x))$.


\change

Enfin, quand dit-on que deux applications $f$ et $g$ sont égales ?

l'application $f$ égale l'application $g$ si et seulement si pour tous les $x$ de l'ensemble de départ on a $f(x)=g(x)$.


%%%%%%%%%%%%%%%%%%%%%%%%%%%%%%%%%%%%%%%%%%%%%%%%%%%%%%%%%%%
\diapo

Le graphe d'une application $f$ est l'ensemble des couples $(x,f(x))$.

C'est un sous-ensemble du produit cartésien $E \times F$.

\change

Dans les cas des fonctions de $\Rr$ dans $\Rr$, vous connaissez bien
cette notion.

\change

Si $f$ est une application de $E$ vers $F$ et $g$ est une application
de $F$ vers $G$, 

\change

alors on définit une troisième application $g\circ f$ qui va de $E$ vers $G$.

Cette application est définie  par la formule  
$g\circ f(x) = g(f(x))$.


%%%%%%%%%%%%%%%%%%%%%%%%%%%%%%%%%%%%%%%%%%%%%%%%%%%%%%%%%%%
\diapo

Une application qui sera utile dans la suite est l'identité de $E$ dans $E$

elle est simple à définir : à $x$ on associe $x$ lui-même !

\change

Définissons deux fonctions, une fonction $f$ définie par $1/x$.

Et une fonction $g$ définie par $x-1/x+1$.

\change

Remarquez que l'ensemble d'arrivée de $f$ 
est le même que l'ensemble de départ de $g$
donc on peut composer $f$ et $g$.

Calculons la fonction $g\circ f$.

$g\circ f(x)$
 
\change

c'est par définition 

$= g \big(f(x)\big) $

\change

donc 
$= g \left(\frac 1x \right) $

\change

$= \frac{\frac 1x - 1}{\frac 1x + 1} $

\change

$= \frac{1-x}{1+x}$ 

\change

et sur cet exemple particulier on trouve 
$= - g(x)$

%%%%%%%%%%%%%%%%%%%%%%%%%%%%%%%%%%%%%%%%%%%%%%%%%%%%%%%%%%%
\diapo

Voici deux concepts nouveaux, importants et délicats : l'image directe et l'image réciproque.

prenons une application $f$ d'un ensemble $E$ vers un ensemble $F$

et soit $A$ une partie de l'ensemble de départ $E$ 


L'image directe de $A$ par $f$ est l'ensemble
des $f(x)$ pour $x$ parcourant $A$.

Cet ensemble se note $f(A)$, c'est un sous-ensemble de l'ensemble d'arrivée $F$.

\change

Sur ce premier dessin $f(A)$ est l'ensemble des points 
pour lesquels une flèches de $A$ arrive.

\change


Sur ce second dessin $f(A)$ est représenté sur l'axe des ordonnées, 

c'est l'ensemble des valeurs prises par $f$ aux  points de $A$.


%%%%%%%%%%%%%%%%%%%%%%%%%%%%%%%%%%%%%%%%%%%%%%%%%%%%%%%%%%%
\diapo

Reprenons notre application $f$ de $E$ vers $F$.

Mais cette fois partons d'une partie $B$ de l'ensemble d'arrivée $F$.


L'image réciproque de $B$ par l'application $f$ est l'ensemble
des $x$ tels que $f(x) \in B$.

C'est un sous-ensemble de l'ensemble de départ $E$ et on le note  $f^{-1}(B)$.

\change

Sur ce dessin pour trouver l'image réciproque de $B$, on cherche tous les points de $E$
dont les flèches arrivent dans $B$.

\change

Sur ce second dessin, $B$ est intervalle représenté sur l'axe des ordonnées,
on cherche tous les points de l'axe des abscisses tel que $f(x)$ soit dans $B$.

Sur cet exemple les points de $f^{-1}(B)$ sont les points de deux intervalles de l'axe des abscisses.
$f^{-1}(B)$ est donc ici l'union des deux intervalles. 


Quelques remarques : l'image réciproque d'un ensemble est une notion plus difficile
qu'il n'y parait. 

N'oubliez que $f^{-1}(B)$ est une notation qui forme un tout. 

En particulier nous ne supposons
pas que $f$ soit une bijection, il ne faut donc pas parler de $f^{-1}$ seulement,
mais uniquement de $f^{-1}$ d'un ensemble.



%%%%%%%%%%%%%%%%%%%%%%%%%%%%%%%%%%%%%%%%%%%%%%%%%%%%%%%%%%%
\diapo

Dernière notion : les d'antécédents.

Fixons $y \in F$. Tout élément $x\in E$ tel que $f(x)=y$ s'appelle un antécédent
de $y$.

Autrement dit l'ensemble des antécédents de $y$ est l'image réciproque du singleton $y$.

Sur les dessins suivants, l'élément $y$ admet $3$ antécédents par $f$. 
Ce sont $x_1$, $x_2$, $x_3$.

Sur le premier dessin, on trouve les antécédents de $y$ à l'aide des flèches qui arrivent en $y$.

Sur le second dessin : on trace la droite horizontale passant par $y$, cette droite
coupe le graphe de $f$ au points d'abscisses $x_1, x_2, x_3$.



%%%%%%%%%%%%%%%%%%%%%%%%%%%%%%%%%%%%%%%%%%%%%%%%%%%%%%%%%%%
\diapo

Encore une fois, ce sont des notions basiques qu'il faut maîtriser.

Entraînez-vous !


\end{document}
\documentclass[class=report,crop=false]{standalone}
\usepackage[screen]{../exo7book}

\begin{document}

%====================================================================
\chapitre{Leçons de choses}
%====================================================================


\insertvideo{hju3f2VVPWw}{partie 1. L'alphabet grec}

\insertvideo{PRZ3cqjE1-4}{partie 2. \LaTeX\ en cinq minutes}

\insertvideo{yTQxjJ73lgM}{partie 3. Formules de trigonométrie : sinus, cosinus, tangente}

\insertvideo{89quOLXleak}{partie 4. Formulaire: trigonométrie circulaire et hyperbolique}

\insertvideo{b0zQuVDQ2Yw}{partie 5. Développements limités}

\insertvideo{QRwIVrdc-jI}{partie 6. Primitives}


%%%%%%%%%%%%%%%%%%%%%%%%%%%%%%%%%%%%%%%%%%%%%%%%%%%%%%%%%%%%%%%%
\section{Alphabet grec}


\begin{center}
%\noindent
\setlength{\arrayrulewidth}{0.05mm}
%\begin{tabular}{|l|l|l|} \hline
\begin{tabular}[t]{|ll|l@{\vrule depth 1.2ex height 3ex width 0mm \ }|}
\hline
   $\alpha$      &               & alpha   \\ \hline
   $\beta$       &               & beta    \\ \hline
   $\gamma$      & $\Gamma$      & gamma   \\ \hline
   $\delta$      & $\Delta$      & delta   \\ \hline
   $\varepsilon$ &               & epsilon \\ \hline
   $\zeta$       &               & zeta    \\ \hline
   $\eta$        &               & eta     \\ \hline
   $\theta$      & $\Theta$      & theta   \\ \hline
   $\iota$       &               & iota    \\ \hline
   $\kappa$      &               & kappa   \\ \hline
   $\lambda$     & $\Lambda$     & lambda  \\ \hline
   $\mu$         &               & mu      \\ \hline
\end{tabular}
\hspace*{2cm}
\begin{tabular}[t]{|ll|l@{\vrule depth 1.2ex height 3ex width 0mm \ }|}
\hline
   $\nu$         &               & nu      \\ \hline
   $\xi$         &               & xi      \\ \hline
   $o$           &               & omicron \\ \hline
   $\pi$         & $\Pi$         & pi      \\ \hline
   $\rho,\varrho$ &              & rho     \\ \hline
   $\sigma$      & $\Sigma$      & sigma   \\ \hline
   $\tau$        &               & tau     \\ \hline
   $\upsilon$    &               & upsilon \\ \hline
   $\phi,\varphi$& $\Phi$        & phi     \\ \hline
   $\chi$        &               & chi     \\ \hline
   $\psi$        & $\Psi$        & psi     \\ \hline
   $\omega$      & $\Omega$      & omega   \\ \hline
\end{tabular}\hfill
\end{center}

\medskip

On rencontre aussi ``nabla'' $\nabla$, l'opérateur de dérivée
partielle $\partial$ (dites ``d rond''), et aussi la première
lettre de l'alphabet hébreu ``aleph'' $\aleph$.


%%%%%%%%%%%%%%%%%%%%%%%%%%%%%%%%%%%%%%%%%%%%%%%%%%%%%%%%%%%%%%%%
\newpage
\section{Écrire des mathématiques: \LaTeX\ en cinq minutes}

\subsection{Les bases}

Pour écrire des mathématiques, il existe un langage pratique et universel,
le langage \LaTeX\ (prononcé [latek]). Il est utile pour rédiger des textes contenant des formules,
mais aussi accepté sur certains blogs et vous permet d'écrire des maths
dans un courriel ou un texto.

\bigskip

Une formule s'écrit entre deux dollars \verb?$\pi^2$? qui donne $\pi^2$ ou entre double dollars
si l'on veut la centrer sur une nouvelle ligne ; \verb?$$\lim u_n = +\infty$$? affichera:
$$\lim u_n = + \infty$$

Dans la suite on omettra les balises dollars.


\subsection{Premières commandes}

Les exposants s'obtiennent avec la commande \verb?^? et les indices avec \verb?_?:
 $a^2$ s'écrit \verb?a^2? ; $u_n$ s'écrit \verb?u_n? ; $\alpha_i^2$ s'écrit \verb?\alpha_i^2?.
Les accolades \verb?{ }? permettent de grouper du texte: \verb?2^{10}? pour $2^{10}$ ;
 \verb?a_{i,j}? pour $a_{i,j}$.


Il y a ensuite toute une liste de commandes (qui commencent par \verb?\?) dont voici les plus utiles:


\begin{tabular}{cccc@{\vrule depth 1.2ex height 4ex width 0mm \ }}

\verb?\sqrt? & racine &
   $\sqrt{a}$ & \verb?\sqrt{a}? \\
&& $\sqrt{1+\sqrt{2}}$ & \verb?\sqrt{1+\sqrt{2}}?  \\
&&  $\sqrt[3]{x}$ & \verb?\sqrt[3]{x}?  \\

\verb?\frac? & fraction &
   $\dfrac{a}{b}$ & \verb?\frac{a}{b}? \\
&& $\dfrac{\pi^3}{12}$ & \verb?\frac{\pi^3}{12}?   \\
&& $\dfrac{1}{2+ \frac{3}{4}}$ & \verb?\frac{1}{2 + \frac{3}{4}}?  \\
&& $\gamma^{\frac{1}{n}}$ & \verb?\gamma^{\frac{1}{n}}? \\


\verb?\lim? & limite &
  $\lim_{n \to + \infty} u_n = 0$ & \verb?\lim_{n \to +\infty} u_n = 0?  \\
&& $\lim_{x \to 0^+} f(x) < \epsilon$ & \verb?\lim_{x \to 0^+} f(x) < \epsilon?  \\

\verb?\sum? & somme &
   $\displaystyle \sum_{i=1}^n \frac{1}{i}$ & \verb?\sum_{i=1}^n \frac{1}{i}? \\
&& $\displaystyle \sum_{i \ge 0} a_i$ & \verb?\sum_{i \ge 0} a_i?  \\

\verb?\int? & intégrale &
  $\displaystyle \int_a^b \phi(t) dt$ & \verb?\int_a^b \phi(t) dt?  \\

\end{tabular}

\subsection{D'autres commandes}

Voici d'autres commandes, assez naturelles pour les anglophones.

\begin{center}
\begin{tabular}{cc}
$f : E \to F$ & \verb?f : E \to F?  \\
$+\infty$ & \verb?+\infty? \\
$a \le 0$ & \verb?a \le 0? \\
$a > 0$ & \verb?a > 0?  \\
$a \ge 1$ & \verb?a \ge 1? \\
$\delta$ & \verb?\delta? \\
$\Delta$ & \verb?\Delta? \\
\end{tabular}\hspace{2cm}
\begin{tabular}{cc}
$a \in E$ & \verb?a \in E? \\
$A \subset E$ & \verb?A \subset E? \\
$P \implies Q$ & \verb?P \implies Q? \\
$P \iff Q$ & \verb?P \iff Q? \\
$\forall$ & \verb?\forall? \\
$\exists$ & \verb?\exists? \\
$\cup$ & \verb?\cup? \\
$\cap$ & \verb?\cap?  \\
\end{tabular}
\end{center}

\subsection{Pour allez plus loin}

Il est possible de créer ses propres commandes avec \verb?\newcommand?.
Par exemple avec l'instruction

\hfil \verb?\newcommand{\Rr}{\mathbb{R}}?

vous définissez une nouvelle commande \verb?\Rr? qui exécutera l'instruction
\verb?\mathbb{R}? et affichera donc $\Rr$.

Autre exemple, après avoir défini

\hfil \verb?\newcommand{\monintegrale}{\int_0^{+\infty} \frac{\sin t}{t} dt}?

la commande \verb?\monintegrale? affichera $\int_0^{+\infty} \frac{\sin t}{t} dt$.

\bigskip

Pour (beaucoup) plus de détails, consultez le manuel \emph{Une courte (?) introduction à \LaTeX}.


%\subsection{Mini-exercices}

\begin{miniexercices}
\'Ecrire en \LaTeX\ toutes ces formules (qui par ailleurs sont vraies!).

\begin{enumerate}
  \item $\displaystyle \sqrt{a}-\sqrt{b} = \frac{a-b}{\sqrt a + \sqrt b}$
  \item $\displaystyle \sum_{n=1}^{+\infty}{\frac{1}{n^2}} = \frac{\pi^2}{6}$
  \item $\displaystyle \lim_{R\to+\infty} \int_{-R}^{+R} e^{-t^2} dt = \sqrt{\pi}$
  \item $\displaystyle \forall \epsilon>0 \quad \exists \delta \ge 0 \quad (|x-x_0|<\delta \implies |\ln(x)-\ln(x_0)|<\epsilon)$
  \item $\displaystyle \sum_{k=0}^{+\infty} \ \frac{1}{16^k} \left(\frac{4}{8k+1}-\frac{2}{8k+4}-\frac{1}{8k+5}-\frac{1}{8k+6} \right) = \pi$
\end{enumerate}
\end{miniexercices}

%%%%%%%%%%%%%%%%%%%%%%%%%%%%%%%%%%%%%%%%%%%%%%%%%%%%%%%%%%%%%%%%
\newpage
\section{Formules de trigonométrie: sinus, cosinus, tangente}

%--------------------------------------------------------
\subsection{Le cercle trigonométrique}




\myfigure{0.85}{
  \tikzinput{fig_lecons07}
}
Voici le cercle trigonométrique (de rayon $1$),
le sens de lecture est l'inverse du sens des aiguilles d'une montre.
Les angles remarquables sont marqués de $0$ à $2 \pi$ (en radian) et de $0^\circ$ à $360^\circ$.
Les coordonnées des points correspondant à ces angles sont aussi indiquées.


\myfigure{0.85}{
  \tikzinput{fig_lecons02}
}

Le point $M$ a pour coordonnées $(\cos x,\sin x)$.
La droite $(OM)$ coupe la droite d'équation $(x=1)$ en $T$,
l'ordonnée du point $T$ est $\tan x$.

Les formules de base:
\begin{align*}
& \cos^2 x + \sin^2 x = 1 \\
& \cos(x+2\pi)=\cos x \\
& \sin(x+2\pi)=\sin x \\
\end{align*}

\begin{minipage}{0.45\textwidth}
\myfigure{0.8}{
  \tikzinput{fig_lecons06}
}
\end{minipage}
\begin{minipage}{0.42\textwidth}
Nous avons les formules suivantes:
\begin{align*}
\cos (-x) &= \cos x \\
\sin (-x) &= -\sin x \\
\end{align*}
On retrouve graphiquement ces formules à l'aide du dessin des angles $x$ et $-x$.
\end{minipage}

Il en est de même pour les formules suivantes:

\begin{minipage}{0.32\textwidth}
\begin{align*}
\cos (\pi + x) &= -\cos x \\
\sin (\pi + x) &= -\sin x \\
\end{align*}
\end{minipage}
\begin{minipage}{0.32\textwidth}
\begin{align*}
\cos (\pi - x) &= -\cos x \\
\sin (\pi - x) &= \sin x \\
\end{align*}
\end{minipage}
\begin{minipage}{0.32\textwidth}
\begin{align*}
\cos (\frac\pi2 - x) &= \sin x \\
\sin (\frac\pi2 - x) &= \cos x \\
\end{align*}
\end{minipage}
\myfigure{0.8}{
  \tikzinput{fig_lecons04}
  \tikzinput{fig_lecons03}
  \tikzinput{fig_lecons05}
}


{
\renewcommand{\arraystretch}{3}
$$
\begin{array}{c|*{5}{c}}
    x     & \qquad 0 \qquad & \qquad \dfrac\pi6 \qquad & \qquad \dfrac\pi 4\qquad
& \qquad \dfrac \pi 3\qquad  &\qquad  \dfrac \pi 2\qquad  \\
\hline
\cos x  \quad & 1 & \dfrac{\sqrt3}{2} & \dfrac{\sqrt2}{2} & \dfrac12 & 0 \\
\hline
\sin x  \quad & 0 &\dfrac12 & \dfrac{\sqrt2}{2} & \dfrac{\sqrt3}{2} & 1\\
\hline
\tan x  \ & 0 & \dfrac{1}{\sqrt{3}} & 1 & \sqrt{3} &
\end{array}
$$
}

Valeurs que l'on retrouve bien sur le cercle trigonométrique.
\myfigure{1}{
  \tikzinput{fig_lecons08}
}

%--------------------------------------------------------
\subsection{Les fonctions sinus, cosinus, tangente}

La fonction cosinus est périodique de période $2\pi$
et elle paire (donc symétrique par rapport à l'axe des ordonnées).
La fonction sinus est aussi périodique de période de $2\pi$ mais elle impaire
(donc symétrique par rapport à l'origine).



\myfigure{0.57}{
  \tikzinput{fig_lecons10}
}

Voici un zoom sur l'intervalle $[-\pi,\pi]$.
\myfigure{1.2}{
  \tikzinput{fig_lecons11}
}


Pour tout $x$ n'appartenant pas à $\{\ldots, -\frac\pi2, \frac\pi2, \frac{3\pi}{2}, \frac{5\pi}{2},\ldots  \}$
la tangente est définie par
$$\tan x = \frac{\sin x}{\cos x}$$
La fonction $x \mapsto \tan x$ est périodique de période $\pi$ ; c'est une fonction impaire.

\myfigure{0.7}{
  \tikzinput{fig_lecons12}
}


Voici les dérivées:
\begin{align*}
\cos'x&= -\sin x\\
\sin'x&=\cos x\\
\tan' x &= 1+\tan^2x=\frac{1}{\cos^2x}\\
\end{align*}




%--------------------------------------------------------
\subsection{Les formules d'additions}



\begin{align*}
\cos(a+b) &= \cos a \cdot \cos b - \sin a \cdot \sin b \\
\sin(a+b) &= \sin a\cdot \cos b  +  \sin b\cdot\cos a \\
\tan (a+b) &=\dfrac{\tan a + \tan b}{1-\tan a \cdot \tan b}\\
\end{align*}



On en déduit immédiatement :
\begin{align*}
\cos(a-b)&=\cos a\cdot\cos b + \sin a\cdot\sin b\\
\sin(a-b)&=\sin a\cdot\cos b  - \sin b\cdot\cos a \\
\tan (a-b)&=\frac{\tan a - \tan b}{1+\tan a\cdot\tan b} \\
\end{align*}



Il est bon de connaître par c\oe ur les formules suivantes (faire $a=b$ dans les formules d'additions) :
\begin{align*}
\cos 2a &= 2\cos^2a-1\\
    &= 1-2\sin^2a\\
    &=\cos^2a-\sin^2a\\
\sin 2a &= 2\sin a\cdot \cos a\\
\tan 2a &= \frac{2\tan a}{1-\tan^2 a}
\end{align*}

%--------------------------------------------------------
\subsection{Les autres formules}

Voici d'autres formules qui se déduisent des formules d'additions.
Il n'est pas nécessaire de les connaître mais il faut savoir 
les retrouver en cas de besoin.

\begin{align*}
\cos a\cdot\cos b &= \frac{1}{2}\big[ \cos(a+b)+\cos(a-b)\big]\\
\sin a\cdot\sin b &= \frac{1}{2}\big[ \cos(a-b)-\cos(a+b)\big]\\
\sin a\cdot\cos b &= \frac{1}{2}\big[ \sin(a+b)+\sin(a-b)\big]
\end{align*}



Les formules précédentes se reformulent aussi en:
\begin{align*}
\cos p+\cos q &= 2\cos \frac{p+q}{2}\cdot\cos\frac{p-q}{2}\\
\cos p-\cos q &= -2\sin \frac{p+q}{2}\cdot\sin \frac{p-q}{2}\\
\sin p+\sin q &= 2\sin \frac{p+q}{2}\cdot\cos\frac{p-q}{2}\\
\sin p-\sin q &= 2\sin \frac{p-q}{2}\cdot\cos\frac{p+q}{2}\\
\end{align*}


Enfin les formules de la \og tangente de l'arc moitié\fg{} permettent d'exprimer sinus, cosinus et tangente
en fonction de $\tan \frac x2$.
\begin{align*}
\text{Avec }\quad  t&=\tan \frac{x}{2} \quad \text{ on a } \quad
\begin{cases}
    \cos x &= \frac {1-t^2}{1+t^2} \\
    \sin x &= \frac{2t}{1+t^2} \\
    \tan x &= \frac{2t}{1-t^2}
\end{cases}
\end{align*}

Ces formules sont utiles pour le calcul de certaines intégrales par changement de variable,
en utilisant en plus la relation $dx=\dfrac{2dt}{1+t^2}$.



%--------------------------------------------------------
%\subsection{Mini-exercices}

\begin{miniexercices}
\sauteligne
\begin{enumerate}
  \item Montrer que $1+\tan^2x=\frac{1}{\cos^2x}$.
  \item Montrer la formule d'addition de $\tan(a+b)$.
  \item Prouver la formule pour $\cos a\cdot\cos b$.
  \item Prouver la formule pour $\cos p+\cos q$.
  \item Prouver la formule: $\sin x = \dfrac{2\tan \frac{x}{2}}{1+(\tan \frac{x}{2})^2}$.
  \item Montrer que $\cos \frac{\pi}{8}= \frac 12\sqrt{ \sqrt{2} + 2}$. Calculer $\cos \frac{\pi}{16}$,
$\cos \frac{\pi}{32}$,\ldots
  \item Exprimer $\cos(3x)$ en fonction $\cos x$ ; $\sin(3x)$ en fonction $\sin x$ ;
$\tan(3x)$ en fonction $\tan x$.
\end{enumerate}
\end{miniexercices}

%%%%%%%%%%%%%%%%%%%%%%%%%%%%%%%%%%%%%%%%%%%%%%%%%%%%%%%%%%%%%%%%
\newpage
\section{Formulaire: trigonométrie circulaire et hyperbolique}


\begin{multicols}{2}[
\begin{center}
\emph{\textbf{Propri\'et\'es trigonom\'etriques}: remplacer $\cos$ par $\ch$
et $\sin$ par $\ii\cdot\sh$.}
\end{center}
\vspace{0.5em}]

$$\cos^2 x + \sin^2 x = 1$$

\begin{align*}
\cos(a+b)&=\cos a\cdot\cos b - \sin a\cdot\sin b\\
\sin(a+b)&=\sin a\cdot\cos b  +  \sin b\cdot\cos a \\
\tan (a+b)&=\frac{\tan a + \tan b}{1-\tan a\cdot\tan b} \\
\end{align*}


$$\ch^2 x - \sh^2 x = 1$$

\begin{align*}
\ch(a+b)&=\ch a\cdot\ch b + \sh a\cdot\sh b\\
\sh(a+b)&=\sh a\cdot\ch b  +  \sh b\cdot\ch a\\
\tanh (a+b)&=\frac{\tanh a + \tanh b}{1+\tanh a\cdot\tanh b}\\
\end{align*}
\end{multicols}



\begin{multicols}{2}
\begin{align*}
\cos(a-b)&=\cos a\cdot\cos b + \sin a\cdot\sin b\\
\sin(a-b)&=\sin a\cdot\cos b  - \sin b\cdot\cos a \\
\tan (a-b)&=\frac{\tan a - \tan b}{1+\tan a\cdot\tan b} \\
\end{align*}

\begin{align*}
\ch(a-b)&=\ch a\cdot\ch b - \sh a\cdot\sh b\\
\sh(a-b)&=\sh a\cdot\ch b  - \sh b\cdot\ch a\\
\tanh (a-b)&=\frac{\tanh a - \tanh b}{1-\tanh a\cdot\tanh b}\\
\end{align*}

\end{multicols}

%\vspace{0.5cm}
\begin{multicols}{2}
\begin{align*}
\cos 2a &= 2\,\cos^2a-1\\
    &= 1-2\,\sin^2a\\
    &=\cos^2a-\sin^2a\\[3mm]
\sin 2a &= 2\,\sin a\cdot\cos a\\[3mm]
\tan 2a &= \frac{2\,\tan a}{1-\tan^2 a}
\end{align*}

\begin{align*}
\ch 2a &= 2\,\ch^2a-1\\
    &= 1+2\,\sh^2a\\
    &=\ch^2a+\sh^2a\\[3mm]
\sh 2a &= 2\,\sh a\cdot\ch a\\[3mm]
\tanh 2a &= \frac{2\,\tanh a}{1+\tanh^2 a}
\end{align*}
\end{multicols}


\begin{multicols}{2}
\small
\begin{align*}
\cos a\cdot\cos b &= \frac{1}{2}\,\big[ \cos(a+b)+\cos(a-b)\big]\\
\sin a\cdot\sin b &= \frac{1}{2}\,\big[ \cos(a-b)-\cos(a+b)\big]\\
\sin a\cdot\cos b &= \frac{1}{2}\,\big[ \sin(a+b)+\sin(a-b)\big]
\end{align*}

\begin{align*}
\ch a\cdot\ch b &= \frac{1}{2}\,\big[ \ch(a+b)+\ch(a-b)\big]\\
\sh a\cdot\sh b &= \frac{1}{2}\,\big[ \ch(a+b)-\ch(a-b)\big]\\
\sh a\cdot\ch b &= \frac{1}{2}\,\big[ \sh(a+b)+\sh(a-b)\big]
\end{align*}
\end{multicols}



\begin{multicols}{2}
\begin{align*}
\cos p+\cos q &= 2\,\cos \frac{p+q}{2}\cdot\cos\frac{p-q}{2}\\
\cos p-\cos q &= -2\,\sin \frac{p+q}{2}\cdot\sin \frac{p-q}{2}\\
\sin p+\sin q &= 2\,\sin \frac{p+q}{2}\cdot\cos\frac{p-q}{2}\\
\sin p-\sin q &= 2\,\sin \frac{p-q}{2}\cdot\cos\frac{p+q}{2}
\end{align*}\vspace{0.1cm}
\begin{align*}
\ch p+\ch q &= 2\,\ch \frac{p+q}{2}\cdot\ch\frac{p-q}{2}\\
\ch p-\ch q &= 2\,\sh \frac{p+q}{2}\cdot\sh \frac{p-q}{2}\\
\sh p+\sh q &= 2\,\sh \frac{p+q}{2}\cdot\ch\frac{p-q}{2}\\
\sh p-\sh q &= 2\,\sh \frac{p-q}{2}\cdot\ch\frac{p+q}{2}
\end{align*}
\end{multicols}

%**********************************
%\newpage

\begin{multicols}{2}
Avec \quad $t=\tan \frac{x}{2}$ \quad on a \quad
$$
\begin{cases}
    \cos x &= \frac {1-t^2}{1+t^2} \\
    \sin x &= \frac{2t}{1+t^2} \\
    \tan x &= \frac{2t}{1-t^2}
\end{cases}$$

Avec \quad $t=\tanh \frac{x}{2}$ \quad on a \quad
$$
\begin{cases}
    \ch x &= \frac {1+t^2}{1-t^2} \\
    \sh x &= \frac{2t}{1-t^2} \\
    \tanh x &= \frac{2t}{1+t^2}
\end{cases}$$
\end{multicols}

\vspace{2cm}
\begin{multicols}{2}[\begin{center}
\emph{\textbf{D\'eriv\'ees}: la multiplication par $\ii$ n'est plus valable}
\end{center}]


\begin{align*}
\cos'x&= -\sin x\\
\sin'x&=\cos x\\
\tan' x &= 1+\tan^2x=\frac{1}{\cos^2x}\\
\end{align*}\vspace{0.1cm}
\begin{align*}
\ch'x&= \sh x\\
\sh'x&=\ch x\\
\tanh' x &= 1-\tanh^2x=\frac{1}{\ch^2x}\\
\end{align*}
\end{multicols}


\begin{multicols}{2}
\begin{align*}
\text{arccos}'x&=\frac{-1}{\sqrt{1-x^2}} \quad (|x|<1)\\
\text{arcsin}'x&=\frac{1}{\sqrt{1-x^2}} \quad (|x|<1)\\
\text{arctan}'x&=\frac{1}{1+x^2}
\end{align*}%\vspace{0.1cm}
\begin{align*}
\text{Argch}'x&=\frac{1}{\sqrt{x^2-1}} \quad (x>1)\\
\text{Argsh}'x&=\frac{1}{\sqrt{x^2+1}} \\
\text{Argth}'x&=\frac{1}{1-x^2} \quad (|x|<1)
\end{align*}
\end{multicols}


%%%%%%%%%%%%%%%%%%%%%%%%%%%%%%%%%%%%%%%%%%%%%%%%%%%%%%%%%%%%%%%%

%\newpage
\newpage
\section{Formules de développements limités}

\emph{Développements limités usuels (au voisinage de $0$)}

{\small
\begin{align*}
e^x &\ \ =\ \  1 + \frac{x}{1!}+\frac{x^2}{2!}+ \cdots + \frac{x^n}{n!}\ \  + \ \ o(x^n)
 = \sum_{k=0}^n \frac{x^k}{k!} \ \ + \ \ o(x^n)
\\[1.2em]
\text{cos } x &\ \ =\ \
1 - \frac{x^2}{2!}+\frac{x^4}{4!}- \cdots + (-1)^n \cdot \frac{x^{2n}}{(2n)!}\ \  + \ \ o(x^{2n+1})
= \sum_{k=0}^n (-1)^{k} \frac{x^{2k}}{(2k)!}  \ \  + \ \ o(x^{2n+1})
\\[0.5em]
\text{sin } x &\ \ =\ \
x - \frac{x^3}{3!}+\frac{x^5}{5!}- \cdots + (-1)^n \cdot \frac{x^{2n+1}}{(2n+1)!}\ \  + \ \ o(x^{2n+2})
= \sum_{k=0}^n (-1)^{k} \frac{x^{2k+1}}{(2k+1)!}  \ \  + \ \ o(x^{2n+2})
 \\[0.5em]
\text{tan } x &\ \ =\ \
x +  \frac{x^3}{3} + \frac{2}{15}x^5 + \frac{17}{315}x^7\ \  + \ \ o(x^{8})  \\[1.2em]
\text{ch } x &\ \ =\ \
1 + \frac{x^2}{2!}+\frac{x^4}{4!}+ \cdots + \frac{x^{2n}}{(2n)!}\ \  + \ \ o(x^{2n+1})
= \sum_{k=0}^n \frac{x^{2k}}{(2k)!} \ \  + \ \ o(x^{2n+1})
\\[0.5em]
\text{sh } x &\ \ =\ \
x + \frac{x^3}{3!}+\frac{x^5}{5!}+ \cdots + \frac{x^{2n+1}}{(2n+1)!}\ \  + \ \ o(x^{2n+2})
= \sum_{k=0}^n \frac{x^{2k+1}}{(2k+1)!} \ \  + \ \ o(x^{2n+2})
\\[0.5em]
\text{th } x &\ \ =\ \
x -  \frac{x^3}{3} + \frac{2}{15}x^5 - \frac{17}{315}x^7\ \  + \ \ o(x^{8})
\\[0.5em]
\end{align*}
}

{\small
\begin{align*}
\text{ln}\, (1+x)&\ \ =\ \
x - \frac{x^2}{2}+\frac{x^3}{3}- \cdots + (-1)^{n-1}\cdot\frac{x^n}{n}\ \  + \ \ o(x^n)
= \sum_{k=1}^n (-1)^{k+1} \frac{x^k}{k} \ \  + \ \ o(x^n)
\\[0.5em]
(1+x)^\alpha &\ \ =\ \
1 + \alpha x + \frac{\alpha(\alpha - 1)}{2!}x^2+ \cdots +
\frac {\alpha(\alpha-1)\cdots(\alpha - n+1)}{n!} x^n \ \  + \ \ o(x^n)\\
&\ \ =\ \ \sum_{k=0}^n \binom{\alpha}{k} x^k \ \  + \ \ o(x^n)
\\[1.2em]
\frac{1}{1+x}&\ \ =\ \
1 -x+x^2-\cdots+(-1)^n x^n \ \ +\ \ o(x^n)
= \sum_{k=0}^n (-1)^k x^k \ \ +\ \ o(x^n)
\\[0.5em]
\frac{1}{1-x}&\ \ =\ \
1 + x+x^2+ \cdots+ x^n\ \ +\ \ o(x^n) = \sum_{k=0}^n x^k \ \ +\ \ o(x^n)
\\[0.5em]
\sqrt{1+x}&\ \ =\ \
1 + \frac{x}{2} - \frac{1}{8}x^2- \cdots +
(-1)^{n-1}\cdot\frac{1\cdot1\cdot3\cdot5\cdots(2n-3)}{2^n n!}x^n\ \  + \ \ o(x^n)
\\[0.5em]
\frac{1}{\sqrt{1+x}}&\ \ =\ \
1 - \frac{x}{2} + \frac{3}{8}x^2- \cdots +
(-1)^{n}\cdot\frac{1\cdot3\cdot5\cdots(2n-1)}{2^n n!} x^n\ \ + \ \ o(x^n)
\\[1.2em]
\text{arccos } x &\ \ =\ \
\frac{\pi}{2} - x - \frac{1}{2}\frac{x^3}{3} - \frac{1\cdot3}{2\cdot4}\frac{x^5}{5}- \cdots -
\frac{1\cdot3\cdot5\cdots(2n-1)}{2\cdot4\cdot6\cdots(2n)} \frac{x^{2n+1}}{2n+1}\ \ + \ \ o(x^{2n+2})
\\[0.5em]
\text{arcsin } x &\ \ =\ \
x + \frac{1}{2}\frac{x^3}{3} + \frac{1\cdot3}{2\cdot4}\frac{x^5}{5}+ \cdots +
\frac{1\cdot3\cdot5\cdots(2n-1)}{2\cdot4\cdot6\cdots(2n)} \frac{x^{2n+1}}{2n+1}\ \ + \ \ o(x^{2n+2})
\\[0.5em]
\text{arctan } x &\ \ =\ \
x - \frac{x^3}{3}+\frac{x^5}{5}+ \cdots + (-1)^{n}\cdot\frac{x^{2n+1}}{2n+1}\ \  + \ \ o(x^{2n+2})
\end{align*}
}


%%%%%%%%%%%%%%%%%%%%%%%%%%%%%%%%%%%%%%%%%%%%%%%%%%%%%%%%%%%%%%%%

\newpage
\section{Formulaire : primitives}


\emph{$C$ d\'esigne une constante arbitraire. Les intervalles sont \`a pr\'eciser.}

$$\int e^{\alpha t} \, dt= \frac{e^{\alpha t}}{\alpha}+C \quad (\alpha \in \Cc^*)$$

\begin{multicols}{2}[
%\begin{center}
%\textbf{\textsf{\Large{Primitives  usuelles}}}
%\vspace{0.5cm}
%    \textit{$C$ d\'esigne une constante arbitraire. Les intervalles sont \`a pr\'eciser.}
%\vspace{0.5cm}
%$$\int e^{\alpha t} \, dt= \frac{e^{\alpha t}}{\alpha}+C \quad (\alpha \in \Cc^*)$$
%\vspace{0.5cm}
%\end{center}
]

$$\int t^{\alpha} \, dt= \frac{t^{\alpha +1}}{\alpha+1}+C \quad (\alpha \not= -1)$$

\vspace{2mm}

$$\int \frac{dt}{1+t^2} = \text{Arctan }t+C$$

\vspace{6mm}

$$\int \frac{dt}{\sqrt{1-t^2}} = \text{Arcsin }t+C$$

\vspace{4mm}


$$\int \cos t \, dt = \sin t +C$$

$$\int \sin t \, dt = -\cos t +C$$

$$\int \frac{dt}{\cos^2 t} = \tan t +C$$

$$\int \frac{dt}{\sin^2 t} = -\text{cotan } t +C$$

$$\int \frac{dt}{\cos t} = \ln \left| \tan \left( \frac{t}{2}+\frac{\pi}{4} \right) \right| +C$$

$$\int \frac{dt}{\sin t} = \ln \left| \tan \frac{t}{2} \right| +C$$

$$\int \tan t \, dt = - \ln \left| \cos t \right| +C$$

\vspace{-2mm}

$$\int \text{cotan } t \, dt = \ln \left| \sin t \right| +C$$

%%%%%%%%%%%%%%%%%%%%%%%%%%%%%%%%%%%%%%%%%%%
% Changement de colonne

$$\int \frac{dt}{t} = \ln|t|+C$$

$$\int \frac{dt}{1-t^2} = \frac{1}{2}\ln \left| \frac{1+t}{1-t} \right| +C$$

$$\int \frac{dt}{\sqrt{t^2+\alpha}} = \ln \left| t+\sqrt{t^2+\alpha} \right| +C$$

$$\int \text{ch } t \, dt = \text{sh } t +C$$

$$\int \text{sh } t \, dt = \text{ch } t +C$$

$$\int \frac{dt}{\text{ch}^2 t} = \text{th } t +C$$

$$\int \frac{dt}{\text{sh}^2 t} = -\text{coth } t +C$$

$$\int \frac{dt}{\text{ch } t} = 2\text{Arctan }e^t +C$$

\vspace{2mm}

$$\int \frac{dt}{\text{sh } t} = \ln \left| \text{th } \frac{t}{2} \right| +C$$

$$\int \text{th } t \, dt =  \ln \left( \text{ch } t \right) +C$$

$$\int \text{coth } t \, dt = \ln \left| \text{sh } t \right| +C$$
\end{multicols}


% \auteurs{
% Arnaud Bodin
% }

\finchapitre
\end{document}

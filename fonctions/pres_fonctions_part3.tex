
%%%%%%%%%%%%%%%%%% PREAMBULE %%%%%%%%%%%%%%%%%%

\documentclass[aspectratio=169,utf8]{beamer}
%\documentclass[aspectratio=169,handout]{beamer}

\usetheme{Boadilla}
%\usecolortheme{seahorse}
%\usecolortheme[RGB={245,66,24}]{structure}
\useoutertheme{infolines}

% packages
\usepackage{amsfonts,amsmath,amssymb,amsthm}
\usepackage[utf8]{inputenc}
\usepackage[T1]{fontenc}
\usepackage{lmodern}

\usepackage[francais]{babel}
\usepackage{fancybox}
\usepackage{graphicx}

\usepackage{float}
\usepackage{xfrac}

%\usepackage[usenames, x11names]{xcolor}
\usepackage{pgfplots}
\usepackage{datetime}


% ----------------------------------------------------------------------
% Pour les images
\usepackage{tikz}
\usetikzlibrary{calc,shadows,arrows.meta,patterns,matrix}

\newcommand{\tikzinput}[1]{\input{figures/#1.tikz}}
% --- les figures avec échelle éventuel
\newcommand{\myfigure}[2]{% entrée : échelle, fichier(s) figure à inclure
\begin{center}\small%
\tikzstyle{every picture}=[scale=1.0*#1]% mise en échelle + 0% (automatiquement annulé à la fin du groupe)
#2%
\end{center}}



%-----  Package unités -----
\usepackage{siunitx}
\sisetup{locale = FR,detect-all,per-mode = symbol}

%\usepackage{mathptmx}
%\usepackage{fouriernc}
%\usepackage{newcent}
%\usepackage[mathcal,mathbf]{euler}

%\usepackage{palatino}
%\usepackage{newcent}
% \usepackage[mathcal,mathbf]{euler}



% \usepackage{hyperref}
% \hypersetup{colorlinks=true, linkcolor=blue, urlcolor=blue,
% pdftitle={Exo7 - Exercices de mathématiques}, pdfauthor={Exo7}}


%section
% \usepackage{sectsty}
% \allsectionsfont{\bf}
%\sectionfont{\color{Tomato3}\upshape\selectfont}
%\subsectionfont{\color{Tomato4}\upshape\selectfont}

%----- Ensembles : entiers, reels, complexes -----
\newcommand{\Nn}{\mathbb{N}} \newcommand{\N}{\mathbb{N}}
\newcommand{\Zz}{\mathbb{Z}} \newcommand{\Z}{\mathbb{Z}}
\newcommand{\Qq}{\mathbb{Q}} \newcommand{\Q}{\mathbb{Q}}
\newcommand{\Rr}{\mathbb{R}} \newcommand{\R}{\mathbb{R}}
\newcommand{\Cc}{\mathbb{C}} 
\newcommand{\Kk}{\mathbb{K}} \newcommand{\K}{\mathbb{K}}

%----- Modifications de symboles -----
\renewcommand{\epsilon}{\varepsilon}
\renewcommand{\Re}{\mathop{\text{Re}}\nolimits}
\renewcommand{\Im}{\mathop{\text{Im}}\nolimits}
%\newcommand{\llbracket}{\left[\kern-0.15em\left[}
%\newcommand{\rrbracket}{\right]\kern-0.15em\right]}

\renewcommand{\ge}{\geqslant}
\renewcommand{\geq}{\geqslant}
\renewcommand{\le}{\leqslant}
\renewcommand{\leq}{\leqslant}
\renewcommand{\epsilon}{\varepsilon}

%----- Fonctions usuelles -----
\newcommand{\ch}{\mathop{\text{ch}}\nolimits}
\newcommand{\sh}{\mathop{\text{sh}}\nolimits}
\renewcommand{\tanh}{\mathop{\text{th}}\nolimits}
\newcommand{\cotan}{\mathop{\text{cotan}}\nolimits}
\newcommand{\Arcsin}{\mathop{\text{arcsin}}\nolimits}
\newcommand{\Arccos}{\mathop{\text{arccos}}\nolimits}
\newcommand{\Arctan}{\mathop{\text{arctan}}\nolimits}
\newcommand{\Argsh}{\mathop{\text{argsh}}\nolimits}
\newcommand{\Argch}{\mathop{\text{argch}}\nolimits}
\newcommand{\Argth}{\mathop{\text{argth}}\nolimits}
\newcommand{\pgcd}{\mathop{\text{pgcd}}\nolimits} 


%----- Commandes divers ------
\newcommand{\ii}{\mathrm{i}}
\newcommand{\dd}{\text{d}}
\newcommand{\id}{\mathop{\text{id}}\nolimits}
\newcommand{\Ker}{\mathop{\text{Ker}}\nolimits}
\newcommand{\Card}{\mathop{\text{Card}}\nolimits}
\newcommand{\Vect}{\mathop{\text{Vect}}\nolimits}
\newcommand{\Mat}{\mathop{\text{Mat}}\nolimits}
\newcommand{\rg}{\mathop{\text{rg}}\nolimits}
\newcommand{\tr}{\mathop{\text{tr}}\nolimits}


%----- Structure des exercices ------

\newtheoremstyle{styleexo}% name
{2ex}% Space above
{3ex}% Space below
{}% Body font
{}% Indent amount 1
{\bfseries} % Theorem head font
{}% Punctuation after theorem head
{\newline}% Space after theorem head 2
{}% Theorem head spec (can be left empty, meaning ‘normal’)

%\theoremstyle{styleexo}
\newtheorem{exo}{Exercice}
\newtheorem{ind}{Indications}
\newtheorem{cor}{Correction}


\newcommand{\exercice}[1]{} \newcommand{\finexercice}{}
%\newcommand{\exercice}[1]{{\tiny\texttt{#1}}\vspace{-2ex}} % pour afficher le numero absolu, l'auteur...
\newcommand{\enonce}{\begin{exo}} \newcommand{\finenonce}{\end{exo}}
\newcommand{\indication}{\begin{ind}} \newcommand{\finindication}{\end{ind}}
\newcommand{\correction}{\begin{cor}} \newcommand{\fincorrection}{\end{cor}}

\newcommand{\noindication}{\stepcounter{ind}}
\newcommand{\nocorrection}{\stepcounter{cor}}

\newcommand{\fiche}[1]{} \newcommand{\finfiche}{}
\newcommand{\titre}[1]{\centerline{\large \bf #1}}
\newcommand{\addcommand}[1]{}
\newcommand{\video}[1]{}

% Marge
\newcommand{\mymargin}[1]{\marginpar{{\small #1}}}

\def\noqed{\renewcommand{\qedsymbol}{}}


%----- Presentation ------
\setlength{\parindent}{0cm}

%\newcommand{\ExoSept}{\href{http://exo7.emath.fr}{\textbf{\textsf{Exo7}}}}

\definecolor{myred}{rgb}{0.93,0.26,0}
\definecolor{myorange}{rgb}{0.97,0.58,0}
\definecolor{myyellow}{rgb}{1,0.86,0}

\newcommand{\LogoExoSept}[1]{  % input : echelle
{\usefont{U}{cmss}{bx}{n}
\begin{tikzpicture}[scale=0.1*#1,transform shape]
  \fill[color=myorange] (0,0)--(4,0)--(4,-4)--(0,-4)--cycle;
  \fill[color=myred] (0,0)--(0,3)--(-3,3)--(-3,0)--cycle;
  \fill[color=myyellow] (4,0)--(7,4)--(3,7)--(0,3)--cycle;
  \node[scale=5] at (3.5,3.5) {Exo7};
\end{tikzpicture}}
}


\newcommand{\debutmontitre}{
  \author{} \date{} 
  \thispagestyle{empty}
  \hspace*{-10ex}
  \begin{minipage}{\textwidth}
    \titlepage  
  \vspace*{-2.5cm}
  \begin{center}
    \LogoExoSept{2.5}
  \end{center}
  \end{minipage}

  \vspace*{-0cm}
  
  % Astuce pour que le background ne soit pas discrétisé lors de la conversion pdf -> png
\begin{tikzpicture}
        \fill[opacity=0,green!60!black] (0,0)--++(0,0)--++(0,0)--++(0,0)--cycle; 
\end{tikzpicture}

% toc S'affiche trop tot :
% \tableofcontents[hideallsubsections, pausesections]
}

\newcommand{\finmontitre}{
  \end{frame}
  \setcounter{framenumber}{0}
} % ne marche pas pour une raison obscure

%----- Commandes supplementaires ------

% \usepackage[landscape]{geometry}
% \geometry{top=1cm, bottom=3cm, left=2cm, right=10cm, marginparsep=1cm
% }
% \usepackage[a4paper]{geometry}
% \geometry{top=2cm, bottom=2cm, left=2cm, right=2cm, marginparsep=1cm
% }

%\usepackage{standalone}


% New command Arnaud -- november 2011
\setbeamersize{text margin left=24ex}
% si vous modifier cette valeur il faut aussi
% modifier le decalage du titre pour compenser
% (ex : ici =+10ex, titre =-5ex

\theoremstyle{definition}
%\newtheorem{proposition}{Proposition}
%\newtheorem{exemple}{Exemple}
%\newtheorem{theoreme}{Théorème}
%\newtheorem{lemme}{Lemme}
%\newtheorem{corollaire}{Corollaire}
%\newtheorem*{remarque*}{Remarque}
%\newtheorem*{miniexercice}{Mini-exercices}
%\newtheorem{definition}{Définition}

% Commande tikz
\usetikzlibrary{calc}
\usetikzlibrary{patterns,arrows}
\usetikzlibrary{matrix}
\usetikzlibrary{fadings} 

%definition d'un terme
\newcommand{\defi}[1]{{\color{myorange}\textbf{\emph{#1}}}}
\newcommand{\evidence}[1]{{\color{blue}\textbf{\emph{#1}}}}
\newcommand{\assertion}[1]{\emph{\og#1\fg}}  % pour chapitre logique
%\renewcommand{\contentsname}{Sommaire}
\renewcommand{\contentsname}{}
\setcounter{tocdepth}{2}



%------ Encadrement ------

\usepackage{fancybox}


\newcommand{\mybox}[1]{
\setlength{\fboxsep}{7pt}
\begin{center}
\shadowbox{#1}
\end{center}}

\newcommand{\myboxinline}[1]{
\setlength{\fboxsep}{5pt}
\raisebox{-10pt}{
\shadowbox{#1}
}
}

%--------------- Commande beamer---------------
\newcommand{\beameronly}[1]{#1} % permet de mettre des pause dans beamer pas dans poly


\setbeamertemplate{navigation symbols}{}
\setbeamertemplate{footline}  % tiré du fichier beamerouterinfolines.sty
{
  \leavevmode%
  \hbox{%
  \begin{beamercolorbox}[wd=.333333\paperwidth,ht=2.25ex,dp=1ex,center]{author in head/foot}%
    % \usebeamerfont{author in head/foot}\insertshortauthor%~~(\insertshortinstitute)
    \usebeamerfont{section in head/foot}{\bf\insertshorttitle}
  \end{beamercolorbox}%
  \begin{beamercolorbox}[wd=.333333\paperwidth,ht=2.25ex,dp=1ex,center]{title in head/foot}%
    \usebeamerfont{section in head/foot}{\bf\insertsectionhead}
  \end{beamercolorbox}%
  \begin{beamercolorbox}[wd=.333333\paperwidth,ht=2.25ex,dp=1ex,right]{date in head/foot}%
    % \usebeamerfont{date in head/foot}\insertshortdate{}\hspace*{2em}
    \insertframenumber{} / \inserttotalframenumber\hspace*{2ex} 
  \end{beamercolorbox}}%
  \vskip0pt%
}


\definecolor{mygrey}{rgb}{0.5,0.5,0.5}
\setlength{\parindent}{0cm}
%\DeclareTextFontCommand{\helvetica}{\fontfamily{phv}\selectfont}

% background beamer
\definecolor{couleurhaut}{rgb}{0.85,0.9,1}  % creme
\definecolor{couleurmilieu}{rgb}{1,1,1}  % vert pale
\definecolor{couleurbas}{rgb}{0.85,0.9,1}  % blanc
\setbeamertemplate{background canvas}[vertical shading]%
[top=couleurhaut,middle=couleurmilieu,midpoint=0.4,bottom=couleurbas] 
%[top=fondtitre!05,bottom=fondtitre!60]



\makeatletter
\setbeamertemplate{theorem begin}
{%
  \begin{\inserttheoremblockenv}
  {%
    \inserttheoremheadfont
    \inserttheoremname
    \inserttheoremnumber
    \ifx\inserttheoremaddition\@empty\else\ (\inserttheoremaddition)\fi%
    \inserttheorempunctuation
  }%
}
\setbeamertemplate{theorem end}{\end{\inserttheoremblockenv}}

\newenvironment{theoreme}[1][]{%
   \setbeamercolor{block title}{fg=structure,bg=structure!40}
   \setbeamercolor{block body}{fg=black,bg=structure!10}
   \begin{block}{{\bf Th\'eor\`eme }#1}
}{%
   \end{block}%
}


\newenvironment{proposition}[1][]{%
   \setbeamercolor{block title}{fg=structure,bg=structure!40}
   \setbeamercolor{block body}{fg=black,bg=structure!10}
   \begin{block}{{\bf Proposition }#1}
}{%
   \end{block}%
}

\newenvironment{corollaire}[1][]{%
   \setbeamercolor{block title}{fg=structure,bg=structure!40}
   \setbeamercolor{block body}{fg=black,bg=structure!10}
   \begin{block}{{\bf Corollaire }#1}
}{%
   \end{block}%
}

\newenvironment{mydefinition}[1][]{%
   \setbeamercolor{block title}{fg=structure,bg=structure!40}
   \setbeamercolor{block body}{fg=black,bg=structure!10}
   \begin{block}{{\bf Définition} #1}
}{%
   \end{block}%
}

\newenvironment{lemme}[0]{%
   \setbeamercolor{block title}{fg=structure,bg=structure!40}
   \setbeamercolor{block body}{fg=black,bg=structure!10}
   \begin{block}{\bf Lemme}
}{%
   \end{block}%
}

\newenvironment{remarque}[1][]{%
   \setbeamercolor{block title}{fg=black,bg=structure!20}
   \setbeamercolor{block body}{fg=black,bg=structure!5}
   \begin{block}{Remarque #1}
}{%
   \end{block}%
}


\newenvironment{exemple}[1][]{%
   \setbeamercolor{block title}{fg=black,bg=structure!20}
   \setbeamercolor{block body}{fg=black,bg=structure!5}
   \begin{block}{{\bf Exemple }#1}
}{%
   \end{block}%
}


\newenvironment{miniexercice}[0]{%
   \setbeamercolor{block title}{fg=structure,bg=structure!20}
   \setbeamercolor{block body}{fg=black,bg=structure!5}
   \begin{block}{Mini-exercices}
}{%
   \end{block}%
}


\newenvironment{tp}[0]{%
   \setbeamercolor{block title}{fg=structure,bg=structure!40}
   \setbeamercolor{block body}{fg=black,bg=structure!10}
   \begin{block}{\bf Travaux pratiques}
}{%
   \end{block}%
}
\newenvironment{exercicecours}[1][]{%
   \setbeamercolor{block title}{fg=structure,bg=structure!40}
   \setbeamercolor{block body}{fg=black,bg=structure!10}
   \begin{block}{{\bf Exercice }#1}
}{%
   \end{block}%
}
\newenvironment{algo}[1][]{%
   \setbeamercolor{block title}{fg=structure,bg=structure!40}
   \setbeamercolor{block body}{fg=black,bg=structure!10}
   \begin{block}{{\bf Algorithme}\hfill{\color{gray}\texttt{#1}}}
}{%
   \end{block}%
}


\setbeamertemplate{proof begin}{
   \setbeamercolor{block title}{fg=black,bg=structure!20}
   \setbeamercolor{block body}{fg=black,bg=structure!5}
   \begin{block}{{\footnotesize Démonstration}}
   \footnotesize
   \smallskip}
\setbeamertemplate{proof end}{%
   \end{block}}
\setbeamertemplate{qed symbol}{\openbox}


\makeatother
\usecolortheme[RGB={66,15,15}]{structure}

%%%%%%%%%%%%%%%%%%%%%%%%%%%%%%%%%%%%%%%%%%%%%%%%%%%%%%%%%%%%%
%%%%%%%%%%%%%%%%%%%%%%%%%%%%%%%%%%%%%%%%%%%%%%%%%%%%%%%%%%%%%


\begin{document}


\title{{\bf Limites et fonctions continues}}
\subtitle{Continuité en un point}

\begin{frame}
  
  \debutmontitre

  \pause

{\footnotesize
\hfill
\setbeamercovered{transparent=50}
\begin{minipage}{0.6\textwidth}
  \begin{itemize}
    \item<3-> Définitions
    \item<4-> Opérations sur les fonctions continues
    \item<5-> Prolongement par continuité
    \item<6-> Suites et continuité
  \end{itemize}
\end{minipage}
}

\end{frame}

\setcounter{framenumber}{0}


%%%%%%%%%%%%%%%%%%%%%%%%%%%%%%%%%%%%%%%%%%%%%%%%%%%%%%%%%%%%%%%%

%---------------------------------------------------------------
\section{Définition}

\begin{frame}

\begin{mydefinition}
\begin{itemize}
  \item $f:I\to\Rr$ est \defi{continue en $x_0\in I$} si
\vspace*{1ex}

\hspace*{-1.5em}
\begin{minipage}{\textwidth}
\mybox{$
\forall \epsilon>0 \quad \exists \delta>0 \quad \forall x\in I \quad \vert x-x_0\vert <\delta 
\implies \vert f(x)-f(x_0)\vert <\epsilon$}  
\end{minipage}

\vspace*{1ex}
\pause 
c'est-à-dire si $f(x)$ tend vers $f(x_0)$ lorsque $x$ tend vers $x_0$

  \item <4-> $f$ est \defi{continue sur $I$} si $f$ est continue en tout point de $I$
% \vspace*{-1ex}
 \end{itemize}
\end{mydefinition}

 \pause 
\vspace*{-1ex}
\myfigure{.85}{
\tikzinput{fig_fonctions4bis}
} 
 
\end{frame}



\begin{frame}

\centerline{\hspace*{-2em}Les fonctions suivantes \evidence{ne sont pas continues} au point $x_0$}

\myfigure{1}{
\tikzinput{fig_fonctions6}
}
  
\end{frame}


\begin{frame}


\begin{exemple}
\begin{itemize}
\item Les fonctions suivantes sont continues
\begin{itemize}
\item \pause fonction constante sur un intervalle
\item \pause $x\mapsto\sqrt{x}$ sur $[0,+\infty[$
\item \pause $\sin$ et $\cos$ sur $\Rr$
\item \pause $x\mapsto\vert x\vert$ sur $\Rr$
\item \pause $\exp$ sur $\Rr$ \ et \ $\ln$ sur $]0,+\infty[$
\end{itemize}

\item \pause $E$ n'est pas continue aux points $x_0\in\Z$


Mais elle est continue en $x_0\in \Rr \setminus \Zz$
\end{itemize}
\end{exemple}
\myfigure{0.6}{
\tikzinput{fig_fonctionsA09}
}  
\end{frame}

%---------------------------------------------------------------
\section{Propriétés}

\begin{frame}

\begin{lemme}
Si $f$ est continue en $x_0$ et si $f(x_0)\neq 0$, alors il existe $\delta>0$ tel que
\[
\forall x\in ]x_0-\delta,x_0+\delta [ \quad f(x)\neq 0
\]
\end{lemme}
\pause
\myfigure{.85}{
\tikzinput{fig_fonctionsA13}
} 
\pause
\begin{proof}
Supposons $f(x_0)>0$. \pause Par hypothèse

$\forall \epsilon>0 \quad \exists \delta>0  \quad  
x\in \, ]x_0-\delta,x_0+\delta [  \implies f(x_0)-\epsilon < f(x) <f(x_0)+\epsilon$

\pause
On fixe $\epsilon$ tel que $0<\epsilon<f(x_0)$,
\pause
alors
$f(x) > f(x_0)-\epsilon >0$
\end{proof}
  
\end{frame}

\begin{frame}

\begin{proposition}
Soient $f$ et $g$ deux fonctions continues en $x_0\in\Rr$. \pause Alors
\begin{itemize}
  \item $\lambda\cdot f$ est continue en $x_0$ (pour tout $\lambda\in\R$)
  \item\pause  $f+g$ est continue en $x_0$
  \item\pause  $f\times g$ est continue en $x_0$
  \item\pause  si $f(x_0)\neq 0$ alors $\frac1f$ est continue en $x_0$
\end{itemize}
\end{proposition}

\pause 
\begin{exemple}
On en déduit que les fonctions suivantes sont continues
\begin{itemize}
\item\pause  $x\mapsto x^n$ sur $\Rr$ (comme produit $x \cdot x \cdot \ldots$)
\item\pause  les polynômes sur $\Rr$ 
\item\pause  les fractions rationnelles $x\mapsto \frac{P(x)}{Q(x)}$ sur tout intervalle où  $Q(x)\neq 0$
\end{itemize}
\end{exemple}  
  
\end{frame}


\begin{frame}
  
\begin{proposition}
Si $f$ est continue en $x_0$ et si $g$ est continue en $f(x_0)$
alors $g\circ f$ est continue en $x_0$
\end{proposition}  
  
\end{frame}


%---------------------------------------------------------------
\section{Prolongement par continuité}

\begin{frame}

\begin{mydefinition}
Soit $I$ un intervalle, $x_0\in I$ et $f:I\setminus\{x_0\}\to\Rr$ une fonction
\pause 
\begin{itemize}
  \item $f$ est \defi{prolongeable par continuité} en $x_0$ si $f$ a une 
  limite finie en $x_0$. \pause Notons $\ell=\displaystyle\lim_{x_0} f$
  \item<4-> On définit alors le \defi{prolongement par continuité} de $f$ en $x_0$ par
  \vspace*{-1ex}
  \[
  \tilde f(x) =
  \begin{cases}
  f(x) &\text{ si } x\neq x_0\\
  \ell &\text{ si } x=x_0
  \end{cases}
  \] 
  
   \vspace*{-2ex}
  \pause 
 Alors $\tilde f:I\to\Rr$ est continue en $x_0$
\end{itemize}
\end{mydefinition}

\myfigure{.9}{
\tikzinput{fig_fonctionsA15}
}
  
\end{frame}


\begin{frame}

\begin{exemple}
La fonction $f:\Rr^*\to\Rr$, $f(x)=x\sin\left(\frac1x\right)$ admet-elle un prolongement par continuité en $0$ ?  \pause

\medskip

\centerline{$\forall x\in\Rr^* \quad \vert f(x)\vert\leq \vert x\vert  \pause \ \implies \ \displaystyle\lim_0 f = 0$}

\bigskip

\pause
Donc $f$ est donc prolongeable par continuité en $0$ et $\tilde f:\Rr\to\Rr$ est
\[
\tilde f(x) = 
  \begin{cases}
  x\sin\left(\frac1x\right) &\text{ si } x\neq 0\\
  0 &\text{ si } x=0
  \end{cases}
\]

\end{exemple}
  
\end{frame}


%---------------------------------------------------------------
\section{Suites et continuité}

\begin{frame}

\begin{proposition}
Soit $f:I\to\Rr$ une fonction et $x_0\in I$. \pause Alors :
\mybox{$f \text{ continue en } x_0 \ \iff \
\begin{matrix}
\text{pour toute suite $(u_n)$ qui tend vers } x_0\\
\text{la suite $(f(u_n))$ tend vers } f(x_0)
\end{matrix}$}
\end{proposition}
 
\pause 
\begin{remarque}
\'Etude des suites récurrentes :  $u_{n+1}= f(u_n)$\pause
\mybox{si $f$ est continue et $u_n\to \ell$, alors $f(u_n)\to f(\ell)$, 
et donc $f(\ell)=\ell$}
\end{remarque} 
  
\end{frame}



\begin{frame}

\centerline{$f \text{ continue en } x_0 \ \iff \
\begin{matrix}
\text{pour toute suite $(u_n)$ qui converge vers } x_0\\
\text{la suite $(f(u_n))$ converge vers } f(x_0)
\end{matrix}$}
 
\begin{proof}
\begin{itemize}\pause 
  \item[$\implies$] 
  \emph{On suppose que $f$ est continue en $x_0$ et $u_n\to x_0$. 
  On montre $f(u_n)\to f(x_0)$} 
  
\pause  
  Soit $\epsilon >0$
  \pause
\begin{itemize}
  \item Comme $f$ est continue en $x_0$
\[
\exists \delta>0 \quad |x-x_0|<\delta \implies |f(x)-f(x_0)|<\epsilon
\]
\pause 
  \item Comme $u_n\to x_0$ 
$$\begin{array}{rl}
\exists N\in\Nn  \quad  n\geq N & \implies |u_n-x_0|<\delta\\
\pause ~ & \implies |f(u_n)-f(x_0)|<\epsilon
\end{array}$$
\pause
\item  Donc $f(u_n)\to f(x_0)$
\end{itemize}
\end{itemize}
\end{proof}
\end{frame}


\begin{frame} 

\centerline{$f \text{ continue en } x_0 \ \iff \
\begin{matrix}
\text{pour toute suite $(u_n)$ qui converge vers } x_0\\
\text{la suite $(f(u_n))$ converge vers } f(x_0)
\end{matrix}$}



\begin{proof}  \pause
\begin{itemize}
  \item[$\Longleftarrow$] \emph{On suppose que $f$ n'est pas continue en $x_0$. 
  On montre qu'il existe $(u_n)$ qui converge vers $x_0$ et 
  telle que $(f(u_n))$ ne converge pas vers $f(x_0)$} 

  \pause 
  \begin{itemize}
  \item Comme $f$ n'est pas continue en $x_0$ 
\[
\exists \epsilon_0>0 \quad \forall\delta>0 \quad \exists x_\delta \in I \quad |x_\delta-x_0|<\delta \text{ et } |f(x_\delta)-f(x_0)|>\epsilon_0
\]
\pause
  \item pour tout $n\in \Nn^*$, 
on choisit $\delta=1/n$. \pause Alors il existe $u_n$ tel que 
\[
|u_n-x_0|<\frac1n \quad \text{et} \quad |f(u_n)-f(x_0)|>\epsilon_0
\] 
\pause
  \item $(u_n)$ tend vers $x_0$ mais $(f(u_n))$ ne tend pas vers $f(x_0)$

  \end{itemize}
\end{itemize}
\end{proof}

  
\end{frame}



%%%%%%%%%%%%%%%%%%%%%%%%%%%%%%%%%%%%%%%%%%%%%%%%%%%%%%%%%%%%%%%%
\section{Mini-exercices}

\begin{frame}

\begin{miniexercice}
\begin{enumerate}
  \item Déterminer le domaine de définition et de continuité de
$f(x) = 1/\sin x$, $g(x) = 1/\sqrt{x+\frac12}$ et $h(x) = \ln(x^2+x-1)$.  
  
  \item Trouver les couples $(a,b)\in\Rr^2$ tels que la fonction $f$ définie sur $\Rr$ par 
  $f(x) = ax+b$ si $x< 0$ et $f(x) = \exp(x)$  si $x\geq 0$ soit continue sur $\Rr$. 
  Et si on avait $f(x) = \frac{a}{x-1}+b$ pour $x<0$ ?

  \item Soit $f$ une fonction continue telle que $f(x_0)=1$. Montrer qu'il existe $\delta>0$ tel que : 
  pour tout $x\in ]x_0 - \delta, x_0+\delta[ \quad f(x) > \frac12$.
  
  \item \'Etudier la continuité de $f : \Rr \to \Rr$ définie par : $f(0)=0$ et
  $f(x) = \sin(x)\cos\left(\frac1x\right)$ si $x\neq 0$.
  Et pour $g(x)=xE(x)$ ?

  \item La fonction définie par $f(x)=\frac{x^3+8}{|x+2|}$ admet-elle 
  un prolongement par continuité en $-2$ ?
  
  \item Soit la suite définie par $u_0>0$ et $u_{n+1}=\sqrt{u_n}$. Montrer que $(u_n)$ 
  admet une limite $\ell \in \Rr$ lorsque $n\to+\infty$. \`A l'aide de la fonction 
  $f(x)=\sqrt{x}$ calculer cette limite.
\end{enumerate}
\end{miniexercice}

\end{frame}

\end{document}
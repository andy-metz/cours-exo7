
%%%%%%%%%%%%%%%%%% PREAMBULE %%%%%%%%%%%%%%%%%%


\documentclass[12pt]{article}

\usepackage{amsfonts,amsmath,amssymb,amsthm}
\usepackage[utf8]{inputenc}
\usepackage[T1]{fontenc}
\usepackage[francais]{babel}


% packages
\usepackage{amsfonts,amsmath,amssymb,amsthm}
\usepackage[utf8]{inputenc}
\usepackage[T1]{fontenc}
%\usepackage{lmodern}

\usepackage[francais]{babel}
\usepackage{fancybox}
\usepackage{graphicx}

\usepackage{float}

%\usepackage[usenames, x11names]{xcolor}
\usepackage{tikz}
\usepackage{datetime}

\usepackage{mathptmx}
%\usepackage{fouriernc}
%\usepackage{newcent}
\usepackage[mathcal,mathbf]{euler}

%\usepackage{palatino}
%\usepackage{newcent}


% Commande spéciale prompteur

%\usepackage{mathptmx}
%\usepackage[mathcal,mathbf]{euler}
%\usepackage{mathpple,multido}

\usepackage[a4paper]{geometry}
\geometry{top=2cm, bottom=2cm, left=1cm, right=1cm, marginparsep=1cm}

\newcommand{\change}{{\color{red}\rule{\textwidth}{1mm}\\}}

\newcounter{mydiapo}

\newcommand{\diapo}{\newpage
\hfill {\normalsize  Diapo \themydiapo \quad \texttt{[\jobname]}} \\
\stepcounter{mydiapo}}


%%%%%%% COULEURS %%%%%%%%%%

% Pour blanc sur noir :
%\pagecolor[rgb]{0.5,0.5,0.5}
% \pagecolor[rgb]{0,0,0}
% \color[rgb]{1,1,1}



%\DeclareFixedFont{\myfont}{U}{cmss}{bx}{n}{18pt}
\newcommand{\debuttexte}{
%%%%%%%%%%%%% FONTES %%%%%%%%%%%%%
\renewcommand{\baselinestretch}{1.5}
\usefont{U}{cmss}{bx}{n}
\bfseries

% Taille normale : commenter le reste !
%Taille Arnaud
%\fontsize{19}{19}\selectfont

% Taille Barbara
%\fontsize{21}{22}\selectfont

%Taille François
\fontsize{25}{30}\selectfont

%Taille Pascal
%\fontsize{25}{30}\selectfont

%Taille Laura
%\fontsize{30}{35}\selectfont


%\myfont
%\usefont{U}{cmss}{bx}{n}

%\Huge
%\addtolength{\parskip}{\baselineskip}
}


% \usepackage{hyperref}
% \hypersetup{colorlinks=true, linkcolor=blue, urlcolor=blue,
% pdftitle={Exo7 - Exercices de mathématiques}, pdfauthor={Exo7}}


%section
% \usepackage{sectsty}
% \allsectionsfont{\bf}
%\sectionfont{\color{Tomato3}\upshape\selectfont}
%\subsectionfont{\color{Tomato4}\upshape\selectfont}

%----- Ensembles : entiers, reels, complexes -----
\newcommand{\Nn}{\mathbb{N}} \newcommand{\N}{\mathbb{N}}
\newcommand{\Zz}{\mathbb{Z}} \newcommand{\Z}{\mathbb{Z}}
\newcommand{\Qq}{\mathbb{Q}} \newcommand{\Q}{\mathbb{Q}}
\newcommand{\Rr}{\mathbb{R}} \newcommand{\R}{\mathbb{R}}
\newcommand{\Cc}{\mathbb{C}} 
\newcommand{\Kk}{\mathbb{K}} \newcommand{\K}{\mathbb{K}}

%----- Modifications de symboles -----
\renewcommand{\epsilon}{\varepsilon}
\renewcommand{\Re}{\mathop{\text{Re}}\nolimits}
\renewcommand{\Im}{\mathop{\text{Im}}\nolimits}
%\newcommand{\llbracket}{\left[\kern-0.15em\left[}
%\newcommand{\rrbracket}{\right]\kern-0.15em\right]}

\renewcommand{\ge}{\geqslant}
\renewcommand{\geq}{\geqslant}
\renewcommand{\le}{\leqslant}
\renewcommand{\leq}{\leqslant}

%----- Fonctions usuelles -----
\newcommand{\ch}{\mathop{\mathrm{ch}}\nolimits}
\newcommand{\sh}{\mathop{\mathrm{sh}}\nolimits}
\renewcommand{\tanh}{\mathop{\mathrm{th}}\nolimits}
\newcommand{\cotan}{\mathop{\mathrm{cotan}}\nolimits}
\newcommand{\Arcsin}{\mathop{\mathrm{Arcsin}}\nolimits}
\newcommand{\Arccos}{\mathop{\mathrm{Arccos}}\nolimits}
\newcommand{\Arctan}{\mathop{\mathrm{Arctan}}\nolimits}
\newcommand{\Argsh}{\mathop{\mathrm{Argsh}}\nolimits}
\newcommand{\Argch}{\mathop{\mathrm{Argch}}\nolimits}
\newcommand{\Argth}{\mathop{\mathrm{Argth}}\nolimits}
\newcommand{\pgcd}{\mathop{\mathrm{pgcd}}\nolimits} 

\newcommand{\Card}{\mathop{\text{Card}}\nolimits}
\newcommand{\Ker}{\mathop{\text{Ker}}\nolimits}
\newcommand{\id}{\mathop{\text{id}}\nolimits}
\newcommand{\ii}{\mathrm{i}}
\newcommand{\dd}{\mathrm{d}}
\newcommand{\Vect}{\mathop{\text{Vect}}\nolimits}
\newcommand{\Mat}{\mathop{\mathrm{Mat}}\nolimits}
\newcommand{\rg}{\mathop{\text{rg}}\nolimits}
\newcommand{\tr}{\mathop{\text{tr}}\nolimits}
\newcommand{\ppcm}{\mathop{\text{ppcm}}\nolimits}

%----- Structure des exercices ------

\newtheoremstyle{styleexo}% name
{2ex}% Space above
{3ex}% Space below
{}% Body font
{}% Indent amount 1
{\bfseries} % Theorem head font
{}% Punctuation after theorem head
{\newline}% Space after theorem head 2
{}% Theorem head spec (can be left empty, meaning ‘normal’)

%\theoremstyle{styleexo}
\newtheorem{exo}{Exercice}
\newtheorem{ind}{Indications}
\newtheorem{cor}{Correction}


\newcommand{\exercice}[1]{} \newcommand{\finexercice}{}
%\newcommand{\exercice}[1]{{\tiny\texttt{#1}}\vspace{-2ex}} % pour afficher le numero absolu, l'auteur...
\newcommand{\enonce}{\begin{exo}} \newcommand{\finenonce}{\end{exo}}
\newcommand{\indication}{\begin{ind}} \newcommand{\finindication}{\end{ind}}
\newcommand{\correction}{\begin{cor}} \newcommand{\fincorrection}{\end{cor}}

\newcommand{\noindication}{\stepcounter{ind}}
\newcommand{\nocorrection}{\stepcounter{cor}}

\newcommand{\fiche}[1]{} \newcommand{\finfiche}{}
\newcommand{\titre}[1]{\centerline{\large \bf #1}}
\newcommand{\addcommand}[1]{}
\newcommand{\video}[1]{}

% Marge
\newcommand{\mymargin}[1]{\marginpar{{\small #1}}}



%----- Presentation ------
\setlength{\parindent}{0cm}

%\newcommand{\ExoSept}{\href{http://exo7.emath.fr}{\textbf{\textsf{Exo7}}}}

\definecolor{myred}{rgb}{0.93,0.26,0}
\definecolor{myorange}{rgb}{0.97,0.58,0}
\definecolor{myyellow}{rgb}{1,0.86,0}

\newcommand{\LogoExoSept}[1]{  % input : echelle
{\usefont{U}{cmss}{bx}{n}
\begin{tikzpicture}[scale=0.1*#1,transform shape]
  \fill[color=myorange] (0,0)--(4,0)--(4,-4)--(0,-4)--cycle;
  \fill[color=myred] (0,0)--(0,3)--(-3,3)--(-3,0)--cycle;
  \fill[color=myyellow] (4,0)--(7,4)--(3,7)--(0,3)--cycle;
  \node[scale=5] at (3.5,3.5) {Exo7};
\end{tikzpicture}}
}



\theoremstyle{definition}
%\newtheorem{proposition}{Proposition}
%\newtheorem{exemple}{Exemple}
%\newtheorem{theoreme}{Théorème}
\newtheorem{lemme}{Lemme}
\newtheorem{corollaire}{Corollaire}
%\newtheorem*{remarque*}{Remarque}
%\newtheorem*{miniexercice}{Mini-exercices}
%\newtheorem{definition}{Définition}




%definition d'un terme
\newcommand{\defi}[1]{{\color{myorange}\textbf{\emph{#1}}}}
\newcommand{\evidence}[1]{{\color{blue}\textbf{\emph{#1}}}}



 %----- Commandes divers ------

\newcommand{\codeinline}[1]{\texttt{#1}}

%%%%%%%%%%%%%%%%%%%%%%%%%%%%%%%%%%%%%%%%%%%%%%%%%%%%%%%%%%%%%
%%%%%%%%%%%%%%%%%%%%%%%%%%%%%%%%%%%%%%%%%%%%%%%%%%%%%%%%%%%%%



\begin{document}

\debuttexte



%%%%%%%%%%%%%%%%%%%%%%%%%%%%%%%%%%%%%%%%%%%%%%%%%%%%%%%%%%
\diapo


\change

Nous allons étudier les propriétés des fonctions qui sont continues sur un intervalle.


\change

Le principal résultat sera le théorème des valeurs intermédiaires et ses corollaires.

\change

On énoncera un théorème plus précis lorsque l'intervalle est fermé et borné.


%%%%%%%%%%%%%%%%%%%%%%%%%%%%%%%%%%%%%%%%%%%%%%%%%%%%%%%%%%
\diapo


Voici le théorème des valeurs intermédiaires qui est le résultat principal de cette leçon. 
Nous allons le prouver, en déduire plusieurs variantes et des applications, et étudier des exemples.


On part d'une fonction $f$ qui est définie et continue sur un segment $[a,b]$.


Le théorème des valeurs intermédiaires affirme que quelque soit la valeur réelle
 $y$ comprise entre $f(a)$ et $f(b)$, alors il existe un réel $c$ de l'intervalle de départ $[a,b]$ 
 tel que $y=f(c)$.
 
\change

Sur ce dessin la fonction $f$ représentée est continue.

Voici $a$ et $f(a)$, $b$ et $f(b)$.

Pour n'importe quel $y$ compris entre $f(a)$ et $f(b)$,
il y a effectivement au moins un $c$ tel que $y=f(c)$.

Pour cette valeur de $y$ il y en a même trois.

Il n'y a donc en général pas unicité des valeurs $c$.

\change

Voici une situation qui justifie que l'hypothèse "continue" dans l'énoncé du 
théorème des valeurs intermédiaires est nécessaire :

pour cette valeur de $y$, qui est bien comprise entre $f(a)$ et $f(b)$,
il n'y a aucun $c$ qui vérifie $f(c)=y$.

La fonction n'est pas continue car  il n'y a un "saut" ici,
donc on ne peut pas appliquer ce théorème.




%%%%%%%%%%%%%%%%%%%%%%%%%%%%%%%%%%%%%%%%%%%%%%%%%%%%%%%%%%
\diapo

Montrons le théorème dans le cas où $f(a)<f(b)$. On considère alors un réel $y$ tel que 
$f(a)\leq y\leq f(b)$ et on veut montrer qu'il a un antécédent par $f$.

\change

On introduit l'ensemble suivant
\[
A=\Big\{ x\in [a,b] \ \vert \ f(x)\leq y \Big\}.
\]

\change

Voici une figure. $A$, ici en vert, représente les $x$ tels que $f(x)$ reste inférieure à notre valeur fixée $y$.

\change

Tout d'abord l'ensemble $A$ est non vide (car $a\in A$) et il est majoré (car il est 
contenu dans l'intervalle $[a,b] $) :

\change

$A$ admet donc une borne supérieure, que l'on note 
$c=\sup A$. 

\change

Dans la diapo suivante nous allons montrer que $f(c)=y$.

%%%%%%%%%%%%%%%%%%%%%%%%%%%%%%%%%%%%%%%%%%%%%%%%%%%%%%%%%%%
\diapo

\change

On a défini  $c$ comme étant le sup de $A$.
Pour montrer que $f(c)=y$ on commence par prouver que $f(c)\leq y$. 

\change

Comme $c=\sup A$, il existe une suite 
$(u_n)$ d'élément de $A$ telle que $(u_n)$ converge vers $c$. 


\change

D'une part, pour tout $n$, comme $u_n\in A$, alors par définition de 
l'ensemble $A$ on a $f(u_n)\leq y$. 

\change

D'autre part, comme $f$ est continue en $c$, le fait que $u_n$ tende vers $c$
implique que la suite $\left(f(u_n)\right)$ 
tend vers $f(c)$.


\change
 On en déduit donc, par passage à la limite, que $f(c)\leq y$.
 
\change

Montrons à présent que $f(c)\geq y$.


\change

Remarquons tout d'abord que si $c=b$, 
alors on a fini, puisque $f(b)\geq y$. 

\change

Sinon $c<b$. Pour tout $x\in]c,b]$, comme $x\notin A$, alors par définition de l'ensemble $A$
on a $f(x)>y$. 

\change

Or, étant donné que $f$ est continue en $c$, $f$ admet une limite 
en $c$, qui vaut $f(c)$ et on obtient $f(c)\geq y$.


Conclusion : on a explicité un $c$ tel que $y=f(c)$.
Ce qui termine la preuve du théorème des valeurs intermédiaires.


%%%%%%%%%%%%%%%%%%%%%%%%%%%%%%%%%%%%%%%%%%%%%%%%%%%%%%%%%%
\diapo

Voici la version la plus utilisée du théorème des valeurs intermédiaires.

\change

Partons d'une fonction $f$ qui est continue sur un segment $[a,b]$.

"Si $f(a)\cdot f(b)<0$, alors il existe $c$ de l'intervalle ouvert $]a,b[$ tel que $f(c)=0$."


L'hypothèse $f(a)\cdot f(b)<0$ signifie tout simplement que $f(a)$ et $f(b)$ 
sont de signes opposés : l'un positif, l'autre négatif.


\change

Pour la preuve il s'agit d'une application directe du théorème des valeurs intermédiaires avec la valeur $y=0$.

\change

Voici une représentation graphique : ici $f(a)$ est négatif, $f(b)$ positif, donc le produit
$f(a)\cdot f(b)$ est bien négatif, notre fonction continue $f$ s'annule bien entre $a$ et $b$, 
ici en $c$.

Notez de nouveau que la fonction pourrait aussi s'annuler plusieurs fois,
et que l'hypothèse continue est bien sûr essentielle, pour éviter que la fonction fasse un saut.


%%%%%%%%%%%%%%%%%%%%%%%%%%%%%%%%%%%%%%%%%%%%%%%%%%%%%%%%%%
\diapo

Voyons un exemple d'application : nous allons prouver que
tout polynôme dont le de degré est impair possède une racine réelle,
c-à-d s'annule au moins une fois.


\change

Un tel polynôme s'écrit $P(x)=a_nx^n+\cdots+a_1x+a_0$ avec $n$ un entier impair. 

On peut supposer, par exemple, que le coefficient dominant $a_n$ est strictement positif.

\change

Alors on a 
$\displaystyle\lim_{-\infty} P = -\infty$ et $\displaystyle\lim_{+\infty} P = +\infty$. 

\change


En particulier, il existe un réel $a$ tel que $P(a)<0$ 

et un réel $b$ tel que $P(b)>0$.

\change

On conclut grâce au corollaire précédent : la fonction $x \mapsto P(x)$ est une fonction continue,
$P(a)<0$ et $P(b)>0$ donc le polynôme s'annule, au moins une fois.


%%%%%%%%%%%%%%%%%%%%%%%%%%%%%%%%%%%%%%%%%%%%%%%%%%%%%%%%%%
\diapo

Voici une conséquence directe du théorème des valeurs intermédiaires, que l'on appelle aussi
théorème des des valeurs intermédiaires

Corollaire : "Soit $f$ une fonction continue sur un intervalle $I$. 
Alors $f(I)$ est un intervalle."


\change

Attention ! Il serait faux de croire que l'image par une fonction $f$ de l'intervalle $[a,b]$ 
soit l'intervalle $[f(a),f(b)]$.

[petit $m$, grand $M$]
Ce n'est pas ce que dit ce théorème, l'image de l'intervalle $[a,b]$ est un intervalle $[m,M]$. 
Nous verrons juste après ce que sont ces valeurs $m$ et grand $M$.


\change

Voici un exemple graphique :

$f$ est continue sur l'intervalle $I$ qui est l'intervalle $a,b$ en rouge.

L'image de l'intervalle se lit sur l'axe des ordonnées, et est représenté en vert ici : 
c'est bien un intervalle.

Notez que, comme je vous l'ai dit, l'image ne vaut pas l'intervalle $f(a),f(b)$, elle est ici plus grande.




%%%%%%%%%%%%%%%%%%%%%%%%%%%%%%%%%%%%%%%%%%%%%%%%%%%%%%%%%%
\diapo

Passons à la démonstration que l'image d'un intervalle par une fonction continue reste un intervalle.

\change

Soient $y_1,y_2$ deux valeurs de l'image $f(I)$ avec $y_1\leq y_2$. 

Il s'agit de montrer que si $y\in[y_1,y_2]$, alors $y\in f(I)$. Si on le prouve quelque soit $y_1,y_2$ et $y$ 
alors cela prouvera que $f(I)$ est bien un intervalle.


\change

Par hypothèse, comme $y_1$ est dans $f(I)$ alors il existe $x_1\in I$ tels que $y_1 =f(x_1)$, 

de même il existe $x_2 \in I$ tel que $y_2 =f(x_2)$.

\change

Par hypothèse  $y$ est compris entre $y_1=f(x_1)$ et $y_2=f(x_2)$. 

\change

D'après le théorème des valeurs intermédiaires, 
$f$ étant une fonction continue, 
toute valeur entre $f(x_1)$ et $f(x_2)$ se réalise comme une image :

il existe donc $x$ entre $x_1$ et $x_2$ et donc dans $I$ tel que $y=f(x)$

\change


et donc $y$ qui est l'image d'un élément de $I$ appartient à $f(I)$.



%%%%%%%%%%%%%%%%%%%%%%%%%%%%%%%%%%%%%%%%%%%%%%%%%%%%%%%%%%
\diapo

[petit $m$, grand $M$]

Voici une conséquence très importante du théorème des valeurs intermédiaires.

Théorème : Soit $f$ une fonction continue sur un segment $[a,b]$. 
Alors il existe deux réels $m$ et $M$ tels que $f([a,b])=[m,M]$. 

\change

$m$ est le minimum de la fonction sur l'intervalle $[a,b]$ alors que $M$ est le maximum.

\change

On résume souvent ce théorème par la phrase suivante  : 
"l'image d'un segment par une fonction continue est un segment."


\change

Comme on sait déjà par le théorème des valeurs intermédiaires que 
$f([a,b])$ est un intervalle, le théorème précédent signifie exactement que 

"Si $f$ est continue sur le segment $[a,b]$ alors $f$ est bornée 
sur ce segment et elle atteint ses bornes."


Tous ces résultats sont très utiles et il faut donc les connaître !


%%%%%%%%%%%%%%%%%%%%%%%%%%%%%%%%%%%%%%%%%%%%%%%%%%%%%%%%%%%
\diapo


Il y a beaucoup de résultats théoriques fondamentaux dans cette leçon prenez le temps de les recopier, 
de les comprendre et de les apprendre. Et pour vraiment comprendre à fond, 
passez aussi du temps sur les démonstrations !



\end{document}

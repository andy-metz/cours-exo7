
%%%%%%%%%%%%%%%%%% PREAMBULE %%%%%%%%%%%%%%%%%%

\documentclass[aspectratio=169,utf8]{beamer}
%\documentclass[aspectratio=169,handout]{beamer}

\usetheme{Boadilla}
%\usecolortheme{seahorse}
%\usecolortheme[RGB={245,66,24}]{structure}
\useoutertheme{infolines}

% packages
\usepackage{amsfonts,amsmath,amssymb,amsthm}
\usepackage[utf8]{inputenc}
\usepackage[T1]{fontenc}
\usepackage{lmodern}

\usepackage[francais]{babel}
\usepackage{fancybox}
\usepackage{graphicx}

\usepackage{float}
\usepackage{xfrac}

%\usepackage[usenames, x11names]{xcolor}
\usepackage{pgfplots}
\usepackage{datetime}


% ----------------------------------------------------------------------
% Pour les images
\usepackage{tikz}
\usetikzlibrary{calc,shadows,arrows.meta,patterns,matrix}

\newcommand{\tikzinput}[1]{\input{figures/#1.tikz}}
% --- les figures avec échelle éventuel
\newcommand{\myfigure}[2]{% entrée : échelle, fichier(s) figure à inclure
\begin{center}\small%
\tikzstyle{every picture}=[scale=1.0*#1]% mise en échelle + 0% (automatiquement annulé à la fin du groupe)
#2%
\end{center}}



%-----  Package unités -----
\usepackage{siunitx}
\sisetup{locale = FR,detect-all,per-mode = symbol}

%\usepackage{mathptmx}
%\usepackage{fouriernc}
%\usepackage{newcent}
%\usepackage[mathcal,mathbf]{euler}

%\usepackage{palatino}
%\usepackage{newcent}
% \usepackage[mathcal,mathbf]{euler}



% \usepackage{hyperref}
% \hypersetup{colorlinks=true, linkcolor=blue, urlcolor=blue,
% pdftitle={Exo7 - Exercices de mathématiques}, pdfauthor={Exo7}}


%section
% \usepackage{sectsty}
% \allsectionsfont{\bf}
%\sectionfont{\color{Tomato3}\upshape\selectfont}
%\subsectionfont{\color{Tomato4}\upshape\selectfont}

%----- Ensembles : entiers, reels, complexes -----
\newcommand{\Nn}{\mathbb{N}} \newcommand{\N}{\mathbb{N}}
\newcommand{\Zz}{\mathbb{Z}} \newcommand{\Z}{\mathbb{Z}}
\newcommand{\Qq}{\mathbb{Q}} \newcommand{\Q}{\mathbb{Q}}
\newcommand{\Rr}{\mathbb{R}} \newcommand{\R}{\mathbb{R}}
\newcommand{\Cc}{\mathbb{C}} 
\newcommand{\Kk}{\mathbb{K}} \newcommand{\K}{\mathbb{K}}

%----- Modifications de symboles -----
\renewcommand{\epsilon}{\varepsilon}
\renewcommand{\Re}{\mathop{\text{Re}}\nolimits}
\renewcommand{\Im}{\mathop{\text{Im}}\nolimits}
%\newcommand{\llbracket}{\left[\kern-0.15em\left[}
%\newcommand{\rrbracket}{\right]\kern-0.15em\right]}

\renewcommand{\ge}{\geqslant}
\renewcommand{\geq}{\geqslant}
\renewcommand{\le}{\leqslant}
\renewcommand{\leq}{\leqslant}
\renewcommand{\epsilon}{\varepsilon}

%----- Fonctions usuelles -----
\newcommand{\ch}{\mathop{\text{ch}}\nolimits}
\newcommand{\sh}{\mathop{\text{sh}}\nolimits}
\renewcommand{\tanh}{\mathop{\text{th}}\nolimits}
\newcommand{\cotan}{\mathop{\text{cotan}}\nolimits}
\newcommand{\Arcsin}{\mathop{\text{arcsin}}\nolimits}
\newcommand{\Arccos}{\mathop{\text{arccos}}\nolimits}
\newcommand{\Arctan}{\mathop{\text{arctan}}\nolimits}
\newcommand{\Argsh}{\mathop{\text{argsh}}\nolimits}
\newcommand{\Argch}{\mathop{\text{argch}}\nolimits}
\newcommand{\Argth}{\mathop{\text{argth}}\nolimits}
\newcommand{\pgcd}{\mathop{\text{pgcd}}\nolimits} 


%----- Commandes divers ------
\newcommand{\ii}{\mathrm{i}}
\newcommand{\dd}{\text{d}}
\newcommand{\id}{\mathop{\text{id}}\nolimits}
\newcommand{\Ker}{\mathop{\text{Ker}}\nolimits}
\newcommand{\Card}{\mathop{\text{Card}}\nolimits}
\newcommand{\Vect}{\mathop{\text{Vect}}\nolimits}
\newcommand{\Mat}{\mathop{\text{Mat}}\nolimits}
\newcommand{\rg}{\mathop{\text{rg}}\nolimits}
\newcommand{\tr}{\mathop{\text{tr}}\nolimits}


%----- Structure des exercices ------

\newtheoremstyle{styleexo}% name
{2ex}% Space above
{3ex}% Space below
{}% Body font
{}% Indent amount 1
{\bfseries} % Theorem head font
{}% Punctuation after theorem head
{\newline}% Space after theorem head 2
{}% Theorem head spec (can be left empty, meaning ‘normal’)

%\theoremstyle{styleexo}
\newtheorem{exo}{Exercice}
\newtheorem{ind}{Indications}
\newtheorem{cor}{Correction}


\newcommand{\exercice}[1]{} \newcommand{\finexercice}{}
%\newcommand{\exercice}[1]{{\tiny\texttt{#1}}\vspace{-2ex}} % pour afficher le numero absolu, l'auteur...
\newcommand{\enonce}{\begin{exo}} \newcommand{\finenonce}{\end{exo}}
\newcommand{\indication}{\begin{ind}} \newcommand{\finindication}{\end{ind}}
\newcommand{\correction}{\begin{cor}} \newcommand{\fincorrection}{\end{cor}}

\newcommand{\noindication}{\stepcounter{ind}}
\newcommand{\nocorrection}{\stepcounter{cor}}

\newcommand{\fiche}[1]{} \newcommand{\finfiche}{}
\newcommand{\titre}[1]{\centerline{\large \bf #1}}
\newcommand{\addcommand}[1]{}
\newcommand{\video}[1]{}

% Marge
\newcommand{\mymargin}[1]{\marginpar{{\small #1}}}

\def\noqed{\renewcommand{\qedsymbol}{}}


%----- Presentation ------
\setlength{\parindent}{0cm}

%\newcommand{\ExoSept}{\href{http://exo7.emath.fr}{\textbf{\textsf{Exo7}}}}

\definecolor{myred}{rgb}{0.93,0.26,0}
\definecolor{myorange}{rgb}{0.97,0.58,0}
\definecolor{myyellow}{rgb}{1,0.86,0}

\newcommand{\LogoExoSept}[1]{  % input : echelle
{\usefont{U}{cmss}{bx}{n}
\begin{tikzpicture}[scale=0.1*#1,transform shape]
  \fill[color=myorange] (0,0)--(4,0)--(4,-4)--(0,-4)--cycle;
  \fill[color=myred] (0,0)--(0,3)--(-3,3)--(-3,0)--cycle;
  \fill[color=myyellow] (4,0)--(7,4)--(3,7)--(0,3)--cycle;
  \node[scale=5] at (3.5,3.5) {Exo7};
\end{tikzpicture}}
}


\newcommand{\debutmontitre}{
  \author{} \date{} 
  \thispagestyle{empty}
  \hspace*{-10ex}
  \begin{minipage}{\textwidth}
    \titlepage  
  \vspace*{-2.5cm}
  \begin{center}
    \LogoExoSept{2.5}
  \end{center}
  \end{minipage}

  \vspace*{-0cm}
  
  % Astuce pour que le background ne soit pas discrétisé lors de la conversion pdf -> png
\begin{tikzpicture}
        \fill[opacity=0,green!60!black] (0,0)--++(0,0)--++(0,0)--++(0,0)--cycle; 
\end{tikzpicture}

% toc S'affiche trop tot :
% \tableofcontents[hideallsubsections, pausesections]
}

\newcommand{\finmontitre}{
  \end{frame}
  \setcounter{framenumber}{0}
} % ne marche pas pour une raison obscure

%----- Commandes supplementaires ------

% \usepackage[landscape]{geometry}
% \geometry{top=1cm, bottom=3cm, left=2cm, right=10cm, marginparsep=1cm
% }
% \usepackage[a4paper]{geometry}
% \geometry{top=2cm, bottom=2cm, left=2cm, right=2cm, marginparsep=1cm
% }

%\usepackage{standalone}


% New command Arnaud -- november 2011
\setbeamersize{text margin left=24ex}
% si vous modifier cette valeur il faut aussi
% modifier le decalage du titre pour compenser
% (ex : ici =+10ex, titre =-5ex

\theoremstyle{definition}
%\newtheorem{proposition}{Proposition}
%\newtheorem{exemple}{Exemple}
%\newtheorem{theoreme}{Théorème}
%\newtheorem{lemme}{Lemme}
%\newtheorem{corollaire}{Corollaire}
%\newtheorem*{remarque*}{Remarque}
%\newtheorem*{miniexercice}{Mini-exercices}
%\newtheorem{definition}{Définition}

% Commande tikz
\usetikzlibrary{calc}
\usetikzlibrary{patterns,arrows}
\usetikzlibrary{matrix}
\usetikzlibrary{fadings} 

%definition d'un terme
\newcommand{\defi}[1]{{\color{myorange}\textbf{\emph{#1}}}}
\newcommand{\evidence}[1]{{\color{blue}\textbf{\emph{#1}}}}
\newcommand{\assertion}[1]{\emph{\og#1\fg}}  % pour chapitre logique
%\renewcommand{\contentsname}{Sommaire}
\renewcommand{\contentsname}{}
\setcounter{tocdepth}{2}



%------ Encadrement ------

\usepackage{fancybox}


\newcommand{\mybox}[1]{
\setlength{\fboxsep}{7pt}
\begin{center}
\shadowbox{#1}
\end{center}}

\newcommand{\myboxinline}[1]{
\setlength{\fboxsep}{5pt}
\raisebox{-10pt}{
\shadowbox{#1}
}
}

%--------------- Commande beamer---------------
\newcommand{\beameronly}[1]{#1} % permet de mettre des pause dans beamer pas dans poly


\setbeamertemplate{navigation symbols}{}
\setbeamertemplate{footline}  % tiré du fichier beamerouterinfolines.sty
{
  \leavevmode%
  \hbox{%
  \begin{beamercolorbox}[wd=.333333\paperwidth,ht=2.25ex,dp=1ex,center]{author in head/foot}%
    % \usebeamerfont{author in head/foot}\insertshortauthor%~~(\insertshortinstitute)
    \usebeamerfont{section in head/foot}{\bf\insertshorttitle}
  \end{beamercolorbox}%
  \begin{beamercolorbox}[wd=.333333\paperwidth,ht=2.25ex,dp=1ex,center]{title in head/foot}%
    \usebeamerfont{section in head/foot}{\bf\insertsectionhead}
  \end{beamercolorbox}%
  \begin{beamercolorbox}[wd=.333333\paperwidth,ht=2.25ex,dp=1ex,right]{date in head/foot}%
    % \usebeamerfont{date in head/foot}\insertshortdate{}\hspace*{2em}
    \insertframenumber{} / \inserttotalframenumber\hspace*{2ex} 
  \end{beamercolorbox}}%
  \vskip0pt%
}


\definecolor{mygrey}{rgb}{0.5,0.5,0.5}
\setlength{\parindent}{0cm}
%\DeclareTextFontCommand{\helvetica}{\fontfamily{phv}\selectfont}

% background beamer
\definecolor{couleurhaut}{rgb}{0.85,0.9,1}  % creme
\definecolor{couleurmilieu}{rgb}{1,1,1}  % vert pale
\definecolor{couleurbas}{rgb}{0.85,0.9,1}  % blanc
\setbeamertemplate{background canvas}[vertical shading]%
[top=couleurhaut,middle=couleurmilieu,midpoint=0.4,bottom=couleurbas] 
%[top=fondtitre!05,bottom=fondtitre!60]



\makeatletter
\setbeamertemplate{theorem begin}
{%
  \begin{\inserttheoremblockenv}
  {%
    \inserttheoremheadfont
    \inserttheoremname
    \inserttheoremnumber
    \ifx\inserttheoremaddition\@empty\else\ (\inserttheoremaddition)\fi%
    \inserttheorempunctuation
  }%
}
\setbeamertemplate{theorem end}{\end{\inserttheoremblockenv}}

\newenvironment{theoreme}[1][]{%
   \setbeamercolor{block title}{fg=structure,bg=structure!40}
   \setbeamercolor{block body}{fg=black,bg=structure!10}
   \begin{block}{{\bf Th\'eor\`eme }#1}
}{%
   \end{block}%
}


\newenvironment{proposition}[1][]{%
   \setbeamercolor{block title}{fg=structure,bg=structure!40}
   \setbeamercolor{block body}{fg=black,bg=structure!10}
   \begin{block}{{\bf Proposition }#1}
}{%
   \end{block}%
}

\newenvironment{corollaire}[1][]{%
   \setbeamercolor{block title}{fg=structure,bg=structure!40}
   \setbeamercolor{block body}{fg=black,bg=structure!10}
   \begin{block}{{\bf Corollaire }#1}
}{%
   \end{block}%
}

\newenvironment{mydefinition}[1][]{%
   \setbeamercolor{block title}{fg=structure,bg=structure!40}
   \setbeamercolor{block body}{fg=black,bg=structure!10}
   \begin{block}{{\bf Définition} #1}
}{%
   \end{block}%
}

\newenvironment{lemme}[0]{%
   \setbeamercolor{block title}{fg=structure,bg=structure!40}
   \setbeamercolor{block body}{fg=black,bg=structure!10}
   \begin{block}{\bf Lemme}
}{%
   \end{block}%
}

\newenvironment{remarque}[1][]{%
   \setbeamercolor{block title}{fg=black,bg=structure!20}
   \setbeamercolor{block body}{fg=black,bg=structure!5}
   \begin{block}{Remarque #1}
}{%
   \end{block}%
}


\newenvironment{exemple}[1][]{%
   \setbeamercolor{block title}{fg=black,bg=structure!20}
   \setbeamercolor{block body}{fg=black,bg=structure!5}
   \begin{block}{{\bf Exemple }#1}
}{%
   \end{block}%
}


\newenvironment{miniexercice}[0]{%
   \setbeamercolor{block title}{fg=structure,bg=structure!20}
   \setbeamercolor{block body}{fg=black,bg=structure!5}
   \begin{block}{Mini-exercices}
}{%
   \end{block}%
}


\newenvironment{tp}[0]{%
   \setbeamercolor{block title}{fg=structure,bg=structure!40}
   \setbeamercolor{block body}{fg=black,bg=structure!10}
   \begin{block}{\bf Travaux pratiques}
}{%
   \end{block}%
}
\newenvironment{exercicecours}[1][]{%
   \setbeamercolor{block title}{fg=structure,bg=structure!40}
   \setbeamercolor{block body}{fg=black,bg=structure!10}
   \begin{block}{{\bf Exercice }#1}
}{%
   \end{block}%
}
\newenvironment{algo}[1][]{%
   \setbeamercolor{block title}{fg=structure,bg=structure!40}
   \setbeamercolor{block body}{fg=black,bg=structure!10}
   \begin{block}{{\bf Algorithme}\hfill{\color{gray}\texttt{#1}}}
}{%
   \end{block}%
}


\setbeamertemplate{proof begin}{
   \setbeamercolor{block title}{fg=black,bg=structure!20}
   \setbeamercolor{block body}{fg=black,bg=structure!5}
   \begin{block}{{\footnotesize Démonstration}}
   \footnotesize
   \smallskip}
\setbeamertemplate{proof end}{%
   \end{block}}
\setbeamertemplate{qed symbol}{\openbox}


\makeatother
\usecolortheme[RGB={66,15,15}]{structure}

%%%%%%%%%%%%%%%%%%%%%%%%%%%%%%%%%%%%%%%%%%%%%%%%%%%%%%%%%%%%%
%%%%%%%%%%%%%%%%%%%%%%%%%%%%%%%%%%%%%%%%%%%%%%%%%%%%%%%%%%%%%


\begin{document}


\title{{\bf Limites et fonctions continues}}
\subtitle{Notions de fonction}

\begin{frame}
  
  \debutmontitre

  \pause

{\footnotesize
\hfill
\setbeamercovered{transparent=50}
\begin{minipage}{0.6\textwidth}
  \begin{itemize}
    \item<3-> Définitions
    \item<4-> Opérations sur les fonctions
    \item<5-> Fonctions croissantes, décroissantes
    \item<6-> Parité et périodicité
  \end{itemize}
\end{minipage}
}

\end{frame}

\setcounter{framenumber}{0}


%%%%%%%%%%%%%%%%%%%%%%%%%%%%%%%%%%%%%%%%%%%%%%%%%%%%%%%%%%%%%%%%


\section{Motivation}


\begin{frame}

\begin{itemize}
  \item \'Equation $x+\exp x=0$
\pause
  \item Fonction $f(x)=x + \exp x$
\pause
  \item \'Equation $x+\exp x=y$ 
\pause  
  \begin{itemize}
    \item pour chaque $y \in \Rr$ l'équation admet une solution $x$
\pause
    \item cette solution est unique
\pause    
    \item nous saurons dire comment varie $x$ en fonction de $y$
  \end{itemize}  
\end{itemize}


\onslide<2->\myfigure{0.8}{
\tikzinput{fig_fonctionsA01}
}

\end{frame}


%---------------------------------------------------------------
\section{Définitions}

\begin{frame}

\begin{mydefinition}
Une \defi{fonction} d'une variable réelle à valeurs réelles est une application 
$f:U\to \Rr$

\pause
$U\subset\Rr$ est le \defi{domaine de définition}
\end{mydefinition}

\pause 
Le \defi{graphe} de $f$  \[
\Gamma_f=\big\{(x,f(x)) \ \vert \ x\in U\big\}\subset\Rr^2
\] 

\pause

\myfigure{1}{
\tikzinput{fig_fonctionsA03}
} 
\end{frame}



\begin{frame}

\begin{exemple}
\[
\begin{array}{ccc}
f: \ ]-\infty,0[ \,\cup \, ]0,+\infty[ &\longrightarrow& \Rr \\
 x &\longmapsto& \dfrac1x
 \end{array}
\]
\end{exemple}

\pause

\myfigure{1}{
\tikzinput{fig_fonctionsA02}
}
  
\end{frame}
%---------------------------------------------------------------
\section{Opérations sur les fonctions}

\begin{frame}
Soient $f,g:U\to \Rr$ deux fonctions
\begin{mydefinition}

\begin{itemize}
  \item<1-> La \defi{somme} de $f$ et $g$ est la fonction $f+g:U\to \Rr$ 
  définie par $(f+g)(x) = f(x) + g(x)$ pour tout $x\in U$ 
  \item<3-> Le \defi{produit} de $f$ et $g$ est $f\times g:U\to \Rr$ défini par 
  $(f\times g)(x) = f(x) \times g(x)$
  \item<4-> La \defi{multiplication par un scalaire} $\lambda\in\Rr$ 
  de $f$ est $\lambda\cdot f:U\to \Rr$
\end{itemize}
\end{mydefinition}

\onslide<2->\myfigure{0.9}{
\tikzinput{fig_fonctionsA04}
}
  
\end{frame}

%---------------------------------------------------------------
\section{Fonctions majorées, minorées, bornées}

\begin{frame}

\begin{mydefinition}
Une fonction $f:U\to \Rr$ est dite
\begin{itemize}
  \item \defi{majorée} sur $U$ si \quad $\exists M\in\Rr \quad \forall x\in U \quad \ f(x)\leq M$ 
\pause
  \item \defi{minorée} sur $U$ si \quad $\exists m\in\Rr \quad \forall x\in U \quad \ f(x)\geq m$ 
\pause
  \item \defi{bornée} sur $U$ si \quad $\exists M\in\Rr \quad \forall x\in U \quad \ |f(x)|\leq M$
\end{itemize}
\end{mydefinition}
 
 \pause
 
\myfigure{0.9}{
\tikzinput{fig_fonctions1}
} 
  
\end{frame}



%---------------------------------------------------------------
\section{Fonctions croissantes, décroissantes}

\begin{frame}  
\begin{mydefinition}
Une fonction $f:U\to \Rr$ est dite
\medskip
\begin{itemize}
  \item \defi{croissante} si 
\myboxinline{$\forall x,y\in U \quad x\leq y \implies f(x)\leq f(y)$}
  
  \medskip
  
\pause  \pause 
  
  \item \defi{strictement croissante} si \ \ 
  $\forall x,y\in U \quad  x< y \implies f(x)< f(y)$
\end{itemize}
\end{mydefinition}

\onslide<2->\myfigure{0.8}{
\tikzinput{fig_fonctions2}
}
  
\end{frame}

\begin{frame}

\begin{mydefinition}
\begin{itemize}  
  \item $f$ est \defi{décroissante}  si \ \ 
  $\forall x,y\in U \quad  x\leq y \implies f(x)\geq f(y)$
  
  \item $f$ est \defi{strictement décroissante} si 
  
  \hfill
  $\forall x,y\in U \quad  x< y \implies f(x)> f(y)$
  
\pause
  
 \item $f$ est \defi{monotone} (resp. \defi{strictement monotone}) sur $U$ si 
 $f$ est croissante ou décroissante (resp. strictement croissante ou 
 strictement décroissante) sur $U$
\end{itemize}
\end{mydefinition}
  
\end{frame}

\begin{frame}

\begin{exemple}
\begin{itemize}
  \item $x\mapsto \sqrt x$ est strictement croissante sur $[0,+\infty[$
\pause
  \item $\exp: \Rr\to\Rr$ et $\ln :]0,+\infty[\to\Rr$ sont strictement croissantes
\pause  
  \item $\begin{cases} \Rr \longrightarrow \Rr \\ x\longmapsto |x| \end{cases}$ 
  n'est pas monotone
\pause  
  \item $\begin{cases} [0,+\infty[ \longrightarrow \Rr \\ x\longmapsto |x| \end{cases}$ 
  est strictement croissante
\end{itemize}
\end{exemple}

\end{frame}



%---------------------------------------------------------------
\section{Parité et périodicité}

\begin{frame}

\begin{mydefinition}
Soit $I=]-a,a[$ ou $[-a,a]$ ou $\Rr$. On dit que $f:I\to \Rr$ est
\begin{itemize}
  \item \defi{paire} si \ \ $\forall x\in I \quad f(-x)=f(x)$
\pause
  \item \defi{impaire} si \ \ $\forall x\in I \quad f(-x)=-f(x)$
\end{itemize}
\end{mydefinition}

\pause 

\begin{itemize}
  \item le graphe d'une fonction paire est symétrique par rapport à l'axe des ordonnées
  \item le graphe d'une fonction impaire est symétrique par rapport à l'origine
\end{itemize}

\myfigure{0.8}{
\tikzinput{fig_fonctions3}
}  
\end{frame}

\begin{frame}

\begin{exemple}
\begin{itemize}
  \item $x\mapsto x^{2n}$ est paire sur $\Rr$ ($n\in\Nn$)
\pause  
  \item $x\mapsto x^{2n+1}$ est impaire sur $\Rr$ ($n\in\Nn$)
\pause  \pause
  \item $\cos :\Rr\to\Rr$ est paire, \ $\sin :\Rr\to\Rr$ est impaire
\end{itemize} 
\end{exemple}

\onslide<3->
\myfigure{0.8}{
\tikzinput{fig_fonctionsA05}
}

\end{frame}

\begin{frame}

\begin{mydefinition}
Soit $T>0$. Une fonction $f:\Rr \to \Rr$ est \defi{périodique} de période $T$ si 
\mybox{$\forall x\in\Rr \quad \ f(x+T)=f(x)$}
\end{mydefinition}

\pause
Le graphe est invariant par la translation de vecteur $T \vec{i}$

\myfigure{1}{
\tikzinput{fig_fonctionsA06}
}

\end{frame}

\begin{frame}



\begin{exemple}
\begin{itemize}
  \item $\cos :\Rr\to\Rr$ et $\sin :\Rr\to\Rr$ sont $2\pi$-périodiques
  \item tangente est $\pi$-périodique
\end{itemize}
\end{exemple}

\bigskip


\myfigure{0.57}{
\tikzinput{fig_fonctionsA07}
}

\end{frame}


%%%%%%%%%%%%%%%%%%%%%%%%%%%%%%%%%%%%%%%%%%%%%%%%%%%%%%%%%%%%%%%%
\section{Mini-exercices}

\begin{frame}

\begin{miniexercice}
\begin{enumerate}
  \item Soit $U=]-\infty,0[$ et $f : U \to \Rr$ définie par $f(x)= 1/x$. $f$ est-elle  monotone ?
  Et sur $U=]0,+\infty[$ ? Et sur $U = ]-\infty,0[\,\cup \, ]0,+\infty[$ ?
  
  \item Pour deux fonctions paires que peut-on dire sur la parité de la somme ? du produit ? et de la 
  composée ? Et pour deux fonctions impaires ? Et si l'une est paire et l'autre impaire ?
  
  \item On note $\{x\}=x-E(x)$ la partie fractionnaire de $x$. 
  Tracer le graphe de la fonction $x\mapsto\{x\}$ et montrer qu'elle est périodique.
  
  \item Soit $f:\Rr\to\Rr$ la fonction définie par $f(x)=\frac{x}{1+x^2}$. 
  Montrer que $|f|$ est majorée par $\frac12$, étudier les variations de $f$ 
  (sans utiliser de dérivée) et tracer son graphe.
  
  \item On considère la fonction $g:\Rr\to\Rr$, $g(x)=\sin\big(\pi f(x)\big)$, 
  où $f$ est définie à la question précédente. Déduire de l'étude de $f$ les variations, 
  la parité, la périodicité de $g$ et tracer son graphe.
\end{enumerate}
\end{miniexercice}

\end{frame}

\end{document}
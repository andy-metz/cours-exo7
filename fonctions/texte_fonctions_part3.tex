
%%%%%%%%%%%%%%%%%% PREAMBULE %%%%%%%%%%%%%%%%%%


\documentclass[12pt]{article}

\usepackage{amsfonts,amsmath,amssymb,amsthm}
\usepackage[utf8]{inputenc}
\usepackage[T1]{fontenc}
\usepackage[francais]{babel}


% packages
\usepackage{amsfonts,amsmath,amssymb,amsthm}
\usepackage[utf8]{inputenc}
\usepackage[T1]{fontenc}
%\usepackage{lmodern}

\usepackage[francais]{babel}
\usepackage{fancybox}
\usepackage{graphicx}

\usepackage{float}

%\usepackage[usenames, x11names]{xcolor}
\usepackage{tikz}
\usepackage{datetime}

\usepackage{mathptmx}
%\usepackage{fouriernc}
%\usepackage{newcent}
\usepackage[mathcal,mathbf]{euler}

%\usepackage{palatino}
%\usepackage{newcent}


% Commande spéciale prompteur

%\usepackage{mathptmx}
%\usepackage[mathcal,mathbf]{euler}
%\usepackage{mathpple,multido}

\usepackage[a4paper]{geometry}
\geometry{top=2cm, bottom=2cm, left=1cm, right=1cm, marginparsep=1cm}

\newcommand{\change}{{\color{red}\rule{\textwidth}{1mm}\\}}

\newcounter{mydiapo}

\newcommand{\diapo}{\newpage
\hfill {\normalsize  Diapo \themydiapo \quad \texttt{[\jobname]}} \\
\stepcounter{mydiapo}}


%%%%%%% COULEURS %%%%%%%%%%

% Pour blanc sur noir :
%\pagecolor[rgb]{0.5,0.5,0.5}
% \pagecolor[rgb]{0,0,0}
% \color[rgb]{1,1,1}



%\DeclareFixedFont{\myfont}{U}{cmss}{bx}{n}{18pt}
\newcommand{\debuttexte}{
%%%%%%%%%%%%% FONTES %%%%%%%%%%%%%
\renewcommand{\baselinestretch}{1.5}
\usefont{U}{cmss}{bx}{n}
\bfseries

% Taille normale : commenter le reste !
%Taille Arnaud
%\fontsize{19}{19}\selectfont

% Taille Barbara
%\fontsize{21}{22}\selectfont

%Taille François
\fontsize{25}{30}\selectfont

%Taille Pascal
%\fontsize{25}{30}\selectfont

%Taille Laura
%\fontsize{30}{35}\selectfont


%\myfont
%\usefont{U}{cmss}{bx}{n}

%\Huge
%\addtolength{\parskip}{\baselineskip}
}


% \usepackage{hyperref}
% \hypersetup{colorlinks=true, linkcolor=blue, urlcolor=blue,
% pdftitle={Exo7 - Exercices de mathématiques}, pdfauthor={Exo7}}


%section
% \usepackage{sectsty}
% \allsectionsfont{\bf}
%\sectionfont{\color{Tomato3}\upshape\selectfont}
%\subsectionfont{\color{Tomato4}\upshape\selectfont}

%----- Ensembles : entiers, reels, complexes -----
\newcommand{\Nn}{\mathbb{N}} \newcommand{\N}{\mathbb{N}}
\newcommand{\Zz}{\mathbb{Z}} \newcommand{\Z}{\mathbb{Z}}
\newcommand{\Qq}{\mathbb{Q}} \newcommand{\Q}{\mathbb{Q}}
\newcommand{\Rr}{\mathbb{R}} \newcommand{\R}{\mathbb{R}}
\newcommand{\Cc}{\mathbb{C}} 
\newcommand{\Kk}{\mathbb{K}} \newcommand{\K}{\mathbb{K}}

%----- Modifications de symboles -----
\renewcommand{\epsilon}{\varepsilon}
\renewcommand{\Re}{\mathop{\text{Re}}\nolimits}
\renewcommand{\Im}{\mathop{\text{Im}}\nolimits}
%\newcommand{\llbracket}{\left[\kern-0.15em\left[}
%\newcommand{\rrbracket}{\right]\kern-0.15em\right]}

\renewcommand{\ge}{\geqslant}
\renewcommand{\geq}{\geqslant}
\renewcommand{\le}{\leqslant}
\renewcommand{\leq}{\leqslant}

%----- Fonctions usuelles -----
\newcommand{\ch}{\mathop{\mathrm{ch}}\nolimits}
\newcommand{\sh}{\mathop{\mathrm{sh}}\nolimits}
\renewcommand{\tanh}{\mathop{\mathrm{th}}\nolimits}
\newcommand{\cotan}{\mathop{\mathrm{cotan}}\nolimits}
\newcommand{\Arcsin}{\mathop{\mathrm{Arcsin}}\nolimits}
\newcommand{\Arccos}{\mathop{\mathrm{Arccos}}\nolimits}
\newcommand{\Arctan}{\mathop{\mathrm{Arctan}}\nolimits}
\newcommand{\Argsh}{\mathop{\mathrm{Argsh}}\nolimits}
\newcommand{\Argch}{\mathop{\mathrm{Argch}}\nolimits}
\newcommand{\Argth}{\mathop{\mathrm{Argth}}\nolimits}
\newcommand{\pgcd}{\mathop{\mathrm{pgcd}}\nolimits} 

\newcommand{\Card}{\mathop{\text{Card}}\nolimits}
\newcommand{\Ker}{\mathop{\text{Ker}}\nolimits}
\newcommand{\id}{\mathop{\text{id}}\nolimits}
\newcommand{\ii}{\mathrm{i}}
\newcommand{\dd}{\mathrm{d}}
\newcommand{\Vect}{\mathop{\text{Vect}}\nolimits}
\newcommand{\Mat}{\mathop{\mathrm{Mat}}\nolimits}
\newcommand{\rg}{\mathop{\text{rg}}\nolimits}
\newcommand{\tr}{\mathop{\text{tr}}\nolimits}
\newcommand{\ppcm}{\mathop{\text{ppcm}}\nolimits}

%----- Structure des exercices ------

\newtheoremstyle{styleexo}% name
{2ex}% Space above
{3ex}% Space below
{}% Body font
{}% Indent amount 1
{\bfseries} % Theorem head font
{}% Punctuation after theorem head
{\newline}% Space after theorem head 2
{}% Theorem head spec (can be left empty, meaning ‘normal’)

%\theoremstyle{styleexo}
\newtheorem{exo}{Exercice}
\newtheorem{ind}{Indications}
\newtheorem{cor}{Correction}


\newcommand{\exercice}[1]{} \newcommand{\finexercice}{}
%\newcommand{\exercice}[1]{{\tiny\texttt{#1}}\vspace{-2ex}} % pour afficher le numero absolu, l'auteur...
\newcommand{\enonce}{\begin{exo}} \newcommand{\finenonce}{\end{exo}}
\newcommand{\indication}{\begin{ind}} \newcommand{\finindication}{\end{ind}}
\newcommand{\correction}{\begin{cor}} \newcommand{\fincorrection}{\end{cor}}

\newcommand{\noindication}{\stepcounter{ind}}
\newcommand{\nocorrection}{\stepcounter{cor}}

\newcommand{\fiche}[1]{} \newcommand{\finfiche}{}
\newcommand{\titre}[1]{\centerline{\large \bf #1}}
\newcommand{\addcommand}[1]{}
\newcommand{\video}[1]{}

% Marge
\newcommand{\mymargin}[1]{\marginpar{{\small #1}}}



%----- Presentation ------
\setlength{\parindent}{0cm}

%\newcommand{\ExoSept}{\href{http://exo7.emath.fr}{\textbf{\textsf{Exo7}}}}

\definecolor{myred}{rgb}{0.93,0.26,0}
\definecolor{myorange}{rgb}{0.97,0.58,0}
\definecolor{myyellow}{rgb}{1,0.86,0}

\newcommand{\LogoExoSept}[1]{  % input : echelle
{\usefont{U}{cmss}{bx}{n}
\begin{tikzpicture}[scale=0.1*#1,transform shape]
  \fill[color=myorange] (0,0)--(4,0)--(4,-4)--(0,-4)--cycle;
  \fill[color=myred] (0,0)--(0,3)--(-3,3)--(-3,0)--cycle;
  \fill[color=myyellow] (4,0)--(7,4)--(3,7)--(0,3)--cycle;
  \node[scale=5] at (3.5,3.5) {Exo7};
\end{tikzpicture}}
}



\theoremstyle{definition}
%\newtheorem{proposition}{Proposition}
%\newtheorem{exemple}{Exemple}
%\newtheorem{theoreme}{Théorème}
\newtheorem{lemme}{Lemme}
\newtheorem{corollaire}{Corollaire}
%\newtheorem*{remarque*}{Remarque}
%\newtheorem*{miniexercice}{Mini-exercices}
%\newtheorem{definition}{Définition}




%definition d'un terme
\newcommand{\defi}[1]{{\color{myorange}\textbf{\emph{#1}}}}
\newcommand{\evidence}[1]{{\color{blue}\textbf{\emph{#1}}}}



 %----- Commandes divers ------

\newcommand{\codeinline}[1]{\texttt{#1}}

%%%%%%%%%%%%%%%%%%%%%%%%%%%%%%%%%%%%%%%%%%%%%%%%%%%%%%%%%%%%%
%%%%%%%%%%%%%%%%%%%%%%%%%%%%%%%%%%%%%%%%%%%%%%%%%%%%%%%%%%%%%



\begin{document}

\debuttexte



%%%%%%%%%%%%%%%%%%%%%%%%%%%%%%%%%%%%%%%%%%%%%%%%%%%%%%%%%%
\diapo

\change

\change

Dans cette leçon nous allons définir ce qu'est la continuité d'une fonction en 
un point.

\change

Nous verrons ensuite que la somme, le produit, etc. de deux fonctions continues 
restent continus.

\change

Puis nous verrons comment étendre le domaine de définition d'une fonction,
tout en conservant la continuité.

\change

Enfin nous terminons par une caractérisation de la continuité par les suites.


%%%%%%%%%%%%%%%%%%%%%%%%%%%%%%%%%%%%%%%%%%%%%%%%%%%%%%%%%%
\diapo

Je considère une fonction $f$ définie sur un intervalle $I$ de $\Rr$, et $x_0$ un point de $I$.


Voici la définition de $f$ \defi{continue au point $x_0$} :


$\forall \epsilon>0 \quad \exists \delta>0 \quad \forall x\in I \quad \vert x-x_0\vert <\delta 
\implies \vert f(x)-f(x_0)\vert <\epsilon$

C'est une définition délicate à maîtriser, prenons un peu de temps pour en discuter :

\change

en terme de limite cette définition signifie que 
$f$ admet une limite en $x_0$  et, comme $f$ est définit au point $x_0$, cette limite vaut nécessairement $f(x_0)$.
 
 
Autrement dit $f(x)$ tend vers $f(x_0)$ lorsque $x$ tend vers $x_0$
 
 
 
\change

Voici comment s'explique la définition de la continuité sur un dessin.


On regarde ce qui se passe autour de $x_0$ sur l'axe des $x$, et autour de $f(x_0)$
sur l'axe des $y$.

Pour chaque $\epsilon>0$, même très petit, que l'on me donne, 


je dois trouver un $\delta$ tel que si 
$x$ est situé à une distance de $x_0$ inférieure à $\delta$, alors
$f(x)$ est situé à une distance de $f(x_0)$ inférieure à $\epsilon$.

 
\change 
 
On dit que $f$ est continue sur l'intervalle $I$ si $f$ est continue 
en tous les $x_0$ de l'intervalle $I$.


%%%%%%%%%%%%%%%%%%%%%%%%%%%%%%%%%%%%%%%%%%%%%%%%%%%%%%%%%%
\diapo

Pour mieux comprendre la définition de la continuité,
voici des fonctions qui ne sont **pas** continues en $x_0$ :

Intuitivement, une fonction est continue sur un intervalle, 
si on peut tracer son graphe 
<< sans lever le crayon >>, c'est-à-dire si elle n'a pas de saut.




%%%%%%%%%%%%%%%%%%%%%%%%%%%%%%%%%%%%%%%%%%%%%%%%%%%%%%%%%%
\diapo


Voici une liste de fonctions qui sont continues :

\change

une fonction constante est continue sur n'importe quel intervalle,

\change

la fonction racine carrée  est continue en tout point positif ou nul.

\change

les fonctions $\sin$ et $\cos$ sont continue sur $\Rr$,

\change

la fonction valeur absolue aussi,

\change

 la fonction exponentielle est continue sur $\Rr$
alors que que la fonction logarithme est continue sur $]0,+\infty[$.

\change


Par contre, la fonction partie entière $E$ n'est pas continue aux points 
$x_0$ qui sont des entiers, puisqu'elle n'admet pas de limite en ces points. 

En un point $x_0$ réel mais pas entier, la fonction partie entière est continue.



%%%%%%%%%%%%%%%%%%%%%%%%%%%%%%%%%%%%%%%%%%%%%%%%%%%%%%%%%%
\diapo


La continuité est une condition assez naturelle, elle assure par exemple que 
si la fonction n'est pas nulle en un point 
alors elle n'est pas nulle autour de ce point. 

\change 

[[montrer dessin]]
On passe donc d'une propriété en un seul point à une propriété 
tout autour de ce point.

Voici l'énoncé :
Si $f$ est continue en $x_0$ et si $f(x_0)\neq 0$, alors il existe $\delta>0$

tel que, quelque soit $x$ compris entre $x_0-\delta$ et $x_0+\delta$,
$f(x)$ reste non nul.

\change

Voici la démonstration :

Supposons par exemple que $f(x_0)>0$

\change

\'Ecrivons la définition de la continuité de $f$ en $x_0$ :
\[
\forall \epsilon>0 \quad \exists \delta>0 \quad \forall x\in I \quad  x\in \, ]x_0-\delta,x_0+\delta [
\implies f(x_0)-\epsilon < f(x) <f(x_0)+\epsilon.
\]

\change

Cette assertion est vraie quel que soit $\epsilon$.

Donc en particulier, je fixe un $\epsilon$ qui vérifie  $\epsilon<f(x_0)$. 

\change

Alors cette inégalité $f(x)>f(x_0)-\epsilon$ [montrer]
implique maintenant que $f(x) > 0$ comme nous le souhaitions.

Ainsi pour tout $x$ de l'intervalle $]x_0-\delta,x_0+\delta [$,
$f(x)$ est non nul.

Si au départ $f(x_0)<0$, la preuve est similaire.


%%%%%%%%%%%%%%%%%%%%%%%%%%%%%%%%%%%%%%%%%%%%%%%%%%%%%%%%%%
\diapo


La continuité se comporte bien avec les opérations élémentaires. 

Les propositions suivantes sont des conséquences immédiates des propositions analogues sur les limites.

\change

Si $f$ est une fonction continue en $x_0$ alors 
pour tout $\lambda\in\R$, $\lambda\cdot f$ est continue en $x_0$ 

\change

Si  $f$ et $g$ sont continues $x_0$ alors
la somme $f+g$ est continue en $x_0$,
 
 \change
 
 c'est la même chose pour le produit :  $f\times g$ est continue en $x_0$

 
 \change
 
 Enfin si $f(x_0)\neq 0$, alors $\frac1f$ est définie et continue en $x_0$.
 
 
 [pause]
 
 Si maintenant on suppose que $f$ et $g$ sont continues en tout point d'un intervalle $I$.
 
 Alors  [montrer chaque item] $\lambda\cdot f$ ,  $f+g$,  $f\times g$ sont continue sur l'intervalle $I$

 et si $f$ ne s'annule en aucun point de l'intervalle $I$ alors $1/f$ est continue sur $I$.
 
 \change
 
 
Cette proposition permet de vérifier que d'autres fonctions usuelles sont continues :

\change

comme la fonction $x\mapsto x$ est continue alors 
la fonction puissance $x\mapsto x^n$ est continue sur $\Rr$ car c'est un produit $x \times x \times \cdots$
de fonctions continues.


\change

Quand on fait la somme de fonctions puissance, et des multiplications par des réels
on obtient que les fonctions polynomiales sont des fonctions continues.

\change

le quotient de deux polynômes fournit aussi une fonction continue en tout point où le dénominateur 
ne s'annule pas.



%%%%%%%%%%%%%%%%%%%%%%%%%%%%%%%%%%%%%%%%%%%%%%%%%%%%%%%%%%
\diapo


La composition conserve la continuité (mais il faut faire attention en quels
points les hypothèses s'appliquent).


Si $f$ est continue en un point $x_0$ et si $g$ est continue *en $f(x_0)$*, 
alors $g\circ f$ est continue en $x_0$.


%%%%%%%%%%%%%%%%%%%%%%%%%%%%%%%%%%%%%%%%%%%%%%%%%%%%%%%%%%
\diapo

Le prolongement par continuité, c'est le moyen d'étendre le domaine de définition
d'une fonction $f$ en un point où elle n'est pas définie tout en préservant la continuité.


Partons d'un intervalle $I$, $x_0$ un point de $I$ et $f:I\setminus\{x_0\}\to\Rr$ une fonction,
qui n'est pour l'instant pas définie en $x_0$.


\change

 On dit que $f$ est \defi{prolongeable par continuité} en $x_0$ si $f$ admet une 
  limite finie en $x_0$. 
  
\change
  
Dans ce cas on note $\ell$ la limite de $f$ en $x_0$.

\change


Et on peut alors étendre la fonction $f$ :
on définit la fonction $\tilde f$  en posant pour tout $x\in I$
  \[
  \tilde f(x) =
  \begin{cases}
  f(x) &\text{ si } x\neq x_0\\
  \ell &\text{ si } x=x_0.
  \end{cases}
  \]
  
  Cette nouvelle fonction $\tilde f$ est définie en $x_0$, et de plus on a tout fait pour que $\tilde f$ soit continue en $x_0$.
  
  On appelle $\tilde f$ le \defi{prolongement par continuité} de $f$ en $x_0$.
  
  [dessin]
  
  Sur le dessin nous avons une fonction qui n'est pas définie en $x_0$ mais qui admet
  des limites à gauche et à droite qui sont les mêmes.
  
  \change
  

  
  Alors il est naturel d'étendre la définition de $f$ en $x_0$ en posant
  pour valeur de $\tilde f$ en $x_0$ cette limite.
  
  La fonction obtenue est maintenant continue en $x_0$.

Dans la pratique, on continuera souvent à noter $f$ à la place de $\tilde f$.

%%%%%%%%%%%%%%%%%%%%%%%%%%%%%%%%%%%%%%%%%%%%%%%%%%%%%%%%%%
\diapo


Passons à un exemple de prolongement par continuité.


Considérons la fonction $f$ définie en dehors de $0$ par $f(x)=x\sin\left(\frac1x\right)$. 

Voyons si $f$ admet un prolongement par continuité en $0$. Ceci revient à se demander si $f$ admet une limite finie en $0$.

\change

Comme $|\sin 1/x| \le 1$ alors pour tout $x$ non nul 
on a $\vert f(x)\vert\leq \vert x\vert$, 

\change

on en déduit que $f$ tend vers $0$ en $0$. 

\change


Elle est donc prolongeable par continuité en $0$ et son prolongement est la fonction $\tilde f$ 
définie sur $\Rr$ tout entier par :


$\tilde f(x) = x\sin\left(\frac1x\right)$ si $ x\neq 0$

et $\tilde f(0)=0$.

La nouvelle fonction $\tilde f$ est définie partout,
elle est continue en dehors de $0$ comme l'était $f$,
mais elle est de plus continue en $0$ par notre choix de $\tilde f(0)=0$.

Ainsi $\tilde f$ est continue partout.



%%%%%%%%%%%%%%%%%%%%%%%%%%%%%%%%%%%%%%%%%%%%%%%%%%%%%%%%%%
\diapo

On termine par un résultat important qui lie les suites et la continuité des fonctions.

Partons d'une fonction $f$, et soit $x_0$ un point de son intervalle de définition $I$. 

\change


Alors nous avons l'équivalence entre les deux assertions suivantes  :

$f \text{ est continue en } x_0$


*et* 

pour toute suite $(u_n)$ qui tend vers $x_0$, la suite $(f(u_n))$ tend vers $f(x_0)$.


Je reprends :

si $f$ est continue en $x_0$ alors 
pour n'importe quelle suite qui tend $x_0$,
on a aussi $(f(u_n))$ qui tend vers $f(x_0)$.

Et réciproquement :

si pour  n'importe quelle suite qui tend $x_0$,
on a aussi $(f(u_n))$ qui tend vers $f(x_0)$,
cela implique que $f$ est continue en $x_0$.




C'est surtout cette implication [$\implies]$ qui sera utile

\change

On l'utilisera intensivement pour l'étude des suites récurrentes : soit une suite
$(u_n)$ définie par une formule de récurrences $u_{n+1}= f(u_n)$.

\change

Le résultat ci-dessus implique que 
si $f$ est continue et $u_n\to \ell$, 
alors $(f(u_n))$ tend vers $f(\ell)$. Mais comme $u_{n+1}$ tend aussi vers $\ell$, on en déduit ainsi que $\ell$ vérifie l'équation $f(\ell)=\ell$.

Ceci donne une contrainte importante sur la limite $\ell$ et permet assez souvent de la calculer.



%%%%%%%%%%%%%%%%%%%%%%%%%%%%%%%%%%%%%%%%%%%%%%%%%%%%%%%%%%
\diapo

On termine par la démonstration de l'équivalence dans la proposition précédente.
Démonstration que vous pouvez éluder en première lecture.

\change

On commence par la preuve de l'implication directe.

On suppose donc que $f$ est continue en $x_0$
et on considère une suite $(u_n)$ qui 
  converge vers $x_0$
  
 Il s'agit de  montrer que $(f(u_n))$ converge vers $f(x_0)$.

 \change
 

  Soit $\epsilon >0$ fixé. 
 
 \change
 
  Comme $f$ est continue en $x_0$, alors par définition, pour ce $\epsilon$,
  il existe un $\delta>0$ tel que
\[
\forall x \in I \quad  |x-x_0|<\delta \implies |f(x)-f(x_0)|<\epsilon.
\]

\change

Comme $(u_n)$ tend vers $x_0$, 
alors pour le $\delta$ que l'on vient d'obtenir, il existe grand $N$ tel que
pour tout $n\geq N$ on a $|u_n-x_0|<\delta.$

\change


On en déduit que, pour  $n\geq N$, comme $|u_n-x_0|<\delta$, 
alors par la continuité on a $|f(u_n)-f(x_0)|<\epsilon$.

\change

Bilan : on a montré que pour $n\ge N$, $|f(u_n)-f(x_0)|<\epsilon$,

c'est donc que l'on a prouvé que $(f(u_n))$ converge vers $f(x_0)$.


%%%%%%%%%%%%%%%%%%%%%%%%%%%%%%%%%%%%%%%%%%%%%%%%%%%%%%%%%%
\diapo

\change

Pour montrer l'implication réciproque [$\Leftarrow$]

on va en fait montrer la contraposée : 
  
  c'est-à-dire on suppose que $f$ n'est pas continue en $x_0$ et on va
  montrer qu'il existe   une suite $(u_n)$ qui converge vers $x_0$ mais telle que $(f(u_n))$ ne converge pas 
  vers $f(x_0)$.
  
 \change
 
 
  
  Par hypothèse, comme $f$ n'est pas continue en $x_0$ :
\[
\exists \epsilon_0>0 \quad \forall\delta>0 \quad \exists x_\delta \in I \quad 
\text{tel que} \quad |x_\delta-x_0|<\delta \text{ et } |f(x_\delta)-f(x_0)|>\epsilon_0.
\]

\change

On construit la suite $(u_n)$ de la façon suivante : pour tout $n\in \Nn^*$, 
on choisit dans l'assertion précédente $\delta=1/n$ 

\change

et on obtient qu'il existe $u_n$ (qui est $x_\delta$ avec $\delta = 1/n$) tel que 
\[
|u_n-x_0|<\frac1n \quad \text{et} \quad |f(u_n)-f(x_0)|>\epsilon_0.
\]

\change

La suite $(u_n)$ converge vers $x_0$ alors que la suite $(f(u_n))$ ne peut pas converger vers $f(x_0)$.

Nous avons montré l'implication réciproque,
ce qui termine la démonstration de l'équivalence.



%%%%%%%%%%%%%%%%%%%%%%%%%%%%%%%%%%%%%%%%%%%%%%%%%%%%%%%%%%%
\diapo

Voici toute une série de petits exercices pour vous entraîner !


\end{document}

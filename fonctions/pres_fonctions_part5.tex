
%%%%%%%%%%%%%%%%%% PREAMBULE %%%%%%%%%%%%%%%%%%

\documentclass[aspectratio=169,utf8]{beamer}
%\documentclass[aspectratio=169,handout]{beamer}

\usetheme{Boadilla}
%\usecolortheme{seahorse}
%\usecolortheme[RGB={245,66,24}]{structure}
\useoutertheme{infolines}

% packages
\usepackage{amsfonts,amsmath,amssymb,amsthm}
\usepackage[utf8]{inputenc}
\usepackage[T1]{fontenc}
\usepackage{lmodern}

\usepackage[francais]{babel}
\usepackage{fancybox}
\usepackage{graphicx}

\usepackage{float}
\usepackage{xfrac}

%\usepackage[usenames, x11names]{xcolor}
\usepackage{pgfplots}
\usepackage{datetime}


% ----------------------------------------------------------------------
% Pour les images
\usepackage{tikz}
\usetikzlibrary{calc,shadows,arrows.meta,patterns,matrix}

\newcommand{\tikzinput}[1]{\input{figures/#1.tikz}}
% --- les figures avec échelle éventuel
\newcommand{\myfigure}[2]{% entrée : échelle, fichier(s) figure à inclure
\begin{center}\small%
\tikzstyle{every picture}=[scale=1.0*#1]% mise en échelle + 0% (automatiquement annulé à la fin du groupe)
#2%
\end{center}}



%-----  Package unités -----
\usepackage{siunitx}
\sisetup{locale = FR,detect-all,per-mode = symbol}

%\usepackage{mathptmx}
%\usepackage{fouriernc}
%\usepackage{newcent}
%\usepackage[mathcal,mathbf]{euler}

%\usepackage{palatino}
%\usepackage{newcent}
% \usepackage[mathcal,mathbf]{euler}



% \usepackage{hyperref}
% \hypersetup{colorlinks=true, linkcolor=blue, urlcolor=blue,
% pdftitle={Exo7 - Exercices de mathématiques}, pdfauthor={Exo7}}


%section
% \usepackage{sectsty}
% \allsectionsfont{\bf}
%\sectionfont{\color{Tomato3}\upshape\selectfont}
%\subsectionfont{\color{Tomato4}\upshape\selectfont}

%----- Ensembles : entiers, reels, complexes -----
\newcommand{\Nn}{\mathbb{N}} \newcommand{\N}{\mathbb{N}}
\newcommand{\Zz}{\mathbb{Z}} \newcommand{\Z}{\mathbb{Z}}
\newcommand{\Qq}{\mathbb{Q}} \newcommand{\Q}{\mathbb{Q}}
\newcommand{\Rr}{\mathbb{R}} \newcommand{\R}{\mathbb{R}}
\newcommand{\Cc}{\mathbb{C}} 
\newcommand{\Kk}{\mathbb{K}} \newcommand{\K}{\mathbb{K}}

%----- Modifications de symboles -----
\renewcommand{\epsilon}{\varepsilon}
\renewcommand{\Re}{\mathop{\text{Re}}\nolimits}
\renewcommand{\Im}{\mathop{\text{Im}}\nolimits}
%\newcommand{\llbracket}{\left[\kern-0.15em\left[}
%\newcommand{\rrbracket}{\right]\kern-0.15em\right]}

\renewcommand{\ge}{\geqslant}
\renewcommand{\geq}{\geqslant}
\renewcommand{\le}{\leqslant}
\renewcommand{\leq}{\leqslant}
\renewcommand{\epsilon}{\varepsilon}

%----- Fonctions usuelles -----
\newcommand{\ch}{\mathop{\text{ch}}\nolimits}
\newcommand{\sh}{\mathop{\text{sh}}\nolimits}
\renewcommand{\tanh}{\mathop{\text{th}}\nolimits}
\newcommand{\cotan}{\mathop{\text{cotan}}\nolimits}
\newcommand{\Arcsin}{\mathop{\text{arcsin}}\nolimits}
\newcommand{\Arccos}{\mathop{\text{arccos}}\nolimits}
\newcommand{\Arctan}{\mathop{\text{arctan}}\nolimits}
\newcommand{\Argsh}{\mathop{\text{argsh}}\nolimits}
\newcommand{\Argch}{\mathop{\text{argch}}\nolimits}
\newcommand{\Argth}{\mathop{\text{argth}}\nolimits}
\newcommand{\pgcd}{\mathop{\text{pgcd}}\nolimits} 


%----- Commandes divers ------
\newcommand{\ii}{\mathrm{i}}
\newcommand{\dd}{\text{d}}
\newcommand{\id}{\mathop{\text{id}}\nolimits}
\newcommand{\Ker}{\mathop{\text{Ker}}\nolimits}
\newcommand{\Card}{\mathop{\text{Card}}\nolimits}
\newcommand{\Vect}{\mathop{\text{Vect}}\nolimits}
\newcommand{\Mat}{\mathop{\text{Mat}}\nolimits}
\newcommand{\rg}{\mathop{\text{rg}}\nolimits}
\newcommand{\tr}{\mathop{\text{tr}}\nolimits}


%----- Structure des exercices ------

\newtheoremstyle{styleexo}% name
{2ex}% Space above
{3ex}% Space below
{}% Body font
{}% Indent amount 1
{\bfseries} % Theorem head font
{}% Punctuation after theorem head
{\newline}% Space after theorem head 2
{}% Theorem head spec (can be left empty, meaning ‘normal’)

%\theoremstyle{styleexo}
\newtheorem{exo}{Exercice}
\newtheorem{ind}{Indications}
\newtheorem{cor}{Correction}


\newcommand{\exercice}[1]{} \newcommand{\finexercice}{}
%\newcommand{\exercice}[1]{{\tiny\texttt{#1}}\vspace{-2ex}} % pour afficher le numero absolu, l'auteur...
\newcommand{\enonce}{\begin{exo}} \newcommand{\finenonce}{\end{exo}}
\newcommand{\indication}{\begin{ind}} \newcommand{\finindication}{\end{ind}}
\newcommand{\correction}{\begin{cor}} \newcommand{\fincorrection}{\end{cor}}

\newcommand{\noindication}{\stepcounter{ind}}
\newcommand{\nocorrection}{\stepcounter{cor}}

\newcommand{\fiche}[1]{} \newcommand{\finfiche}{}
\newcommand{\titre}[1]{\centerline{\large \bf #1}}
\newcommand{\addcommand}[1]{}
\newcommand{\video}[1]{}

% Marge
\newcommand{\mymargin}[1]{\marginpar{{\small #1}}}

\def\noqed{\renewcommand{\qedsymbol}{}}


%----- Presentation ------
\setlength{\parindent}{0cm}

%\newcommand{\ExoSept}{\href{http://exo7.emath.fr}{\textbf{\textsf{Exo7}}}}

\definecolor{myred}{rgb}{0.93,0.26,0}
\definecolor{myorange}{rgb}{0.97,0.58,0}
\definecolor{myyellow}{rgb}{1,0.86,0}

\newcommand{\LogoExoSept}[1]{  % input : echelle
{\usefont{U}{cmss}{bx}{n}
\begin{tikzpicture}[scale=0.1*#1,transform shape]
  \fill[color=myorange] (0,0)--(4,0)--(4,-4)--(0,-4)--cycle;
  \fill[color=myred] (0,0)--(0,3)--(-3,3)--(-3,0)--cycle;
  \fill[color=myyellow] (4,0)--(7,4)--(3,7)--(0,3)--cycle;
  \node[scale=5] at (3.5,3.5) {Exo7};
\end{tikzpicture}}
}


\newcommand{\debutmontitre}{
  \author{} \date{} 
  \thispagestyle{empty}
  \hspace*{-10ex}
  \begin{minipage}{\textwidth}
    \titlepage  
  \vspace*{-2.5cm}
  \begin{center}
    \LogoExoSept{2.5}
  \end{center}
  \end{minipage}

  \vspace*{-0cm}
  
  % Astuce pour que le background ne soit pas discrétisé lors de la conversion pdf -> png
\begin{tikzpicture}
        \fill[opacity=0,green!60!black] (0,0)--++(0,0)--++(0,0)--++(0,0)--cycle; 
\end{tikzpicture}

% toc S'affiche trop tot :
% \tableofcontents[hideallsubsections, pausesections]
}

\newcommand{\finmontitre}{
  \end{frame}
  \setcounter{framenumber}{0}
} % ne marche pas pour une raison obscure

%----- Commandes supplementaires ------

% \usepackage[landscape]{geometry}
% \geometry{top=1cm, bottom=3cm, left=2cm, right=10cm, marginparsep=1cm
% }
% \usepackage[a4paper]{geometry}
% \geometry{top=2cm, bottom=2cm, left=2cm, right=2cm, marginparsep=1cm
% }

%\usepackage{standalone}


% New command Arnaud -- november 2011
\setbeamersize{text margin left=24ex}
% si vous modifier cette valeur il faut aussi
% modifier le decalage du titre pour compenser
% (ex : ici =+10ex, titre =-5ex

\theoremstyle{definition}
%\newtheorem{proposition}{Proposition}
%\newtheorem{exemple}{Exemple}
%\newtheorem{theoreme}{Théorème}
%\newtheorem{lemme}{Lemme}
%\newtheorem{corollaire}{Corollaire}
%\newtheorem*{remarque*}{Remarque}
%\newtheorem*{miniexercice}{Mini-exercices}
%\newtheorem{definition}{Définition}

% Commande tikz
\usetikzlibrary{calc}
\usetikzlibrary{patterns,arrows}
\usetikzlibrary{matrix}
\usetikzlibrary{fadings} 

%definition d'un terme
\newcommand{\defi}[1]{{\color{myorange}\textbf{\emph{#1}}}}
\newcommand{\evidence}[1]{{\color{blue}\textbf{\emph{#1}}}}
\newcommand{\assertion}[1]{\emph{\og#1\fg}}  % pour chapitre logique
%\renewcommand{\contentsname}{Sommaire}
\renewcommand{\contentsname}{}
\setcounter{tocdepth}{2}



%------ Encadrement ------

\usepackage{fancybox}


\newcommand{\mybox}[1]{
\setlength{\fboxsep}{7pt}
\begin{center}
\shadowbox{#1}
\end{center}}

\newcommand{\myboxinline}[1]{
\setlength{\fboxsep}{5pt}
\raisebox{-10pt}{
\shadowbox{#1}
}
}

%--------------- Commande beamer---------------
\newcommand{\beameronly}[1]{#1} % permet de mettre des pause dans beamer pas dans poly


\setbeamertemplate{navigation symbols}{}
\setbeamertemplate{footline}  % tiré du fichier beamerouterinfolines.sty
{
  \leavevmode%
  \hbox{%
  \begin{beamercolorbox}[wd=.333333\paperwidth,ht=2.25ex,dp=1ex,center]{author in head/foot}%
    % \usebeamerfont{author in head/foot}\insertshortauthor%~~(\insertshortinstitute)
    \usebeamerfont{section in head/foot}{\bf\insertshorttitle}
  \end{beamercolorbox}%
  \begin{beamercolorbox}[wd=.333333\paperwidth,ht=2.25ex,dp=1ex,center]{title in head/foot}%
    \usebeamerfont{section in head/foot}{\bf\insertsectionhead}
  \end{beamercolorbox}%
  \begin{beamercolorbox}[wd=.333333\paperwidth,ht=2.25ex,dp=1ex,right]{date in head/foot}%
    % \usebeamerfont{date in head/foot}\insertshortdate{}\hspace*{2em}
    \insertframenumber{} / \inserttotalframenumber\hspace*{2ex} 
  \end{beamercolorbox}}%
  \vskip0pt%
}


\definecolor{mygrey}{rgb}{0.5,0.5,0.5}
\setlength{\parindent}{0cm}
%\DeclareTextFontCommand{\helvetica}{\fontfamily{phv}\selectfont}

% background beamer
\definecolor{couleurhaut}{rgb}{0.85,0.9,1}  % creme
\definecolor{couleurmilieu}{rgb}{1,1,1}  % vert pale
\definecolor{couleurbas}{rgb}{0.85,0.9,1}  % blanc
\setbeamertemplate{background canvas}[vertical shading]%
[top=couleurhaut,middle=couleurmilieu,midpoint=0.4,bottom=couleurbas] 
%[top=fondtitre!05,bottom=fondtitre!60]



\makeatletter
\setbeamertemplate{theorem begin}
{%
  \begin{\inserttheoremblockenv}
  {%
    \inserttheoremheadfont
    \inserttheoremname
    \inserttheoremnumber
    \ifx\inserttheoremaddition\@empty\else\ (\inserttheoremaddition)\fi%
    \inserttheorempunctuation
  }%
}
\setbeamertemplate{theorem end}{\end{\inserttheoremblockenv}}

\newenvironment{theoreme}[1][]{%
   \setbeamercolor{block title}{fg=structure,bg=structure!40}
   \setbeamercolor{block body}{fg=black,bg=structure!10}
   \begin{block}{{\bf Th\'eor\`eme }#1}
}{%
   \end{block}%
}


\newenvironment{proposition}[1][]{%
   \setbeamercolor{block title}{fg=structure,bg=structure!40}
   \setbeamercolor{block body}{fg=black,bg=structure!10}
   \begin{block}{{\bf Proposition }#1}
}{%
   \end{block}%
}

\newenvironment{corollaire}[1][]{%
   \setbeamercolor{block title}{fg=structure,bg=structure!40}
   \setbeamercolor{block body}{fg=black,bg=structure!10}
   \begin{block}{{\bf Corollaire }#1}
}{%
   \end{block}%
}

\newenvironment{mydefinition}[1][]{%
   \setbeamercolor{block title}{fg=structure,bg=structure!40}
   \setbeamercolor{block body}{fg=black,bg=structure!10}
   \begin{block}{{\bf Définition} #1}
}{%
   \end{block}%
}

\newenvironment{lemme}[0]{%
   \setbeamercolor{block title}{fg=structure,bg=structure!40}
   \setbeamercolor{block body}{fg=black,bg=structure!10}
   \begin{block}{\bf Lemme}
}{%
   \end{block}%
}

\newenvironment{remarque}[1][]{%
   \setbeamercolor{block title}{fg=black,bg=structure!20}
   \setbeamercolor{block body}{fg=black,bg=structure!5}
   \begin{block}{Remarque #1}
}{%
   \end{block}%
}


\newenvironment{exemple}[1][]{%
   \setbeamercolor{block title}{fg=black,bg=structure!20}
   \setbeamercolor{block body}{fg=black,bg=structure!5}
   \begin{block}{{\bf Exemple }#1}
}{%
   \end{block}%
}


\newenvironment{miniexercice}[0]{%
   \setbeamercolor{block title}{fg=structure,bg=structure!20}
   \setbeamercolor{block body}{fg=black,bg=structure!5}
   \begin{block}{Mini-exercices}
}{%
   \end{block}%
}


\newenvironment{tp}[0]{%
   \setbeamercolor{block title}{fg=structure,bg=structure!40}
   \setbeamercolor{block body}{fg=black,bg=structure!10}
   \begin{block}{\bf Travaux pratiques}
}{%
   \end{block}%
}
\newenvironment{exercicecours}[1][]{%
   \setbeamercolor{block title}{fg=structure,bg=structure!40}
   \setbeamercolor{block body}{fg=black,bg=structure!10}
   \begin{block}{{\bf Exercice }#1}
}{%
   \end{block}%
}
\newenvironment{algo}[1][]{%
   \setbeamercolor{block title}{fg=structure,bg=structure!40}
   \setbeamercolor{block body}{fg=black,bg=structure!10}
   \begin{block}{{\bf Algorithme}\hfill{\color{gray}\texttt{#1}}}
}{%
   \end{block}%
}


\setbeamertemplate{proof begin}{
   \setbeamercolor{block title}{fg=black,bg=structure!20}
   \setbeamercolor{block body}{fg=black,bg=structure!5}
   \begin{block}{{\footnotesize Démonstration}}
   \footnotesize
   \smallskip}
\setbeamertemplate{proof end}{%
   \end{block}}
\setbeamertemplate{qed symbol}{\openbox}


\makeatother
\usecolortheme[RGB={66,15,15}]{structure}

% Commande spécifique à ce chapitre
\newcounter{saveenumi}

%%%%%%%%%%%%%%%%%%%%%%%%%%%%%%%%%%%%%%%%%%%%%%%%%%%%%%%%%%%%%
%%%%%%%%%%%%%%%%%%%%%%%%%%%%%%%%%%%%%%%%%%%%%%%%%%%%%%%%%%%%%


\begin{document}


\title{{\bf Limites et fonctions continues}}
\subtitle{Fonctions monotones et bijections}

\begin{frame}
  
  \debutmontitre

  \pause

{\footnotesize
\hfill
\setbeamercovered{transparent=50}
\begin{minipage}{0.6\textwidth}
  \begin{itemize}
    \item<3-> Injection, surjection, bijection
    \item<4-> Théorème de la bijection
  \end{itemize}
\end{minipage}
}

\end{frame}

\setcounter{framenumber}{0}


%%%%%%%%%%%%%%%%%%%%%%%%%%%%%%%%%%%%%%%%%%%%%%%%%%%%%%%%%%%%%%%%


\section{Rappels : injection, surjection, bijection}

\begin{frame}

\begin{mydefinition}
Soit $f:E\to F$ une fonction, $E, F\subset \Rr$
\begin{itemize}
  \item $f$ est \defi{injective} si \quad $\forall x,x'\in E \ \ f(x)=f(x') \implies x=x'$ 
\uncover<2->{  \item $f$ est \defi{surjective} si \quad $\forall y\in F \ \ \exists x\in E \ \ y=f(x)$ }
\uncover<3->{   \item $f$ est \defi{bijective} si $f$ est injective et surjective, 
  c'est-à-dire \\
  \hfil\hfil $\forall y\in F \ \ \exists! x\in E \ \ y=f(x)$ }
\end{itemize}
\end{mydefinition}


\myfigure{.55}{\hspace*{-7em}
\tikzinput{fig_fonctions10a}\ \ 
\uncover<2->{\tikzinput{fig_fonctions10b} }
\ 
\uncover<3->{\tikzinput{fig_fonctions10c} }
}

\end{frame}



\begin{frame}


\begin{proposition}
Si $f :  E \to F$ est bijective alors il existe une 
unique application $g : F \to E$ telle que $g\circ f = \id_E$ et $f\circ g = \id_F$
\end{proposition}

La fonction $g$ est la \defi{bijection réciproque} de $f$ et se note $f^{-1}$

\bigskip
\pause

Remarques :
% \ \\[-1em]
\begin{itemize}
  \item $\id_E : E \to E$ est définie par $x \mapsto x$
\pause  
  \item $g \circ f = \id_E$ se reformule ainsi : $\forall x \in E\quad  g\big(f(x)\big) = x$
\pause  
  \item $f \circ g = \id_F$  s'écrit : $\forall y \in F\quad  f\big(g(y)\big) = y$
\pause  
  \item Les graphes de $f$ et $f^{-1}$ sont symétriques 
\end{itemize}

\vspace*{-4ex}\hfill
\begin{minipage}{0.39\textwidth}
\myfigure{.55}{\tikzinput{fig_fonctions10c} }  
\end{minipage}






  
\end{frame}


%---------------------------------------------------------------
\section{Fonctions monotones et bijections}

\begin{frame}

\begin{theoreme}[de la bijection]
Soit $f$ une fonction définie sur un intervalle $I$ de $\Rr$. Si $f$ est continue 
et strictement monotone sur $I$, alors

\begin{enumerate}
\uncover<2->{\item $f$ établit une bijection de $I$ dans l'intervalle image $J=f(I)$}

\uncover<3->{\item la fonction réciproque $f^{-1}:J\to I$ est continue et strictement monotone 
sur $J$ et a le même sens de variation que $f$}
\end{enumerate}
\end{theoreme}



\myfigure{.8}{
\tikzinput{fig_fonctionsA19}
}
  
\end{frame}



\begin{frame}

\begin{exemple}[$f(x)=x^2$]
\begin{itemize}
  \item Sur $\Rr$, $f$ n'est pas bijective


  \item \uncover<2->{$
f_1 :
\left\{\begin{array}{c}
]-\infty,0] \longrightarrow [0,+\infty[ \\
x \longmapsto x^2
\end{array}\right.
\quad \text{et } \quad
f_2 : 
\left\{\begin{array}{c}
[0,+\infty[ \longrightarrow [0,+\infty[ \\
x \longmapsto x^2
\end{array}\right.
$
}
\vspace*{-1.5ex}
\item  \uncover<4->{ $f_1$ et $f_2$ sont strictement monotones : elles sont donc bijectives}
  
 \uncover<5->{   \item Pour $y\geq 0$,  $f_1^{-1}(y)=-\sqrt{y}\leq0$ et $f_2^{-1}(y)=\sqrt{y}\geq0$}
\end{itemize}



\vspace*{-3ex}
\uncover<3->{  
\myfigure{.75}{
\tikzinput{fig_fonctions11}
}}
\vspace*{-3.5ex}
\end{exemple}
  
\end{frame}





\begin{frame}

\begin{exemple}

\begin{itemize}
  \item Soit $n\ge 1$. Soit $f : [0,+\infty[ \to [0,+\infty[$ définie par $f(x)=x^n$
\pause  
  \item Alors $f$ est continue et strictement croissante
\pause   
  \item Comme $\lim_{+\infty} f = +\infty$ alors $f$ est une bijection
\pause   
  \item Sa bijection réciproque $f^{-1}$ est notée : $x \mapsto x^{\frac{1}{n}}$
(ou aussi $x \mapsto \sqrt[n]{x}$) : c'est la fonction racine $n$-ème
\pause   
  \item La fonction racine $n$-ème est continue et strictement croissante 
\end{itemize}
\end{exemple}

\end{frame}

%---------------------------------------------------------------
\section{Démonstration}

\begin{frame}

\begin{lemme}
Soit $f:I\to \Rr$. Si $f$ est strictement monotone sur $I$, alors $f$ est injective sur $I$
\end{lemme}

\pause

\begin{proof}
\begin{itemize}
  \item Soient $x,x' \in I$ tels que $f(x)=f(x')$. Montrons que $x=x'$
\pause  
  \item 
  \begin{itemize}
    \item Si on avait $x<x'$ 
    
    \item alors on aurait $f(x)<f(x')$ si $f$ strictement croissante 
    
    \item ou $f(x)>f(x')$ si $f$ strictement décroissante 
    
    \item Comme c'est impossible, on en déduit que $x\geq x'$
   \end{itemize}   
\pause  
  
  \item En échangeant les rôles de $x$ et de $x'$, 
on montre de même que $x\leq x'$
\pause    
  \item On en conclut que $x=x'$ et donc que $f$ est injective
  
\end{itemize}

\end{proof}
\end{frame}

\begin{frame}

\begin{lemme}
Soit $f : I \to \Rr$ une fonction continue 
et strictement monotone sur $I$ alors $f$
établit une bijection de $I$ dans l'intervalle image $J=f(I)$  
\end{lemme}

\pause

\begin{proof}
\begin{itemize}
  \item D'après le lemme précédent, $f$ est injective sur $I$
\pause  
  \item En restreignant son ensemble d'arrivée à son image $J=f(I)$,
  $f : I \to J$ est automatiquement surjective
\pause 
  \item On obtient que $f$ établit une bijection de $I$ dans $J$
\pause  
  \item Comme $f$ est continue, par le théorème des valeurs intermédiaires,
l'ensemble $J$ est un intervalle
 
\end{itemize}
  
\end{proof}

\end{frame}

% 
% \begin{frame}
% 
% \begin{proof}[Démonstration du théorème]
% \begin{enumerate}
% \item 
% \item Supposons pour fixer les idées que $f$ est strictement croissante.
% \begin{enumerate}
% \item Montrons que $f^{-1}$ est strictement croissante sur $J$. Soient $y,y'\in J$ tels que $y<y'$. Notons $x=f^{-1}(y)\in I$ et $x'=f^{-1}(y')\in I$. Alors $y=f(x)$, $y'=f(x')$ et donc
% \begin{align*}
% y<y'& \implies f(x)<f(x')\\
% 	& \implies x<x' \qquad \text{ (car $f$ est strictement croissante)}\\
% 	& \implies f^{-1}(y)<f^{-1}(y'),
% \end{align*} 
% c'est-à-dire $f^{-1}$ est strictement croissante sur $J$.
% 
% \setcounter{saveenumi}{\theenumi}
% 
% \end{enumerate}
% \end{enumerate}
% 
% \end{proof}
%   
% \end{frame}
% 
% 
% \begin{frame}
% 
% \begin{proof}
% 
% \begin{enumerate}
%   \setcounter{enumi}{\thesaveenumi}
% 
% \item Montrons que $f^{-1}$ est continue sur $J$. On se limite au cas où $I$ est de la forme $]a,b[$, les autres cas se montrent de la même manière. Soit $y_0\in J$. On note $x_0=f^{-1}(y_0)\in I$. Soit $\epsilon>0$. On peut toujours supposer que $[x_0-\epsilon,x_0+\epsilon]\subset I$. On cherche un réel $\delta>0$ tel que pour tout $y\in J$ on ait
% \[
% y_0-\delta<y<y_0+\delta \implies f^{-1}(y_0)-\epsilon<f^{-1}(y)<f^{-1}(y_0)+\epsilon
% \]
% c'est-à-dire tel que pour tout $x\in I$ on ait
% \[
% y_0-\delta<f(x)<y_0+\delta \implies f^{-1}(y_0)-\epsilon<x<f^{-1}(y_0)+\epsilon.
% \]
% Or, comme $f$ est strictement croissante, on a pour tout $x\in I$
% \begin{align*}
% f(x_0-\epsilon)<f(x)<f(x_0+\epsilon) & \implies x_0-\epsilon<x<x_0+\epsilon\\
% & \implies f^{-1}(y_0)-\epsilon<x<f^{-1}(y_0)+\epsilon.
% \end{align*} 
% Comme $f(x_0-\epsilon)<y_0<f(x_0+\epsilon) $, on peut choisir le réel $\delta>0$ tel que
% \[
% f(x_0-\epsilon)<y_0-\delta \quad \text{ et } \quad f(x_0+\epsilon) > y_0+\delta
% \]
% et on a bien alors pour tout $x\in I$
% \begin{align*}
% y_0-\delta<f(x)<y_0+\delta & \implies f(x_0-\epsilon)<f(x)<f(x_0+\epsilon)\\
% & \implies f^{-1}(y_0)-\epsilon<x<f^{-1}(y_0)+\epsilon.
% \end{align*} 
% La fonction $f^{-1}$ est donc continue sur $J$.
% \end{enumerate}
% \end{proof}
% 
%   
% \end{frame}

%%%%%%%%%%%%%%%%%%%%%%%%%%%%%%%%%%%%%%%%%%%%%%%%%%%%%%%%%%%%%%%%
\section{Mini-exercices}

\begin{frame}

\begin{miniexercice}
\begin{enumerate}
  \item Montrer que chacune des hypothèses \og continue \fg\  et \og strictement monotone \fg\ 
  est nécessaire dans l'énoncé du théorème.

  \item Soit $f : \Rr \to \Rr$ définie par $f(x)=x^3+x$. Montrer que $f$ est bijective, 
  tracer le graphe de $f$ et de $f^{-1}$.
  
  \item Soit $n \ge 1$. Montrer que $f(x)=1+x+x^2+\cdots+x^n$ définit une bijection 
  de l'intervalle $[0,1]$ vers un intervalle à préciser.
  
 
  \item Existe-t-il une fonction continue : $f: [0,1[ \to ]0,1[$ qui soit bijective ? 
  $f: [0,1[ \to ]0,1[$ qui soit injective ?  $f: ]0,1[ \to [0,1]$ qui soit surjective ?
  
  \item Pour $y \in \Rr$ on considère l'équation $x + \exp x =y$. 
  Montrer qu'il existe une unique solution $y$. Comment varie $y$ en fonction de $x$ ?
  Comme varie $x$ en fonction de $y$ ?
    
\end{enumerate}
\end{miniexercice}

\end{frame}

\end{document}
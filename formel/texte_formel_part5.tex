
%%%%%%%%%%%%%%%%%% PREAMBULE %%%%%%%%%%%%%%%%%%


\documentclass[12pt]{article}

\usepackage{amsfonts,amsmath,amssymb,amsthm}
\usepackage[utf8]{inputenc}
\usepackage[T1]{fontenc}
\usepackage[francais]{babel}


% packages
\usepackage{amsfonts,amsmath,amssymb,amsthm}
\usepackage[utf8]{inputenc}
\usepackage[T1]{fontenc}
%\usepackage{lmodern}

\usepackage[francais]{babel}
\usepackage{fancybox}
\usepackage{graphicx}

\usepackage{float}

%\usepackage[usenames, x11names]{xcolor}
\usepackage{tikz}
\usepackage{datetime}

\usepackage{mathptmx}
%\usepackage{fouriernc}
%\usepackage{newcent}
\usepackage[mathcal,mathbf]{euler}

%\usepackage{palatino}
%\usepackage{newcent}


% Commande spéciale prompteur

%\usepackage{mathptmx}
%\usepackage[mathcal,mathbf]{euler}
%\usepackage{mathpple,multido}

\usepackage[a4paper]{geometry}
\geometry{top=2cm, bottom=2cm, left=1cm, right=1cm, marginparsep=1cm}

\newcommand{\change}{{\color{red}\rule{\textwidth}{1mm}\\}}

\newcounter{mydiapo}

\newcommand{\diapo}{\newpage
\hfill {\normalsize  Diapo \themydiapo \quad \texttt{[\jobname]}} \\
\stepcounter{mydiapo}}


%%%%%%% COULEURS %%%%%%%%%%

% Pour blanc sur noir :
%\pagecolor[rgb]{0.5,0.5,0.5}
% \pagecolor[rgb]{0,0,0}
% \color[rgb]{1,1,1}



%\DeclareFixedFont{\myfont}{U}{cmss}{bx}{n}{18pt}
\newcommand{\debuttexte}{
%%%%%%%%%%%%% FONTES %%%%%%%%%%%%%
\renewcommand{\baselinestretch}{1.5}
\usefont{U}{cmss}{bx}{n}
\bfseries

% Taille normale : commenter le reste !
%Taille Arnaud
%\fontsize{19}{19}\selectfont

% Taille Barbara
%\fontsize{21}{22}\selectfont

%Taille François
\fontsize{25}{30}\selectfont

%Taille Pascal
%\fontsize{25}{30}\selectfont

%Taille Laura
%\fontsize{30}{35}\selectfont


%\myfont
%\usefont{U}{cmss}{bx}{n}

%\Huge
%\addtolength{\parskip}{\baselineskip}
}


% \usepackage{hyperref}
% \hypersetup{colorlinks=true, linkcolor=blue, urlcolor=blue,
% pdftitle={Exo7 - Exercices de mathématiques}, pdfauthor={Exo7}}


%section
% \usepackage{sectsty}
% \allsectionsfont{\bf}
%\sectionfont{\color{Tomato3}\upshape\selectfont}
%\subsectionfont{\color{Tomato4}\upshape\selectfont}

%----- Ensembles : entiers, reels, complexes -----
\newcommand{\Nn}{\mathbb{N}} \newcommand{\N}{\mathbb{N}}
\newcommand{\Zz}{\mathbb{Z}} \newcommand{\Z}{\mathbb{Z}}
\newcommand{\Qq}{\mathbb{Q}} \newcommand{\Q}{\mathbb{Q}}
\newcommand{\Rr}{\mathbb{R}} \newcommand{\R}{\mathbb{R}}
\newcommand{\Cc}{\mathbb{C}} 
\newcommand{\Kk}{\mathbb{K}} \newcommand{\K}{\mathbb{K}}

%----- Modifications de symboles -----
\renewcommand{\epsilon}{\varepsilon}
\renewcommand{\Re}{\mathop{\text{Re}}\nolimits}
\renewcommand{\Im}{\mathop{\text{Im}}\nolimits}
%\newcommand{\llbracket}{\left[\kern-0.15em\left[}
%\newcommand{\rrbracket}{\right]\kern-0.15em\right]}

\renewcommand{\ge}{\geqslant}
\renewcommand{\geq}{\geqslant}
\renewcommand{\le}{\leqslant}
\renewcommand{\leq}{\leqslant}

%----- Fonctions usuelles -----
\newcommand{\ch}{\mathop{\mathrm{ch}}\nolimits}
\newcommand{\sh}{\mathop{\mathrm{sh}}\nolimits}
\renewcommand{\tanh}{\mathop{\mathrm{th}}\nolimits}
\newcommand{\cotan}{\mathop{\mathrm{cotan}}\nolimits}
\newcommand{\Arcsin}{\mathop{\mathrm{Arcsin}}\nolimits}
\newcommand{\Arccos}{\mathop{\mathrm{Arccos}}\nolimits}
\newcommand{\Arctan}{\mathop{\mathrm{Arctan}}\nolimits}
\newcommand{\Argsh}{\mathop{\mathrm{Argsh}}\nolimits}
\newcommand{\Argch}{\mathop{\mathrm{Argch}}\nolimits}
\newcommand{\Argth}{\mathop{\mathrm{Argth}}\nolimits}
\newcommand{\pgcd}{\mathop{\mathrm{pgcd}}\nolimits} 

\newcommand{\Card}{\mathop{\text{Card}}\nolimits}
\newcommand{\Ker}{\mathop{\text{Ker}}\nolimits}
\newcommand{\id}{\mathop{\text{id}}\nolimits}
\newcommand{\ii}{\mathrm{i}}
\newcommand{\dd}{\mathrm{d}}
\newcommand{\Vect}{\mathop{\text{Vect}}\nolimits}
\newcommand{\Mat}{\mathop{\mathrm{Mat}}\nolimits}
\newcommand{\rg}{\mathop{\text{rg}}\nolimits}
\newcommand{\tr}{\mathop{\text{tr}}\nolimits}
\newcommand{\ppcm}{\mathop{\text{ppcm}}\nolimits}

%----- Structure des exercices ------

\newtheoremstyle{styleexo}% name
{2ex}% Space above
{3ex}% Space below
{}% Body font
{}% Indent amount 1
{\bfseries} % Theorem head font
{}% Punctuation after theorem head
{\newline}% Space after theorem head 2
{}% Theorem head spec (can be left empty, meaning ‘normal’)

%\theoremstyle{styleexo}
\newtheorem{exo}{Exercice}
\newtheorem{ind}{Indications}
\newtheorem{cor}{Correction}


\newcommand{\exercice}[1]{} \newcommand{\finexercice}{}
%\newcommand{\exercice}[1]{{\tiny\texttt{#1}}\vspace{-2ex}} % pour afficher le numero absolu, l'auteur...
\newcommand{\enonce}{\begin{exo}} \newcommand{\finenonce}{\end{exo}}
\newcommand{\indication}{\begin{ind}} \newcommand{\finindication}{\end{ind}}
\newcommand{\correction}{\begin{cor}} \newcommand{\fincorrection}{\end{cor}}

\newcommand{\noindication}{\stepcounter{ind}}
\newcommand{\nocorrection}{\stepcounter{cor}}

\newcommand{\fiche}[1]{} \newcommand{\finfiche}{}
\newcommand{\titre}[1]{\centerline{\large \bf #1}}
\newcommand{\addcommand}[1]{}
\newcommand{\video}[1]{}

% Marge
\newcommand{\mymargin}[1]{\marginpar{{\small #1}}}



%----- Presentation ------
\setlength{\parindent}{0cm}

%\newcommand{\ExoSept}{\href{http://exo7.emath.fr}{\textbf{\textsf{Exo7}}}}

\definecolor{myred}{rgb}{0.93,0.26,0}
\definecolor{myorange}{rgb}{0.97,0.58,0}
\definecolor{myyellow}{rgb}{1,0.86,0}

\newcommand{\LogoExoSept}[1]{  % input : echelle
{\usefont{U}{cmss}{bx}{n}
\begin{tikzpicture}[scale=0.1*#1,transform shape]
  \fill[color=myorange] (0,0)--(4,0)--(4,-4)--(0,-4)--cycle;
  \fill[color=myred] (0,0)--(0,3)--(-3,3)--(-3,0)--cycle;
  \fill[color=myyellow] (4,0)--(7,4)--(3,7)--(0,3)--cycle;
  \node[scale=5] at (3.5,3.5) {Exo7};
\end{tikzpicture}}
}



\theoremstyle{definition}
%\newtheorem{proposition}{Proposition}
%\newtheorem{exemple}{Exemple}
%\newtheorem{theoreme}{Théorème}
\newtheorem{lemme}{Lemme}
\newtheorem{corollaire}{Corollaire}
%\newtheorem*{remarque*}{Remarque}
%\newtheorem*{miniexercice}{Mini-exercices}
%\newtheorem{definition}{Définition}




%definition d'un terme
\newcommand{\defi}[1]{{\color{myorange}\textbf{\emph{#1}}}}
\newcommand{\evidence}[1]{{\color{blue}\textbf{\emph{#1}}}}



 %----- Commandes divers ------

\newcommand{\codeinline}[1]{\texttt{#1}}

%%%%%%%%%%%%%%%%%%%%%%%%%%%%%%%%%%%%%%%%%%%%%%%%%%%%%%%%%%%%%
%%%%%%%%%%%%%%%%%%%%%%%%%%%%%%%%%%%%%%%%%%%%%%%%%%%%%%%%%%%%%


\begin{document}

\debuttexte


%%%%%%%%%%%%%%%%%%%%%%%%%%%%%%%%%%%%%%%%%%%%%%%%%%%%%%%%%%%
\diapo

L'algèbre linéaire est la partie des mathématiques qui 
%s'occupe 
traite des matrices et des structures vectorielles.
Beaucoup de problèmes se ramènent ou se réduisent à %'approchent par 
des problèmes linéaires, pour lesquels il existe  
des 
%solutions 
méthodes de résolution efficaces.

\change

\change
Nous commencerons par visiter les opérations de base sur les matrices

\change
Il y aura un tp non corrigé sur la réduction de Gauss,  et le calcul de l'inverse des matrices.

\change
Nous verrons comment manipuler les applications linéaires
et en particulier calculer leur image et noyau.

\change
Enfin nous utiliserons l'algèbre linéaire pour résoudre des problèmes de moindres carrés.



%%%%%%%%%%%%%%%%%%%%%%%%%%%%%%%%%%%%%%%%%%%%%%%%%%%%%%%%%%%
\diapo

Nous commençons par des opérations basiques sur les matrices.

Il s'agit de définir plusieurs matrices, de voir quels sont les produits possibles,

de calculer ensuite une puissance de matrice, notez déjà qu'ici $I$ désigne la matrice identité,

nous verrons de plus comment extraire un coefficient.

Enfin nous terminerons ce tp en calculant l'inverse de la matrice $A$ puis la trace de son inverse.

%%%%%%%%%%%%%%%%%%%%%%%%%%%%%%%%%%%%%%%%%%%%%%%%%%%%%%%%%%%
\diapo

On définit une matrice ayant $n$ lignes et $p$ colonnes par cette commande

où l'on décrit la matrice ligne par ligne par des listes.
 
 
\change

Par exemple voici la matrice $A$ qui est une matrice $3 \times 3$,
sa première ligne est $(1,2,3)$ sa deuxième $(-1,0,1)$ et sa troisième $(0,1,0)$.

La matrice $B$ a deux lignes et trois colonnes.

$u$ est une matrices à $3$ lignes et une colonne, on peut la considérer comme un vecteur colonne,

idem pour $v$.


\change
Le produit de matrices s'écrit tout simplement \codeinline{A * B}.

\change
Souvenons nous que le produit $A \times B$ de deux matrices 
n'est possible que si le nombre de colonnes de $A$ est égal au nombre de lignes de $B$.

\change
  Les produits possibles seront donc ici : 
  $B\times A$, $A \times u$, $A \times v$,  $B \times u$ et $B \times v$.
  
  Mais le produit $A \times B$ n'est pas défini.

%%%%%%%%%%%%%%%%%%%%%%%%%%%%%%%%%%%%%%%%%%%%%%%%%%%%%%%%%%%
\diapo

On définit la matrice identité $I$ de taille $3\times 3$ par cette
 commande \codeinline{I = identity\_matrix(3)}.


\change

Le calcul de $(A-I)^{7}$ se fait naturellement ainsi


\change
et renvoie cette matrice.


\change  
Le coefficient en position $(2,3)$ d'une matrice $M$ 
  est celui situé sur la deuxième ligne et la troisième colonne.
  Donc ici c'est $153$.  
  
Mais avec Sage les indices démarrent à $0$,

on récupère donc le coefficient $153$ comme celui indexé
par la ligne qui porte le numéro $1$ et la colonne qui porte le numéro $2$.
 
Attention au décalage !
  
\change
L'inverse d'une matrice   \codeinline{A} s'écrit simplement
  \codeinline{A\^{}-1}
  
\change
Cela donne ici cette matrice.

\change
La trace est par définition la somme des éléments situés sur la diagonale principale : 

\change
les éléments de la diagonale principale sont ceux d'indices $(i,i)$ pour $i$ variant de 0 à 2.


On fait donc la somme de tous ces éléments,

\change
cela renvoie la trace qui vaut ici $-\frac14$.


%%%%%%%%%%%%%%%%%%%%%%%%%%%%%%%%%%%%%%%%%%%%%%%%%%%%%%%%%%%
\diapo

Revenons sur quelques points.

 Une matrice $A$ se comporte en Sage comme une liste à double entrée et 
  on accède aux coefficients par la commande \codeinline{A[i,j]} (\codeinline{i}
  pour les lignes, \codeinline{j} pour les colonnes).
  
\change  
  Attention ! Les listes étant indexées à partir du rang $0$, les
  indices \codeinline{i}, \codeinline{j} ont pour première valeur $0$.


\change
On peut aussi définir des vecteurs par
  \codeinline{vector([x1,x2,...,xn])}.
  Un vecteur peut représenter en même temps un vecteur ligne, 
  ou un vecteur colonne. On peut donc multiplier une matrice 
  par un vecteur (à gauche ou à droite).
  
\change   
 Si on définit un vecteur, par exemple  le vecteur $(1,2,3,4)$ alors on peut le transformer 

 \change
en sa  matrice ligne correspondante.
 
 \change
puis, en utilisant la transposition, en sa matrice colonne correspondante.   


%%%%%%%%%%%%%%%%%%%%%%%%%%%%%%%%%%%%%%%%%%%%%%%%%%%%%%%%%%%
\diapo

Voici un tp que nous n'allons pas corriger ici mais qui est corrigé dans le polycopié.

Il s'agit de mettre en \oe uvre la méthode de Gauss.
Cette méthode est valable pour des matrices de taille quelconque, 
mais ici on se limite au cas des matrices carrées inversibles.

Il faut d'abord implémenter les opérations élémentaires sur les lignes.

Obtenir une matrice échelonnée,

puis échelonnée réduite ,

pour être enfin en mesure de calculer l'inverse d'une matrice.

Encore une fois ce tp n'est pas corrigé, mais je vous encourage vivement à le travailler !



%%%%%%%%%%%%%%%%%%%%%%%%%%%%%%%%%%%%%%%%%%%%%%%%%%%%%%%%%%%
\diapo

Attardons nous maintenant à l'étude d'applications linéaires.

Soit $f : \Rr^3 \to \Rr^4$ l'application linéaire définie par
cette expression.

On demande d'abord de l'exprimer en termes matriciels.

Puis de calculer son noyau, puis son image.

Nous serons alors en mesure de répondre aux questions relatives à l'injectivité et à la surjectivité de l'application linéaire $f$.
%  $f$ est-elle injective ? est-elle surjective ?


%%%%%%%%%%%%%%%%%%%%%%%%%%%%%%%%%%%%%%%%%%%%%%%%%%%%%%%%%%%
\diapo

La matrice associée à l'application linéaire $f$ est clairement la matrice 
  $$A = \begin{pmatrix}1&0&2\\5&2&2\\2&1&0\\1&1&-2\\\end{pmatrix}.$$
  
\change
  L'application linéaire $f$ étant alors définie par $X \mapsto Y = AX$.

\change 
  Le noyau de l'application linéaire (ou ce qui qui revient au même de la matrice) 
  s'obtient par la commande \codeinline{right\_kernel()}.
  
\change
L'image se définit par la commande
  \codeinline{column\_space()} car l'image est
  exactement le sous-espace vectoriel engendré par les vecteurs colonnes de la matrice. 
  
 Attention, les méthodes \codeinline{kernel} et \codeinline{image} ne conviennent pas, car elles définissent le noyau et l'image "à gauche".  
   
\change
Notons "Ker" le noyau et "Im" l'image et demandons à la machine de nous en calculer une base.

\change
Voici les résultats :

Le noyau est un espace vectoriel de dimension $1$ dans $\Rr^3$,

\change
engendré par le vecteur $(2, -4, -1)$. 

\change
Le noyau n'étant pas trivial, l'application $f$ n'est pas injective.
  
\change
L'image quant à elle est un espace vectoriel de dimension $2$ dans $\Rr^4$, 

\change
dont une base   
  est formée des deux vecteurs :
  $$ 
  \left(1, 1, 0, -1\right)\qquad \mbox{et} \left(0,2, 1, 1\right).$$
  
\change
L'image n'étant pas $\Rr^4$ tout entier, l'application $f$ n'est pas non plus surjective.
  
 

%%%%%%%%%%%%%%%%%%%%%%%%%%%%%%%%%%%%%%%%%%%%%%%%%%%%%%%%%%%
\diapo

\change
Un système linéaire contenant plus d'équations que d'inconnues 

\change
n'a en général pas de solution.

\change
On aimerait pourtant parfois trouver 
ce qui s'approche le plus d'une solution.

\change
Formalisons ceci : 
soit $A \in M_{n,p}(\mathbb R)$ avec $n \ge p$ une matrice à coefficients réels, 

\change
$X$ un vecteur inconnu de taille $p$ 

\change
et $B$ un vecteur de taille $n$.

\change
Le système $AX=B$ n'a en général pas de solution, 

mais on aimerait que le vecteur 
$AX-B$ soit le plus proche du vecteur nul.

\change
On appelle \defi{solution des moindres carrés}, un vecteur $X$ 
tel que $\| AX -B \|$ soit minimale, c'est-à-dire 
on veut rendre minimum la somme des carrés des coordonnées de $AX-B$.

\change
Cette solution des moindres carrées est explicite et est donnée par la formule :
\begin{equation}
X = (A^TA)^{-1} A^T B
\end{equation}
\begin{center}
L'inverse du produit de $A^T$ par $A$ \quad fois $A^T$ \quad fois $B$
\end{center}

Si la matrice $A$ est de rang maximal, ce qu'on supposera toujours, 
alors on peut montrer que la matrice $A^TA$ de la formule est inversible.


%%%%%%%%%%%%%%%%%%%%%%%%%%%%%%%%%%%%%%%%%%%%%%%%%%%%%%%%%%%
\diapo


Nous allons appliquer cette méthode des moindres carrés à deux problèmes d'interpolation.

On se donne tout d'abord une liste de points.

Le premier sujet consiste à déterminer une droite qui approche au mieux ces points, 
qui bien sûr ne sont pas alignés.

Le plus difficile est de bien comprendre comment on transforme ce problème en un problème linéaire.

~

Une fois cette étape franchie, il n'est pas plus compliqué
 d'étendre cette question à la recherche d'une interpolation polynomiale, par exemple de trouver 
quel est le polynôme de degré $3$ qui approche 
au mieux ces points.
  


%%%%%%%%%%%%%%%%%%%%%%%%%%%%%%%%%%%%%%%%%%%%%%%%%%%%%%%%%%%
\diapo


Voici une représentation graphique de la droite demandée, c'est celle qui minimise la distance entre les points et la droite.


\change

Et voici la courbe du polynôme de degré $3$ qui évidemment approche encore mieux les points.
 
 %%%%%%%%%%%%%%%%%%%%%%%%%%%%%%%%%%%%%%%%%%%%%%%%%%%%%%%%%%%
\diapo
  
\change
Comme les points ne sont pas alignés, il n'existe 
  pas de droite passant par tous ces points.
  
\change  
  On cherche une droite d'équation $y = a +bx$ [!! et pas ax+b!!] qui minimise la distance
  entre les points et la droite.
  
\change
Pour la trouver on pose [!!!!] $f(x) = a + bx$. 

et nous devons déterminer les coefficients $a$ et $b$ qui conviennent. 


\change
Pour nos $n$ points $(x_i,y_i)$, on aurait souhaité, dans l'idéal, $f(x_i)=y_i$ ce qui est ici impossible. 

\change
Mais cela nous donne la liste de conditions suivante 

\change
c'est-à-dire 
  $
  \left\{ 
  \begin{array}{rcl}
  a +b x_1 &=& y_1 \\
  a +b x_2 &=& y_2 \\
  \vdots && \vdots\\
  a +b x_n &=& y_n \\
  \end{array}
  \right.$

Je vous rappelle que les $x_i$ et les $y_i$ sont des données, et que
$a$ et $b$ sont les inconnues.

\change
Cette présentation montre clairement que ce problème peut s'écrire de manière matricielle.
% C'est maintenant un problème linéaire, qui s'écrit sous forme matricielle

$
  \begin{pmatrix}
  1 & x_1 \\
  1 & x_2 \\
  \vdots & \vdots \\
  1 & x_n \\  
  \end{pmatrix}
  \begin{pmatrix}
  a \\ b  
  \end{pmatrix}
  = 
  \begin{pmatrix}
  y_1\\
  y_2 \\
  \vdots \\
  y_n
  \end{pmatrix}
  $
  
\change
Que l'on notera $AX = B.$

\change  
Résolvons (de façon non exacte) le système $AX = B$, par la formule des moindres carrés :

\change
$X = (A^TA)^{-1} A^T B$.

\change
On obtient ainsi $a$ et $b$ qui sont les coordonnées du vecteur $X$
et caractérisent la droite de régression linéaire d'équation [!!!!] $y=a+bx$.


  
%%%%%%%%%%%%%%%%%%%%%%%%%%%%%%%%%%%%%%%%%%%%%%%%%%%%%%%%%%%
\diapo 

 
Passons au code :

tout d'abord définissons une fonction moindres carrés qui 
résout le problème général des moindres carrés
par la formule $X = (A^TA)^{-1} A^T B$.


\change

Voici la liste des points, ils ne sont pas alignés.

\change
On regroupe ces données dans la matrice $A$, chaque ligne est formée de $1$ et de  
l'abscisse du point.

Le vecteur colonne $B$ est la liste des ordonnées.

Il ne nous reste qu'à appliquer la formule des moindres carrés pour trouver $X$...

\change
...c'est-à-dire trouver $a$ et $b$. On obtient cette équation de droite.

%%%%%%%%%%%%%%%%%%%%%%%%%%%%%%%%%%%%%%%%%%%%%%%%%%%%%%%%%%%
\diapo

Pour l'interpolation polynomiale,
l'idée est la même, mais cette fois les inconnues
sont les coefficients de la fonction
$f(x) = a + bx +c x^2 + dx^3$.
  
On aimerait comme précédemment avoir $f(x_i)=y_i$ pour tout $i$.
  
Ce qui équivaut à ce système linéaire :

en termes de matrices  

on arrive à l'équation $AX=B$.

On ne trouve pas de solution exacte $X$ mais une solution approchée par la méthode des moindres carrées. Ce qui nous donne les coefficients $a,b,c,d$ et l'interpolation polynomiale de degré $3$.



\end{document}

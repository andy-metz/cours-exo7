
%%%%%%%%%%%%%%%%%% PREAMBULE %%%%%%%%%%%%%%%%%%

\documentclass[aspectratio=169,utf8]{beamer}
%\documentclass[aspectratio=169,handout]{beamer}

\usetheme{Boadilla}
%\usecolortheme{seahorse}
%\usecolortheme[RGB={245,66,24}]{structure}
\useoutertheme{infolines}

% packages
\usepackage{amsfonts,amsmath,amssymb,amsthm}
\usepackage[utf8]{inputenc}
\usepackage[T1]{fontenc}
\usepackage{lmodern}

\usepackage[francais]{babel}
\usepackage{fancybox}
\usepackage{graphicx}

\usepackage{float}
\usepackage{xfrac}

%\usepackage[usenames, x11names]{xcolor}
\usepackage{pgfplots}
\usepackage{datetime}


% ----------------------------------------------------------------------
% Pour les images
\usepackage{tikz}
\usetikzlibrary{calc,shadows,arrows.meta,patterns,matrix}

\newcommand{\tikzinput}[1]{\input{figures/#1.tikz}}
% --- les figures avec échelle éventuel
\newcommand{\myfigure}[2]{% entrée : échelle, fichier(s) figure à inclure
\begin{center}\small%
\tikzstyle{every picture}=[scale=1.0*#1]% mise en échelle + 0% (automatiquement annulé à la fin du groupe)
#2%
\end{center}}



%-----  Package unités -----
\usepackage{siunitx}
\sisetup{locale = FR,detect-all,per-mode = symbol}

%\usepackage{mathptmx}
%\usepackage{fouriernc}
%\usepackage{newcent}
%\usepackage[mathcal,mathbf]{euler}

%\usepackage{palatino}
%\usepackage{newcent}
% \usepackage[mathcal,mathbf]{euler}



% \usepackage{hyperref}
% \hypersetup{colorlinks=true, linkcolor=blue, urlcolor=blue,
% pdftitle={Exo7 - Exercices de mathématiques}, pdfauthor={Exo7}}


%section
% \usepackage{sectsty}
% \allsectionsfont{\bf}
%\sectionfont{\color{Tomato3}\upshape\selectfont}
%\subsectionfont{\color{Tomato4}\upshape\selectfont}

%----- Ensembles : entiers, reels, complexes -----
\newcommand{\Nn}{\mathbb{N}} \newcommand{\N}{\mathbb{N}}
\newcommand{\Zz}{\mathbb{Z}} \newcommand{\Z}{\mathbb{Z}}
\newcommand{\Qq}{\mathbb{Q}} \newcommand{\Q}{\mathbb{Q}}
\newcommand{\Rr}{\mathbb{R}} \newcommand{\R}{\mathbb{R}}
\newcommand{\Cc}{\mathbb{C}} 
\newcommand{\Kk}{\mathbb{K}} \newcommand{\K}{\mathbb{K}}

%----- Modifications de symboles -----
\renewcommand{\epsilon}{\varepsilon}
\renewcommand{\Re}{\mathop{\text{Re}}\nolimits}
\renewcommand{\Im}{\mathop{\text{Im}}\nolimits}
%\newcommand{\llbracket}{\left[\kern-0.15em\left[}
%\newcommand{\rrbracket}{\right]\kern-0.15em\right]}

\renewcommand{\ge}{\geqslant}
\renewcommand{\geq}{\geqslant}
\renewcommand{\le}{\leqslant}
\renewcommand{\leq}{\leqslant}
\renewcommand{\epsilon}{\varepsilon}

%----- Fonctions usuelles -----
\newcommand{\ch}{\mathop{\text{ch}}\nolimits}
\newcommand{\sh}{\mathop{\text{sh}}\nolimits}
\renewcommand{\tanh}{\mathop{\text{th}}\nolimits}
\newcommand{\cotan}{\mathop{\text{cotan}}\nolimits}
\newcommand{\Arcsin}{\mathop{\text{arcsin}}\nolimits}
\newcommand{\Arccos}{\mathop{\text{arccos}}\nolimits}
\newcommand{\Arctan}{\mathop{\text{arctan}}\nolimits}
\newcommand{\Argsh}{\mathop{\text{argsh}}\nolimits}
\newcommand{\Argch}{\mathop{\text{argch}}\nolimits}
\newcommand{\Argth}{\mathop{\text{argth}}\nolimits}
\newcommand{\pgcd}{\mathop{\text{pgcd}}\nolimits} 


%----- Commandes divers ------
\newcommand{\ii}{\mathrm{i}}
\newcommand{\dd}{\text{d}}
\newcommand{\id}{\mathop{\text{id}}\nolimits}
\newcommand{\Ker}{\mathop{\text{Ker}}\nolimits}
\newcommand{\Card}{\mathop{\text{Card}}\nolimits}
\newcommand{\Vect}{\mathop{\text{Vect}}\nolimits}
\newcommand{\Mat}{\mathop{\text{Mat}}\nolimits}
\newcommand{\rg}{\mathop{\text{rg}}\nolimits}
\newcommand{\tr}{\mathop{\text{tr}}\nolimits}


%----- Structure des exercices ------

\newtheoremstyle{styleexo}% name
{2ex}% Space above
{3ex}% Space below
{}% Body font
{}% Indent amount 1
{\bfseries} % Theorem head font
{}% Punctuation after theorem head
{\newline}% Space after theorem head 2
{}% Theorem head spec (can be left empty, meaning ‘normal’)

%\theoremstyle{styleexo}
\newtheorem{exo}{Exercice}
\newtheorem{ind}{Indications}
\newtheorem{cor}{Correction}


\newcommand{\exercice}[1]{} \newcommand{\finexercice}{}
%\newcommand{\exercice}[1]{{\tiny\texttt{#1}}\vspace{-2ex}} % pour afficher le numero absolu, l'auteur...
\newcommand{\enonce}{\begin{exo}} \newcommand{\finenonce}{\end{exo}}
\newcommand{\indication}{\begin{ind}} \newcommand{\finindication}{\end{ind}}
\newcommand{\correction}{\begin{cor}} \newcommand{\fincorrection}{\end{cor}}

\newcommand{\noindication}{\stepcounter{ind}}
\newcommand{\nocorrection}{\stepcounter{cor}}

\newcommand{\fiche}[1]{} \newcommand{\finfiche}{}
\newcommand{\titre}[1]{\centerline{\large \bf #1}}
\newcommand{\addcommand}[1]{}
\newcommand{\video}[1]{}

% Marge
\newcommand{\mymargin}[1]{\marginpar{{\small #1}}}

\def\noqed{\renewcommand{\qedsymbol}{}}


%----- Presentation ------
\setlength{\parindent}{0cm}

%\newcommand{\ExoSept}{\href{http://exo7.emath.fr}{\textbf{\textsf{Exo7}}}}

\definecolor{myred}{rgb}{0.93,0.26,0}
\definecolor{myorange}{rgb}{0.97,0.58,0}
\definecolor{myyellow}{rgb}{1,0.86,0}

\newcommand{\LogoExoSept}[1]{  % input : echelle
{\usefont{U}{cmss}{bx}{n}
\begin{tikzpicture}[scale=0.1*#1,transform shape]
  \fill[color=myorange] (0,0)--(4,0)--(4,-4)--(0,-4)--cycle;
  \fill[color=myred] (0,0)--(0,3)--(-3,3)--(-3,0)--cycle;
  \fill[color=myyellow] (4,0)--(7,4)--(3,7)--(0,3)--cycle;
  \node[scale=5] at (3.5,3.5) {Exo7};
\end{tikzpicture}}
}


\newcommand{\debutmontitre}{
  \author{} \date{} 
  \thispagestyle{empty}
  \hspace*{-10ex}
  \begin{minipage}{\textwidth}
    \titlepage  
  \vspace*{-2.5cm}
  \begin{center}
    \LogoExoSept{2.5}
  \end{center}
  \end{minipage}

  \vspace*{-0cm}
  
  % Astuce pour que le background ne soit pas discrétisé lors de la conversion pdf -> png
\begin{tikzpicture}
        \fill[opacity=0,green!60!black] (0,0)--++(0,0)--++(0,0)--++(0,0)--cycle; 
\end{tikzpicture}

% toc S'affiche trop tot :
% \tableofcontents[hideallsubsections, pausesections]
}

\newcommand{\finmontitre}{
  \end{frame}
  \setcounter{framenumber}{0}
} % ne marche pas pour une raison obscure

%----- Commandes supplementaires ------

% \usepackage[landscape]{geometry}
% \geometry{top=1cm, bottom=3cm, left=2cm, right=10cm, marginparsep=1cm
% }
% \usepackage[a4paper]{geometry}
% \geometry{top=2cm, bottom=2cm, left=2cm, right=2cm, marginparsep=1cm
% }

%\usepackage{standalone}


% New command Arnaud -- november 2011
\setbeamersize{text margin left=24ex}
% si vous modifier cette valeur il faut aussi
% modifier le decalage du titre pour compenser
% (ex : ici =+10ex, titre =-5ex

\theoremstyle{definition}
%\newtheorem{proposition}{Proposition}
%\newtheorem{exemple}{Exemple}
%\newtheorem{theoreme}{Théorème}
%\newtheorem{lemme}{Lemme}
%\newtheorem{corollaire}{Corollaire}
%\newtheorem*{remarque*}{Remarque}
%\newtheorem*{miniexercice}{Mini-exercices}
%\newtheorem{definition}{Définition}

% Commande tikz
\usetikzlibrary{calc}
\usetikzlibrary{patterns,arrows}
\usetikzlibrary{matrix}
\usetikzlibrary{fadings} 

%definition d'un terme
\newcommand{\defi}[1]{{\color{myorange}\textbf{\emph{#1}}}}
\newcommand{\evidence}[1]{{\color{blue}\textbf{\emph{#1}}}}
\newcommand{\assertion}[1]{\emph{\og#1\fg}}  % pour chapitre logique
%\renewcommand{\contentsname}{Sommaire}
\renewcommand{\contentsname}{}
\setcounter{tocdepth}{2}



%------ Encadrement ------

\usepackage{fancybox}


\newcommand{\mybox}[1]{
\setlength{\fboxsep}{7pt}
\begin{center}
\shadowbox{#1}
\end{center}}

\newcommand{\myboxinline}[1]{
\setlength{\fboxsep}{5pt}
\raisebox{-10pt}{
\shadowbox{#1}
}
}

%--------------- Commande beamer---------------
\newcommand{\beameronly}[1]{#1} % permet de mettre des pause dans beamer pas dans poly


\setbeamertemplate{navigation symbols}{}
\setbeamertemplate{footline}  % tiré du fichier beamerouterinfolines.sty
{
  \leavevmode%
  \hbox{%
  \begin{beamercolorbox}[wd=.333333\paperwidth,ht=2.25ex,dp=1ex,center]{author in head/foot}%
    % \usebeamerfont{author in head/foot}\insertshortauthor%~~(\insertshortinstitute)
    \usebeamerfont{section in head/foot}{\bf\insertshorttitle}
  \end{beamercolorbox}%
  \begin{beamercolorbox}[wd=.333333\paperwidth,ht=2.25ex,dp=1ex,center]{title in head/foot}%
    \usebeamerfont{section in head/foot}{\bf\insertsectionhead}
  \end{beamercolorbox}%
  \begin{beamercolorbox}[wd=.333333\paperwidth,ht=2.25ex,dp=1ex,right]{date in head/foot}%
    % \usebeamerfont{date in head/foot}\insertshortdate{}\hspace*{2em}
    \insertframenumber{} / \inserttotalframenumber\hspace*{2ex} 
  \end{beamercolorbox}}%
  \vskip0pt%
}


\definecolor{mygrey}{rgb}{0.5,0.5,0.5}
\setlength{\parindent}{0cm}
%\DeclareTextFontCommand{\helvetica}{\fontfamily{phv}\selectfont}

% background beamer
\definecolor{couleurhaut}{rgb}{0.85,0.9,1}  % creme
\definecolor{couleurmilieu}{rgb}{1,1,1}  % vert pale
\definecolor{couleurbas}{rgb}{0.85,0.9,1}  % blanc
\setbeamertemplate{background canvas}[vertical shading]%
[top=couleurhaut,middle=couleurmilieu,midpoint=0.4,bottom=couleurbas] 
%[top=fondtitre!05,bottom=fondtitre!60]



\makeatletter
\setbeamertemplate{theorem begin}
{%
  \begin{\inserttheoremblockenv}
  {%
    \inserttheoremheadfont
    \inserttheoremname
    \inserttheoremnumber
    \ifx\inserttheoremaddition\@empty\else\ (\inserttheoremaddition)\fi%
    \inserttheorempunctuation
  }%
}
\setbeamertemplate{theorem end}{\end{\inserttheoremblockenv}}

\newenvironment{theoreme}[1][]{%
   \setbeamercolor{block title}{fg=structure,bg=structure!40}
   \setbeamercolor{block body}{fg=black,bg=structure!10}
   \begin{block}{{\bf Th\'eor\`eme }#1}
}{%
   \end{block}%
}


\newenvironment{proposition}[1][]{%
   \setbeamercolor{block title}{fg=structure,bg=structure!40}
   \setbeamercolor{block body}{fg=black,bg=structure!10}
   \begin{block}{{\bf Proposition }#1}
}{%
   \end{block}%
}

\newenvironment{corollaire}[1][]{%
   \setbeamercolor{block title}{fg=structure,bg=structure!40}
   \setbeamercolor{block body}{fg=black,bg=structure!10}
   \begin{block}{{\bf Corollaire }#1}
}{%
   \end{block}%
}

\newenvironment{mydefinition}[1][]{%
   \setbeamercolor{block title}{fg=structure,bg=structure!40}
   \setbeamercolor{block body}{fg=black,bg=structure!10}
   \begin{block}{{\bf Définition} #1}
}{%
   \end{block}%
}

\newenvironment{lemme}[0]{%
   \setbeamercolor{block title}{fg=structure,bg=structure!40}
   \setbeamercolor{block body}{fg=black,bg=structure!10}
   \begin{block}{\bf Lemme}
}{%
   \end{block}%
}

\newenvironment{remarque}[1][]{%
   \setbeamercolor{block title}{fg=black,bg=structure!20}
   \setbeamercolor{block body}{fg=black,bg=structure!5}
   \begin{block}{Remarque #1}
}{%
   \end{block}%
}


\newenvironment{exemple}[1][]{%
   \setbeamercolor{block title}{fg=black,bg=structure!20}
   \setbeamercolor{block body}{fg=black,bg=structure!5}
   \begin{block}{{\bf Exemple }#1}
}{%
   \end{block}%
}


\newenvironment{miniexercice}[0]{%
   \setbeamercolor{block title}{fg=structure,bg=structure!20}
   \setbeamercolor{block body}{fg=black,bg=structure!5}
   \begin{block}{Mini-exercices}
}{%
   \end{block}%
}


\newenvironment{tp}[0]{%
   \setbeamercolor{block title}{fg=structure,bg=structure!40}
   \setbeamercolor{block body}{fg=black,bg=structure!10}
   \begin{block}{\bf Travaux pratiques}
}{%
   \end{block}%
}
\newenvironment{exercicecours}[1][]{%
   \setbeamercolor{block title}{fg=structure,bg=structure!40}
   \setbeamercolor{block body}{fg=black,bg=structure!10}
   \begin{block}{{\bf Exercice }#1}
}{%
   \end{block}%
}
\newenvironment{algo}[1][]{%
   \setbeamercolor{block title}{fg=structure,bg=structure!40}
   \setbeamercolor{block body}{fg=black,bg=structure!10}
   \begin{block}{{\bf Algorithme}\hfill{\color{gray}\texttt{#1}}}
}{%
   \end{block}%
}


\setbeamertemplate{proof begin}{
   \setbeamercolor{block title}{fg=black,bg=structure!20}
   \setbeamercolor{block body}{fg=black,bg=structure!5}
   \begin{block}{{\footnotesize Démonstration}}
   \footnotesize
   \smallskip}
\setbeamertemplate{proof end}{%
   \end{block}}
\setbeamertemplate{qed symbol}{\openbox}


\makeatother
\usecolortheme[RGB={192,41,0}]{structure}

% Commande spécifique à ce chapitre
\newcommand{\Sage}{\texttt{Sage}}

\usepackage{textcomp}

\usepackage{listings}
\lstset{
  upquote=true,
  columns=flexible,
  keepspaces=true,
  basicstyle=\ttfamily,
  commentstyle=\color{gray},
  language=Python,
  showstringspaces=false,
  aboveskip=0em,  
  belowskip=0em,
  escapeinside=||,
  breaklines=true,
  postbreak=\raisebox{0ex}[0ex][0ex]{\qquad\ensuremath{\color{red}\hookrightarrow\space}},
}

\lstset{
  literate={é}{{\'e}}1
           {è}{{\`e}}1
           {à}{{\`a}}1
}

\newcommand{\codeinline}[1]{\lstinline!#1!}
   
%%%%%%%%%%%%%%%%%%%%%%%%%%%%%%%%%%%%%%%%%%%%%%%%%%%%%%%%%%%%%
%%%%%%%%%%%%%%%%%%%%%%%%%%%%%%%%%%%%%%%%%%%%%%%%%%%%%%%%%%%%%


\begin{document}

\renewcommand*{\theenumii}{\alph{enumii}}

\title{{\bf Calcul formel}}
\subtitle{Calculs d'intégrales}

\begin{frame}
  
  \debutmontitre

  \pause

{\footnotesize
\hfill
\setbeamercovered{transparent=50}
\begin{minipage}{0.6\textwidth}
  \begin{itemize}
    \item<3-> \Sage\ comme une super-calculatrice 
    \item<4-> Intégration par parties
    \item<5-> Changement de variable
  \end{itemize}
\end{minipage}
}

\end{frame}

\setcounter{framenumber}{0}





%%%%%%%%%%%%%%%%%%%%%%%%%%%%%%%%%%%%%%%%%%%%%%%%%%%%%%%%%%%%%%%%
\section{\Sage\ comme une super-calculatrice}

\begin{frame}
\begin{tp}
Calculer à la main les primitives des fonctions suivantes. 
Vérifier vos résultats à l'ordinateur.

\medskip

\begin{enumerate}
  \item $f_1(x) = x^4 + \frac{1}{x^2} + \exp(x)+ \cos(x)$
  
  \medskip
  
  \item $f_2(x) = x \sin(x^2)$
  
  \medskip
   
  \item $f_3(x) = \frac{\alpha}{\sqrt{1-x^2}} + \frac{\beta}{1+x^2} + \frac{\gamma}{1+x}$
  
  \medskip
  
  \item $f_4(x) = \frac{x^4}{x^2-1}$ 
  
%  \medskip
  
%  \item $f_5(x) = x^n \ln x$ pour tout $n\ge0$
\end{enumerate}
\end{tp}

\end{frame}

\begin{frame}
\begin{enumerate}
  \item $\displaystyle\int x^4 + \frac{1}{x^2} + \exp(x)+ \cos(x) \; \dd x 
  \uncover<3->{= \frac15 x^5 - \frac1x + \exp(x) + \sin(x)}$
  
  \pause
  \codeinline{integral(x^4 + 1/x^2 + exp(x)+ cos(x),x)}
  
  \medskip
  \pause\pause
  
  
  \item $\displaystyle\int  x \sin(x^2) \; \dd x 
  \uncover<6->{= - \frac12\cos(x^2)}$
  
  \pause
  \codeinline{integral(x*sin(x^2),x)}
  
  \medskip
  \pause\pause
   
  \item $\displaystyle\int \tfrac{a}{\sqrt{1-x^2}} + \tfrac{b}{1+x^2} + \tfrac{c}{1+x} \; \dd x
  \uncover<9->{= a\arcsin(x) + b\arctan(x) + c\ln(x + 1)}$
  
  \pause
  \codeinline{var('x,a,b,c')}
  
  \codeinline{integral(a/sqrt(1-x^2) + b/(1+x^2) + c/(1+x),x)}
  
  \medskip
  \pause\pause
  
  \item $\displaystyle\int \frac{x^4}{x^2-1} \; \dd x 
  \uncover<14->{= \frac13 x^3 + x - \frac12 \ln(x + 1) + \frac12 \ln(x - 1)}$ 
  
  \pause
  \codeinline{f(x) = x^4/(x^2-1)}
  
  \codeinline{f.partial_fraction(x)} 
  
  \pause
  
  $f(x) = x^2 + 1 - \frac12\frac{1}{x + 1} + \frac12\frac{1}{x - 1}$ 
  
  \pause  
  
  \codeinline{integral(f(x),x)}
  
%  \medskip
%  
%  \item $\int x^n \ln x \; \dd x$ pour tout $n\ge0$
%  
%  \codeinline{var('x,n')}
%  
%  \codeinline{assume(n>0)}
%  
%  \codeinline{f(x) = x^n*ln(x)}
%  
%  \codeinline{integral(f(x),x)}  
\end{enumerate}

\end{frame}


\begin{frame}
\begin{tp}
Calculer à la main et à l'ordinateur les intégrales suivantes.
\begin{enumerate}
  \item $\displaystyle I_1 = \int_1^3 x^2 + \sqrt{x} + \frac1x + \sqrt[3]{x} \; \dd x$
  \item $\displaystyle I_2 = \int_0^\frac{\pi}{2} \sin x (1+\cos x)^4 \; \dd x$
  \item $\displaystyle I_3 = \int_0^{\frac\pi6} \sin^3 x \; \dd x$
  %\item $\displaystyle I_4 = \int_0^{\sqrt3} \frac{3x^3 - 2x^2 - 5x - 6}{(x + 1)^2(x^2 + 1)} \; \dd x$
\end{enumerate}
\end{tp}

\pause
\begin{enumerate}
\item \codeinline{integral(x^2 + sqrt(x) + 1/x + x^(1/3),x,1,3)} 

\pause

$I_1 = \frac94 3^{\frac13} + 2\sqrt3 + \ln 3 + \frac{29}{4}$

\pause

\item \codeinline{integral(sin(x)*(1+cos(x))^4,x,0,pi/2)} \quad $I_2 = \frac{31}{5}$

\pause

\item \codeinline{integral(sin(x)^3,x,0,pi/6)} \quad $I_3 = \frac23 - \frac{3\sqrt{3}}{8}$

\end{enumerate}
%f(x) = (3*x^3 - 2*x^2 - 5*x - 6)/((x^2 + 1)*(x + 1)^2)
%monint = integral(f(x),x,0,sqrt(3))
%print(monint)
%g(x) = f.partial_fraction(x)
%monintbis = integral(g(x),x,0,sqrt(3))
%print(monintbis)

\end{frame}

%%%%%%%%%%%%%%%%%%%%%%%%%%%%%%%%%%%%%%%%%%%%%%%%%%%%%%%%%%%%%%%%
\section{Intégration par parties}

\begin{frame}
\evidence{Intégration par parties}

\begin{theoreme}
Soient $u$ et $v$ deux fonctions de classe $\mathcal{C}^1$ sur un intervalle $[a,b]$.
$$\int_a^b u(x) \, v'(x)\;\dd x= \big[uv\big]_a^b - \int_a^b u'(x) \, v(x)\;\dd x$$
\end{theoreme} 

\end{frame}



\begin{frame}
\begin{tp}
Pour chaque $n\ge 0$, nous allons déterminer une primitive $F_n$ de la fonction 
$$f_n(x) = x^n \exp(x).$$

\begin{enumerate}
  \item \Sage\ connaît-il directement la formule ?
  
  \item Calculer une primitive $F_n$ de $f_n$ pour les premières valeurs de $n$.
  
   \item 
   \begin{enumerate}
     \item \'Emettre une conjecture reliant $F_n$ et $F_{n-1}$.
     \item Prouver cette conjecture.
     \item En déduire une fonction récursive qui calcule $F_n$.
   \end{enumerate}
  
  \item 
   \begin{enumerate}
     \item \'Emettre une conjecture pour une formule directe de $F_n$.
     \item Prouver cette conjecture.
     \item En déduire une expression qui calcule $F_n$.
   \end{enumerate}  
\end{enumerate} 
\end{tp}
\end{frame}



\begin{frame}[fragile]
\vspace*{-1ex}

$$F_n = \int x^n \exp(x) \; \dd x$$

\vspace*{-1ex}
\pause

\begin{enumerate}
  \item \Sage\ ne connaît pas la formule !
  
\pause  
  
  \item Premières primitives $F_n$
\end{enumerate}

\vspace*{-1ex}
\pause

{\small
\begin{algo}[integrale-ipp.sage (1)]
\begin{lstlisting}
var('x,n')
f = x^n*exp(x)|\vspace*{-0.5ex}|
   
def F(n):
    return integral(x^n*exp(x),x)|\vspace*{-0.5ex}|
    
for n in range(0,10):
    print F(n)
\end{lstlisting}
\end{algo}

\pause
\vspace*{-2ex}

  $$\begin{array}{rcl}
    F_0(x) &=& \exp(x) \\
    F_1(x) &=& (x - 1)\exp(x) \\
    F_2(x) &=& (x^2 - 2x + 2)\exp(x) \\
    F_3(x) &=& (x^3 - 3x^2 + 6x - 6)\exp(x) \\
    F_4(x) &=& (x^4 - 4x^3 + 12x^2 - 24x + 24)\exp(x) \\
  %  F_5(x) &=& (x^5 - 5x^4 + 20x^3 - 60x^2 + 120x - 120)\exp(x) \\
  \end{array}$$ 
}  
\end{frame}



\begin{frame}[fragile]
\vspace*{-1ex}

$$F_n = \int x^n \exp(x) \; \dd x$$

\vspace*{-1ex}
\pause 


\begin{enumerate}
\setcounter{enumi}{2}
  \item
    \begin{itemize}
      \item $F_n(x) = x^n \exp(x) - nF_{n-1}(x)$
    \pause  
      
      \item \Sage\ ne sait pas (encore) vérifier formellement cette identité 
    (avec $n$ une variable formelle)
    
    \pause
    
      \item Preuve par intégration par parties 
    ($u=x^n$, $v'=\exp(x)$ donc $u'=nx^{n-1}$, $v = \exp(x)$)
    
    
    \vspace*{-0.5ex}
    \pause
    
    $$\begin{array}{rcl}
    F_n(x) 
    & = & \int x^n \exp(x)\; \dd x \\[1mm]
    \pause 
    & = & \big[x^n\exp(x)\big] - \int nx^{n-1}\exp(x)\; \dd x\\[1mm]
    \pause
    & = & x^n\exp(x) - n F_{n-1}(x)
    \end{array}$$
    
    \vspace*{-1.5ex}
    
    \end{itemize}
\end{enumerate}


\pause

\begin{algo}[integrale-ipp.sage (2)]
\begin{lstlisting}
def FF(n):
    if n==0:
        return exp(x)
    else:
        return x^n*exp(x) - n*FF(n-1)
\end{lstlisting}
\end{algo}

\end{frame}


 
\begin{frame}[fragile]

\vspace*{-1ex}

$$F_n = \int x^n \exp(x) \; \dd x$$

\vspace*{-1ex}

\begin{enumerate}
\setcounter{enumi}{3}
  \item $\displaystyle F_n(x) = \exp(x)\sum_{k=0}^n (-1)^{n-k} \frac{n!}{k!} x^k$ ?
  
  \pause
  \vspace*{2ex} 
  
   \begin{itemize}
      \item ~ 
      
    \vspace*{-5ex}    
        
    $\begin{array}{rcl} 
    G_n(x) 
    &=& \displaystyle \exp(x)\sum_{k=0}^{n} (-1)^{n-k} \frac{n!}{k!} x^k\\
    \pause
    &=& \displaystyle \exp(x)\sum_{k=0}^{n-1} (-1)^{n-k} \frac{n!}{k!} x^k  + \exp(x)x^n \\
   \pause
    &=& -n G_{n-1}(x)+ \exp(x)x^n       
    \end{array}$
    
    \pause
    \item $G_n$ vérifie la même relation de récurrence que $F_n$,
    de plus $G_0(x)=\exp x = F_0(x)$, donc pour tout $n$, $G_n=F_n$
    \vspace*{-1ex}
    
    \end{itemize}
\end{enumerate}

\medskip 
\pause
 
\begin{algo}[integrale-ipp.sage (3)]
\begin{lstlisting}
def FFF(n): 
    return exp(x)*sum((-1)^(n-k)*x^k*factorial(n)/factorial(k),k,0,n)
\end{lstlisting}
\end{algo}
\end{frame}


%%%%%%%%%%%%%%%%%%%%%%%%%%%%%%%%%%%%%%%%%%%%%%%%%%%%%%%%%%%%%%%%
\section{Changement de variable}

\begin{frame}

\evidence{Changement de variable}

\begin{theoreme}
Soit $f$ une fonction définie sur un intervalle $I$ et $\varphi : J \to I$ 
une bijection de classe $\mathcal{C}^1$. Pour tout $a,b\in I$,

$$\int_a^b f(x) \; \dd x 
= \int_{\varphi^{-1}(a)}^{\varphi^{-1}(b)} f\big(\varphi(u)\big)\cdot\varphi'(u) \; \dd u$$
\end{theoreme}

\end{frame}


\begin{frame}
\begin{tp}
\begin{enumerate}
  \item Calcul de la primitive $\displaystyle\int \sqrt{1-x^2} \; \dd x$.
  \begin{enumerate}
    \item Poser $f(x) = \sqrt{1-x^2}$ et le changement de variable $x = \sin u$, c'est-à-dire on pose 
    $\varphi(u) = \sin u$.
    \item Calculer $f\big( \varphi(u) \big)$ et $\varphi'(u)$.
    \item Calculer la primitive de $g(u) = f\big(\varphi(u)\big)\cdot\varphi'(u)$.
    \item En déduire une primitive de $f(x) = \sqrt{1-x^2}$.
  \end{enumerate}
  
  % \item Calcul de la primitive $\displaystyle\int \frac{\dd x}{1+\left(\frac{x+1}{x}\right)^{1/3}}$.
  % \begin{enumerate}  
    % \item Poser le changement de variable défini par l'équation $u^3 = \frac{x+1}{x}$.
    
    % \item En déduire le changement de variable $\varphi(u)$ et $\varphi^{-1}(x)$.
  % \end{enumerate}
  
  
  % \item \'Ecrire une function \codeinline{integrale_chgtvar(f,eqn)} qui calcule
  % une primitive de $f(x)$ par le changement de variable défini par l'équation 
  % \codeinline{eqn} reliant $u$ et $x$.
  
  % En déduire $\displaystyle\int (\cos x + 1)^n \cdot \sin x \; \dd x$, en posant $u= \cos x + 1$.
  
  % \item Même travail avec \codeinline{integrale_chgtvar_bornes(f,a,b,eqn)}
  % qui calcule l'intégrale de $f(x)$ entre $a$ et $b$ par changement de variable.
  
  % En déduire $\displaystyle\int_0^{\frac\pi4} \frac{\dd x}{2+\sin^2 x}$, en posant $u = \tan x$.
\end{enumerate}
\end{tp}

\end{frame}

\begin{frame}[fragile]

\small 

  \vspace*{-1ex} 
  
$$\displaystyle\int \sqrt{1-x^2} \; \dd x$$

  \vspace*{-3ex} 
  
  \begin{enumerate}
    \item 
%  \begin{enumerate}
%    \item ~
%  \end{enumerate}
  
  \end{enumerate}    
    
    \begin{minipage}{0.59\textwidth}
    $f(x) = \sqrt{1-x^2}$
    
    $x = \sin u$ 
       
    $\varphi(u) = \sin u$
    \end{minipage}
    \begin{minipage}{0.33\textwidth}
    \codeinline{var('x,u')}
    
    \codeinline{f = sqrt(1-x^2)}
    
    \codeinline{phi = sin(u)}
    \end{minipage} 
    
    \bigskip
    \pause
   
    %\item 
    \begin{minipage}{0.59\textwidth}    
    $f\big( \varphi(u) \big) = \sqrt{1-\sin^2 u}  = |\cos u|$
    
    $\varphi'(u) = \cos u$
    \end{minipage} 
    \begin{minipage}{0.33\textwidth}
    \codeinline{frondphi = f(x=phi)}

    \codeinline{dphi = diff(phi,u)}    
    \end{minipage}
    
    \bigskip         
    \pause
     
    %\item 
    \begin{minipage}{0.59\textwidth}    
    $g(u) = f\big(\varphi(u)\big)\cdot\varphi'(u) = \cos^2 u = \frac{1+\cos(2u)}{2}$
    
    $\int g(u)\; \dd u = \frac{u}{2} + \frac{\sin(2u)}{4}$
    \end{minipage} 
    \begin{minipage}{0.33\textwidth}
    \codeinline{g = frondphi*dphi}
    
    %\codeinline{g = g.simplify_full().simplify_radical()}
    
    \codeinline{G = integral(g,u)}
    \end{minipage}
    
    
    \bigskip 
    \pause
             
    %\item 
    \begin{minipage}{0.59\textwidth}
    $\begin{array}{rcl}
    \int \sqrt{1-x^2} \; \dd x &=& \frac{\arcsin x}{2} + \frac{\sin(2\arcsin x)}{4}\\
    
    
    &=& \frac{\arcsin x}{2} + \frac x2 \sqrt{1-x^2}\end{array}$
    \end{minipage} 
    \begin{minipage}{0.33\textwidth}
    \codeinline{F = G(u = arcsin(x))}
    \end{minipage}
    
   


\end{frame}

\begin{frame}
\begin{tp}
\begin{enumerate}
		  \setcounter{enumi}{1}
  % \item Calcul de la primitive $\displaystyle\int \sqrt{1-x^2} \; \dd x$.
  % \begin{enumerate}
    % \item Poser $f(x) = \sqrt{1-x^2}$ et le changement de variable $x = \sin u$, c'est-à-dire on pose 
    % $\varphi(u) = \sin u$.
    % \item Calculer $f\big( \varphi(u) \big)$ et $\varphi'(u)$.
    % \item Calculer la primitive de $g(u) = f\big(\varphi(u)\big)\cdot\varphi'(u)$.
    % \item En déduire une primitive de $f(x) = \sqrt{1-x^2}$.
  % \end{enumerate}
  
  \item Calcul de la primitive $\displaystyle\int \frac{\dd x}{1+\left(\frac{x+1}{x}\right)^{1/3}}$.
  \begin{enumerate}  
    \item Poser le changement de variable défini par l'équation $u^3 = \frac{x+1}{x}$.
    
    \item En déduire le changement de variable $\varphi(u)$ et $\varphi^{-1}(x)$.
  \end{enumerate}
  
  
  \item \'Ecrire une fonction \codeinline{integrale_chgtvar(f,eqn)} qui calcule
   une primitive de $f(x)$ par le changement de variable défini par l'équation 
   \codeinline{eqn} reliant $u$ et $x$.
  
  En déduire $\displaystyle\int (\cos x + 1)^n \cdot \sin x \; \dd x$, en posant $u= \cos x + 1$.
  
  \item Même travail avec \codeinline{integrale_chgtvar_bornes(f,a,b,eqn)}
   qui calcule l'intégrale de $f(x)$ entre $a$ et $b$ par changement de variable.
  
  En déduire $\displaystyle\int_0^{\frac\pi4} \frac{\dd x}{2+\sin^2 x}$, en posant $u = \tan x$.
\end{enumerate}
\end{tp}

\end{frame}

\begin{frame}[fragile]




\begin{algo}[integrales-chgtvar.sage]
\begin{lstlisting}
f = 1/(1+((x+1)/x)^(1/3))
eqn = u^3 == (x+1)/x
phi = solve(eqn,x)[0].rhs()      # x en fonction de u 
phi_inv = solve(eqn,u)[0].rhs()  # u en fonction de x
frondphi = f(x=phi)
dphi = diff(phi,u)
g = frondphi*dphi
g = g.simplify_full()
G = integrate(g,u)
F = G(u = phi_inv)
F = F.simplify_full()
\end{lstlisting}
\end{algo}
\end{frame}


%
%
%\begin{frame}
%\begin{tp}
%\begin{enumerate}
%		  \setcounter{enumi}{2}
%  % \item Calcul de la primitive $\displaystyle\int \sqrt{1-x^2} \; \dd x$.
%  % \begin{enumerate}
%    % \item Poser $f(x) = \sqrt{1-x^2}$ et le changement de variable $x = \sin u$, c'est-à-dire on pose 
%    % $\varphi(u) = \sin u$.
%    % \item Calculer $f\big( \varphi(u) \big)$ et $\varphi'(u)$.
%    % \item Calculer la primitive de $g(u) = f\big(\varphi(u)\big)\cdot\varphi'(u)$.
%    % \item En déduire une primitive de $f(x) = \sqrt{1-x^2}$.
%  % \end{enumerate}
%  
%  % \item Calcul de la primitive $\displaystyle\int \frac{\dd x}{1+\left(\frac{x+1}{x}\right)^{1/3}}$.
%  % \begin{enumerate}  
%    % \item Poser le changement de variable défini par l'équation $u^3 = \frac{x+1}{x}$.
%    
%    % \item En déduire le changement de variable $\varphi(u)$ et $\varphi^{-1}(x)$.
%  % \end{enumerate}
%  
%  
%  \item \'Ecrire une fonction \codeinline{integrale_chgtvar(f,eqn)} qui calcule
%  une primitive de $f(x)$ par le changement de variable défini par l'équation 
%  \codeinline{eqn} reliant $u$ et $x$.
%  
%  En déduire $\displaystyle\int (\cos x + 1)^n \cdot \sin x \; \dd x$, en posant $u= \cos x + 1$.
%  
%  % \item Même travail avec \codeinline{integrale_chgtvar_bornes(f,a,b,eqn)}
%  % qui calcule l'intégrale de $f(x)$ entre $a$ et $b$ par changement de variable.
%  
%%  En déduire $\displaystyle\int_0^{\frac\pi4} \frac{\dd x}{2+\sin^2 x}$, en posant $u = \tan x$.
%\end{enumerate}
%\end{tp}
%
%\end{frame}
%
%
%\begin{frame}[fragile]
%\begin{algo}[integrales-chgtvar.sage (3)]
%\begin{lstlisting}
%def integrale_chgtvar(f,eqn):
%    phi = solve(eqn,x)[0].rhs()     # x en fonction de u 
%    phi_inv = solve(eqn,u)[0].rhs() # u en fonction de x 
%    frondphi = f(x=phi)             
%    dphi = diff(phi,u)              
%    g = frondphi*dphi               
%    g = g.simplify_full()
%    G = integral(g,u)               
%    F = G(u = phi_inv)             
%    F = F.simplify_full()
%    return F
%f = (cos(x)+1)^n*sin(x)
%eqn = u == cos(x)+1
%print(integrale_chgtvar(f,eqn))
%\end{lstlisting}
%\end{algo}
%\end{frame}
%
%
%
%
%\begin{frame}
%\begin{tp}
%\begin{enumerate}
%		  \setcounter{enumi}{3}
%  % \item Calcul de la primitive $\displaystyle\int \sqrt{1-x^2} \; \dd x$.
%  % \begin{enumerate}
%    % \item Poser $f(x) = \sqrt{1-x^2}$ et le changement de variable $x = \sin u$, c'est-à-dire on pose 
%    % $\varphi(u) = \sin u$.
%    % \item Calculer $f\big( \varphi(u) \big)$ et $\varphi'(u)$.
%    % \item Calculer la primitive de $g(u) = f\big(\varphi(u)\big)\cdot\varphi'(u)$.
%    % \item En déduire une primitive de $f(x) = \sqrt{1-x^2}$.
%  % \end{enumerate}
%  
%  % \item Calcul de la primitive $\displaystyle\int \frac{\dd x}{1+\left(\frac{x+1}{x}\right)^{1/3}}$.
%  % \begin{enumerate}  
%    % \item Poser le changement de variable défini par l'équation $u^3 = \frac{x+1}{x}$.
%    
%    % \item En déduire le changement de variable $\varphi(u)$ et $\varphi^{-1}(x)$.
%  % \end{enumerate}
%  
%  
%  % \item \'Ecrire une function \codeinline{integrale_chgtvar(f,eqn)} qui calcule
%  % une primitive de $f(x)$ par le changement de variable défini par l'équation 
%  % \codeinline{eqn} reliant $u$ et $x$.
%  
%  % En déduire $\displaystyle\int (\cos x + 1)^n \cdot \sin x \; \dd x$, en posant $u= \cos x + 1$.
%  
%  \item Même travail avec \codeinline{integrale_chgtvar_bornes(f,a,b,eqn)}
%  qui calcule l'intégrale de $f(x)$ entre $a$ et $b$ par changement de variable.
%  
%  En déduire $\displaystyle\int_0^{\frac\pi4} \frac{\dd x}{2+\sin^2 x}$, en posant $u = \tan x$.
%\end{enumerate}
%\end{tp}
%
%\end{frame}
%%
%%\begin{frame}[fragile]
%%\begin{algo}[integrales-chgtvar.sage (4)]
%%\begin{lstlisting}
%%def integrale_chgtvar_bornes(f,a,b,eqn):
%%    phi = solve(eqn,x)[0].rhs()     # x en fonction de u 
%%    phi_inv = solve(eqn,u)[0].rhs() # u en fonction de x 
%%    frondphi = f(x=phi)             
%%    dphi = diff(phi,u)           
%%    g = frondphi*dphi            
%%    g = g.simplify_full()
%%    a_inv = phi_inv(x=a)           
%%    b_inv = phi_inv(x=b)
%%    new_I = integral(g,u,a_inv,b_inv) 
%%    return new_I
%%f = 1/(2+sin(x)^2)
%%eqn = u == tan(x)
%%a = 0
%%b = pi/4
%%print 'Nous : ', integrale_chgtvar_bornes(f,a,b,eqn)   
%%\end{lstlisting}
%%\end{algo}
%%\end{frame}
%


\end{document}

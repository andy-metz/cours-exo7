
%%%%%%%%%%%%%%%%%% PREAMBULE %%%%%%%%%%%%%%%%%%

\documentclass[aspectratio=169,utf8]{beamer}
%\documentclass[aspectratio=169,handout]{beamer}

\usetheme{Boadilla}
%\usecolortheme{seahorse}
%\usecolortheme[RGB={245,66,24}]{structure}
\useoutertheme{infolines}

% packages
\usepackage{amsfonts,amsmath,amssymb,amsthm}
\usepackage[utf8]{inputenc}
\usepackage[T1]{fontenc}
\usepackage{lmodern}

\usepackage[francais]{babel}
\usepackage{fancybox}
\usepackage{graphicx}

\usepackage{float}
\usepackage{xfrac}

%\usepackage[usenames, x11names]{xcolor}
\usepackage{pgfplots}
\usepackage{datetime}


% ----------------------------------------------------------------------
% Pour les images
\usepackage{tikz}
\usetikzlibrary{calc,shadows,arrows.meta,patterns,matrix}

\newcommand{\tikzinput}[1]{\input{figures/#1.tikz}}
% --- les figures avec échelle éventuel
\newcommand{\myfigure}[2]{% entrée : échelle, fichier(s) figure à inclure
\begin{center}\small%
\tikzstyle{every picture}=[scale=1.0*#1]% mise en échelle + 0% (automatiquement annulé à la fin du groupe)
#2%
\end{center}}



%-----  Package unités -----
\usepackage{siunitx}
\sisetup{locale = FR,detect-all,per-mode = symbol}

%\usepackage{mathptmx}
%\usepackage{fouriernc}
%\usepackage{newcent}
%\usepackage[mathcal,mathbf]{euler}

%\usepackage{palatino}
%\usepackage{newcent}
% \usepackage[mathcal,mathbf]{euler}



% \usepackage{hyperref}
% \hypersetup{colorlinks=true, linkcolor=blue, urlcolor=blue,
% pdftitle={Exo7 - Exercices de mathématiques}, pdfauthor={Exo7}}


%section
% \usepackage{sectsty}
% \allsectionsfont{\bf}
%\sectionfont{\color{Tomato3}\upshape\selectfont}
%\subsectionfont{\color{Tomato4}\upshape\selectfont}

%----- Ensembles : entiers, reels, complexes -----
\newcommand{\Nn}{\mathbb{N}} \newcommand{\N}{\mathbb{N}}
\newcommand{\Zz}{\mathbb{Z}} \newcommand{\Z}{\mathbb{Z}}
\newcommand{\Qq}{\mathbb{Q}} \newcommand{\Q}{\mathbb{Q}}
\newcommand{\Rr}{\mathbb{R}} \newcommand{\R}{\mathbb{R}}
\newcommand{\Cc}{\mathbb{C}} 
\newcommand{\Kk}{\mathbb{K}} \newcommand{\K}{\mathbb{K}}

%----- Modifications de symboles -----
\renewcommand{\epsilon}{\varepsilon}
\renewcommand{\Re}{\mathop{\text{Re}}\nolimits}
\renewcommand{\Im}{\mathop{\text{Im}}\nolimits}
%\newcommand{\llbracket}{\left[\kern-0.15em\left[}
%\newcommand{\rrbracket}{\right]\kern-0.15em\right]}

\renewcommand{\ge}{\geqslant}
\renewcommand{\geq}{\geqslant}
\renewcommand{\le}{\leqslant}
\renewcommand{\leq}{\leqslant}
\renewcommand{\epsilon}{\varepsilon}

%----- Fonctions usuelles -----
\newcommand{\ch}{\mathop{\text{ch}}\nolimits}
\newcommand{\sh}{\mathop{\text{sh}}\nolimits}
\renewcommand{\tanh}{\mathop{\text{th}}\nolimits}
\newcommand{\cotan}{\mathop{\text{cotan}}\nolimits}
\newcommand{\Arcsin}{\mathop{\text{arcsin}}\nolimits}
\newcommand{\Arccos}{\mathop{\text{arccos}}\nolimits}
\newcommand{\Arctan}{\mathop{\text{arctan}}\nolimits}
\newcommand{\Argsh}{\mathop{\text{argsh}}\nolimits}
\newcommand{\Argch}{\mathop{\text{argch}}\nolimits}
\newcommand{\Argth}{\mathop{\text{argth}}\nolimits}
\newcommand{\pgcd}{\mathop{\text{pgcd}}\nolimits} 


%----- Commandes divers ------
\newcommand{\ii}{\mathrm{i}}
\newcommand{\dd}{\text{d}}
\newcommand{\id}{\mathop{\text{id}}\nolimits}
\newcommand{\Ker}{\mathop{\text{Ker}}\nolimits}
\newcommand{\Card}{\mathop{\text{Card}}\nolimits}
\newcommand{\Vect}{\mathop{\text{Vect}}\nolimits}
\newcommand{\Mat}{\mathop{\text{Mat}}\nolimits}
\newcommand{\rg}{\mathop{\text{rg}}\nolimits}
\newcommand{\tr}{\mathop{\text{tr}}\nolimits}


%----- Structure des exercices ------

\newtheoremstyle{styleexo}% name
{2ex}% Space above
{3ex}% Space below
{}% Body font
{}% Indent amount 1
{\bfseries} % Theorem head font
{}% Punctuation after theorem head
{\newline}% Space after theorem head 2
{}% Theorem head spec (can be left empty, meaning ‘normal’)

%\theoremstyle{styleexo}
\newtheorem{exo}{Exercice}
\newtheorem{ind}{Indications}
\newtheorem{cor}{Correction}


\newcommand{\exercice}[1]{} \newcommand{\finexercice}{}
%\newcommand{\exercice}[1]{{\tiny\texttt{#1}}\vspace{-2ex}} % pour afficher le numero absolu, l'auteur...
\newcommand{\enonce}{\begin{exo}} \newcommand{\finenonce}{\end{exo}}
\newcommand{\indication}{\begin{ind}} \newcommand{\finindication}{\end{ind}}
\newcommand{\correction}{\begin{cor}} \newcommand{\fincorrection}{\end{cor}}

\newcommand{\noindication}{\stepcounter{ind}}
\newcommand{\nocorrection}{\stepcounter{cor}}

\newcommand{\fiche}[1]{} \newcommand{\finfiche}{}
\newcommand{\titre}[1]{\centerline{\large \bf #1}}
\newcommand{\addcommand}[1]{}
\newcommand{\video}[1]{}

% Marge
\newcommand{\mymargin}[1]{\marginpar{{\small #1}}}

\def\noqed{\renewcommand{\qedsymbol}{}}


%----- Presentation ------
\setlength{\parindent}{0cm}

%\newcommand{\ExoSept}{\href{http://exo7.emath.fr}{\textbf{\textsf{Exo7}}}}

\definecolor{myred}{rgb}{0.93,0.26,0}
\definecolor{myorange}{rgb}{0.97,0.58,0}
\definecolor{myyellow}{rgb}{1,0.86,0}

\newcommand{\LogoExoSept}[1]{  % input : echelle
{\usefont{U}{cmss}{bx}{n}
\begin{tikzpicture}[scale=0.1*#1,transform shape]
  \fill[color=myorange] (0,0)--(4,0)--(4,-4)--(0,-4)--cycle;
  \fill[color=myred] (0,0)--(0,3)--(-3,3)--(-3,0)--cycle;
  \fill[color=myyellow] (4,0)--(7,4)--(3,7)--(0,3)--cycle;
  \node[scale=5] at (3.5,3.5) {Exo7};
\end{tikzpicture}}
}


\newcommand{\debutmontitre}{
  \author{} \date{} 
  \thispagestyle{empty}
  \hspace*{-10ex}
  \begin{minipage}{\textwidth}
    \titlepage  
  \vspace*{-2.5cm}
  \begin{center}
    \LogoExoSept{2.5}
  \end{center}
  \end{minipage}

  \vspace*{-0cm}
  
  % Astuce pour que le background ne soit pas discrétisé lors de la conversion pdf -> png
\begin{tikzpicture}
        \fill[opacity=0,green!60!black] (0,0)--++(0,0)--++(0,0)--++(0,0)--cycle; 
\end{tikzpicture}

% toc S'affiche trop tot :
% \tableofcontents[hideallsubsections, pausesections]
}

\newcommand{\finmontitre}{
  \end{frame}
  \setcounter{framenumber}{0}
} % ne marche pas pour une raison obscure

%----- Commandes supplementaires ------

% \usepackage[landscape]{geometry}
% \geometry{top=1cm, bottom=3cm, left=2cm, right=10cm, marginparsep=1cm
% }
% \usepackage[a4paper]{geometry}
% \geometry{top=2cm, bottom=2cm, left=2cm, right=2cm, marginparsep=1cm
% }

%\usepackage{standalone}


% New command Arnaud -- november 2011
\setbeamersize{text margin left=24ex}
% si vous modifier cette valeur il faut aussi
% modifier le decalage du titre pour compenser
% (ex : ici =+10ex, titre =-5ex

\theoremstyle{definition}
%\newtheorem{proposition}{Proposition}
%\newtheorem{exemple}{Exemple}
%\newtheorem{theoreme}{Théorème}
%\newtheorem{lemme}{Lemme}
%\newtheorem{corollaire}{Corollaire}
%\newtheorem*{remarque*}{Remarque}
%\newtheorem*{miniexercice}{Mini-exercices}
%\newtheorem{definition}{Définition}

% Commande tikz
\usetikzlibrary{calc}
\usetikzlibrary{patterns,arrows}
\usetikzlibrary{matrix}
\usetikzlibrary{fadings} 

%definition d'un terme
\newcommand{\defi}[1]{{\color{myorange}\textbf{\emph{#1}}}}
\newcommand{\evidence}[1]{{\color{blue}\textbf{\emph{#1}}}}
\newcommand{\assertion}[1]{\emph{\og#1\fg}}  % pour chapitre logique
%\renewcommand{\contentsname}{Sommaire}
\renewcommand{\contentsname}{}
\setcounter{tocdepth}{2}



%------ Encadrement ------

\usepackage{fancybox}


\newcommand{\mybox}[1]{
\setlength{\fboxsep}{7pt}
\begin{center}
\shadowbox{#1}
\end{center}}

\newcommand{\myboxinline}[1]{
\setlength{\fboxsep}{5pt}
\raisebox{-10pt}{
\shadowbox{#1}
}
}

%--------------- Commande beamer---------------
\newcommand{\beameronly}[1]{#1} % permet de mettre des pause dans beamer pas dans poly


\setbeamertemplate{navigation symbols}{}
\setbeamertemplate{footline}  % tiré du fichier beamerouterinfolines.sty
{
  \leavevmode%
  \hbox{%
  \begin{beamercolorbox}[wd=.333333\paperwidth,ht=2.25ex,dp=1ex,center]{author in head/foot}%
    % \usebeamerfont{author in head/foot}\insertshortauthor%~~(\insertshortinstitute)
    \usebeamerfont{section in head/foot}{\bf\insertshorttitle}
  \end{beamercolorbox}%
  \begin{beamercolorbox}[wd=.333333\paperwidth,ht=2.25ex,dp=1ex,center]{title in head/foot}%
    \usebeamerfont{section in head/foot}{\bf\insertsectionhead}
  \end{beamercolorbox}%
  \begin{beamercolorbox}[wd=.333333\paperwidth,ht=2.25ex,dp=1ex,right]{date in head/foot}%
    % \usebeamerfont{date in head/foot}\insertshortdate{}\hspace*{2em}
    \insertframenumber{} / \inserttotalframenumber\hspace*{2ex} 
  \end{beamercolorbox}}%
  \vskip0pt%
}


\definecolor{mygrey}{rgb}{0.5,0.5,0.5}
\setlength{\parindent}{0cm}
%\DeclareTextFontCommand{\helvetica}{\fontfamily{phv}\selectfont}

% background beamer
\definecolor{couleurhaut}{rgb}{0.85,0.9,1}  % creme
\definecolor{couleurmilieu}{rgb}{1,1,1}  % vert pale
\definecolor{couleurbas}{rgb}{0.85,0.9,1}  % blanc
\setbeamertemplate{background canvas}[vertical shading]%
[top=couleurhaut,middle=couleurmilieu,midpoint=0.4,bottom=couleurbas] 
%[top=fondtitre!05,bottom=fondtitre!60]



\makeatletter
\setbeamertemplate{theorem begin}
{%
  \begin{\inserttheoremblockenv}
  {%
    \inserttheoremheadfont
    \inserttheoremname
    \inserttheoremnumber
    \ifx\inserttheoremaddition\@empty\else\ (\inserttheoremaddition)\fi%
    \inserttheorempunctuation
  }%
}
\setbeamertemplate{theorem end}{\end{\inserttheoremblockenv}}

\newenvironment{theoreme}[1][]{%
   \setbeamercolor{block title}{fg=structure,bg=structure!40}
   \setbeamercolor{block body}{fg=black,bg=structure!10}
   \begin{block}{{\bf Th\'eor\`eme }#1}
}{%
   \end{block}%
}


\newenvironment{proposition}[1][]{%
   \setbeamercolor{block title}{fg=structure,bg=structure!40}
   \setbeamercolor{block body}{fg=black,bg=structure!10}
   \begin{block}{{\bf Proposition }#1}
}{%
   \end{block}%
}

\newenvironment{corollaire}[1][]{%
   \setbeamercolor{block title}{fg=structure,bg=structure!40}
   \setbeamercolor{block body}{fg=black,bg=structure!10}
   \begin{block}{{\bf Corollaire }#1}
}{%
   \end{block}%
}

\newenvironment{mydefinition}[1][]{%
   \setbeamercolor{block title}{fg=structure,bg=structure!40}
   \setbeamercolor{block body}{fg=black,bg=structure!10}
   \begin{block}{{\bf Définition} #1}
}{%
   \end{block}%
}

\newenvironment{lemme}[0]{%
   \setbeamercolor{block title}{fg=structure,bg=structure!40}
   \setbeamercolor{block body}{fg=black,bg=structure!10}
   \begin{block}{\bf Lemme}
}{%
   \end{block}%
}

\newenvironment{remarque}[1][]{%
   \setbeamercolor{block title}{fg=black,bg=structure!20}
   \setbeamercolor{block body}{fg=black,bg=structure!5}
   \begin{block}{Remarque #1}
}{%
   \end{block}%
}


\newenvironment{exemple}[1][]{%
   \setbeamercolor{block title}{fg=black,bg=structure!20}
   \setbeamercolor{block body}{fg=black,bg=structure!5}
   \begin{block}{{\bf Exemple }#1}
}{%
   \end{block}%
}


\newenvironment{miniexercice}[0]{%
   \setbeamercolor{block title}{fg=structure,bg=structure!20}
   \setbeamercolor{block body}{fg=black,bg=structure!5}
   \begin{block}{Mini-exercices}
}{%
   \end{block}%
}


\newenvironment{tp}[0]{%
   \setbeamercolor{block title}{fg=structure,bg=structure!40}
   \setbeamercolor{block body}{fg=black,bg=structure!10}
   \begin{block}{\bf Travaux pratiques}
}{%
   \end{block}%
}
\newenvironment{exercicecours}[1][]{%
   \setbeamercolor{block title}{fg=structure,bg=structure!40}
   \setbeamercolor{block body}{fg=black,bg=structure!10}
   \begin{block}{{\bf Exercice }#1}
}{%
   \end{block}%
}
\newenvironment{algo}[1][]{%
   \setbeamercolor{block title}{fg=structure,bg=structure!40}
   \setbeamercolor{block body}{fg=black,bg=structure!10}
   \begin{block}{{\bf Algorithme}\hfill{\color{gray}\texttt{#1}}}
}{%
   \end{block}%
}


\setbeamertemplate{proof begin}{
   \setbeamercolor{block title}{fg=black,bg=structure!20}
   \setbeamercolor{block body}{fg=black,bg=structure!5}
   \begin{block}{{\footnotesize Démonstration}}
   \footnotesize
   \smallskip}
\setbeamertemplate{proof end}{%
   \end{block}}
\setbeamertemplate{qed symbol}{\openbox}


\makeatother
\usecolortheme[RGB={56,98,238}]{structure}
 
% Commande spécifique à ce chapitre
\definecolor{Green4}{rgb}{0.28,0.67,0.10}
   
%%%%%%%%%%%%%%%%%%%%%%%%%%%%%%%%%%%%%%%%%%%%%%%%%%%%%%%%%%%%%
%%%%%%%%%%%%%%%%%%%%%%%%%%%%%%%%%%%%%%%%%%%%%%%%%%%%%%%%%%%%%


\begin{document}


\title{{\bf Matrices et applications linéaires}}
\subtitle{Matrice d'une application linéaire}

\begin{frame}
  
  \debutmontitre

  \pause

{\footnotesize
\hfill
\setbeamercovered{transparent=50}
\begin{minipage}{0.6\textwidth}
  \begin{itemize}
    \item<3-> Matrice associée à une application linéaire
    \item<4-> Opérations sur les applications linéaires \\ et les matrices
    \item<5-> Matrice d'un endomorphisme
    \item<6-> Matrice d'un isomorphisme
  \end{itemize}
\end{minipage}
}

\end{frame}

\setcounter{framenumber}{0}


%%%%%%%%%%%%%%%%%%%%%%%%%%%%%%%%%%%%%%%%%%%%%%%%%%%%%%%%%%%%%%%%
\section{Matrice associée à une application linéaire}

\begin{frame}

\begin{itemize}
  \item $E$ et $F$ deux $\Kk$-espaces vectoriels de dimension finie
  \pause
  \item $\mathcal{B}=(e_1, \dots ,e_p)$ une base de $E$
  \pause
  \item $\mathcal{B}'=(f_1, \dots ,f_n)$ une base de $F$
  \pause
  \item $f : E \to F$ une application linéaire
\end{itemize}

\bigskip
\pause

\begin{enumerate}
  \item $f$ est déterminée de façon unique par l'image
d'une base de $E$, donc par les vecteurs 
$f(e_1), f(e_2), \ldots, f(e_p)$
\pause
  
  \item $f(e_j)$ se décompose de manière unique dans la base $\mathcal{B}'$
  \pause
  
  \item Il existe $a_{1,j},a_{2,j}, \ldots , a_{n,j} \in \Kk$
tels que 
\vspace*{-2ex}
$$f (e_j)=a_{1,j}f_1+a_{2,j}f_2+\dots +a_{n,j}f_n 
\pause = \left(\begin{smallmatrix}a_{1,j}\\a_{2,j}\\ \vdots \\a_{n,j}\end{smallmatrix}\right)_{\mathcal{B}'}$$

\end{enumerate}

\end{frame}


\begin{frame}  

\begin{itemize}
  \item $f : E \to F$ une application linéaire
  
  \item $\mathcal{B}=(e_1, \dots ,e_p)$ une base de $E$, $\mathcal{B}'=(f_1, \dots ,f_n)$ une base de $F$

  \item $f (e_j)=a_{1,j}f_1+a_{2,j}f_2+\dots +a_{n,j}f_n 
= \left(\begin{smallmatrix}a_{1,j}\\a_{2,j}\\ \vdots \\a_{n,j}\end{smallmatrix}\right)_{\mathcal{B}'}$

%   \item L'application linéaire $f$ est entièrement déterminée par les
% coefficients $(a_{i,j})_{(i,j) \in \{1,\ldots,n\} \times \{1,\ldots,p\}}$
\end{itemize}
\vspace*{-1.5ex}
\pause
\begin{mydefinition}
La \defi{matrice} de $f$ 
par rapport aux bases ${\color{blue}\mathcal{B}}$ et ${\color{Green4}\mathcal{B}'}$ est 
la matrice $(a_{i,j}) \in M_{n,p}(\Kk)$
\vspace*{-4ex}
\pause
$$\Mat_{{\color{blue}\mathcal{B}},{\color{Green4}\mathcal{B}'}}(f) = 
 \bordermatrix{    
                                  & \uncover<4->{f({\color{blue}e_1})}& \uncover<7->{\ldots} & \uncover<7->{f({\color{blue}e_j})}  &\uncover<9->{\ldots}  & \uncover<9->{f({\color{blue}e_p})} \cr
  \uncover<5->{\color{Green4}f_1} & \uncover<6->{a_{11}} &        & \uncover<8->{a_{1j}} & \uncover<9->{\ldots} & \uncover<9->{a_{1p}}\cr
  \uncover<5->{\color{Green4}f_2} & \uncover<6->{a_{21}} &        & \uncover<8->{a_{2j}} & \uncover<9->{\ldots} & \uncover<9->{a_{2p}}\cr
  \uncover<5->{\color{Green4}\vdots} & \uncover<6->{\vdots} & \uncover<7->{\vdots} & \uncover<8->{\vdots} &        & \uncover<9->{\vdots}\cr
  \uncover<5->{\color{Green4}f_n} & \uncover<6->{a_{n1}} &        & \uncover<8->{a_{nj}} & \uncover<9->{\ldots} & \uncover<9->{a_{np}}
 }
$$
\end{mydefinition}
\vspace*{-1ex}

\begin{itemize}

  \item   \uncover<10->{la $j$-ème colonne est constituée des coordonnées du vecteur
$f({\color{blue}e_j})$ dans la base 
${\color{Green4}\mathcal{B}'}=({\color{Green4}f_1}, {\color{Green4}f_2}, \ldots ,{\color{Green4}f_n})$
}
 
  \item   \uncover<11->{les vecteurs colonnes 
sont l'image par $f$ des vecteurs de la base de départ $\color{blue}\mathcal{B}$, 
exprimée dans la base d'arrivée $\color{Green4}\mathcal{B}'$
}
\end{itemize}
\end{frame}


\begin{frame}
\begin{exemple}
$$\begin{array}{rcl}
f \quad : \quad \Rr^3& \longrightarrow & \Rr^2\\
(x_1,x_2,x_3)&\longmapsto & (x_1+x_2-x_3, x_1-2x_2+3x_3)\\
\end{array}$$
\pause
\vspace*{-3ex}
\begin{itemize}
  \item $f  : \left(\begin{smallmatrix} x_1\\x_2\\x_3 \end{smallmatrix}\right) 
\mapsto \left(\begin{smallmatrix} x_1+x_2-x_3 \\ x_1-2x_2+3x_3 \end{smallmatrix}\right)$
\pause
  \item Soit $\mathcal{B} = (e_1,e_2,e_3)$ la base canonique de $\Rr^3$
\pause  
  \item Soit $\mathcal{B}' = (f_1,f_2)$ la base canonique de $\Rr^2$
\pause  
  \item $e_1 = \left(\begin{smallmatrix}1\\0\\0\end{smallmatrix}\right) \quad
e_2 = \left(\begin{smallmatrix}0\\1\\0\end{smallmatrix}\right) \quad
e_3 = \left(\begin{smallmatrix}0\\0\\1\end{smallmatrix}\right) \qquad
\pause
f_1 = \left(\begin{smallmatrix}1\\0\end{smallmatrix}\right) \quad
f_2 = \left(\begin{smallmatrix}0\\1\end{smallmatrix}\right)$
\end{itemize}
\pause
Quelle est la matrice de $f$ dans les bases $\mathcal{B}$ et $\mathcal{B}'$ ?
\vspace*{-1ex}
  \begin{itemize}
  \pause
    \item $f(e_1) = f(1,0,0) \pause = (1,1) \pause =f_1+f_2$ 
  \pause  
    \item $f(e_2) = f(0,1,0) =(1,-2)=f_1-2f_2$
  \pause 
    \item $f(e_3) = f(0,0,1) =(-1,3)=-f_1+3f_2$
  \pause
  \end{itemize}
  $$\Mat_{{\color{blue}\mathcal{B}},{\color{Green4}\mathcal{B}'}}(f)
  = \bordermatrix{
                                   & \uncover<14->{f({\color{blue}e_1})} & \uncover<17->{f({\color{blue}e_2})} & \uncover<19->{f({\color{blue}e_3})} \cr
  \uncover<15->{\color{Green4}f_1} & \uncover<16->{1}                    & \uncover<18->{1}                    & \uncover<20->{-1} \cr
  \uncover<15->{\color{Green4}f_2} & \uncover<16->{1}                    & \uncover<18->{-2}                   & \uncover<20->{3} 
  }
  $$
\end{exemple}
\end{frame}


\begin{frame}
\begin{exemple}
Même application linéaire :
\vspace*{-1ex}
$$\begin{array}{rcl}
f \quad : \quad \Rr^3& \longrightarrow & \Rr^2\\
(x_1,x_2,x_3)&\longmapsto & (x_1+x_2-x_3, x_1-2x_2+3x_3)\\
\end{array}$$
\pause
\vspace*{-2ex}
\begin{itemize}  
  \item $\epsilon_1 = \left(\begin{smallmatrix}1\\1\\0\end{smallmatrix}\right) \quad
\epsilon_2 = \left(\begin{smallmatrix}1 \\ 0 \\ 1 \end{smallmatrix}\right) \quad
\epsilon_3 = \left(\begin{smallmatrix}0 \\ 1 \\ 1 \end{smallmatrix}\right)\qquad
\pause
\phi_1 = \left(\begin{smallmatrix}1\\0\end{smallmatrix}\right) \quad
\phi_2 = \left(\begin{smallmatrix}1\\1\end{smallmatrix}\right)
$
\pause
  \item Nouvelle base de départ $\mathcal{B}_0 =(\epsilon_1,\epsilon_2,\epsilon_3)$
\pause

  \item Nouvelle base d'arrivée $\mathcal{B}_0' = (\phi_1,\phi_2)$

\pause  
  \item Quelle est la matrice de $f$ dans les bases $\mathcal{B}_0$
et $\mathcal{B}_0'$ ?

\pause 
  \item 
  \begin{itemize}
    \item $f(\epsilon_1) = f(1,1,0) \pause= (2,-1) \pause= 3\phi_1-\phi_2$
    \pause
    \item $f(\epsilon_2) = f(1,0,1) = (0,4) = -4\phi_1+4\phi_2$
    \pause
    \item $f(\epsilon_3) = f(0,1,1) = (0,1) = -\phi_1+\phi_2$
  \end{itemize}
\end{itemize}
  \pause
  \vspace*{-0.0ex}
   $$\Mat_{{\color{blue}\mathcal{B}_0},{\color{Green4}\mathcal{B}_0'}}(f)
  = \bordermatrix{
                         & f({\color{blue}\epsilon_1}) & f({\color{blue}\epsilon_2}) & f({\color{blue}\epsilon_3}) \cr
  {\color{Green4}\phi_1} &  3 & -4 & -1\cr
  {\color{Green4}\phi_2} &  -1 & 4 & 1
  }
  $$ 


\end{exemple}
\end{frame}

%%%%%%%%%%%%%%%%%%%%%%%%%%%%%%%%%%%%%%%%%%%%%%%%%%%%%%%%%%%%%%%%
\section{Opérations sur les applications linéaires et les matrices}

\begin{frame}

\begin{itemize}
  \item Soient $f,g : E \to F$ deux applications linéaires
  \pause
  \item Soit $\mathcal{B}$ une base de $E$
  \pause
  \item Soit $\mathcal{B}'$ une base de $F$
\end{itemize}

\pause
\begin{proposition}
\begin{itemize}
  \item $\Mat_{\mathcal{B},\mathcal{B}'} (f+g) = \Mat_{\mathcal{B},\mathcal{B}'} (f)
  + \Mat_{\mathcal{B},\mathcal{B}'} (g)$
  \pause
  \item $\Mat_{\mathcal{B},\mathcal{B}'} ( \lambda f) = \lambda \Mat_{\mathcal{B},\mathcal{B}'} (f)$
\end{itemize}
\end{proposition}

\pause
$$A = \Mat_{\mathcal{B},\mathcal{B}'} (f)
\qquad
B = \Mat_{\mathcal{B},\mathcal{B}'} (g)
$$
\pause
$$
C = \Mat_{\mathcal{B},\mathcal{B}'} (f+g)
\qquad
D = \Mat_{\mathcal{B},\mathcal{B}'} (\lambda f)$$

\pause
$$\text{Alors} \qquad \qquad C = A+B \qquad \pause\qquad D = \lambda A$$
\end{frame}


\begin{frame}

\begin{itemize}
  \item Soient $f : E \to F$  et $g : F \to G$ deux applications linéaires
  \pause
  \item Soient
$\mathcal{B}$ une base de $E$, $\mathcal{B}'$ une base de $F$
et $\mathcal{B}''$ une base de $G$

\end{itemize}

\pause
\begin{proposition}
\label{prop:multmatlin}
\mybox{$\Mat_{\mathcal{B},\mathcal{B}''} (g \circ f)
= \Mat_{\mathcal{B}',\mathcal{B}''} (g) \times \Mat_{\mathcal{B},\mathcal{B}'} (f)$}
\end{proposition}

\pause
$$A = \Mat_{\mathcal{B},\mathcal{B}'} (f)
\qquad
B = \Mat_{\mathcal{B}',\mathcal{B}''} (g)
\qquad
C = \Mat_{\mathcal{B},\mathcal{B}''} (g \circ f)$$
 
$$\text{Alors} \quad C = B\times A$$ 

\end{frame}



\begin{frame}
\begin{exemple}

\begin{itemize}
  \item $f : \Rr^2 \to \Rr^3$
  
  $A = \Mat_{\mathcal{B},\mathcal{B}'} (f)
= \begin{pmatrix}
1&0\cr
1&1\cr
0&2\cr
\end{pmatrix}
\in M_{3,2}$
\pause

  \item $g : \Rr^3 \to \Rr^2$ 
  
  $B = \Mat_{\mathcal{B}',\mathcal{B}''} (g)
= \begin{pmatrix}
2&-1&0\cr
3&1&2\cr
\end{pmatrix}
\in M_{2,3}$

\pause
  \item 
  
  $g \circ f : \Rr^2 \to \Rr^2$
  
\end{itemize}  
\pause
  $$\Mat_{\mathcal{B},\mathcal{B}''} (g \circ f) \pause = C \pause = B \times A 
\pause  = 
\begin{pmatrix}
2&-1&0\cr
3&1&2\cr
\end{pmatrix}
\times
\begin{pmatrix}
1&0\cr
1&1\cr
0&2\cr
\end{pmatrix}
\pause = 
\begin{pmatrix}
1&-1\cr
4&5\cr
\end{pmatrix}
$$
  


\end{exemple}
\end{frame}

%%%%%%%%%%%%%%%%%%%%%%%%%%%%%%%%%%%%%%%%%%%%%%%%%%%%%%%%%%%%%%%%
\section{Matrice d'un endomorphisme}


\begin{frame}

\begin{itemize}

  \item $f : E \to E$ est un endomorphisme
  \pause
  \item Deux bases distinctes pour l'espace vectoriel $E$ : $\Mat_{\mathcal{B},\mathcal{B}'} (f)$
  \pause  
  \item Même base $\mathcal{B}$ au départ et à l'arrivée : $\Mat_{\mathcal{B}} (f)$

\end{itemize}

\pause
\begin{exemple}
\begin{itemize}
  \item \evidence{identité} $\id : E \to E$, $\id(x)=x$ 
\hfill\pause
  $\Mat_{\mathcal{B}} (\id) = I_n$
\hspace*{1em}

\pause
  \item \evidence{homothétie} $h_\lambda : E \to E$, $h_\lambda(x) = \lambda \cdot x$
\hfill\pause
  $\Mat_{\mathcal{B}} (h_\lambda) = \lambda I_n$
\hspace*{1em}

\pause
  \item \evidence{symétrie centrale} $s : E \to E$, $s(x) = - x$ 
\hfill\pause
  $\Mat_{\mathcal{B}} (s) = - I_n$
\hspace*{1em}
  
  \pause
  \item \evidence{rotation}   
  \begin{itemize}
    \item $r_\theta : \Rr^2 \longrightarrow \Rr^2$
    \pause
    \item rotation d'angle $\theta$, centrée à l'origine 
    \pause
    \item $\Rr^2$ muni de la base canonique $\mathcal{B}$
    \pause
    \item $r_\theta (x,y) = (x \cos  \theta - y \sin\theta,x\sin \, \theta + y \cos\theta)$
    \pause
    
$$\Mat_{\mathcal{B}} (r_\theta) = 
  \begin{pmatrix}
\cos\theta & -\sin\theta\\
\sin\theta & \cos\theta
\end{pmatrix}$$
  \end{itemize} 
\end{itemize}
  
\end{exemple}
\end{frame}

\begin{frame}

\begin{corollaire}
Soient $f :  E \to E$ une application linéaire et $\mathcal{B}$ une base de $E$

Quel que soit $p \in \Nn$
$$\Mat_{\mathcal{B}} (f^p) = \big( \Mat_{\mathcal{B}} (f) \big)^p$$
\end{corollaire}

\pause
\bigskip

La matrice associée à $f^p = \underbrace{f \circ f \circ\cdots \circ f}_{p \text{ occurrences}} $
est $A^p = \underbrace{A \times A \times \cdots \times A}_{p \text{ facteurs}}$

\bigskip
\pause

\begin{exemple}
\vspace*{-2ex}
$$\Mat_{\mathcal{B}} (r_\theta^p) 
\pause= \big( \Mat_{\mathcal{B}} (r_\theta) \big)^p
\pause=  \begin{pmatrix}
\cos\theta & -\sin\theta\\
\sin\theta & \cos\theta
\end{pmatrix}^p
\pause \!\!= \begin{pmatrix}
\cos(p\theta) & -\sin(p\theta)\\
\sin(p\theta) & \cos(p\theta)
\end{pmatrix}$$
\end{exemple}
\end{frame}


%%%%%%%%%%%%%%%%%%%%%%%%%%%%%%%%%%%%%%%%%%%%%%%%%%%%%%%%%%%%%%%%
\section{Matrice d'un isomorphisme}

\begin{frame}

Un \defi{isomorphisme} $f : E \to F$ est une application linéaire bijective
\pause

\begin{theoreme}
\label{th:invmatlin}
\begin{itemize}
  \item Soient $E$ et $F$ deux espaces vectoriels de même dimension finie
  \item Soit $f : E \to F$ une application linéaire
  \item Soient $\mathcal{B}$ une base de $E$, $\mathcal{B}'$ une base de $F$
  \item Soit $A = \Mat_{\mathcal{B},\mathcal{B}'} (f)$
\end{itemize}
\pause

\begin{enumerate}
  \item $f$ est bijective si et seulement si la matrice $A$ est inversible
 \pause 
  
  \item  Si $f : E \to F$ est bijective, alors la matrice de 
  $f^{-1} : F \to E$ est la matrice $A^{-1}$
  \pause
  \vspace*{-2ex}
  $$\Mat_{\mathcal{B}',\mathcal{B}}(f^{-1}) = \bigg( \Mat_{\mathcal{B},\mathcal{B}'}(f) \bigg)^{-1}$$
\end{enumerate}
\end{theoreme}
\end{frame}


\begin{frame}

\begin{itemize}
  \item $f : E \to E$
  \item Même base $\mathcal{B}$ au départ et à l'arrivée
  \item $A = \Mat_{\mathcal{B}}(f)$
\end{itemize}

\pause
\begin{corollaire}
\begin{itemize}
  \item $f$ est bijective si et seulement si $A$ est inversible
  
  \item Si $f$ est bijective, alors la matrice associée à 
  $f^{-1}$ dans la base $\mathcal{B}$ est $A^{-1}$
\end{itemize}
\end{corollaire}

\pause
\centerline{$\Mat_{\mathcal{B}}(f^{-1}) = \big(\Mat_{\mathcal{B}}(f) \big)^{-1}$}
\end{frame}


\begin{frame}
\begin{exemple}
\begin{itemize}
  \item Soient $r : \Rr^2 \to \Rr^2$ la rotation d'angle $\frac\pi6$ (centrée à l'origine)
  \pause
 $$A = \Mat_{\mathcal{B}} (r) =  
  \begin{pmatrix}
\cos\theta & -\sin\theta\\
\sin\theta & \cos\theta
\end{pmatrix}
\pause=
\begin{pmatrix}
\frac{\sqrt{3}}{2} & -\frac12\\
\frac12& \frac{\sqrt{3}}{2}
\end{pmatrix}$$
\pause
    \item Soit $s$ la réflexion par rapport à $(y=x)$ 
    \pause:
    $B = \Mat_{\mathcal{B}} (s) = \begin{pmatrix} 0&1\\1&0 \end{pmatrix}$
  
  \pause
  \item $\Mat_{\mathcal{B}} (s \circ r) \pause= B \times A \pause= \begin{pmatrix}
\frac12&\frac{\sqrt{3}}{2} & \\
\frac{\sqrt{3}}{2}&-\frac12& 
\end{pmatrix}$

  \pause
  
  \item $\Mat_{\mathcal{B}} \big((s \circ r)^{-1}\big) \pause= (BA)^{-1} \pause=  \begin{pmatrix}
\frac12&\frac{\sqrt{3}}{2} & \\
\frac{\sqrt{3}}{2}&-\frac12& 
\end{pmatrix}^{-1} 
\pause= \begin{pmatrix}
\frac12&\frac{\sqrt{3}}{2} & \\
\frac{\sqrt{3}}{2}&-\frac12& 
\end{pmatrix}$ 

\end{itemize}
\end{exemple}
\end{frame}

%%%%%%%%%%%%%%%%%%%%%%%%%%%%%%%%%%%%%%%%%%%%%%%%%%%%%%%%%%%%%%%%
\section{Mini-exercices}

\begin{frame}

\begin{miniexercice}
\begin{enumerate}
  \item Calculer la matrice associée aux applications linéaires 
  $f_i : \Rr^2 \to \Rr^2$ dans la base canonique :
  \begin{enumerate}
  \item $f_1$ la symétrie par rapport à l'axe $(Oy)$,
  \item $f_2$ la symétrie par rapport à l'axe $(y=x)$,
  \item $f_3$ le projection orthogonale sur l'axe $(Oy)$,
  \item $f_4$ la rotation d'angle $\frac\pi4$.
  \end{enumerate}
  Calculer quelques matrices associées à $f_i \circ f_j$ et, 
  lorsque c'est possible, à $f_i^{-1}$.
  
  \item Même travail pour $f_i : \Rr^3 \to \Rr^3$ :
  \begin{enumerate}
  \item $f_1$ l'homothétie de rapport $\lambda$,
  \item $f_2$ la réflexion orthogonale par rapport au plan $(Oxz)$,
  \item $f_3$ la rotation d'axe $(Oz)$ d'angle $-\frac{\pi}{2}$,
  \item $f_4$ la projection orthogonale sur le plan $(Oyz)$.
  \end{enumerate}  
\end{enumerate}
\end{miniexercice}

\end{frame}

\end{document}
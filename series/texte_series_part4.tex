
%%%%%%%%%%%%%%%%%% PREAMBULE %%%%%%%%%%%%%%%%%%


\documentclass[12pt]{article}

\usepackage{amsfonts,amsmath,amssymb,amsthm}
\usepackage[utf8]{inputenc}
\usepackage[T1]{fontenc}
\usepackage[francais]{babel}


% packages
\usepackage{amsfonts,amsmath,amssymb,amsthm}
\usepackage[utf8]{inputenc}
\usepackage[T1]{fontenc}
%\usepackage{lmodern}

\usepackage[francais]{babel}
\usepackage{fancybox}
\usepackage{graphicx}

\usepackage{float}

%\usepackage[usenames, x11names]{xcolor}
\usepackage{tikz}
\usepackage{datetime}

\usepackage{mathptmx}
%\usepackage{fouriernc}
%\usepackage{newcent}
\usepackage[mathcal,mathbf]{euler}

%\usepackage{palatino}
%\usepackage{newcent}


% Commande spéciale prompteur

%\usepackage{mathptmx}
%\usepackage[mathcal,mathbf]{euler}
%\usepackage{mathpple,multido}

\usepackage[a4paper]{geometry}
\geometry{top=2cm, bottom=2cm, left=1cm, right=1cm, marginparsep=1cm}

\newcommand{\change}{{\color{red}\rule{\textwidth}{1mm}\\}}

\newcounter{mydiapo}

\newcommand{\diapo}{\newpage
\hfill {\normalsize  Diapo \themydiapo \quad \texttt{[\jobname]}} \\
\stepcounter{mydiapo}}


%%%%%%% COULEURS %%%%%%%%%%

% Pour blanc sur noir :
%\pagecolor[rgb]{0.5,0.5,0.5}
% \pagecolor[rgb]{0,0,0}
% \color[rgb]{1,1,1}



%\DeclareFixedFont{\myfont}{U}{cmss}{bx}{n}{18pt}
\newcommand{\debuttexte}{
%%%%%%%%%%%%% FONTES %%%%%%%%%%%%%
\renewcommand{\baselinestretch}{1.5}
\usefont{U}{cmss}{bx}{n}
\bfseries

% Taille normale : commenter le reste !
%Taille Arnaud
%\fontsize{19}{19}\selectfont

% Taille Barbara
%\fontsize{21}{22}\selectfont

%Taille François
\fontsize{25}{30}\selectfont

%Taille Pascal
%\fontsize{25}{30}\selectfont

%Taille Laura
%\fontsize{30}{35}\selectfont


%\myfont
%\usefont{U}{cmss}{bx}{n}

%\Huge
%\addtolength{\parskip}{\baselineskip}
}


% \usepackage{hyperref}
% \hypersetup{colorlinks=true, linkcolor=blue, urlcolor=blue,
% pdftitle={Exo7 - Exercices de mathématiques}, pdfauthor={Exo7}}


%section
% \usepackage{sectsty}
% \allsectionsfont{\bf}
%\sectionfont{\color{Tomato3}\upshape\selectfont}
%\subsectionfont{\color{Tomato4}\upshape\selectfont}

%----- Ensembles : entiers, reels, complexes -----
\newcommand{\Nn}{\mathbb{N}} \newcommand{\N}{\mathbb{N}}
\newcommand{\Zz}{\mathbb{Z}} \newcommand{\Z}{\mathbb{Z}}
\newcommand{\Qq}{\mathbb{Q}} \newcommand{\Q}{\mathbb{Q}}
\newcommand{\Rr}{\mathbb{R}} \newcommand{\R}{\mathbb{R}}
\newcommand{\Cc}{\mathbb{C}} 
\newcommand{\Kk}{\mathbb{K}} \newcommand{\K}{\mathbb{K}}

%----- Modifications de symboles -----
\renewcommand{\epsilon}{\varepsilon}
\renewcommand{\Re}{\mathop{\text{Re}}\nolimits}
\renewcommand{\Im}{\mathop{\text{Im}}\nolimits}
%\newcommand{\llbracket}{\left[\kern-0.15em\left[}
%\newcommand{\rrbracket}{\right]\kern-0.15em\right]}

\renewcommand{\ge}{\geqslant}
\renewcommand{\geq}{\geqslant}
\renewcommand{\le}{\leqslant}
\renewcommand{\leq}{\leqslant}

%----- Fonctions usuelles -----
\newcommand{\ch}{\mathop{\mathrm{ch}}\nolimits}
\newcommand{\sh}{\mathop{\mathrm{sh}}\nolimits}
\renewcommand{\tanh}{\mathop{\mathrm{th}}\nolimits}
\newcommand{\cotan}{\mathop{\mathrm{cotan}}\nolimits}
\newcommand{\Arcsin}{\mathop{\mathrm{Arcsin}}\nolimits}
\newcommand{\Arccos}{\mathop{\mathrm{Arccos}}\nolimits}
\newcommand{\Arctan}{\mathop{\mathrm{Arctan}}\nolimits}
\newcommand{\Argsh}{\mathop{\mathrm{Argsh}}\nolimits}
\newcommand{\Argch}{\mathop{\mathrm{Argch}}\nolimits}
\newcommand{\Argth}{\mathop{\mathrm{Argth}}\nolimits}
\newcommand{\pgcd}{\mathop{\mathrm{pgcd}}\nolimits} 

\newcommand{\Card}{\mathop{\text{Card}}\nolimits}
\newcommand{\Ker}{\mathop{\text{Ker}}\nolimits}
\newcommand{\id}{\mathop{\text{id}}\nolimits}
\newcommand{\ii}{\mathrm{i}}
\newcommand{\dd}{\mathrm{d}}
\newcommand{\Vect}{\mathop{\text{Vect}}\nolimits}
\newcommand{\Mat}{\mathop{\mathrm{Mat}}\nolimits}
\newcommand{\rg}{\mathop{\text{rg}}\nolimits}
\newcommand{\tr}{\mathop{\text{tr}}\nolimits}
\newcommand{\ppcm}{\mathop{\text{ppcm}}\nolimits}

%----- Structure des exercices ------

\newtheoremstyle{styleexo}% name
{2ex}% Space above
{3ex}% Space below
{}% Body font
{}% Indent amount 1
{\bfseries} % Theorem head font
{}% Punctuation after theorem head
{\newline}% Space after theorem head 2
{}% Theorem head spec (can be left empty, meaning ‘normal’)

%\theoremstyle{styleexo}
\newtheorem{exo}{Exercice}
\newtheorem{ind}{Indications}
\newtheorem{cor}{Correction}


\newcommand{\exercice}[1]{} \newcommand{\finexercice}{}
%\newcommand{\exercice}[1]{{\tiny\texttt{#1}}\vspace{-2ex}} % pour afficher le numero absolu, l'auteur...
\newcommand{\enonce}{\begin{exo}} \newcommand{\finenonce}{\end{exo}}
\newcommand{\indication}{\begin{ind}} \newcommand{\finindication}{\end{ind}}
\newcommand{\correction}{\begin{cor}} \newcommand{\fincorrection}{\end{cor}}

\newcommand{\noindication}{\stepcounter{ind}}
\newcommand{\nocorrection}{\stepcounter{cor}}

\newcommand{\fiche}[1]{} \newcommand{\finfiche}{}
\newcommand{\titre}[1]{\centerline{\large \bf #1}}
\newcommand{\addcommand}[1]{}
\newcommand{\video}[1]{}

% Marge
\newcommand{\mymargin}[1]{\marginpar{{\small #1}}}



%----- Presentation ------
\setlength{\parindent}{0cm}

%\newcommand{\ExoSept}{\href{http://exo7.emath.fr}{\textbf{\textsf{Exo7}}}}

\definecolor{myred}{rgb}{0.93,0.26,0}
\definecolor{myorange}{rgb}{0.97,0.58,0}
\definecolor{myyellow}{rgb}{1,0.86,0}

\newcommand{\LogoExoSept}[1]{  % input : echelle
{\usefont{U}{cmss}{bx}{n}
\begin{tikzpicture}[scale=0.1*#1,transform shape]
  \fill[color=myorange] (0,0)--(4,0)--(4,-4)--(0,-4)--cycle;
  \fill[color=myred] (0,0)--(0,3)--(-3,3)--(-3,0)--cycle;
  \fill[color=myyellow] (4,0)--(7,4)--(3,7)--(0,3)--cycle;
  \node[scale=5] at (3.5,3.5) {Exo7};
\end{tikzpicture}}
}



\theoremstyle{definition}
%\newtheorem{proposition}{Proposition}
%\newtheorem{exemple}{Exemple}
%\newtheorem{theoreme}{Théorème}
\newtheorem{lemme}{Lemme}
\newtheorem{corollaire}{Corollaire}
%\newtheorem*{remarque*}{Remarque}
%\newtheorem*{miniexercice}{Mini-exercices}
%\newtheorem{definition}{Définition}




%definition d'un terme
\newcommand{\defi}[1]{{\color{myorange}\textbf{\emph{#1}}}}
\newcommand{\evidence}[1]{{\color{blue}\textbf{\emph{#1}}}}



 %----- Commandes divers ------

\newcommand{\codeinline}[1]{\texttt{#1}}

%%%%%%%%%%%%%%%%%%%%%%%%%%%%%%%%%%%%%%%%%%%%%%%%%%%%%%%%%%%%%
%%%%%%%%%%%%%%%%%%%%%%%%%%%%%%%%%%%%%%%%%%%%%%%%%%%%%%%%%%%%%


\begin{document}

\debuttexte

%%%%%%%%%%%%%%%%%%%%%%%%%%%%%%%%%%%%%%%%%%%%%%%%%%%%%%%%%%%
\diapo

Nous poursuivons le chapitre sur les séries par une leçon consacrée aux séries absolument convergentes et à la règle de d'Alembert.

\change

\change
Nous commencerons par définir ce qu'est une série absolument convergente.

\change
Puis nous établirons et nous appliquerons la règle du quotient de d'Alembert, 

\change
ainsi que la règle des racines de Cauchy.

% \change
% Nous comparerons ensuite ces deux méthode
% 
% \change
% et enfin nous terminerons avec la règle de Raabe-Duhamel.


%%%%%%%%%%%%%%%%%%%%%%%%%%%%%%%%%%%%%%%%%%%%%%%%%%%%%%%%%%%
\diapo

On dit qu'une série $\sum u_k$ de nombres réels (ou complexes) est 
\defi{absolument convergente} si la série $\sum |u_k|$ 
est convergente.  

\change
Par exemple la série $\sum_{k\ge1} \frac{\cos k}{k^2}$ est absolument convergente.

\change
Car $\left\vert\frac{\cos k}{k^2}\right\vert \le  \frac{1}{k^2}$. Comme la série $\sum_{k\ge1} \frac{1}{k^2}$ converge alors $\sum_{k\ge1} \left\vert\frac{\cos k}{k^2}\right\vert$ converge aussi.

\change
La série harmonique alternée $\sum_{k=0}^{+\infty} \frac{(-1)^k}{k+1}$  n'est pas absolument convergente. 

\change
Car la série $\sum_{k\ge0}  \left\vert\frac{(-1)^k}{k+1}\right\vert = \sum_{k\ge0} \frac{1}{k+1}$ diverge.

\change
Une série, telle que la série harmonique alternée, qui est convergente, 
mais qui n'est pas absolument convergente, s'appelle une série \defi{semi-convergente}.

%%%%%%%%%%%%%%%%%%%%%%%%%%%%%%%%%%%%%%%%%%%%%%%%%%%%%%%%%%%
\diapo

\^Etre absolument convergent est plus fort qu'être convergent. En effet, on a le théorème suivant :

Toute série absolument convergente est convergente.

\change
Démontrons ce résultat en utilisant le critère de Cauchy. 

Soit $\sum u_k$ une série absolument convergente.

\change
Par définition, la série $\sum |u_k|$ est convergente.

\change
Donc la suite des restes $(R'_n)$ avec $R'_n = \displaystyle\sum_{k=n+1}^{+\infty} |u_k|$ est une suite qui tend vers $0$.

\change
En particulier c'est donc une suite de Cauchy.

\change
Ce qui se traduit ainsi.

Soit $\epsilon>0$ fixé. 

\change
Il existe $n_0 \in \Nn$ tel que pour tout $n \ge n_0$ et pour tout $p \ge 0$ :
$$|u_n|+|u_{n+1}|+\cdots+|u_{n+p}| < \epsilon.$$

\change
Par suite, pour $n \ge n_0$ et  $p \ge 0$ on a :
$$\big|u_n+u_{n+1}+\cdots+u_{n+p}\big| \le |u_n|+|u_{n+1}|+\cdots+|u_{n+p}| < \epsilon.$$

\change
Donc, d'après le critère de Cauchy, la série $\sum u_k$ est convergente.

%%%%%%%%%%%%%%%%%%%%%%%%%%%%%%%%%%%%%%%%%%%%%%%%%%%%%%%%%%%
\diapo

La règle du quotient de d'Alembert est un moyen efficace pour montrer 
si une série de nombres réels ou complexes converge ou pas.

\change
Soit $\sum u_k$ une série dont les termes généraux sont des nombres réels (ou complexes) non nuls.

\change
1) S'il existe une constante $0<q<1$ et un entier $k_0$ tels que, pour tout
$k \ge k_0$,  
$$
\left|\frac{u_{k+1}}{u_k}\right| \le q $$
alors $\sum u_k$ converge.

\change
La série est même absolument convergente.

\change
2) S'il existe un entier $k_0$ tel que, pour tout $k \ge k_0$, 
$$
\left|\frac{u_{k+1}}{u_k}\right| \ge 1
$$
alors $\sum u_k$ diverge.

%%%%%%%%%%%%%%%%%%%%%%%%%%%%%%%%%%%%%%%%%%%%%%%%%%%%%%%%%%%
\diapo

Le plus souvent, la situation que l'on étudie est lorsque 
la suite $\frac{u_{k+1}}{u_k}$ converge ; la position
de la limite par rapport à $1$ détermine alors la nature de la série.

\change
Voici une application directe et la plus utilisée, 
pour les séries de nombres réels, strictement positifs :

Soit $\sum u_k$ une série à termes strictement positifs, telle que 
$\frac{u_{k+1}}{u_k}$ converge vers $\ell$.

\change
Si $\ell<1$ alors $\sum u_k$ converge.

\change
Si $\ell>1$ alors $\sum u_k$ diverge.

\change
Si $\ell=1$ on ne peut pas conclure en général.


%%%%%%%%%%%%%%%%%%%%%%%%%%%%%%%%%%%%%%%%%%%%%%%%%%%%%%%%%%%
\diapo

Démontrons le théorème de la règle du quotient de d'Alembert, et rappelons tout d'abord que la série géométrique $\sum q^k$ converge si $|q|<1$.

\change
Dans le premier cas du théorème, l'hypothèse $\left|\frac{u_{k+1}}{u_k}\right| \le q$ ...

\change
implique $|u_{k_0+1}| \le |u_{k_0}| q$, 

\change
puis $|u_{k_0+2}| \le |u_{k_0}| q^2$, et ainsi de suite.

\change
On vérifie par récurrence que, pour tout $k\ge k_0$ :
$$|u_k| \le |u_{k_0}| q^{-k_0} \cdot q^k = c \cdot q^k,$$
où $c$ est une constante.

\change
Comme $0 < q < 1$, la série $\sum q^k$ converge, 

\change
d'où le résultat par le théorème de comparaison : la série $\sum |u_k|$ converge.

\change
Si $\left|\frac{u_{k+1}}{u_k}\right| \ge 1$, 

\change
la suite $(|u_k|)$ est croissante : 

\change
elle ne peut donc pas tendre vers $0$ 

\change
et la série diverge.  

%%%%%%%%%%%%%%%%%%%%%%%%%%%%%%%%%%%%%%%%%%%%%%%%%%%%%%%%%%%
\diapo

Appliquons la règle du quotient de d'Alembert pour déterminer la nature de quelques séries.

\change
Tout d'abord, pour tout $x \in \Rr$ fixé, la \defi{série exponentielle}
$$\sum_{k=0}^{+\infty} \frac{x^k}{k!}\quad \text{ converge.}$$

\change
En effet pour $u_k = \frac{x^k}{k!}$ on a 

\change
$$\left|\frac{u_{k+1}}{u_k}\right|
= \frac{\left|\frac{x^{k+1}}{(k+1)!}\right|}{\left|\frac{x^k}{k!}\right|}
=\frac{|x|}{k+1} \to 0 
\quad \text{lorsque } k \to +\infty.$$

\change
La limite étant $\ell = 0 < 1$ alors par la règle du quotient de d'Alembert,
la série est absolument convergente, donc convergente.

\change
Par définition la somme est $\exp(x)$ : 
$$\exp(x) = \sum_{k=0}^{+\infty} \frac{x^k}{k!}.$$

\change
Autre exemple, la série $\displaystyle\sum_{k\ge0} \frac{k!}{1\cdot 3\,\cdots\,(2k-1)}$ converge,

\change
car $\frac{u_{k+1}}{u_k}=\frac{k+1}{2k+1}$ tend vers $\frac{1}{2}<1$.
  
\change
La série $\displaystyle\sum_{k\ge0} \frac{(2k)!}{(k!)^2}$ diverge, 

\change
car $\frac{u_{k+1}}{u_k}=\frac{(2k+1)(2k+2)}{(k+1)^2}$ tend vers $4>1$.

%%%%%%%%%%%%%%%%%%%%%%%%%%%%%%%%%%%%%%%%%%%%%%%%%%%%%%%%%%%
\diapo

Quelques remarques sur la règle du quotient de d'Alembert.

\change
Tout d'abord, le théorème ne peut pas s'appliquer si certains $u_k$ sont nuls, contrairement à la règle des racines de Cauchy que l'on verra après.
  
\change
Notez bien que le théorème ne permet pas toujours de conclure. En particulier, faites aussi bien attention que l'hypothèse est  $\left|\frac{u_{k+1}}{u_k}\right| \le q <1$, ce qui est plus fort que $\left|\frac{u_{k+1}}{u_k}\right| <1$.
  
\change
De même le corollaire ne permet pas de conclure lorsque $\frac{u_{k+1}}{u_k} \to 1$.

\change
Par exemple pour les séries $\sum u_k= \sum \frac{1}{k}$ et $\sum v_k = \sum \frac{1}{k^2}$ 

\change
nous avons $\frac{u_{k+1}}{u_k} = \frac{k}{k+1} \to 1$,

\change
de même que $\frac{v_{k+1}}{v_k} = \frac{k^2}{(k+1)^2 } \to 1$.

\change
Cependant la série $\sum \frac{1}{k}$ diverge alors que $\sum \frac{1}{k^2}$ converge.


%%%%%%%%%%%%%%%%%%%%%%%%%%%%%%%%%%%%%%%%%%%%%%%%%%%%%%%%%%%
\diapo

Terminons par un exemple plus compliqué.

Trouver tous les $z\in \Cc$ tels que la série 
$\sum_{k\ge0} \binom{k}{3} z^k$ soit absolument convergente.
\medskip

\change
Soit $u_k=\binom{k}{3} z^k$. 

\change
Alors, pour $z\neq0$,
$$\dfrac{|u_{k+1}|}{|u_k|}=\dfrac{\binom{k+1}{3}|z|^{k+1}}{\binom{k}{3}|z|^{k}}$$

\change
qui vaut
$$
\dfrac{\dfrac{(k+1)k(k-1)}{3!}}{\dfrac{k(k-1)(k-2)}{3!}}|z|$$

\change
c-à-d après simplification :
$$
\frac{k+1}{k-2}|z|.$$

\change
Ce qui tend vers $ |z| $ lorsque $k\to+\infty.$

\change
Si $|z|<1$ alors pour $k$ assez grand $\frac{|u_{k+1}|}{|u_k|} < q <1$.

\change
Donc la série $\sum u_k$ est alors absolument convergente.

\change
Si $|z|\ge 1$ alors le quotient $\frac{|u_{k+1}|}{|u_k|}$, qui vaut $\frac{k+1}{k-2}|z|$

\change
est supérieur ou égal à $\frac{k+1}{k-2}$ qui est strictement supérieur à $ 1$ pour tout $k$. 

\change
Donc la série $\sum u_k$  diverge.


%%%%%%%%%%%%%%%%%%%%%%%%%%%%%%%%%%%%%%%%%%%%%%%%%%%%%%%%%%%
\diapo

Passons à présent à la règle des racines de Cauchy.

\change
Théorème.

Soit $\sum u_k$ une série de nombres réels ou complexes.

\change
S'il existe une constante $0<q<1$ et un entier $k_0$ tels que, pour tout
$k \ge k_0$,  
$$
\sqrt[k]{|u_k|} \le q ,\quad \text{ alors }\quad\sum u_k\quad \text{converge.}
$$

\change
La série est même absolument convergente.

\change
Si par contre il existe un entier $k_0$ tel que, pour tout $k \ge k_0$, 
$$
\sqrt[k]{|u_k|} \ge 1,\quad \text{ alors }\quad\sum u_k \quad\text{diverge.}
$$


%%%%%%%%%%%%%%%%%%%%%%%%%%%%%%%%%%%%%%%%%%%%%%%%%%%%%%%%%%%
\diapo

Le plus souvent vous l'appliquerez avec un terme général strictement positif.

\change
Soit $\sum u_k$ une série à termes positifs, telle que 
$\sqrt[k]{u_k}$ converge vers $\ell$.

\change
Si $\ell<1$ alors $\sum u_k$ converge.

\change
Si $\ell>1$ alors $\sum u_k$ diverge.

\change
Si $\ell=1$ on ne peut pas conclure en général.

\change
Dans la pratique, il faut savoir bien manipuler les racines $k$-ème : 
$\sqrt[k]{u_k} $ qui se note aussi $ (u_k)^{\frac1k}$ est égal par définition à $\exp\left(\tfrac1k \ln u_k\right)$.


%%%%%%%%%%%%%%%%%%%%%%%%%%%%%%%%%%%%%%%%%%%%%%%%%%%%%%%%%%%
\diapo

Rappelons que la nature de la série ne dépend pas de ses premiers
termes. 

\change
Dans le premier cas du théorème, 

\change
$\sqrt[k]{|u_k|} \le q$ implique  $|u_k| \le q^k$. 

\change
Comme $0<q<1$, la série $\sum q^k$ converge. 

\change
D'où le résultat par le théorème de comparaison.  

\change
Dans le second cas, $\sqrt[k]{|u_k|} \ge 1$, 

\change
on a $|u_k| \ge 1$. Le terme général ne tend pas vers $0$, 

\change
donc la série diverge.  

\change
Enfin pour le dernier point du corollaire, 

\change
on pose $u_k=\frac{1}{k}$, $v_k=\frac{1}{k^2}$.

\change
On a $\sqrt[k]{u_k}\to 1$ de même que $\sqrt[k]{v_k}\to 1$.

\change
Mais $\sum u_k$ diverge alors que $\sum v_k$ converge.


%%%%%%%%%%%%%%%%%%%%%%%%%%%%%%%%%%%%%%%%%%%%%%%%%%%%%%%%%%%
\diapo

Par exemple, 
$$\sum \left(\frac{2k+1}{3k+4}\right)^k\quad \text{ converge,}$$

\change
car $\sqrt[k]{u_k} = \frac{2k+1}{3k+4}$ tend vers $\frac{2}{3}<1$.

\change
Par contre  
$$\sum \frac{2^k}{k^\alpha}\quad \text{ diverge,}$$
quel que soit $\alpha >0$.
  
\change
En effet,
$$\sqrt[k]{u_k} = \frac{\sqrt[k]{2^k}}{\big(\sqrt[k]{k}\big)^\alpha}
  = \frac{2}{\big(k^\frac{1}{k}\big)^\alpha}
  = \frac{2}{\big(\exp(\frac{1}{k}\ln k)\big)^\alpha} \to 2>1.$$


%%%%%%%%%%%%%%%%%%%%%%%%%%%%%%%%%%%%%%%%%%%%%%%%%%%%%%%%%%%
\diapo
Regardons en détail un exemple plus élaboré.

Déterminer tous les $z\in \Cc$ tels que la série  
$\sum_{k\ge1} \left( 1+\frac{1}{k}\right)^{k^2} z^k$ soit absolument convergente.

\change
Notons $u_k =\left( 1+\frac{1}{k}\right)^{k^2} z^k$.

\change
On a alors $$\sqrt[k]{|u_k|}=\left( 1+\frac{1}{k}\right)^k |z| $$
qui tend vers $e  \times |z|.$

\change
Cette limite vérifie $e|z| < 1$ si et seulement si $|z|<\frac{1}{e}$.

On a alors la discussion suivante.

\change
Si $|z|<\frac{1}{e}$ alors la série $\sum u_k$ est absolument 
convergente.
  
\change
Si $|z|>\frac{1}{e}$, on a pour $k$ assez grand $\sqrt[k]{|u_k|}>1$, donc la série $\sum u_k$ diverge.
  
\change
Si $|z|=\frac{1}{e}$ la règle des racines de Cauchy ne permet pas de conclure : il faut étudier le terme général à la main. C'est l'objet de la diapo suivante.

%%%%%%%%%%%%%%%%%%%%%%%%%%%%%%%%%%%%%%%%%%%%%%%%%%%%%%%%%%%
\diapo
Poursuivons l'étude de  la série $\sum \left( 1+\frac{1}{k}\right)^{k^2} z^k$ lorsque $|z|=\frac{1}{e}$.

\change
On obtient dans ce cas 
  $$|u_k|=\left(  1+\frac{1}{k}\right)^{k^2}\left(\frac{1}{e}\right)^k$$
  
\change
Donc
$$
\ln|u_k| = k^2\ln\left(1+\tfrac{1}{k}\right)+k\ln \tfrac{1}{e} 
$$

\change
c'est-à-dire
$$
 k\left[k\ln (1+\tfrac{1}{k})-1\right] 
$$

\change
ce qui donne en effectuant un DL
$$
k \left[ k\left(\tfrac{1}{k}-\tfrac{1}{2}\left( \tfrac{1}{k}\right)^2+
o\big( \tfrac{1}{k^2}\big) \right)-1\right]
$$

\change
Après développement et simplification, on obtient
$$
-\tfrac{1}{2} +o(1)
$$

\change
et donc $\ln|u_k| $ tend vers $-\tfrac{1}{2}$.

\change
Donc $|u_k| \to e^{-\frac{1}{2}}\neq 0$. Ainsi $\sum |u_k|$ diverge. 

% 
% %%%%%%%%%%%%%%%%%%%%%%%%%%%%%%%%%%%%%%%%%%%%%%%%%%%%%%%%%%%
% \diapo
% 
% Cette section peut être passée lors d'une première lecture.
% 
% Nous allons comparer la règle du quotient de d'Alembert avec la règle des racines de Cauchy.
% 
% \change
% Nous allons voir que la règle des racines de Cauchy est plus puissante que la règle du quotient de d'Alembert. Cependant dans la pratique la règle du quotient de d'Alembert reste la plus utilisée.
% 
% 
% \change 
% Proposition.
% 
% Soit $(u_k)$ une suite à termes strictement positifs.
% 
% \change
% $$
% \text{Si } \quad \lim_{k\to+\infty} \frac{u_{k+1}}{u_k} = \ell
% \qquad\text{alors}\qquad
% \lim_{k\to+\infty} \sqrt[k]{u_k} = \ell\;.
% $$
% 
% Autrement dit, si on peut appliquer la règle du quotient de d'Alembert, alors on peut aussi appliquer 
% la règle des racines de Cauchy.
% 
% \change
% Voici un exemple où la règle des racines de Cauchy permet de conclure, mais pas la règle du quotient de d'Alembert.
% 
% \change
% Définissons la suite $u_k$ par :
% $$
% u_k = \left\{
% \begin{array}{ll}
% \frac{2^n}{3^n}&\text{ si } k=2n\\[2ex]
% \frac{2^n}{3^{n+1}}&\text{ si } k=2n+1
% \end{array}
% \right.
% $$
% 
% \change
% Le rapport $\frac{u_{k+1}}{u_k}$ vaut $\frac{1}{3}$ si $k$ est pair,
% $2$ si $k$ est impair. 
% 
% \change
% La règle du quotient de d'Alembert ne s'applique donc pas.
% 
% \change 
% Pourtant, $\sqrt[k]{u_k}$ converge vers $\sqrt{\frac{2}{3}}<1$.
% 
% \change
% Donc la règle des racines de Cauchy s'applique et la série $\sum u_k$ converge.  
% 
% 
% %%%%%%%%%%%%%%%%%%%%%%%%%%%%%%%%%%%%%%%%%%%%%%%%%%%%%%%%%%%
% \diapo
% 
% Cette section peut être passée lors d'une première lecture.
% 
% La règle du quotient de d'Alembert et la règle des racines de Cauchy 
% ne s'appliquent pas aux séries de Riemann  
% $$\sum_{k\ge1} \frac{1}{k^\alpha}$$
% 
% \change
% car $\frac{k^\alpha}{(k+1)^\alpha}\to 1$
% et $\sqrt[k]{u_k} \to 1$.
% 
% Il nous faut raffiner la règle de d'Alembert
% pour pouvoir conclure. Cependant nous reviendrons 
% sur la convergence des séries de Riemann par d'autres techniques.
% 
% \change
% Théorème. Règle de Raabe-Duhamel
% 
% Soit $(u_k)$ une suite de nombres réels (ou complexes) non nuls.
% 
% \change
% Si $\forall k\geq k_0$ on a $\left|\frac{u_{k+1}}{u_k}\right|\le 1-\frac{\beta}{k}$, avec $\beta>1$,
% 
% \change
% alors la série $\sum u_k$ est absolument convergente.
% 
% \change
% Si $\forall k\geq k_0$ on a $\left|\frac{u_{k+1}}{u_k}\right|\ge 1-\frac{1}{k}$, 
% 
% \change
% alors la série $\sum u_k$ n'est  pas absolument convergente.
% 
% \change
% \textbf{Attention !} Il existe des séries convergentes, 
% quoique $\left|\frac{u_{k+1}}{u_k}\right|\ge 1-\frac{1}{k}$.
% 
% Par le deuxième point une telle série ne peut pas être absolument convergente.
% 
% \change
% En effet, prenons $u_k=(-1)^k \frac{1}{k}$. Alors :
% $$\frac{|u_{k+1}|}{|u_k|}= \frac{k}{k+1}= 1-\frac{1}{k+1}\ge 1-\frac{1}{k}.$$
% 
% 
% %%%%%%%%%%%%%%%%%%%%%%%%%%%%%%%%%%%%%%%%%%%%%%%%%%%%%%%%%%%
% \diapo
% 
% Nous pouvons maintenant savoir quelles sont les séries de Riemann
% qui convergent.
% 
% Proposition.
% 
% Soit $\alpha>0$. Alors la série $\sum_{k\ge1} \frac{1}{k^\alpha}$
% converge si et seulement si $\alpha>1$.

%%%%%%%%%%%%%%%%%%%%%%%%%%%%%%%%%%%%%%%%%%%%%%%%%%%%%%%%%%%
\diapo

Voici quelques exercices pour vous familiariser avec les séries absolument convergentes, et appliquer les règles de d'Alembert et de Cauchy.

\end{document}

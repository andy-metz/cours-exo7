
%%%%%%%%%%%%%%%%%% PREAMBULE %%%%%%%%%%%%%%%%%%

\documentclass[aspectratio=169,utf8]{beamer}
%\documentclass[aspectratio=169,handout]{beamer}

\usetheme{Boadilla}
%\usecolortheme{seahorse}
%\usecolortheme[RGB={245,66,24}]{structure}
\useoutertheme{infolines}

% packages
\usepackage{amsfonts,amsmath,amssymb,amsthm}
\usepackage[utf8]{inputenc}
\usepackage[T1]{fontenc}
\usepackage{lmodern}

\usepackage[francais]{babel}
\usepackage{fancybox}
\usepackage{graphicx}

\usepackage{float}
\usepackage{xfrac}

%\usepackage[usenames, x11names]{xcolor}
\usepackage{pgfplots}
\usepackage{datetime}


% ----------------------------------------------------------------------
% Pour les images
\usepackage{tikz}
\usetikzlibrary{calc,shadows,arrows.meta,patterns,matrix}

\newcommand{\tikzinput}[1]{\input{figures/#1.tikz}}
% --- les figures avec échelle éventuel
\newcommand{\myfigure}[2]{% entrée : échelle, fichier(s) figure à inclure
\begin{center}\small%
\tikzstyle{every picture}=[scale=1.0*#1]% mise en échelle + 0% (automatiquement annulé à la fin du groupe)
#2%
\end{center}}



%-----  Package unités -----
\usepackage{siunitx}
\sisetup{locale = FR,detect-all,per-mode = symbol}

%\usepackage{mathptmx}
%\usepackage{fouriernc}
%\usepackage{newcent}
%\usepackage[mathcal,mathbf]{euler}

%\usepackage{palatino}
%\usepackage{newcent}
% \usepackage[mathcal,mathbf]{euler}



% \usepackage{hyperref}
% \hypersetup{colorlinks=true, linkcolor=blue, urlcolor=blue,
% pdftitle={Exo7 - Exercices de mathématiques}, pdfauthor={Exo7}}


%section
% \usepackage{sectsty}
% \allsectionsfont{\bf}
%\sectionfont{\color{Tomato3}\upshape\selectfont}
%\subsectionfont{\color{Tomato4}\upshape\selectfont}

%----- Ensembles : entiers, reels, complexes -----
\newcommand{\Nn}{\mathbb{N}} \newcommand{\N}{\mathbb{N}}
\newcommand{\Zz}{\mathbb{Z}} \newcommand{\Z}{\mathbb{Z}}
\newcommand{\Qq}{\mathbb{Q}} \newcommand{\Q}{\mathbb{Q}}
\newcommand{\Rr}{\mathbb{R}} \newcommand{\R}{\mathbb{R}}
\newcommand{\Cc}{\mathbb{C}} 
\newcommand{\Kk}{\mathbb{K}} \newcommand{\K}{\mathbb{K}}

%----- Modifications de symboles -----
\renewcommand{\epsilon}{\varepsilon}
\renewcommand{\Re}{\mathop{\text{Re}}\nolimits}
\renewcommand{\Im}{\mathop{\text{Im}}\nolimits}
%\newcommand{\llbracket}{\left[\kern-0.15em\left[}
%\newcommand{\rrbracket}{\right]\kern-0.15em\right]}

\renewcommand{\ge}{\geqslant}
\renewcommand{\geq}{\geqslant}
\renewcommand{\le}{\leqslant}
\renewcommand{\leq}{\leqslant}
\renewcommand{\epsilon}{\varepsilon}

%----- Fonctions usuelles -----
\newcommand{\ch}{\mathop{\text{ch}}\nolimits}
\newcommand{\sh}{\mathop{\text{sh}}\nolimits}
\renewcommand{\tanh}{\mathop{\text{th}}\nolimits}
\newcommand{\cotan}{\mathop{\text{cotan}}\nolimits}
\newcommand{\Arcsin}{\mathop{\text{arcsin}}\nolimits}
\newcommand{\Arccos}{\mathop{\text{arccos}}\nolimits}
\newcommand{\Arctan}{\mathop{\text{arctan}}\nolimits}
\newcommand{\Argsh}{\mathop{\text{argsh}}\nolimits}
\newcommand{\Argch}{\mathop{\text{argch}}\nolimits}
\newcommand{\Argth}{\mathop{\text{argth}}\nolimits}
\newcommand{\pgcd}{\mathop{\text{pgcd}}\nolimits} 


%----- Commandes divers ------
\newcommand{\ii}{\mathrm{i}}
\newcommand{\dd}{\text{d}}
\newcommand{\id}{\mathop{\text{id}}\nolimits}
\newcommand{\Ker}{\mathop{\text{Ker}}\nolimits}
\newcommand{\Card}{\mathop{\text{Card}}\nolimits}
\newcommand{\Vect}{\mathop{\text{Vect}}\nolimits}
\newcommand{\Mat}{\mathop{\text{Mat}}\nolimits}
\newcommand{\rg}{\mathop{\text{rg}}\nolimits}
\newcommand{\tr}{\mathop{\text{tr}}\nolimits}


%----- Structure des exercices ------

\newtheoremstyle{styleexo}% name
{2ex}% Space above
{3ex}% Space below
{}% Body font
{}% Indent amount 1
{\bfseries} % Theorem head font
{}% Punctuation after theorem head
{\newline}% Space after theorem head 2
{}% Theorem head spec (can be left empty, meaning ‘normal’)

%\theoremstyle{styleexo}
\newtheorem{exo}{Exercice}
\newtheorem{ind}{Indications}
\newtheorem{cor}{Correction}


\newcommand{\exercice}[1]{} \newcommand{\finexercice}{}
%\newcommand{\exercice}[1]{{\tiny\texttt{#1}}\vspace{-2ex}} % pour afficher le numero absolu, l'auteur...
\newcommand{\enonce}{\begin{exo}} \newcommand{\finenonce}{\end{exo}}
\newcommand{\indication}{\begin{ind}} \newcommand{\finindication}{\end{ind}}
\newcommand{\correction}{\begin{cor}} \newcommand{\fincorrection}{\end{cor}}

\newcommand{\noindication}{\stepcounter{ind}}
\newcommand{\nocorrection}{\stepcounter{cor}}

\newcommand{\fiche}[1]{} \newcommand{\finfiche}{}
\newcommand{\titre}[1]{\centerline{\large \bf #1}}
\newcommand{\addcommand}[1]{}
\newcommand{\video}[1]{}

% Marge
\newcommand{\mymargin}[1]{\marginpar{{\small #1}}}

\def\noqed{\renewcommand{\qedsymbol}{}}


%----- Presentation ------
\setlength{\parindent}{0cm}

%\newcommand{\ExoSept}{\href{http://exo7.emath.fr}{\textbf{\textsf{Exo7}}}}

\definecolor{myred}{rgb}{0.93,0.26,0}
\definecolor{myorange}{rgb}{0.97,0.58,0}
\definecolor{myyellow}{rgb}{1,0.86,0}

\newcommand{\LogoExoSept}[1]{  % input : echelle
{\usefont{U}{cmss}{bx}{n}
\begin{tikzpicture}[scale=0.1*#1,transform shape]
  \fill[color=myorange] (0,0)--(4,0)--(4,-4)--(0,-4)--cycle;
  \fill[color=myred] (0,0)--(0,3)--(-3,3)--(-3,0)--cycle;
  \fill[color=myyellow] (4,0)--(7,4)--(3,7)--(0,3)--cycle;
  \node[scale=5] at (3.5,3.5) {Exo7};
\end{tikzpicture}}
}


\newcommand{\debutmontitre}{
  \author{} \date{} 
  \thispagestyle{empty}
  \hspace*{-10ex}
  \begin{minipage}{\textwidth}
    \titlepage  
  \vspace*{-2.5cm}
  \begin{center}
    \LogoExoSept{2.5}
  \end{center}
  \end{minipage}

  \vspace*{-0cm}
  
  % Astuce pour que le background ne soit pas discrétisé lors de la conversion pdf -> png
\begin{tikzpicture}
        \fill[opacity=0,green!60!black] (0,0)--++(0,0)--++(0,0)--++(0,0)--cycle; 
\end{tikzpicture}

% toc S'affiche trop tot :
% \tableofcontents[hideallsubsections, pausesections]
}

\newcommand{\finmontitre}{
  \end{frame}
  \setcounter{framenumber}{0}
} % ne marche pas pour une raison obscure

%----- Commandes supplementaires ------

% \usepackage[landscape]{geometry}
% \geometry{top=1cm, bottom=3cm, left=2cm, right=10cm, marginparsep=1cm
% }
% \usepackage[a4paper]{geometry}
% \geometry{top=2cm, bottom=2cm, left=2cm, right=2cm, marginparsep=1cm
% }

%\usepackage{standalone}


% New command Arnaud -- november 2011
\setbeamersize{text margin left=24ex}
% si vous modifier cette valeur il faut aussi
% modifier le decalage du titre pour compenser
% (ex : ici =+10ex, titre =-5ex

\theoremstyle{definition}
%\newtheorem{proposition}{Proposition}
%\newtheorem{exemple}{Exemple}
%\newtheorem{theoreme}{Théorème}
%\newtheorem{lemme}{Lemme}
%\newtheorem{corollaire}{Corollaire}
%\newtheorem*{remarque*}{Remarque}
%\newtheorem*{miniexercice}{Mini-exercices}
%\newtheorem{definition}{Définition}

% Commande tikz
\usetikzlibrary{calc}
\usetikzlibrary{patterns,arrows}
\usetikzlibrary{matrix}
\usetikzlibrary{fadings} 

%definition d'un terme
\newcommand{\defi}[1]{{\color{myorange}\textbf{\emph{#1}}}}
\newcommand{\evidence}[1]{{\color{blue}\textbf{\emph{#1}}}}
\newcommand{\assertion}[1]{\emph{\og#1\fg}}  % pour chapitre logique
%\renewcommand{\contentsname}{Sommaire}
\renewcommand{\contentsname}{}
\setcounter{tocdepth}{2}



%------ Encadrement ------

\usepackage{fancybox}


\newcommand{\mybox}[1]{
\setlength{\fboxsep}{7pt}
\begin{center}
\shadowbox{#1}
\end{center}}

\newcommand{\myboxinline}[1]{
\setlength{\fboxsep}{5pt}
\raisebox{-10pt}{
\shadowbox{#1}
}
}

%--------------- Commande beamer---------------
\newcommand{\beameronly}[1]{#1} % permet de mettre des pause dans beamer pas dans poly


\setbeamertemplate{navigation symbols}{}
\setbeamertemplate{footline}  % tiré du fichier beamerouterinfolines.sty
{
  \leavevmode%
  \hbox{%
  \begin{beamercolorbox}[wd=.333333\paperwidth,ht=2.25ex,dp=1ex,center]{author in head/foot}%
    % \usebeamerfont{author in head/foot}\insertshortauthor%~~(\insertshortinstitute)
    \usebeamerfont{section in head/foot}{\bf\insertshorttitle}
  \end{beamercolorbox}%
  \begin{beamercolorbox}[wd=.333333\paperwidth,ht=2.25ex,dp=1ex,center]{title in head/foot}%
    \usebeamerfont{section in head/foot}{\bf\insertsectionhead}
  \end{beamercolorbox}%
  \begin{beamercolorbox}[wd=.333333\paperwidth,ht=2.25ex,dp=1ex,right]{date in head/foot}%
    % \usebeamerfont{date in head/foot}\insertshortdate{}\hspace*{2em}
    \insertframenumber{} / \inserttotalframenumber\hspace*{2ex} 
  \end{beamercolorbox}}%
  \vskip0pt%
}


\definecolor{mygrey}{rgb}{0.5,0.5,0.5}
\setlength{\parindent}{0cm}
%\DeclareTextFontCommand{\helvetica}{\fontfamily{phv}\selectfont}

% background beamer
\definecolor{couleurhaut}{rgb}{0.85,0.9,1}  % creme
\definecolor{couleurmilieu}{rgb}{1,1,1}  % vert pale
\definecolor{couleurbas}{rgb}{0.85,0.9,1}  % blanc
\setbeamertemplate{background canvas}[vertical shading]%
[top=couleurhaut,middle=couleurmilieu,midpoint=0.4,bottom=couleurbas] 
%[top=fondtitre!05,bottom=fondtitre!60]



\makeatletter
\setbeamertemplate{theorem begin}
{%
  \begin{\inserttheoremblockenv}
  {%
    \inserttheoremheadfont
    \inserttheoremname
    \inserttheoremnumber
    \ifx\inserttheoremaddition\@empty\else\ (\inserttheoremaddition)\fi%
    \inserttheorempunctuation
  }%
}
\setbeamertemplate{theorem end}{\end{\inserttheoremblockenv}}

\newenvironment{theoreme}[1][]{%
   \setbeamercolor{block title}{fg=structure,bg=structure!40}
   \setbeamercolor{block body}{fg=black,bg=structure!10}
   \begin{block}{{\bf Th\'eor\`eme }#1}
}{%
   \end{block}%
}


\newenvironment{proposition}[1][]{%
   \setbeamercolor{block title}{fg=structure,bg=structure!40}
   \setbeamercolor{block body}{fg=black,bg=structure!10}
   \begin{block}{{\bf Proposition }#1}
}{%
   \end{block}%
}

\newenvironment{corollaire}[1][]{%
   \setbeamercolor{block title}{fg=structure,bg=structure!40}
   \setbeamercolor{block body}{fg=black,bg=structure!10}
   \begin{block}{{\bf Corollaire }#1}
}{%
   \end{block}%
}

\newenvironment{mydefinition}[1][]{%
   \setbeamercolor{block title}{fg=structure,bg=structure!40}
   \setbeamercolor{block body}{fg=black,bg=structure!10}
   \begin{block}{{\bf Définition} #1}
}{%
   \end{block}%
}

\newenvironment{lemme}[0]{%
   \setbeamercolor{block title}{fg=structure,bg=structure!40}
   \setbeamercolor{block body}{fg=black,bg=structure!10}
   \begin{block}{\bf Lemme}
}{%
   \end{block}%
}

\newenvironment{remarque}[1][]{%
   \setbeamercolor{block title}{fg=black,bg=structure!20}
   \setbeamercolor{block body}{fg=black,bg=structure!5}
   \begin{block}{Remarque #1}
}{%
   \end{block}%
}


\newenvironment{exemple}[1][]{%
   \setbeamercolor{block title}{fg=black,bg=structure!20}
   \setbeamercolor{block body}{fg=black,bg=structure!5}
   \begin{block}{{\bf Exemple }#1}
}{%
   \end{block}%
}


\newenvironment{miniexercice}[0]{%
   \setbeamercolor{block title}{fg=structure,bg=structure!20}
   \setbeamercolor{block body}{fg=black,bg=structure!5}
   \begin{block}{Mini-exercices}
}{%
   \end{block}%
}


\newenvironment{tp}[0]{%
   \setbeamercolor{block title}{fg=structure,bg=structure!40}
   \setbeamercolor{block body}{fg=black,bg=structure!10}
   \begin{block}{\bf Travaux pratiques}
}{%
   \end{block}%
}
\newenvironment{exercicecours}[1][]{%
   \setbeamercolor{block title}{fg=structure,bg=structure!40}
   \setbeamercolor{block body}{fg=black,bg=structure!10}
   \begin{block}{{\bf Exercice }#1}
}{%
   \end{block}%
}
\newenvironment{algo}[1][]{%
   \setbeamercolor{block title}{fg=structure,bg=structure!40}
   \setbeamercolor{block body}{fg=black,bg=structure!10}
   \begin{block}{{\bf Algorithme}\hfill{\color{gray}\texttt{#1}}}
}{%
   \end{block}%
}


\setbeamertemplate{proof begin}{
   \setbeamercolor{block title}{fg=black,bg=structure!20}
   \setbeamercolor{block body}{fg=black,bg=structure!5}
   \begin{block}{{\footnotesize Démonstration}}
   \footnotesize
   \smallskip}
\setbeamertemplate{proof end}{%
   \end{block}}
\setbeamertemplate{qed symbol}{\openbox}


\makeatother
\usecolortheme[RGB={102,102,0}]{structure}
  
%%%%%%%%%%%%%%%%%%%%%%%%%%%%%%%%%%%%%%%%%%%%%%%%%%%%%%%%%%%%%
%%%%%%%%%%%%%%%%%%%%%%%%%%%%%%%%%%%%%%%%%%%%%%%%%%%%%%%%%%%%%


\begin{document}


\title{{\bf Courbes paramétrées}}
\subtitle{Points singuliers -- Branches infinies}

\begin{frame}
  
  \debutmontitre

  \pause

{\footnotesize
\hfill
\setbeamercovered{transparent=50}
\begin{minipage}{0.6\textwidth}
  \begin{itemize}
    \item<3-> Tangente en un point singulier
    \item<4-> Position par rapport à la tangente
    \item<5-> Branches infinies   
  \end{itemize}
\end{minipage}
}

\end{frame}

\setcounter{framenumber}{0}



%%%%%%%%%%%%%%%%%%%%%%%%%%%%%%%%%%%%%%%%%%%%%%%%%%%%%%%%%%%%%%%%
\section{Tangente en un point singulier}

\begin{frame}


\begin{itemize}
  \item $M(t_0)$ d'une courbe paramétrée $M(t) = \big( x(t),y(t)\big)$
est dit \defi{point singulier} si le vecteur dérivé en ce point est nul
\pause  
  \item c'est équivalent à $\overrightarrow{\frac{\dd M}{\dd t}}(t_0)=\vec{0}$
\pause   
  \item ou encore $\big( x'(t_0), y'(t_0)\big) = (0,0)$
\end{itemize}

\pause 
\mybox
{
\begin{minipage}{0.8\textwidth}
\begin{center}
En un point $M(t_0)$ singulier, on étudie 
$\displaystyle \lim_{t\rightarrow t_0}\frac{y(t)-y(t_0)}{x(t)-x(t_0)}$
\pause 
\begin{itemize}
  \item Si cette limite est un réel $\ell$, la tangente en $M(t_0)$ 
existe et a pour coefficient directeur $\ell$
\pause   
  \item Si cette limite existe mais est infinie, la tangente en $M(t_0)$ 
existe et est verticale
\pause 
\end{itemize}
\end{center}  
\end{minipage}
}
\end{frame}


\begin{frame}
\begin{exemple}
Déterminer la tangente en tout point de la courbe $\left\{
\begin{array}{l}
x(t)=3t^2\\
y(t)=2t^3\\
\end{array}\right.$


\begin{minipage}{0.65\textwidth}
\begin{itemize}
  \uncover<2->{\item \textbf{Calcul du vecteur dérivé}  }
  \uncover<3->{\qquad $\overrightarrow{\frac{\dd M}{\dd t}}(t)=\left(\begin{smallmatrix}
6t\\6t^2\end{smallmatrix}\right)$}

  \uncover<4->{\item \textbf{Tangente en $M(t)$, $t\neq0$} }
  \begin{itemize}
    \uncover<5->{\item la tangente dirigée 
  par $\overrightarrow{\frac{\dd M}{\dd t}}(t)
=\left(\begin{smallmatrix}6t\\6t^2\end{smallmatrix}\right)$}
    \uncover<6->{\item donc par $\left(\begin{smallmatrix}1\\t\end{smallmatrix}\right)$}
    \uncover<7->{\item une équation de la tangente en $M(t)$ est donc $t(x-3t^2)-(y-2t^3)=0$}
    \uncover<8->{\item ou encore $y=tx-t^3$}
  \end{itemize}
\end{itemize}
\end{minipage}
\begin{minipage}{0.3\textwidth}
\vspace*{2ex}
\myfigure{0.6}{
\tikzinput{fig_courbes_part3_01}
}  
\end{minipage}

\vspace*{-2ex}
\begin{itemize}
  \uncover<9->{\item \textbf{Tangente en $M(0)$} }
  \begin{itemize}
    \uncover<10->{\item Pour $t\neq 0$, $\frac{y(t)-y(0)}{x(t)-x(0)}=\frac{2t^3}{3t^2}=\frac{2t}{3}\xrightarrow{t\to 0} 0$}
    
    \uncover<11->{\item La tangente en $M(0)$ est l'axe des abscisses}

  \end{itemize}
\end{itemize} 

\end{exemple}

\end{frame}



%%%%%%%%%%%%%%%%%%%%%%%%%%%%%%%%%%%%%%%%%%%%%%%%%%%%%%%%%%%%%%%%
\section{Position d'une courbe par rapport à sa tangente}

\begin{frame}
\myfigure{1.1}{
\begin{tabular}{cc}
\tikzinput{fig_courbes_part3_02}&
\pause
\tikzinput{fig_courbes_part3_03}\\[5mm]
\pause
\tikzinput{fig_courbes_part3_04}&
\pause
\tikzinput{fig_courbes_part3_05}\\
\end{tabular}
}

\end{frame}


\begin{frame}

\mybox{$\displaystyle M(t) = M(0) + t^p \overrightarrow{v} + t^q \overrightarrow{w} +t^q \overrightarrow{\epsilon}(t)$}
\pause\vspace*{-2ex}
{\small
\begin{itemize}
  \item $p<q$ sont des entiers
  \pause\item $\overrightarrow{v}$ et $\overrightarrow{w}$ sont des vecteurs non colinéaires 
  \pause\item $\overrightarrow{\epsilon}(t)$ est un vecteur, tel que $\|\overrightarrow{\epsilon}(t)\| \to 0$ lorsque $t\to t_0$
\end{itemize}
}
\pause\vspace*{-3ex}
\myfigure{0.65}{
\begin{tabular}{cc}
\tikzinput{fig_courbes_part3_06}&\pause\pause
\tikzinput{fig_courbes_part3_07}\\[3mm]\pause
\tikzinput{fig_courbes_part3_08}&\pause
\tikzinput{fig_courbes_part3_09}\\
\end{tabular}
}
\end{frame}


\begin{frame}
\begin{exemple}
\'Etudier le point singulier à l'origine de 
$\left\{\begin{array}{l} x(t) = t^5\\ y(t) = t^3\end{array}\right.$

\pause
%\medskip
\begin{minipage}{0.45\textwidth}
\myfigure{1.1}{
\tikzinput{fig_courbes_part3_10}
}  
\end{minipage}
\begin{minipage}{0.5\textwidth}

\begin{itemize}
  \pause\item $\displaystyle M(t) = t^3 \begin{pmatrix}0\\1\end{pmatrix} + t^5 
\begin{pmatrix}1\\0\end{pmatrix}$
  
  \pause\item Ainsi $p=3$, $q=5$
  
  \pause\item $\overrightarrow{v}=\left(\begin{smallmatrix}0\\1\end{smallmatrix}\right)$,
$\overrightarrow{w}=\left(\begin{smallmatrix}1\\0\end{smallmatrix}\right)$

  \pause\item C'est un point d'inflexion
\end{itemize}

\end{minipage}




\end{exemple}
	
\end{frame}

\begin{frame}
\begin{exemple}
\'Etudier le point singulier à l'origine de 
$\left\{\begin{array}{l} x(t) = 2t^2\\ y(t) = t^2 - t^3\end{array}\right.$

\bigskip

\pause
\begin{minipage}{0.45\textwidth}
\myfigure{0.8}{
\tikzinput{fig_courbes_part3_11}
}  
\end{minipage}
\begin{minipage}{0.5\textwidth}

\begin{itemize}
  \pause\item $M(0)=(0,0)$ est bien un point singulier
  
  \pause\item $\displaystyle M(t) = t^2 \begin{pmatrix}2\\1\end{pmatrix} + t^3 
\begin{pmatrix}0\\-1\end{pmatrix}$ 

  \pause\item Ainsi $p=2$, $q=3$
  
  \pause\item $\overrightarrow{v}=\left(\begin{smallmatrix}2\\1\end{smallmatrix}\right)$,
$\overrightarrow{w}=\left(\begin{smallmatrix}0\\-1\end{smallmatrix}\right)$

  \pause\item C'est un point de rebroussement de première espèce
\end{itemize}

\end{minipage}

\end{exemple}

\end{frame}

%%%%%%%%%%%%%%%%%%%%%%%%%%%%%%%%%%%%%%%%%%%%%%%%%%%%%%%%%%%%%%%%
\section{Branches infinies}

\begin{frame}

\begin{mydefinition}
Il y a \defi{branche infinie} en $t_0$ dès que l'une au moins des deux 
fonctions $|x|$ ou $|y|$ tend vers l'infini quand $t$ tend vers $t_0$ 
\end{mydefinition}
\pause
\begin{enumerate}
\item Si $x(t) \to +\infty$ et $y(t) \to \ell \in \Rr$ (quand $t \to t_0$)
la droite $y=\ell$ est \defi{asymptote horizontale}


\uncover<4->{\item Si $x(t) \to \ell \in \Rr$ et $y(t) \to +\infty$ (quand $t \to t_0$)
la droite d'équation $x=\ell$ est \defi{asymptote verticale} 
} 

\end{enumerate}

\myfigure{0.5}{
\uncover<5->{\tikzinput{fig_courbes_part3_14}}
\uncover<3->{\tikzinput{fig_courbes_part3_15}}
}


\end{frame}


\begin{frame}

\begin{mydefinition}
La droite d'équation 
$y=ax+b$ est \defi{asymptote oblique} à la courbe $\big(x(t),y(t) \big)$ si :

\begin{enumerate}
  \uncover<3->{\item $\frac{y(t)}{x(t)}$ tend vers un réel non nul $a$}

  \uncover<4->{\item $y(t)-ax(t)$ tend vers un réel $b$ (nul ou pas)}
\end{enumerate}
\end{mydefinition}

\bigskip

\begin{minipage}{0.49\textwidth}
\uncover<2->{\myfigure{0.65}{
\tikzinput{fig_courbes_part3_16}
}}
\end{minipage}
\begin{minipage}{0.5\textwidth}
\vspace*{-3ex}
\uncover<5->{\mybox{$\displaystyle \lim_{t \to t_0}(y(t)-(ax(t)+b))=0$}}
\end{minipage}


\end{frame}


\begin{frame}
\begin{exemple}
\vspace*{-0.5ex}
\'Etudier les branches infinies de la courbe
$\left\{\begin{array}{l} x(t) = \frac{t}{t-1} \\ y(t) = \frac{3t}{t^2-1} \end{array}\right.$
%Déterminer la position de la courbe par rapport à ses asymptotes.

\vspace*{-2ex}
\begin{itemize}
 
  \uncover<2->{\item \textbf{\'Etude en $-1^-$}}
  \begin{itemize}
    \uncover<3->{\item Lorsque $t\to-1^-$, $x(t) \to \frac12$ et $y(t) \to -\infty$}
    \uncover<4->{\item Asymptote verticale $x=\frac12$  }
  \end{itemize}
  
  
  \uncover<2->{\item \textbf{\'Etude en $-1^+$}}
   \begin{itemize}
    \uncover<5->{\item Lorsque $t\to-1^+$, $x(t) \to \frac12$ et $y(t) \to +\infty$}
    \uncover<6->{\item Asymptote verticale $x=\frac12$}
  \end{itemize}
  
 
   \uncover<2->{\item \textbf{\'Etude en $+1^-$}}
   \begin{itemize}
     \uncover<7->{\item Lorsque $t\to +1^-$, $x(t) \to -\infty$ et $y(t) \to -\infty$}
     
     \uncover<8->{\item $\displaystyle \frac{y(t)}{x(t)} = \frac{\frac{3t}{t^2-1}}{\frac{t}{t-1}}
  = \frac{3}{t+1} \longrightarrow a=\frac{3}{2} \quad \text{ lorsque }t\to +1^-$}
     
     \uncover<9->{\item $\displaystyle y(t)-\frac32 x(t)  \longrightarrow b=-\frac{3}{4} \quad \text{ lorsque }t\to +1^-$}
     
     \uncover<10->{\item Asymptote $y = \frac32 x -\frac{3}{4}$}
   \end{itemize}
  \uncover<2->{\item \textbf{\'Etude en $+1^+$}} \quad   \uncover<11->{Même asymptote}
\end{itemize}
\vspace*{-1.5ex}
\end{exemple}

\end{frame}

\begin{frame}


\myfigure{0.7}{
\tikzinput{fig_courbes_part3_18}
}  

	
\end{frame}



%%%%%%%%%%%%%%%%%%%%%%%%%%%%%%%%%%%%%%%%%%%%%%%%%%%%%%%%%%%%%%%%
\section{Mini-exercices}

\begin{frame}
\begin{miniexercice}
\begin{enumerate}
  \item Déterminer la tangente et le type de point singulier à l'origine 
  dans chacun des cas :
  $(t^5,t^3+t^4)$, $(t^2-t^3,t^2+t^3)$, $(t^2+t^3,t^4)$, $(t^3,t^6+t^7)$.
 
  \item Trouver les branches infinies de la courbe définie par 
  $x(t) = 1-\frac{1}{1+t^2}$, $y(t)=t$. Déterminer l'asymptote, ainsi que la position
  de la courbe par rapport à cette asymptote.
  
  \item Mêmes questions pour les asymptotes de la courbe définie par 
  $x(t) = \frac{1}{t}+\frac{1}{t-1}$, $y(t)=\frac{1}{t-1}$.
  
  \item Déterminer le type de point singulier de l'astroïde définie par $x(t) = \cos^3 t$,
  $y(t) = \sin^3 t$. Pourquoi l'astroïde n'a-t-elle pas de branche infinie ?
  
  \item Déterminer le type de point singulier de la cycloïde définie par 
  $x(t) = r(t-\sin t)$,  $y(t) = r(1-\cos t)$. Pourquoi la cycloïde
  n'a-t-elle pas d'asymptote ?
\end{enumerate}
\end{miniexercice}
\end{frame}

\end{document}

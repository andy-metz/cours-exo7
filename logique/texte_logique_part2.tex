
%%%%%%%%%%%%%%%%%% PREAMBULE %%%%%%%%%%%%%%%%%%


\documentclass[12pt]{article}

\usepackage{amsfonts,amsmath,amssymb,amsthm}
\usepackage[utf8]{inputenc}
\usepackage[T1]{fontenc}
\usepackage[francais]{babel}


% packages
\usepackage{amsfonts,amsmath,amssymb,amsthm}
\usepackage[utf8]{inputenc}
\usepackage[T1]{fontenc}
%\usepackage{lmodern}

\usepackage[francais]{babel}
\usepackage{fancybox}
\usepackage{graphicx}

\usepackage{float}

%\usepackage[usenames, x11names]{xcolor}
\usepackage{tikz}
\usepackage{datetime}

\usepackage{mathptmx}
%\usepackage{fouriernc}
%\usepackage{newcent}
\usepackage[mathcal,mathbf]{euler}

%\usepackage{palatino}
%\usepackage{newcent}


% Commande spéciale prompteur

%\usepackage{mathptmx}
%\usepackage[mathcal,mathbf]{euler}
%\usepackage{mathpple,multido}

\usepackage[a4paper]{geometry}
\geometry{top=2cm, bottom=2cm, left=1cm, right=1cm, marginparsep=1cm}

\newcommand{\change}{{\color{red}\rule{\textwidth}{1mm}\\}}

\newcounter{mydiapo}

\newcommand{\diapo}{\newpage
\hfill {\normalsize  Diapo \themydiapo \quad \texttt{[\jobname]}} \\
\stepcounter{mydiapo}}


%%%%%%% COULEURS %%%%%%%%%%

% Pour blanc sur noir :
%\pagecolor[rgb]{0.5,0.5,0.5}
% \pagecolor[rgb]{0,0,0}
% \color[rgb]{1,1,1}



%\DeclareFixedFont{\myfont}{U}{cmss}{bx}{n}{18pt}
\newcommand{\debuttexte}{
%%%%%%%%%%%%% FONTES %%%%%%%%%%%%%
\renewcommand{\baselinestretch}{1.5}
\usefont{U}{cmss}{bx}{n}
\bfseries

% Taille normale : commenter le reste !
%Taille Arnaud
%\fontsize{19}{19}\selectfont

% Taille Barbara
%\fontsize{21}{22}\selectfont

%Taille François
\fontsize{25}{30}\selectfont

%Taille Pascal
%\fontsize{25}{30}\selectfont

%Taille Laura
%\fontsize{30}{35}\selectfont


%\myfont
%\usefont{U}{cmss}{bx}{n}

%\Huge
%\addtolength{\parskip}{\baselineskip}
}


% \usepackage{hyperref}
% \hypersetup{colorlinks=true, linkcolor=blue, urlcolor=blue,
% pdftitle={Exo7 - Exercices de mathématiques}, pdfauthor={Exo7}}


%section
% \usepackage{sectsty}
% \allsectionsfont{\bf}
%\sectionfont{\color{Tomato3}\upshape\selectfont}
%\subsectionfont{\color{Tomato4}\upshape\selectfont}

%----- Ensembles : entiers, reels, complexes -----
\newcommand{\Nn}{\mathbb{N}} \newcommand{\N}{\mathbb{N}}
\newcommand{\Zz}{\mathbb{Z}} \newcommand{\Z}{\mathbb{Z}}
\newcommand{\Qq}{\mathbb{Q}} \newcommand{\Q}{\mathbb{Q}}
\newcommand{\Rr}{\mathbb{R}} \newcommand{\R}{\mathbb{R}}
\newcommand{\Cc}{\mathbb{C}} 
\newcommand{\Kk}{\mathbb{K}} \newcommand{\K}{\mathbb{K}}

%----- Modifications de symboles -----
\renewcommand{\epsilon}{\varepsilon}
\renewcommand{\Re}{\mathop{\text{Re}}\nolimits}
\renewcommand{\Im}{\mathop{\text{Im}}\nolimits}
%\newcommand{\llbracket}{\left[\kern-0.15em\left[}
%\newcommand{\rrbracket}{\right]\kern-0.15em\right]}

\renewcommand{\ge}{\geqslant}
\renewcommand{\geq}{\geqslant}
\renewcommand{\le}{\leqslant}
\renewcommand{\leq}{\leqslant}

%----- Fonctions usuelles -----
\newcommand{\ch}{\mathop{\mathrm{ch}}\nolimits}
\newcommand{\sh}{\mathop{\mathrm{sh}}\nolimits}
\renewcommand{\tanh}{\mathop{\mathrm{th}}\nolimits}
\newcommand{\cotan}{\mathop{\mathrm{cotan}}\nolimits}
\newcommand{\Arcsin}{\mathop{\mathrm{Arcsin}}\nolimits}
\newcommand{\Arccos}{\mathop{\mathrm{Arccos}}\nolimits}
\newcommand{\Arctan}{\mathop{\mathrm{Arctan}}\nolimits}
\newcommand{\Argsh}{\mathop{\mathrm{Argsh}}\nolimits}
\newcommand{\Argch}{\mathop{\mathrm{Argch}}\nolimits}
\newcommand{\Argth}{\mathop{\mathrm{Argth}}\nolimits}
\newcommand{\pgcd}{\mathop{\mathrm{pgcd}}\nolimits} 

\newcommand{\Card}{\mathop{\text{Card}}\nolimits}
\newcommand{\Ker}{\mathop{\text{Ker}}\nolimits}
\newcommand{\id}{\mathop{\text{id}}\nolimits}
\newcommand{\ii}{\mathrm{i}}
\newcommand{\dd}{\mathrm{d}}
\newcommand{\Vect}{\mathop{\text{Vect}}\nolimits}
\newcommand{\Mat}{\mathop{\mathrm{Mat}}\nolimits}
\newcommand{\rg}{\mathop{\text{rg}}\nolimits}
\newcommand{\tr}{\mathop{\text{tr}}\nolimits}
\newcommand{\ppcm}{\mathop{\text{ppcm}}\nolimits}

%----- Structure des exercices ------

\newtheoremstyle{styleexo}% name
{2ex}% Space above
{3ex}% Space below
{}% Body font
{}% Indent amount 1
{\bfseries} % Theorem head font
{}% Punctuation after theorem head
{\newline}% Space after theorem head 2
{}% Theorem head spec (can be left empty, meaning ‘normal’)

%\theoremstyle{styleexo}
\newtheorem{exo}{Exercice}
\newtheorem{ind}{Indications}
\newtheorem{cor}{Correction}


\newcommand{\exercice}[1]{} \newcommand{\finexercice}{}
%\newcommand{\exercice}[1]{{\tiny\texttt{#1}}\vspace{-2ex}} % pour afficher le numero absolu, l'auteur...
\newcommand{\enonce}{\begin{exo}} \newcommand{\finenonce}{\end{exo}}
\newcommand{\indication}{\begin{ind}} \newcommand{\finindication}{\end{ind}}
\newcommand{\correction}{\begin{cor}} \newcommand{\fincorrection}{\end{cor}}

\newcommand{\noindication}{\stepcounter{ind}}
\newcommand{\nocorrection}{\stepcounter{cor}}

\newcommand{\fiche}[1]{} \newcommand{\finfiche}{}
\newcommand{\titre}[1]{\centerline{\large \bf #1}}
\newcommand{\addcommand}[1]{}
\newcommand{\video}[1]{}

% Marge
\newcommand{\mymargin}[1]{\marginpar{{\small #1}}}



%----- Presentation ------
\setlength{\parindent}{0cm}

%\newcommand{\ExoSept}{\href{http://exo7.emath.fr}{\textbf{\textsf{Exo7}}}}

\definecolor{myred}{rgb}{0.93,0.26,0}
\definecolor{myorange}{rgb}{0.97,0.58,0}
\definecolor{myyellow}{rgb}{1,0.86,0}

\newcommand{\LogoExoSept}[1]{  % input : echelle
{\usefont{U}{cmss}{bx}{n}
\begin{tikzpicture}[scale=0.1*#1,transform shape]
  \fill[color=myorange] (0,0)--(4,0)--(4,-4)--(0,-4)--cycle;
  \fill[color=myred] (0,0)--(0,3)--(-3,3)--(-3,0)--cycle;
  \fill[color=myyellow] (4,0)--(7,4)--(3,7)--(0,3)--cycle;
  \node[scale=5] at (3.5,3.5) {Exo7};
\end{tikzpicture}}
}



\theoremstyle{definition}
%\newtheorem{proposition}{Proposition}
%\newtheorem{exemple}{Exemple}
%\newtheorem{theoreme}{Théorème}
\newtheorem{lemme}{Lemme}
\newtheorem{corollaire}{Corollaire}
%\newtheorem*{remarque*}{Remarque}
%\newtheorem*{miniexercice}{Mini-exercices}
%\newtheorem{definition}{Définition}




%definition d'un terme
\newcommand{\defi}[1]{{\color{myorange}\textbf{\emph{#1}}}}
\newcommand{\evidence}[1]{{\color{blue}\textbf{\emph{#1}}}}



 %----- Commandes divers ------

\newcommand{\codeinline}[1]{\texttt{#1}}

\renewcommand{\implies}{\; \Rightarrow\; } %\implies apparait mal (de meme que Longrightarrow)

%%%%%%%%%%%%%%%%%%%%%%%%%%%%%%%%%%%%%%%%%%%%%%%%%%%%%%%%%%%%%
%%%%%%%%%%%%%%%%%%%%%%%%%%%%%%%%%%%%%%%%%%%%%%%%%%%%%%%%%%%%%



\begin{document}

\debuttexte

%%%%%%%%%%%%%%%%%%%%%%%%%%%%%%%%%%%%%%%%%%%%%%%%%%%%%%%%%%%
\diapo

\change

Nous allons aborder plusieurs type de raisonnements.

\change

Tout d'abord le plus classique est le raisonnement direct.

\change

ensuite nous verrons le raisonnement au cas par cas,

\change

puis celui par contraposition


\change

et le raisonnement par l'absurde.

\change

Nous verrons aussi la notion de contre-exemple,

\change 

et terminons par le raisonnement par récurrence.


%%%%%%%%%%%%%%%%%%%%%%%%%%%%%%%%%%%%%%%%%%%%%%%%%%%%%%%%%%%
\diapo

Le raisonnement direct est celui que vous utilisez le plus souvent.

Pour montrer que l'assertion \og $P \implies Q$\fg\ est vraie :

on suppose que $P$ est vraie et on doit montrer qu'alors $Q$ est vraie.

\change

Par exemple on nous demande de montrer que si
$a,b$ sont deux nombres rationnels alors 
$a+b$ est encore un nombre rationnel.

\change

Le schéma de la démonstration est le suivant :

On suppose que $a$ et $b$ sont des nombres rationnels.

On utilises ces hypothèses et on fait un petit calcul.

\change

On arrive à montrer que $a+b$ est bien un nombre rationnel.


\change

Complétons le raisonnement :

être un rationnel signifie par définition que $a$ est le quotient de deux entiers.

Même chose pour $b$.

On écrit alors $a+b$ comme la somme de deux fractions,
que l'on réduit au même dénominateur.

Maintenant $a+b$ est bien un entier
divisé par un entier,
c'est donc bien un nombre rationnel.


%%%%%%%%%%%%%%%%%%%%%%%%%%%%%%%%%%%%%%%%%%%%%%%%%%%%%%%%%%%
\diapo

Passons à la démonstration au cas par cas.
Pour démontrer qu'une assertion $P(x)$ est vraie
pour tous les $x$ d'un ensemble $E$.

On le démontre d'abord pour les $x$ appartenant à un sous-ensemble 
$A$ puis pour tous les autres.

\change

Voyons ce que cela donne sur un exemple :

on doit montrer que pour tout réel, $|x-1| \le x^2-x+1$.


\change

Comme c'est souvent la cas avec la valeur absolue on distingue deux cas :

\change

Tout d'abord le cas où $x \ge 1$

\change 

Puis le cas où $x<1$. 

On aura donc traité tous les $x$ réels.

\change

Voyons le premier cas.

Comme $x\ge 1$ alors $x-1 \ge 0$ donc $|x-1|=+(x-1)$.

On peut donc calculer la différence :

 $ x^2-x+1 - |x-1|=$
 
$=   x^2-x+1 - (x-1)$
   
   $=x^2 -2x + 2$
   
   $=(x-1)^2 + 1$


qui est bien positive.

\change

Reste le cas où $x < 1$.

Alors $x-1 < 0$ et donc $|x-1|=-(x-1)$.

La différence

 $x^2-x+1 - |x-1| = x^2-x+1 + (x-1) = x^2$

est toujours positive dans ce cas aussi.

\change

La conclusion est toujours la même et comme on a traité tous les cas possibles
alors on a bien montré que 
la différence $x^2-x+1 - |x-1|$ est positive 
pour tous les réels $x$ ou ce qui est équivalent  $|x-1| \le x^2-x+1$.


%%%%%%%%%%%%%%%%%%%%%%%%%%%%%%%%%%%%%%%%%%%%%%%%%%%%%%%%%%%
\diapo

La contraposition est basée sur l'équivalence entre 
l'assertion 

\og $P \implies Q$\fg\


et l'assertion  \og $\text{non}(Q) \implies \text{non}(P)$\fg

\change

Pour montrer qu'une assertion du type \og $P \implies Q$\fg\
est vraie. On part de l'hypothèse que $\text{non}(Q)$
est vraie 

et on montre qu'alors $\text{non}(P)$ est vraie.

\change

Appliquons cette méthode pour démontrer que si
$n^2$ est pair alors $n$ est pair



\change

Voici le schéma de la démonstration :

Nous supposons donc que $n$ n'est pas pair

\change

et nous devons aboutir à $n^2$ n'est pas pair.

\change

En détails : comme
nous supposons que $n$ n'est pas pair alors
$n$ est impair.

$n$ s'écrit donc sous la forme $2k+1$

Calculons $n^2$, c'est donc $(2k+1)^2$ 

et en développant on trouve $4k^2 + 4k + 1$

qui est bien un nombre impair.

Donc $n^2$ est bien impair.

\change

Nous avons montré $n$ non pair implique $n^2$ non pair.

Par contraposition nous avons montré l'assertion équivalente
$n^2$ pair implique $n$ pair.



%%%%%%%%%%%%%%%%%%%%%%%%%%%%%%%%%%%%%%%%%%%%%%%%%%%%%%%%%%%
\diapo

Une variante du raisonnement par contraposition est le raisonnement par l'absurde.

Pour montrer  l'assertion \og $P \implies Q$\fg\ 

on suppose à la fois que $P$ est vraie et que $Q$ est fausse 
et on cherche une contradiction

\change

Montrons par exemple que pour deux réels \textbf{positifs} $a$ et $b$ : si $\frac{a}{1+b}=\frac{b}{1+a}$
alors $a=b$

\change

Nous raisonnons par l'absurde en supposant 
que  $\frac{a}{1+b}=\frac{b}{1+a}$ \textbf{ET} que $a \neq b$


\change

Nous devons aboutir à une contradiction.


\change 

Le raisonnement se déroule alors ainsi 

Comme  $\frac{a}{1+b}=\frac{b}{1+a}$

 Alors $a(1+a)=b(1+b)$, donc $a+a^2=b+b^2$, 

on en déduit $a^2-b^2 = b-a$ et donc $(a-b)(a+b)= -(a-b)$


Maintenant on sait que $a \neq b$ donc on peut diviser par $a-b$ qui est non nul.

On obtient  $a+b=-1$

Mais $a$ et $b$ sont positif, leur somme ne peut donner un nombre négatif.

On obtient donc la contradiction\\
recherchée

\change

En conclusion on bien montré que
si $\frac{a}{1+b}=\frac{b}{1+a}$ 
alors cela implique que $a=b$

%%%%%%%%%%%%%%%%%%%%%%%%%%%%%%%%%%%%%%%%%%%%%%%%%%%%%%%%%%%
\diapo


On souhaite parfois montrer qu'une assertion du type 

\og $\forall x \in E \quad P(x)$\fg\ est fausse

cela revient donc à montrer que sa négation est vraie :

\og $\exists x \in E \quad \text{non} P(x)$\fg\ 

Exhiber un tel $x$ c'est produire un contre-exemple.

\change
Par exemple l'assertion 

\og Tout entier positif est somme de trois carrés\fg

est fausse

\change

On appelle un carré tout entier qui est le carré d'un autre entier.

On peut commencer à tester cette assertion sur les plus petits entiers

par exemple $6$ est bien la somme de trois carrés.


\change

Cependant ce n'est pas vrai pour tous les entiers.

Par exemple $7$ n'est pas la somme de trois carrés

\change


en effet les seuls carrés qui pourraient apparaître sont
$0^2$, $1^2$, $2^2$ (les autres sont plus grand que $7$)

Mais avec ces trois carrés on ne peut arriver à une somme 
égale à $7$.

Il n'est donc pas vrai que tout entier est somme de trois carrés.

Évidemment rien ne vous oblige à\\
choisir $7$ comme contre-exemple,

$15$ conviendrait également.


%%%%%%%%%%%%%%%%%%%%%%%%%%%%%%%%%%%%%%%%%%%%%%%%%%%%%%%%%%%
\diapo

Le principe de récurrence permet de démontrer
qu'une assertion $P(n)$ est vraie pour tout entier $n$.

La rédaction d'une récurrence est assez rigide et se déroule en
trois étapes.


\change

Tout d'abord l'étape d'initialisation 

on prouve que l'assertion est vrai au rang $0$

\change

Ensuite l'étape d'hérédité

Pour laquelle 
  \begin{itemize}
    \item on fixe un entier $n\ge 0$
    \item on fait l'hypothèse que pour ce $n$, $P(n)$ est vraie
    \item on doit alors démontrer que $P(n+1)$ est vraie
   \end{itemize}

\change

On n'oublie pas de conclure que par le principe de récurrence,

$P(n)$ est vraie pour tous les entiers $n$


%%%%%%%%%%%%%%%%%%%%%%%%%%%%%%%%%%%%%%%%%%%%%%%%%%%%%%%%%%%
\diapo


Nous allons montrer par récurrence que 
$2^n > n$ pour tout les entiers $n$


\change

Nous notons $P(n)$ l'assertion $2^n > n$ 

Pour l'initialisation, on vérifie que pour $n=0$ on a bien
$2^0$ qui vaut $1$ est donc strictement plus grand que $0$

\change

Pour l'étape d'hérédité, on fixe un entier $n$
on suppose que l'assertion $P$ est vraie au rang $n$
et on doit démontrer quelle est vraie au rang suivant.

\change

Le calcul se déroule ainsi :

$$\begin{array}{rcl}
 2^{n+1} &=& 2^n + 2^n \\
         &>& n + 2^n \qquad \text{ car par } P(n) : 2^n > n \\
         &>& n + 1 \qquad \text{ car } 2^n \ge 1 \\
\end{array}$$

et donc l'assertion $P(n+1)$ est vraie.


\change


Il ne reste qu'à conclure : Par le principe de récurrence 

$P(n)$ est vraie pour tout $n$

c'est-à-dire $2^n > n$ pour tout $n$.




%%%%%%%%%%%%%%%%%%%%%%%%%%%%%%%%%%%%%%%%%%%%%%%%%%%%%%%%%%%
\diapo

Appliquez ce que venez d'apprendre pour démontrer 
les petites propriétés suivantes.


\end{document}
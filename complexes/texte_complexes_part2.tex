
%%%%%%%%%%%%%%%%%% PREAMBULE %%%%%%%%%%%%%%%%%%


\documentclass[12pt]{article}

\usepackage{amsfonts,amsmath,amssymb,amsthm}
\usepackage[utf8]{inputenc}
\usepackage[T1]{fontenc}
\usepackage[francais]{babel}


% packages
\usepackage{amsfonts,amsmath,amssymb,amsthm}
\usepackage[utf8]{inputenc}
\usepackage[T1]{fontenc}
%\usepackage{lmodern}

\usepackage[francais]{babel}
\usepackage{fancybox}
\usepackage{graphicx}

\usepackage{float}

%\usepackage[usenames, x11names]{xcolor}
\usepackage{tikz}
\usepackage{datetime}

\usepackage{mathptmx}
%\usepackage{fouriernc}
%\usepackage{newcent}
\usepackage[mathcal,mathbf]{euler}

%\usepackage{palatino}
%\usepackage{newcent}


% Commande spéciale prompteur

%\usepackage{mathptmx}
%\usepackage[mathcal,mathbf]{euler}
%\usepackage{mathpple,multido}

\usepackage[a4paper]{geometry}
\geometry{top=2cm, bottom=2cm, left=1cm, right=1cm, marginparsep=1cm}

\newcommand{\change}{{\color{red}\rule{\textwidth}{1mm}\\}}

\newcounter{mydiapo}

\newcommand{\diapo}{\newpage
\hfill {\normalsize  Diapo \themydiapo \quad \texttt{[\jobname]}} \\
\stepcounter{mydiapo}}


%%%%%%% COULEURS %%%%%%%%%%

% Pour blanc sur noir :
%\pagecolor[rgb]{0.5,0.5,0.5}
% \pagecolor[rgb]{0,0,0}
% \color[rgb]{1,1,1}



%\DeclareFixedFont{\myfont}{U}{cmss}{bx}{n}{18pt}
\newcommand{\debuttexte}{
%%%%%%%%%%%%% FONTES %%%%%%%%%%%%%
\renewcommand{\baselinestretch}{1.5}
\usefont{U}{cmss}{bx}{n}
\bfseries

% Taille normale : commenter le reste !
%Taille Arnaud
%\fontsize{19}{19}\selectfont

% Taille Barbara
%\fontsize{21}{22}\selectfont

%Taille François
\fontsize{25}{30}\selectfont

%Taille Pascal
%\fontsize{25}{30}\selectfont

%Taille Laura
%\fontsize{30}{35}\selectfont


%\myfont
%\usefont{U}{cmss}{bx}{n}

%\Huge
%\addtolength{\parskip}{\baselineskip}
}


% \usepackage{hyperref}
% \hypersetup{colorlinks=true, linkcolor=blue, urlcolor=blue,
% pdftitle={Exo7 - Exercices de mathématiques}, pdfauthor={Exo7}}


%section
% \usepackage{sectsty}
% \allsectionsfont{\bf}
%\sectionfont{\color{Tomato3}\upshape\selectfont}
%\subsectionfont{\color{Tomato4}\upshape\selectfont}

%----- Ensembles : entiers, reels, complexes -----
\newcommand{\Nn}{\mathbb{N}} \newcommand{\N}{\mathbb{N}}
\newcommand{\Zz}{\mathbb{Z}} \newcommand{\Z}{\mathbb{Z}}
\newcommand{\Qq}{\mathbb{Q}} \newcommand{\Q}{\mathbb{Q}}
\newcommand{\Rr}{\mathbb{R}} \newcommand{\R}{\mathbb{R}}
\newcommand{\Cc}{\mathbb{C}} 
\newcommand{\Kk}{\mathbb{K}} \newcommand{\K}{\mathbb{K}}

%----- Modifications de symboles -----
\renewcommand{\epsilon}{\varepsilon}
\renewcommand{\Re}{\mathop{\text{Re}}\nolimits}
\renewcommand{\Im}{\mathop{\text{Im}}\nolimits}
%\newcommand{\llbracket}{\left[\kern-0.15em\left[}
%\newcommand{\rrbracket}{\right]\kern-0.15em\right]}

\renewcommand{\ge}{\geqslant}
\renewcommand{\geq}{\geqslant}
\renewcommand{\le}{\leqslant}
\renewcommand{\leq}{\leqslant}

%----- Fonctions usuelles -----
\newcommand{\ch}{\mathop{\mathrm{ch}}\nolimits}
\newcommand{\sh}{\mathop{\mathrm{sh}}\nolimits}
\renewcommand{\tanh}{\mathop{\mathrm{th}}\nolimits}
\newcommand{\cotan}{\mathop{\mathrm{cotan}}\nolimits}
\newcommand{\Arcsin}{\mathop{\mathrm{Arcsin}}\nolimits}
\newcommand{\Arccos}{\mathop{\mathrm{Arccos}}\nolimits}
\newcommand{\Arctan}{\mathop{\mathrm{Arctan}}\nolimits}
\newcommand{\Argsh}{\mathop{\mathrm{Argsh}}\nolimits}
\newcommand{\Argch}{\mathop{\mathrm{Argch}}\nolimits}
\newcommand{\Argth}{\mathop{\mathrm{Argth}}\nolimits}
\newcommand{\pgcd}{\mathop{\mathrm{pgcd}}\nolimits} 

\newcommand{\Card}{\mathop{\text{Card}}\nolimits}
\newcommand{\Ker}{\mathop{\text{Ker}}\nolimits}
\newcommand{\id}{\mathop{\text{id}}\nolimits}
\newcommand{\ii}{\mathrm{i}}
\newcommand{\dd}{\mathrm{d}}
\newcommand{\Vect}{\mathop{\text{Vect}}\nolimits}
\newcommand{\Mat}{\mathop{\mathrm{Mat}}\nolimits}
\newcommand{\rg}{\mathop{\text{rg}}\nolimits}
\newcommand{\tr}{\mathop{\text{tr}}\nolimits}
\newcommand{\ppcm}{\mathop{\text{ppcm}}\nolimits}

%----- Structure des exercices ------

\newtheoremstyle{styleexo}% name
{2ex}% Space above
{3ex}% Space below
{}% Body font
{}% Indent amount 1
{\bfseries} % Theorem head font
{}% Punctuation after theorem head
{\newline}% Space after theorem head 2
{}% Theorem head spec (can be left empty, meaning ‘normal’)

%\theoremstyle{styleexo}
\newtheorem{exo}{Exercice}
\newtheorem{ind}{Indications}
\newtheorem{cor}{Correction}


\newcommand{\exercice}[1]{} \newcommand{\finexercice}{}
%\newcommand{\exercice}[1]{{\tiny\texttt{#1}}\vspace{-2ex}} % pour afficher le numero absolu, l'auteur...
\newcommand{\enonce}{\begin{exo}} \newcommand{\finenonce}{\end{exo}}
\newcommand{\indication}{\begin{ind}} \newcommand{\finindication}{\end{ind}}
\newcommand{\correction}{\begin{cor}} \newcommand{\fincorrection}{\end{cor}}

\newcommand{\noindication}{\stepcounter{ind}}
\newcommand{\nocorrection}{\stepcounter{cor}}

\newcommand{\fiche}[1]{} \newcommand{\finfiche}{}
\newcommand{\titre}[1]{\centerline{\large \bf #1}}
\newcommand{\addcommand}[1]{}
\newcommand{\video}[1]{}

% Marge
\newcommand{\mymargin}[1]{\marginpar{{\small #1}}}



%----- Presentation ------
\setlength{\parindent}{0cm}

%\newcommand{\ExoSept}{\href{http://exo7.emath.fr}{\textbf{\textsf{Exo7}}}}

\definecolor{myred}{rgb}{0.93,0.26,0}
\definecolor{myorange}{rgb}{0.97,0.58,0}
\definecolor{myyellow}{rgb}{1,0.86,0}

\newcommand{\LogoExoSept}[1]{  % input : echelle
{\usefont{U}{cmss}{bx}{n}
\begin{tikzpicture}[scale=0.1*#1,transform shape]
  \fill[color=myorange] (0,0)--(4,0)--(4,-4)--(0,-4)--cycle;
  \fill[color=myred] (0,0)--(0,3)--(-3,3)--(-3,0)--cycle;
  \fill[color=myyellow] (4,0)--(7,4)--(3,7)--(0,3)--cycle;
  \node[scale=5] at (3.5,3.5) {Exo7};
\end{tikzpicture}}
}



\theoremstyle{definition}
%\newtheorem{proposition}{Proposition}
%\newtheorem{exemple}{Exemple}
%\newtheorem{theoreme}{Théorème}
\newtheorem{lemme}{Lemme}
\newtheorem{corollaire}{Corollaire}
%\newtheorem*{remarque*}{Remarque}
%\newtheorem*{miniexercice}{Mini-exercices}
%\newtheorem{definition}{Définition}




%definition d'un terme
\newcommand{\defi}[1]{{\color{myorange}\textbf{\emph{#1}}}}
\newcommand{\evidence}[1]{{\color{blue}\textbf{\emph{#1}}}}



 %----- Commandes divers ------

\newcommand{\codeinline}[1]{\texttt{#1}}

%%%%%%%%%%%%%%%%%%%%%%%%%%%%%%%%%%%%%%%%%%%%%%%%%%%%%%%%%%%%%
%%%%%%%%%%%%%%%%%%%%%%%%%%%%%%%%%%%%%%%%%%%%%%%%%%%%%%%%%%%%%



\begin{document}

\debuttexte
%%%%%%%%%%%%%%%%%%%%%%%%%%%%%%%%%%%%%%%%%%%%%%%%%%%%%%%%%%%
\diapo

\change

Dans cette leçon 

\change

nous allons définir et calculer 
les racines carrées d'un nombre complexe

\change

Cela nous permettra ensuite de résoudre
les équations du second degré

qu'elles soient a coefficients réels ou complexes



%%%%%%%%%%%%%%%%%%%%%%%%%%%%%%%%%%%%%%%%%%%%%%%%%%%%%%%%%%%
\diapo

Par définition un racine carré du nombre complexe $z$
est un nombre complexe $\omega$ tel que $\omega^2=z$

\change

nous allons voir qu'un nombre complexe $z$ admet deux
racines carrées

et si $\omega$ est une racine carrée alors l'autre racine carrée
est $-\omega$

\change

Voici quelques exemples 

Pour un réel positif $x$ ses racines carrées sont $+\sqrt x$ et $- \sqrt x$

\change

Les racines carrées de $-1$ sont $+\ii$ et $-\ii$

\change

Enfin, vérifiez par le calcul -en élevant au carré- que les racines carrées de $\ii$
sont $\frac{\sqrt{2}}{2}(1+\ii)$ et son opposé


%%%%%%%%%%%%%%%%%%%%%%%%%%%%%%%%%%%%%%%%%%%%%%%%%%%%%%%%%%%
\diapo

Nous expliquons maintenant comment calculer les racines carrées d'un nombre complexe $z$



On cherche donc $\omega$ sous la forme $x+\ii y$ 

\change

L'équation $\omega^2=z$

\change

est équivalente à $\left( x + \ii y \right)^2 = a + \ii b$

\change

on développe 

 et on identifie les parties réelles entre elles et les parties imaginaires entre elles

\change

Il y a ensuite une astuce à retenir, comme
$\omega^2=z$ alors  $|\omega|^2 = \|z\|$ 

cela ne change rien aux équivalences...

\change

...si l'on rajoute l'équation
$x^2+y^2$ (qui est égale à $|\omega|^2$) 
égal $\sqrt{a^2+b^2}$ (qui est égal à $|z|$)

\change

En ajoutant la première et la troisième ligne on obtient 

$2 x^2 = \sqrt{a^2 + b^2} + a$

et avec la troisième ligne moins la première on obtient

$2 y^2 = \sqrt{a^2 + b^2} - a$

\change

Cela permet d'obtenir $x$ et $y$ aux signes près

\change

Pour obtenir les bons signes on utilise l'équation
$2xy=b$

\change

Si $b$ est positif alors $x$ et $y$ sont de même signe

donc si on prend $+$ pour $x$ il faut prendre $+$ pour $y$

si on prend $-$ pour $x$ il faut prendre $-$ pour $y$

on obtient bien les deux racines carrées cherchées

\change

Si $b$ est négatif alors $x$ et $y$ sont de signes contraires

et on trouve ceci pour les racines carrées




%%%%%%%%%%%%%%%%%%%%%%%%%%%%%%%%%%%%%%%%%%%%%%%%%%%%%%%%%%%
\diapo

Il n'est pas nécessaire d'apprendre les formules.

Cela va très vite de refaire les calculs à chaque fois.

Par exemple calculons les racines carrées de $\ii$

\change

on cherche $\omega$ tel que $\omega^2=\ii$

\change

c'est-à-dire $( x + \ii y )^2 = \ii$

ce qui donne en identifiant les parties réelles $x^2-y^2=0$ 

et en identifiant les parties imaginaires $2xy=1$

\change

on n'oublie pas de rajouter l'égalité provenant des modules
ici $x^2+y^2$ égale le module de $\ii$ donc $1$


\change

par somme et différence de ces deux équations
cela nous donne les trois équations suivantes

\change

ce qui permet de trouver $x$ et $y$ au signe près

\change

l'équation $2xy=1$ nous indique que $x$ et $y$ sont de même
signe et on obtient exactement deux solutions

\change

en écrivant $\frac 1 {\sqrt 2}$ égal $\frac{\sqrt{2}}{2}$ on obtient
que les racines carrées de $\ii$ sont $\frac{\sqrt{2}}{2}(1+\ii)$ et $-\frac{\sqrt{2}}{2}(1+\ii)$


%%%%%%%%%%%%%%%%%%%%%%%%%%%%%%%%%%%%%%%%%%%%%%%%%%%%%%%%%%%
\diapo
Nous avons tous les ingrédients pour résoudre
 l'équation $az^2 + bz + c = 0$
d'inconnue $z$

où les coefficients $a, b, c$ sont des nombres complexes

Cette équation possède deux solutions 

 $ z_1 = \dfrac{- b + \delta}{2 a} \quad \text{ et } \quad z_2 = \dfrac{- b - \delta}{2a}$

\change

Qu'est ce que ce petit $\delta$ ?

On calcule d'abord le discriminant, grand $\Delta = b^2-4ac$

on calcule ensuite une racine carrée  de grand $\Delta$

on appelle cette racine carrée petit $\delta$

\change

Par exemple pour l'équation de degré deux à coefficients complexes 
$z^2 + z + \frac{1 - \ii }{4}=0$

\change

le discriminant grand $\Delta$ égal $b^2-4ac$ vaut ici $+\ii $ 

\change

dont on a déjà calculé une racine carrée.

On prend n'importe laquelle de ses racines carrées,

par exemple petit $\delta = \frac{\sqrt2}{2}(1 +  \ii)$ 

\change

les deux solutions sont alors $z = \dfrac{- 1 \pm \frac{\sqrt2}{2}(1 +  \ii)}{2}$


\change

Si $a, b, c$ sont des réels on retrouve les formules que vous connaissez deja


\change

Si grand $\Delta=0$ alors les deux solutions $z_1$ et $z_2$ sont confondues


%%%%%%%%%%%%%%%%%%%%%%%%%%%%%%%%%%%%%%%%%%%%%%%%%%%%%%%%%%%
\diapo

\change

La démonstration pour trouver $z_1$ et $z_2$ est la suivante

On factorise par $a$ et on reconnaît dans le terme $z^2 + \frac ba z$ le début du carré
$(z+\frac {b}{2a})^2$

Il faut ensuite corriger en soustrayant par $-\frac{b^2}{4a^2}$

\change

Ensuite on reconnaît au numérateur le discriminant grand $ \Delta$
qui est aussi petit $\delta$ au carré

\change

Nous sommes en présence d'une identité remarquable du type 
$a^2-b^2 = (a-b)(a+b)$

qui nous permet de factoriser.

\change

Maintenant notre polynôme $az^2+bz+c$ s'annule 
si et seulement si un des facteurs est nul

donc si et seulement si $z=z_1$ ou $z=z_2$

Conclusion : les racines de $az^2+bz+c$ 
sont bien $z_1$ et $z_2$




%%%%%%%%%%%%%%%%%%%%%%%%%%%%%%%%%%%%%%%%%%%%%%%%%%%%%%%%%%%
\diapo

Terminons cette partie par l'énoncé du théorème de d'Alembert-Gauss
dont nous admettons la preuve

On considère un polynôme $P(z)$ de degré $n$
qui s'écrit donc

$a_n z^n+a_{n-1}z^{n-1}+\cdots + a_2 z^2 + a_1z+a_0$

où les $a_i$ sont des nombres complexes


Alors l'équation $P(z)=0$, d'inconnue $z$, admet $n$ solutions ;

 et ces $n$ solutions sont des nombres complexes.

\change

Une autre formulation de ce même théorème est que le polynôme $P(z)$
peut se factoriser sous la forme 

$P(z) = a_n(z-z_1)(z-z_2)\cdots(z-z_n)$

Il peut y avoir des solutions multiples, lorsque deux racines (ou plus)
coïncident.

%%%%%%%%%%%%%%%%%%%%%%%%%%%%%%%%%%%%%%%%%%%%%%%%%%%%%%%%%%%
\diapo

A vous de travailler en répondant aux questions suivantes

\end{document}
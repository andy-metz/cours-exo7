
%%%%%%%%%%%%%%%%%% PREAMBULE %%%%%%%%%%%%%%%%%%


\documentclass[12pt]{article}

\usepackage{amsfonts,amsmath,amssymb,amsthm}
\usepackage[utf8]{inputenc}
\usepackage[T1]{fontenc}
\usepackage[francais]{babel}


% packages
\usepackage{amsfonts,amsmath,amssymb,amsthm}
\usepackage[utf8]{inputenc}
\usepackage[T1]{fontenc}
%\usepackage{lmodern}

\usepackage[francais]{babel}
\usepackage{fancybox}
\usepackage{graphicx}

\usepackage{float}

%\usepackage[usenames, x11names]{xcolor}
\usepackage{tikz}
\usepackage{datetime}

\usepackage{mathptmx}
%\usepackage{fouriernc}
%\usepackage{newcent}
\usepackage[mathcal,mathbf]{euler}

%\usepackage{palatino}
%\usepackage{newcent}


% Commande spéciale prompteur

%\usepackage{mathptmx}
%\usepackage[mathcal,mathbf]{euler}
%\usepackage{mathpple,multido}

\usepackage[a4paper]{geometry}
\geometry{top=2cm, bottom=2cm, left=1cm, right=1cm, marginparsep=1cm}

\newcommand{\change}{{\color{red}\rule{\textwidth}{1mm}\\}}

\newcounter{mydiapo}

\newcommand{\diapo}{\newpage
\hfill {\normalsize  Diapo \themydiapo \quad \texttt{[\jobname]}} \\
\stepcounter{mydiapo}}


%%%%%%% COULEURS %%%%%%%%%%

% Pour blanc sur noir :
%\pagecolor[rgb]{0.5,0.5,0.5}
% \pagecolor[rgb]{0,0,0}
% \color[rgb]{1,1,1}



%\DeclareFixedFont{\myfont}{U}{cmss}{bx}{n}{18pt}
\newcommand{\debuttexte}{
%%%%%%%%%%%%% FONTES %%%%%%%%%%%%%
\renewcommand{\baselinestretch}{1.5}
\usefont{U}{cmss}{bx}{n}
\bfseries

% Taille normale : commenter le reste !
%Taille Arnaud
%\fontsize{19}{19}\selectfont

% Taille Barbara
%\fontsize{21}{22}\selectfont

%Taille François
\fontsize{25}{30}\selectfont

%Taille Pascal
%\fontsize{25}{30}\selectfont

%Taille Laura
%\fontsize{30}{35}\selectfont


%\myfont
%\usefont{U}{cmss}{bx}{n}

%\Huge
%\addtolength{\parskip}{\baselineskip}
}


% \usepackage{hyperref}
% \hypersetup{colorlinks=true, linkcolor=blue, urlcolor=blue,
% pdftitle={Exo7 - Exercices de mathématiques}, pdfauthor={Exo7}}


%section
% \usepackage{sectsty}
% \allsectionsfont{\bf}
%\sectionfont{\color{Tomato3}\upshape\selectfont}
%\subsectionfont{\color{Tomato4}\upshape\selectfont}

%----- Ensembles : entiers, reels, complexes -----
\newcommand{\Nn}{\mathbb{N}} \newcommand{\N}{\mathbb{N}}
\newcommand{\Zz}{\mathbb{Z}} \newcommand{\Z}{\mathbb{Z}}
\newcommand{\Qq}{\mathbb{Q}} \newcommand{\Q}{\mathbb{Q}}
\newcommand{\Rr}{\mathbb{R}} \newcommand{\R}{\mathbb{R}}
\newcommand{\Cc}{\mathbb{C}} 
\newcommand{\Kk}{\mathbb{K}} \newcommand{\K}{\mathbb{K}}

%----- Modifications de symboles -----
\renewcommand{\epsilon}{\varepsilon}
\renewcommand{\Re}{\mathop{\text{Re}}\nolimits}
\renewcommand{\Im}{\mathop{\text{Im}}\nolimits}
%\newcommand{\llbracket}{\left[\kern-0.15em\left[}
%\newcommand{\rrbracket}{\right]\kern-0.15em\right]}

\renewcommand{\ge}{\geqslant}
\renewcommand{\geq}{\geqslant}
\renewcommand{\le}{\leqslant}
\renewcommand{\leq}{\leqslant}

%----- Fonctions usuelles -----
\newcommand{\ch}{\mathop{\mathrm{ch}}\nolimits}
\newcommand{\sh}{\mathop{\mathrm{sh}}\nolimits}
\renewcommand{\tanh}{\mathop{\mathrm{th}}\nolimits}
\newcommand{\cotan}{\mathop{\mathrm{cotan}}\nolimits}
\newcommand{\Arcsin}{\mathop{\mathrm{Arcsin}}\nolimits}
\newcommand{\Arccos}{\mathop{\mathrm{Arccos}}\nolimits}
\newcommand{\Arctan}{\mathop{\mathrm{Arctan}}\nolimits}
\newcommand{\Argsh}{\mathop{\mathrm{Argsh}}\nolimits}
\newcommand{\Argch}{\mathop{\mathrm{Argch}}\nolimits}
\newcommand{\Argth}{\mathop{\mathrm{Argth}}\nolimits}
\newcommand{\pgcd}{\mathop{\mathrm{pgcd}}\nolimits} 

\newcommand{\Card}{\mathop{\text{Card}}\nolimits}
\newcommand{\Ker}{\mathop{\text{Ker}}\nolimits}
\newcommand{\id}{\mathop{\text{id}}\nolimits}
\newcommand{\ii}{\mathrm{i}}
\newcommand{\dd}{\mathrm{d}}
\newcommand{\Vect}{\mathop{\text{Vect}}\nolimits}
\newcommand{\Mat}{\mathop{\mathrm{Mat}}\nolimits}
\newcommand{\rg}{\mathop{\text{rg}}\nolimits}
\newcommand{\tr}{\mathop{\text{tr}}\nolimits}
\newcommand{\ppcm}{\mathop{\text{ppcm}}\nolimits}

%----- Structure des exercices ------

\newtheoremstyle{styleexo}% name
{2ex}% Space above
{3ex}% Space below
{}% Body font
{}% Indent amount 1
{\bfseries} % Theorem head font
{}% Punctuation after theorem head
{\newline}% Space after theorem head 2
{}% Theorem head spec (can be left empty, meaning ‘normal’)

%\theoremstyle{styleexo}
\newtheorem{exo}{Exercice}
\newtheorem{ind}{Indications}
\newtheorem{cor}{Correction}


\newcommand{\exercice}[1]{} \newcommand{\finexercice}{}
%\newcommand{\exercice}[1]{{\tiny\texttt{#1}}\vspace{-2ex}} % pour afficher le numero absolu, l'auteur...
\newcommand{\enonce}{\begin{exo}} \newcommand{\finenonce}{\end{exo}}
\newcommand{\indication}{\begin{ind}} \newcommand{\finindication}{\end{ind}}
\newcommand{\correction}{\begin{cor}} \newcommand{\fincorrection}{\end{cor}}

\newcommand{\noindication}{\stepcounter{ind}}
\newcommand{\nocorrection}{\stepcounter{cor}}

\newcommand{\fiche}[1]{} \newcommand{\finfiche}{}
\newcommand{\titre}[1]{\centerline{\large \bf #1}}
\newcommand{\addcommand}[1]{}
\newcommand{\video}[1]{}

% Marge
\newcommand{\mymargin}[1]{\marginpar{{\small #1}}}



%----- Presentation ------
\setlength{\parindent}{0cm}

%\newcommand{\ExoSept}{\href{http://exo7.emath.fr}{\textbf{\textsf{Exo7}}}}

\definecolor{myred}{rgb}{0.93,0.26,0}
\definecolor{myorange}{rgb}{0.97,0.58,0}
\definecolor{myyellow}{rgb}{1,0.86,0}

\newcommand{\LogoExoSept}[1]{  % input : echelle
{\usefont{U}{cmss}{bx}{n}
\begin{tikzpicture}[scale=0.1*#1,transform shape]
  \fill[color=myorange] (0,0)--(4,0)--(4,-4)--(0,-4)--cycle;
  \fill[color=myred] (0,0)--(0,3)--(-3,3)--(-3,0)--cycle;
  \fill[color=myyellow] (4,0)--(7,4)--(3,7)--(0,3)--cycle;
  \node[scale=5] at (3.5,3.5) {Exo7};
\end{tikzpicture}}
}



\theoremstyle{definition}
%\newtheorem{proposition}{Proposition}
%\newtheorem{exemple}{Exemple}
%\newtheorem{theoreme}{Théorème}
\newtheorem{lemme}{Lemme}
\newtheorem{corollaire}{Corollaire}
%\newtheorem*{remarque*}{Remarque}
%\newtheorem*{miniexercice}{Mini-exercices}
%\newtheorem{definition}{Définition}




%definition d'un terme
\newcommand{\defi}[1]{{\color{myorange}\textbf{\emph{#1}}}}
\newcommand{\evidence}[1]{{\color{blue}\textbf{\emph{#1}}}}



 %----- Commandes divers ------

\newcommand{\codeinline}[1]{\texttt{#1}}

%%%%%%%%%%%%%%%%%%%%%%%%%%%%%%%%%%%%%%%%%%%%%%%%%%%%%%%%%%%%%
%%%%%%%%%%%%%%%%%%%%%%%%%%%%%%%%%%%%%%%%%%%%%%%%%%%%%%%%%%%%%

\usepackage{mathptmx}

\begin{document}

\debuttexte

%%%%%%%%%%%%%%%%%%%%%%%%%%%%%%%%%%%%%%%%%%%%%%%%%%%%%%%%%%%
\diapo

\change

Nous continuons l'étude des nombres complexes

\change

avec la définition et les propriétés de l'argument


\change

Nous montrons ensuite la formule de Moivre
qui permettra d'introduire la notation exponentielle

\change

Cela nous permettra aussi de calculer les racines $n$-ième
d'un nombre complexe

\change

Nous appliquerons tous ces résultats afin d'établir des formules trigonométriques.



%%%%%%%%%%%%%%%%%%%%%%%%%%%%%%%%%%%%%%%%%%%%%%%%%%%%%%%%%%%
\diapo

Pour un nombre complexe $z$, un argument est un réel $\theta$
tel que  

$z=|z|(\cos \theta + \ii \sin \theta)$

\change

On le note $\arg(z)$

\change

Cet argument existe et correspond à l'angle
entre l'axe des abscisses et la demi-droite $(Oz)$.

On sait aussi que le module mesure lui la distance entre le point et l'origine.

\change

Si $\theta$ est un argument de $z$ alors

$\theta+2\pi$ est aussi un argument de $z$

et $\theta+4\pi$ également.

Plus généralement on dit que 

$\theta' = \theta \pmod {2\pi}$

si il existe un entier  $k$ tel que  $\theta' = \theta + 2 k \pi$

\change


L'argument est déterminé modulo $2\pi$ 

\change

Si l'on souhaite avoir une écriture unique de l'argument on peut par exemple imposer
que $\theta$ appartiennent à l'intervalle $]-\pi,+\pi]$.



%%%%%%%%%%%%%%%%%%%%%%%%%%%%%%%%%%%%%%%%%%%%%%%%%%%%%%%%%%%
\diapo

Une première propriété de l'argument est que l'argument de
$z\times z'$ égale l'argument de $z$ plus l'argument de $z'$
(modulo $2\pi$ bien sûr)

\change

La preuve se déroule ainsi,

on écrit $z$ sous la forme $\left| z \right|  \left( \cos \theta + \ii  \sin \theta \right) $

et pareil pour $z'$

\change

Ensuite on développe

\change

La partie réelle est donc 

$\cos \theta \cos \theta' - \sin \theta     \sin \theta' $

A l'aide des formules trigonométriques, on reconnaît que la partie réelle est

$\cos(\theta + \theta')$

de même la partie imaginaire est $\sin(\theta + \theta')$

Cela signifie exactement que $\theta+\theta'$ est l'argument de $z\times z'$.

(pause)

\change

Par récurrence on en déduirait que l'argument de $z$
à la puissance $n$ est $n$ fois l'argument de $z$.

\change

On a aussi

 $\arg \left( 1 / z \right) = - \arg (z) \pmod {2\pi}$

Cela se déduit du premier point en prenant $z' = \frac 1 z$.

\change

Enfin l'argument de $\bar z$ vaut également $-\arg(z)$.


%%%%%%%%%%%%%%%%%%%%%%%%%%%%%%%%%%%%%%%%%%%%%%%%%%%%%%%%%%%
\diapo

Voici la formule de Moivre ; elle s'énonce ainsi

$
  \left( \cos \theta + \ii \sin \theta \right)^n = \cos \left( n \theta \right)
  + \ii  \sin \left( n \theta \right)$


\change


Nous allons la prouver par récurrence :
  
pour $n=0$ les termes de gauche et de droite  valent $1$ donc la formule est vraie

\change

Supposons la formule vraie au rang $n-1$

alors 

on décompose 
$$
    \left( \cos \theta + \ii  \sin \theta \right)^n$$

en $$( \cos \theta + \ii \sin \theta)^{n-1}
 \times \left( \cos \theta + \ii  \sin \theta \right)
$$

\change

On applique l'hypothèse de récurrence au premier facteur
pour obtenir

$$ \left( \cos \left(
    \left( n - 1 \right) \theta \right) + \ii  \sin \left( \left( n - 1 \right)
    \theta \right) \right) \times \left( \cos \theta + \ii  \sin \theta \right)
$$


\change

On développe


\change

La partie réelle s'écrit 

$\left( \cos \left( \left( n - 1 \right) \theta \right) \cos \theta
    - \sin \left( \left( n - 1 \right) \theta \right) \sin \theta \right) $

on reconnaît la formule pour $\cos( (n-1)\theta + \theta )$.

Donc la partie réelle est $\cos (n\theta)$.

De même la partie imaginaire est $\sin(n\theta)$.


\change

Par le principe de récurrence la formule est vraie pour tout $n$





%%%%%%%%%%%%%%%%%%%%%%%%%%%%%%%%%%%%%%%%%%%%%%%%%%%%%%%%%%%
\diapo


Nous noterons $e^{\ii \theta}$ le nombre complexe $\cos \theta + \ii \sin \theta$

\change

Cela permet d'écrire tout nombre complexe sous la forme module-argument

$z = \rho e^{\ii \theta}$

où le réel positif $\rho$ est le module de $z$

et $\theta$ un argument de $z$

Cette écriture est très pratique car les règles de calculs sont les mêmes
qu'avec l'exponentielle classique.

\change

Ainsi le produit de deux nombres complexes 
$z = \rho e^{\ii  \theta}$  et  $z' = \rho' e^{\ii  \theta'}$
vaut $\rho \rho' e^{\ii  (\theta + \theta')}$

\change

La formule de Moivre devient juste l'identité

$\left(e^{\ii\theta}\right)^n = e^{\ii n \theta}$

\change

Donc $z^n$ s'écrit aussi $\rho^n e^{\ii n \theta}$

\change

L'inverse de $\rho e^{\ii  \theta}$
est tout simplement $\frac{1}{\rho} e^{- \ii \theta}$

\change

Enfin le conjugué de $\rho e^{\ii  \theta}$ est $\rho e^{-\ii \theta}$


\change

Cette écriture est presque unique : plus précisément 

deux nombres complexes non nuls sont égaux si et seulement si
ils ont les mêmes modules
et leurs arguments sont égaux modulo $2\pi$.




%%%%%%%%%%%%%%%%%%%%%%%%%%%%%%%%%%%%%%%%%%%%%%%%%%%%%%%%%%%
\diapo


Nous généralisons maintenant le concept de racine carrée : une racine $n$-ième
de $z$ est un nombre complexe $\omega$ tel que 
$\omega^n=z$

\change
Nous pouvons calculer toutes les racines $n$-ième


en fait $z$ admet $n$ racines $n$-ième 

et si $z$ s'écrit $\rho e^{\ii \theta}$ alors ses racines $n$-ième sont les

$\omega_k = \rho^{1/n} e^{\frac{\ii\theta + 2\ii k\pi}{n}}$

pour $k$ allant de $0$ à $n-1$


\change

Passons à la preuve.

On cherche $\omega$ sous la forme $\omega=re^{\ii t}$ tel que $\omega^n=z$.

\change

ce qui équivaut à 

$\rho e^{\ii \theta}  = r^n e^{\ii nt}$

par la formule de Moivre

\change

Les modules sont donc égaux et les arguments aussi (modulo $2\pi$)

\change

Donc $\rho = r^n$

\change

et $nt = \theta \pmod {2\pi}$

\change

C'est-à-dire $nt = \theta + 2k\pi$
pour un certain entier $k$

\change

Nous avons donc $r = \rho^{1/n}$

\change

et $t = \frac{\theta}{n} + \frac{2k\pi}{n}$

\change

Ainsi les solutions sont les 
$\omega_k = \rho^{1/n} e^{\frac{\ii\theta + 2\ii k\pi}{n}}$

pour $k$ parcourant $\Zz$


\change

Mais pour $k=n$, cela revient à ajouter $2\pi$ à l'argument
dont $\omega_n=\omega_0$.

De même $\omega_{n+1}=\omega_1$

Il y a donc exactement $n$ solutions distinctes

$\omega_0,\omega_1,\ldots,\omega_{n-1}$



%%%%%%%%%%%%%%%%%%%%%%%%%%%%%%%%%%%%%%%%%%%%%%%%%%%%%%%%%%%
\diapo

Il est bon de refaire les calculs de la démonstration à chaque fois.

Par exemple si $z=1$ qui est donc $z=e^0$
et $n=3$ alors les racines $3$-ième de l'unité sont

$\{1,e^{2\ii \pi/3},e^{4\ii \pi/3}\}$

\change

Lorsque l'on place ces trois points, ils forment un triangle équilatéral.

\change

De même si $z=-1$ qui est aussi $e^{\ii \pi}$, les calculs montrent que les 
racines $3$-ième de $-1$ sont 

$\{ e^{\frac{\ii\pi}{3}}, -1, e^{-\frac{\ii\pi}{3}} \}$

\change

qui forment aussi un triangle équilatéral.

%%%%%%%%%%%%%%%%%%%%%%%%%%%%%%%%%%%%%%%%%%%%%%%%%%%%%%%%%%%
\diapo

Les $5$ racines $5$-ième de l'unité sont 

$1,e^{\frac{2\ii\pi}{5}}, e^{\frac{4\ii\pi}{5}}, e^{\frac{6\ii\pi}{5}},e^{\frac{8\ii\pi}{5}}$

\change

Ces $5$ nombres forment un pentagone régulier.


%%%%%%%%%%%%%%%%%%%%%%%%%%%%%%%%%%%%%%%%%%%%%%%%%%%%%%%%%%%
\diapo

Appliquons maintenant les nombres complexes à la trigonométrie.

Nous avons défini $e^{\ii  \theta}$ par $\cos \theta + \ii  \sin \theta$

\change

Cela donne donc $e^{-\ii  \theta} = \cos \theta - \ii  \sin \theta$

car $\cos(-\theta) = \cos\theta$

et $\sin(-\theta)= - \sin \theta$

\change

En ajoutant ces deux formules on obtient la formule d'Euler pour $\cos \theta$
à savoir

$ \cos \theta = \dfrac{e^{\ii  \theta} + e^{- \ii  \theta}}{2}$



En soustrayant on obtient la formule d'Euler pour $\sin \theta$ :

$\sin \theta = \dfrac{e^{\ii  \theta} - e^{- \ii  \theta}}{2 \ii }$

%%%%%%%%%%%%%%%%%%%%%%%%%%%%%%%%%%%%%%%%%%%%%%%%%%%%%%%%%%%
\diapo

Nous souhaitons ici développer $\sin n\theta$ ou $\cos n\theta$

c'est-à-dire les exprimer en fonction de plusieurs $\cos \theta$ et 
$\sin \theta$ élevé à certaines puissances

\change

Montrons comment faire sur un exemple, avec $n=3$

On écrit d'abord la formule de Moivre

$\cos 3 \theta + \ii  \sin 3 \theta = \left( \cos \theta + \ii  \sin \theta
  \right)^3$

\change

On développe à l'aide de la formule du binôme de Newton

$\cos^3 \theta + 3 \ii  \cos^2 \theta \sin \theta - 3 \cos \theta \sin^2
  \theta - \ii  \sin^3 \theta$

\change

on regroupe la partie réelle d'un coté, la partie imaginaire de l'autre


\change

Il ne reste plus qu'à identifier 
$\cos 3\theta$ avec la partie réelle

Donc 
 $\cos 3\theta = \cos^3 \theta - 3 \cos \theta \sin^2 \theta$

Et ensuite on identifie $\sin 3 \theta$ avec la partie imaginaire


Ainsi 
$\sin 3 \theta = 3 \cos^2 \theta \sin \theta - \sin^3 \theta$


On a bien exprimé $\cos 3\theta$ (et $\sin 3\theta$) uniquement avec des puissances
de $\sin \theta$ et de $\cos \theta$.


%%%%%%%%%%%%%%%%%%%%%%%%%%%%%%%%%%%%%%%%%%%%%%%%%%%%%%%%%%%
\diapo


La linéarisation est l'opération inverse du développement

Il s'agit d'exprimer $(\cos \theta)^n$ (ou $(\sin \theta)^n$)

en fonction de plusieurs $\cos(k\theta)$ et $\sin(k\theta)$

\change

Voyons comment faire avec l'exemple de la linéarisation de $\sin^3 \theta$

\change

On écrit d'abord la formule d'Euler

$\sin \theta = \dfrac{e^{\ii  \theta} - e^{- \ii  \theta}}{2 \ii }$

que l'on élève au cube

\change

On factorise par $\frac{1}{(2\ii)^3}$ c'est-à-dire par $\frac{1 }{-8\ii}$

et on développe à l'aide de la formule de binôme

$\frac{1}{- 8 \ii }  \left( (e^{\ii  \theta})^3 - 3 (e^{\ii  \theta})^2e^{- \ii  \theta} 
+ 3 e^{\ii  \theta}(e^{-\ii  \theta})^2 - (e^{- \ii  \theta})^3 \right)$

\change



Le premier terme est $e^{\ii  \theta}$ au cube c'est donc bien $e^{3 \ii  \theta}$

Le second terme est $-3 (e^{\ii  \theta})^2e^{- \ii  \theta} $
c'est bien $- 3 e^{\ii  \theta}$

etc.

\change

On regroupe les termes conjugués 

$e^{3 \ii  \theta}$ avec $e^{-3 \ii  \theta}$

et $e^{\ii  \theta}$ avec $e^{- \ii  \theta}$

\change

On reconnaît, avec les formules d'Euler
$\sin 3\theta$ et $\sin \theta$


Donc $\sin^3 \theta = - \frac{\sin 3 \theta}{4}
  + \frac{3 \sin \theta}{4}$


Nous avons bien exprimé $\sin^3 \theta$ en fonction de deux sinus
et ceci sans exposants.


%%%%%%%%%%%%%%%%%%%%%%%%%%%%%%%%%%%%%%%%%%%%%%%%%%%%%%%%%%%
\diapo

A vous de vérifier votre compréhension du cours avec les petits exercices
suivants.


\end{document}
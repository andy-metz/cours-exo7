
%%%%%%%%%%%%%%%%%% PREAMBULE %%%%%%%%%%%%%%%%%%

\documentclass[aspectratio=169,utf8]{beamer}
%\documentclass[aspectratio=169,handout]{beamer}

\usetheme{Boadilla}
%\usecolortheme{seahorse}
%\usecolortheme[RGB={245,66,24}]{structure}
\useoutertheme{infolines}

% packages
\usepackage{amsfonts,amsmath,amssymb,amsthm}
\usepackage[utf8]{inputenc}
\usepackage[T1]{fontenc}
\usepackage{lmodern}

\usepackage[francais]{babel}
\usepackage{fancybox}
\usepackage{graphicx}

\usepackage{float}
\usepackage{xfrac}

%\usepackage[usenames, x11names]{xcolor}
\usepackage{pgfplots}
\usepackage{datetime}


% ----------------------------------------------------------------------
% Pour les images
\usepackage{tikz}
\usetikzlibrary{calc,shadows,arrows.meta,patterns,matrix}

\newcommand{\tikzinput}[1]{\input{figures/#1.tikz}}
% --- les figures avec échelle éventuel
\newcommand{\myfigure}[2]{% entrée : échelle, fichier(s) figure à inclure
\begin{center}\small%
\tikzstyle{every picture}=[scale=1.0*#1]% mise en échelle + 0% (automatiquement annulé à la fin du groupe)
#2%
\end{center}}



%-----  Package unités -----
\usepackage{siunitx}
\sisetup{locale = FR,detect-all,per-mode = symbol}

%\usepackage{mathptmx}
%\usepackage{fouriernc}
%\usepackage{newcent}
%\usepackage[mathcal,mathbf]{euler}

%\usepackage{palatino}
%\usepackage{newcent}
% \usepackage[mathcal,mathbf]{euler}



% \usepackage{hyperref}
% \hypersetup{colorlinks=true, linkcolor=blue, urlcolor=blue,
% pdftitle={Exo7 - Exercices de mathématiques}, pdfauthor={Exo7}}


%section
% \usepackage{sectsty}
% \allsectionsfont{\bf}
%\sectionfont{\color{Tomato3}\upshape\selectfont}
%\subsectionfont{\color{Tomato4}\upshape\selectfont}

%----- Ensembles : entiers, reels, complexes -----
\newcommand{\Nn}{\mathbb{N}} \newcommand{\N}{\mathbb{N}}
\newcommand{\Zz}{\mathbb{Z}} \newcommand{\Z}{\mathbb{Z}}
\newcommand{\Qq}{\mathbb{Q}} \newcommand{\Q}{\mathbb{Q}}
\newcommand{\Rr}{\mathbb{R}} \newcommand{\R}{\mathbb{R}}
\newcommand{\Cc}{\mathbb{C}} 
\newcommand{\Kk}{\mathbb{K}} \newcommand{\K}{\mathbb{K}}

%----- Modifications de symboles -----
\renewcommand{\epsilon}{\varepsilon}
\renewcommand{\Re}{\mathop{\text{Re}}\nolimits}
\renewcommand{\Im}{\mathop{\text{Im}}\nolimits}
%\newcommand{\llbracket}{\left[\kern-0.15em\left[}
%\newcommand{\rrbracket}{\right]\kern-0.15em\right]}

\renewcommand{\ge}{\geqslant}
\renewcommand{\geq}{\geqslant}
\renewcommand{\le}{\leqslant}
\renewcommand{\leq}{\leqslant}
\renewcommand{\epsilon}{\varepsilon}

%----- Fonctions usuelles -----
\newcommand{\ch}{\mathop{\text{ch}}\nolimits}
\newcommand{\sh}{\mathop{\text{sh}}\nolimits}
\renewcommand{\tanh}{\mathop{\text{th}}\nolimits}
\newcommand{\cotan}{\mathop{\text{cotan}}\nolimits}
\newcommand{\Arcsin}{\mathop{\text{arcsin}}\nolimits}
\newcommand{\Arccos}{\mathop{\text{arccos}}\nolimits}
\newcommand{\Arctan}{\mathop{\text{arctan}}\nolimits}
\newcommand{\Argsh}{\mathop{\text{argsh}}\nolimits}
\newcommand{\Argch}{\mathop{\text{argch}}\nolimits}
\newcommand{\Argth}{\mathop{\text{argth}}\nolimits}
\newcommand{\pgcd}{\mathop{\text{pgcd}}\nolimits} 


%----- Commandes divers ------
\newcommand{\ii}{\mathrm{i}}
\newcommand{\dd}{\text{d}}
\newcommand{\id}{\mathop{\text{id}}\nolimits}
\newcommand{\Ker}{\mathop{\text{Ker}}\nolimits}
\newcommand{\Card}{\mathop{\text{Card}}\nolimits}
\newcommand{\Vect}{\mathop{\text{Vect}}\nolimits}
\newcommand{\Mat}{\mathop{\text{Mat}}\nolimits}
\newcommand{\rg}{\mathop{\text{rg}}\nolimits}
\newcommand{\tr}{\mathop{\text{tr}}\nolimits}


%----- Structure des exercices ------

\newtheoremstyle{styleexo}% name
{2ex}% Space above
{3ex}% Space below
{}% Body font
{}% Indent amount 1
{\bfseries} % Theorem head font
{}% Punctuation after theorem head
{\newline}% Space after theorem head 2
{}% Theorem head spec (can be left empty, meaning ‘normal’)

%\theoremstyle{styleexo}
\newtheorem{exo}{Exercice}
\newtheorem{ind}{Indications}
\newtheorem{cor}{Correction}


\newcommand{\exercice}[1]{} \newcommand{\finexercice}{}
%\newcommand{\exercice}[1]{{\tiny\texttt{#1}}\vspace{-2ex}} % pour afficher le numero absolu, l'auteur...
\newcommand{\enonce}{\begin{exo}} \newcommand{\finenonce}{\end{exo}}
\newcommand{\indication}{\begin{ind}} \newcommand{\finindication}{\end{ind}}
\newcommand{\correction}{\begin{cor}} \newcommand{\fincorrection}{\end{cor}}

\newcommand{\noindication}{\stepcounter{ind}}
\newcommand{\nocorrection}{\stepcounter{cor}}

\newcommand{\fiche}[1]{} \newcommand{\finfiche}{}
\newcommand{\titre}[1]{\centerline{\large \bf #1}}
\newcommand{\addcommand}[1]{}
\newcommand{\video}[1]{}

% Marge
\newcommand{\mymargin}[1]{\marginpar{{\small #1}}}

\def\noqed{\renewcommand{\qedsymbol}{}}


%----- Presentation ------
\setlength{\parindent}{0cm}

%\newcommand{\ExoSept}{\href{http://exo7.emath.fr}{\textbf{\textsf{Exo7}}}}

\definecolor{myred}{rgb}{0.93,0.26,0}
\definecolor{myorange}{rgb}{0.97,0.58,0}
\definecolor{myyellow}{rgb}{1,0.86,0}

\newcommand{\LogoExoSept}[1]{  % input : echelle
{\usefont{U}{cmss}{bx}{n}
\begin{tikzpicture}[scale=0.1*#1,transform shape]
  \fill[color=myorange] (0,0)--(4,0)--(4,-4)--(0,-4)--cycle;
  \fill[color=myred] (0,0)--(0,3)--(-3,3)--(-3,0)--cycle;
  \fill[color=myyellow] (4,0)--(7,4)--(3,7)--(0,3)--cycle;
  \node[scale=5] at (3.5,3.5) {Exo7};
\end{tikzpicture}}
}


\newcommand{\debutmontitre}{
  \author{} \date{} 
  \thispagestyle{empty}
  \hspace*{-10ex}
  \begin{minipage}{\textwidth}
    \titlepage  
  \vspace*{-2.5cm}
  \begin{center}
    \LogoExoSept{2.5}
  \end{center}
  \end{minipage}

  \vspace*{-0cm}
  
  % Astuce pour que le background ne soit pas discrétisé lors de la conversion pdf -> png
\begin{tikzpicture}
        \fill[opacity=0,green!60!black] (0,0)--++(0,0)--++(0,0)--++(0,0)--cycle; 
\end{tikzpicture}

% toc S'affiche trop tot :
% \tableofcontents[hideallsubsections, pausesections]
}

\newcommand{\finmontitre}{
  \end{frame}
  \setcounter{framenumber}{0}
} % ne marche pas pour une raison obscure

%----- Commandes supplementaires ------

% \usepackage[landscape]{geometry}
% \geometry{top=1cm, bottom=3cm, left=2cm, right=10cm, marginparsep=1cm
% }
% \usepackage[a4paper]{geometry}
% \geometry{top=2cm, bottom=2cm, left=2cm, right=2cm, marginparsep=1cm
% }

%\usepackage{standalone}


% New command Arnaud -- november 2011
\setbeamersize{text margin left=24ex}
% si vous modifier cette valeur il faut aussi
% modifier le decalage du titre pour compenser
% (ex : ici =+10ex, titre =-5ex

\theoremstyle{definition}
%\newtheorem{proposition}{Proposition}
%\newtheorem{exemple}{Exemple}
%\newtheorem{theoreme}{Théorème}
%\newtheorem{lemme}{Lemme}
%\newtheorem{corollaire}{Corollaire}
%\newtheorem*{remarque*}{Remarque}
%\newtheorem*{miniexercice}{Mini-exercices}
%\newtheorem{definition}{Définition}

% Commande tikz
\usetikzlibrary{calc}
\usetikzlibrary{patterns,arrows}
\usetikzlibrary{matrix}
\usetikzlibrary{fadings} 

%definition d'un terme
\newcommand{\defi}[1]{{\color{myorange}\textbf{\emph{#1}}}}
\newcommand{\evidence}[1]{{\color{blue}\textbf{\emph{#1}}}}
\newcommand{\assertion}[1]{\emph{\og#1\fg}}  % pour chapitre logique
%\renewcommand{\contentsname}{Sommaire}
\renewcommand{\contentsname}{}
\setcounter{tocdepth}{2}



%------ Encadrement ------

\usepackage{fancybox}


\newcommand{\mybox}[1]{
\setlength{\fboxsep}{7pt}
\begin{center}
\shadowbox{#1}
\end{center}}

\newcommand{\myboxinline}[1]{
\setlength{\fboxsep}{5pt}
\raisebox{-10pt}{
\shadowbox{#1}
}
}

%--------------- Commande beamer---------------
\newcommand{\beameronly}[1]{#1} % permet de mettre des pause dans beamer pas dans poly


\setbeamertemplate{navigation symbols}{}
\setbeamertemplate{footline}  % tiré du fichier beamerouterinfolines.sty
{
  \leavevmode%
  \hbox{%
  \begin{beamercolorbox}[wd=.333333\paperwidth,ht=2.25ex,dp=1ex,center]{author in head/foot}%
    % \usebeamerfont{author in head/foot}\insertshortauthor%~~(\insertshortinstitute)
    \usebeamerfont{section in head/foot}{\bf\insertshorttitle}
  \end{beamercolorbox}%
  \begin{beamercolorbox}[wd=.333333\paperwidth,ht=2.25ex,dp=1ex,center]{title in head/foot}%
    \usebeamerfont{section in head/foot}{\bf\insertsectionhead}
  \end{beamercolorbox}%
  \begin{beamercolorbox}[wd=.333333\paperwidth,ht=2.25ex,dp=1ex,right]{date in head/foot}%
    % \usebeamerfont{date in head/foot}\insertshortdate{}\hspace*{2em}
    \insertframenumber{} / \inserttotalframenumber\hspace*{2ex} 
  \end{beamercolorbox}}%
  \vskip0pt%
}


\definecolor{mygrey}{rgb}{0.5,0.5,0.5}
\setlength{\parindent}{0cm}
%\DeclareTextFontCommand{\helvetica}{\fontfamily{phv}\selectfont}

% background beamer
\definecolor{couleurhaut}{rgb}{0.85,0.9,1}  % creme
\definecolor{couleurmilieu}{rgb}{1,1,1}  % vert pale
\definecolor{couleurbas}{rgb}{0.85,0.9,1}  % blanc
\setbeamertemplate{background canvas}[vertical shading]%
[top=couleurhaut,middle=couleurmilieu,midpoint=0.4,bottom=couleurbas] 
%[top=fondtitre!05,bottom=fondtitre!60]



\makeatletter
\setbeamertemplate{theorem begin}
{%
  \begin{\inserttheoremblockenv}
  {%
    \inserttheoremheadfont
    \inserttheoremname
    \inserttheoremnumber
    \ifx\inserttheoremaddition\@empty\else\ (\inserttheoremaddition)\fi%
    \inserttheorempunctuation
  }%
}
\setbeamertemplate{theorem end}{\end{\inserttheoremblockenv}}

\newenvironment{theoreme}[1][]{%
   \setbeamercolor{block title}{fg=structure,bg=structure!40}
   \setbeamercolor{block body}{fg=black,bg=structure!10}
   \begin{block}{{\bf Th\'eor\`eme }#1}
}{%
   \end{block}%
}


\newenvironment{proposition}[1][]{%
   \setbeamercolor{block title}{fg=structure,bg=structure!40}
   \setbeamercolor{block body}{fg=black,bg=structure!10}
   \begin{block}{{\bf Proposition }#1}
}{%
   \end{block}%
}

\newenvironment{corollaire}[1][]{%
   \setbeamercolor{block title}{fg=structure,bg=structure!40}
   \setbeamercolor{block body}{fg=black,bg=structure!10}
   \begin{block}{{\bf Corollaire }#1}
}{%
   \end{block}%
}

\newenvironment{mydefinition}[1][]{%
   \setbeamercolor{block title}{fg=structure,bg=structure!40}
   \setbeamercolor{block body}{fg=black,bg=structure!10}
   \begin{block}{{\bf Définition} #1}
}{%
   \end{block}%
}

\newenvironment{lemme}[0]{%
   \setbeamercolor{block title}{fg=structure,bg=structure!40}
   \setbeamercolor{block body}{fg=black,bg=structure!10}
   \begin{block}{\bf Lemme}
}{%
   \end{block}%
}

\newenvironment{remarque}[1][]{%
   \setbeamercolor{block title}{fg=black,bg=structure!20}
   \setbeamercolor{block body}{fg=black,bg=structure!5}
   \begin{block}{Remarque #1}
}{%
   \end{block}%
}


\newenvironment{exemple}[1][]{%
   \setbeamercolor{block title}{fg=black,bg=structure!20}
   \setbeamercolor{block body}{fg=black,bg=structure!5}
   \begin{block}{{\bf Exemple }#1}
}{%
   \end{block}%
}


\newenvironment{miniexercice}[0]{%
   \setbeamercolor{block title}{fg=structure,bg=structure!20}
   \setbeamercolor{block body}{fg=black,bg=structure!5}
   \begin{block}{Mini-exercices}
}{%
   \end{block}%
}


\newenvironment{tp}[0]{%
   \setbeamercolor{block title}{fg=structure,bg=structure!40}
   \setbeamercolor{block body}{fg=black,bg=structure!10}
   \begin{block}{\bf Travaux pratiques}
}{%
   \end{block}%
}
\newenvironment{exercicecours}[1][]{%
   \setbeamercolor{block title}{fg=structure,bg=structure!40}
   \setbeamercolor{block body}{fg=black,bg=structure!10}
   \begin{block}{{\bf Exercice }#1}
}{%
   \end{block}%
}
\newenvironment{algo}[1][]{%
   \setbeamercolor{block title}{fg=structure,bg=structure!40}
   \setbeamercolor{block body}{fg=black,bg=structure!10}
   \begin{block}{{\bf Algorithme}\hfill{\color{gray}\texttt{#1}}}
}{%
   \end{block}%
}


\setbeamertemplate{proof begin}{
   \setbeamercolor{block title}{fg=black,bg=structure!20}
   \setbeamercolor{block body}{fg=black,bg=structure!5}
   \begin{block}{{\footnotesize Démonstration}}
   \footnotesize
   \smallskip}
\setbeamertemplate{proof end}{%
   \end{block}}
\setbeamertemplate{qed symbol}{\openbox}


\makeatother
\usecolortheme[RGB={102,102,0}]{structure}

% Commande spécifique à ce chapitre

\newcommand*{\longhookrightarrow}{\ensuremath{\lhook\joinrel%
\relbar\joinrel\relbar\joinrel\relbar\joinrel\relbar%
\joinrel\relbar\joinrel\rightarrow}}



%%%%%%%%%%%%%%%%%%%%%%%%%%%%%%%%%%%%%%%%%%%%%%%%%%%%%%%%%%%%%
%%%%%%%%%%%%%%%%%%%%%%%%%%%%%%%%%%%%%%%%%%%%%%%%%%%%%%%%%%%%%



\begin{document}



\title{{\bf Nombres complexes}}
\subtitle{Définitions et opérations}

\begin{frame}
  
  \debutmontitre

  \pause

{\footnotesize
\hfill
\setbeamercovered{transparent=50}
\begin{minipage}{0.6\textwidth}
  \begin{itemize}
    \item<3-> Définition 
    \item<4-> Partie réelle, partie imaginaire
    \item<5-> Calculs
    \item<6-> Conjugué, module
  \end{itemize}
\end{minipage}
}

\end{frame}

\setcounter{framenumber}{0}


%%%%%%%%%%%%%%%%%%%%%%%%%%%%%%%%%%%%%%%%%%%%%%%%%%%%%%%%%%%%%%%%
\section*{Introduction}

\begin{frame}

$$\Nn \stackrel{x+5=2}{\longhookrightarrow}  \Zz \pause 
\stackrel{2x=-3}\longhookrightarrow \Qq \pause 
\stackrel{x^2=\frac12}\longhookrightarrow \Rr \pause 
\stackrel{x^2=-\sqrt2}\longhookrightarrow \pause 
\Cc$$

\pause


\begin{theoreme}[(d'Alembert-Gauss)]
L'équation 
$$a_nx^n+a_{n-1}x^{n-1}+\cdots + a_2 x^2 + a_1x+a_0=0$$
avec $a_i \in \Cc$ (ou $a_i \in \Rr$)
a ses solutions $x_1,x_2,\ldots,x_n$ dans $\Cc$
\end{theoreme}


\end{frame}

%%%%%%%%%%%%%%%%%%%%%%%%%%%%%%%%%%%%%%%%%%%%%%%%%%%%%%%%%%%%%%%%
\section{Définition}

\begin{frame}
  Un \defi{nombre complexe} est un couple $(a, b) \in \Rr^2$
  noté $a + \ii b$ 

\pause

avec la convention 
\mybox{$\ii ^2 = - 1$}

\pause

\myfigure{0.8}{
\tikzinput{fig_complexes03}
\qquad \uncover<5->{\tikzinput{fig_complexes05}}
}

% On note $\Cc$ l'ensemble des nombres complexes. Si $b = 0$, alors $z = a$
%   est dit \defi{réel} et on considère $\Rr$ comme un sous-ensemble de
%   $\Cc$.

\pause

\qquad $z = a + \ii b$ \qquad $z' = a' + \ii b'$ 
\begin{itemize}
  \item \evidence{addition} \quad  $(a + \ii b) + (a' + \ii b') = 
(a + a') + \ii  (b + b')$ 
\pause \pause
  \item \evidence{multiplication} \  $(a + \ii b) \times (a' + \ii b')
  = (aa' - bb') + \ii  (ab' + ba')$
\end{itemize}
\end{frame}


%%%%%%%%%%%%%%%%%%%%%%%%%%%%%%%%%%%%%%%%%%%%%%%%%%%%%%%%%%%%%%%%
\section{Partie réelle et partie imaginaire}

\begin{frame}

\hspace{2em}
\begin{minipage}{0.5\linewidth}
$z = a + \ii b \in \Cc$ \\[0.5em]
$\Re(z) = a$ est la \defi{partie réelle}  \\[0.5em]
$\Im(z) = b$ est la \defi{partie imaginaire}
\end{minipage}%
\pause
\hspace{-1em}
\begin{minipage}{0.2\linewidth}
\myfigure{0.9}{
\tikzinput{fig_complexes04} 
}
\end{minipage}
% Par identification de $\Cc$ \`a $\Rr^2$,
% l'écriture $z = \Re(z) + \ii  \Im(z)$
% est unique :
\pause
$$ z = z'\quad \Longleftrightarrow \quad \left\{ \begin{array}{l}
     \Re (z) = \Re (z')\\
     \text{ et }\\
     \Im (z) = \Im (z')
   \end{array} \right. $$

% En particulier un nombre complexe est réel si et seulement si sa partie imaginaire
% s'annule. Un nombre complexe est nul si et et seulement 
% si sa partie réelle et sa partie imaginaire s'annule.

\end{frame}

%%%%%%%%%%%%%%%%%%%%%%%%%%%%%%%%%%%%%%%%%%%%%%%%%%%%%%%%%%%%%%%%
\section{Calculs}

\begin{frame}
% Quelques définitions et calculs sur les nombres complexes.
\begin{itemize}
  \item L'\defi{opposé} de $z = a + \ii b$ est $- z = (- a) + \ii (-b) = - a - \ii b$
  
  \item La \defi{multiplication par un scalaire} $\lambda \in \Rr$ : $\lambda \times 
  z = (\lambda a) + \ii(\lambda b)$
\end{itemize}

\bigskip

\myfigure{0.9}{
\tikzinput{fig_complexes06} 
}
\end{frame}


%%%%%%%%%%%%%%%%%%%%%%%%%%%%%%%%%%%%%%%%%%%%%%%%%%%%%%%%%%%%%%%%
\begin{frame}

\vspace*{1cm}

 \begin{itemize} 
  \item L'\defi{inverse} : si $z \neq 0$, il existe un unique $z' \in \Cc$ tel
  que $z \times z' = 1$
\pause
\[ z' = \frac{1}{z} = \frac{a - \ii b}{a^2 + b^2}  \]

\pause
%\only<7>
{
  \item % La division : 
$\dfrac{z}{z'}=z \times \dfrac{1}{z'}$\\[1em]
     
\pause

  \item  si \quad  $zz' = 0$ \quad  alors \quad  $z = 0$ ou $z' = 0$ \qquad  (\defi{intégrité})\\[1em]

\pause
  
  \item % Puissances : 
$z{\,}^n = z \times \cdots \times z$ \quad ; \quad  $z{\,}^0 = 1$ \quad ; \quad $z{\,}^{- n} = \big( \frac 1 z \big)^n$
}

\end{itemize}

\pause
\bigskip
\bigskip

\begin{remarque}
  Ne \alert{jamais} écrire $z \geqslant 0$ ou $z \leqslant z'$
\end{remarque}

% \pause
% \uncover<2-6>{
% 
% \begin{proof}
% $z=a+\ii b$, on cherche $z'=a'+\ii b'$ tel que $z \times z'=1$\\[0.5em]
% \pause
% Soit donc \`a r\'esoudre $(a+ \ii b)(a'+\ii b') =1$ :
% % En développant et identifiant les parties réelles et imaginaires alors
% \[ \left\{ \begin{array}{cc}
%        aa' - bb' = 1 & \uncover<3,4->{\color{blue}\times a}
%        \\
%        ab' + ba' = 0 & \uncover<3,4->{\color{blue}\times b}  
%      \end{array} \right. 
% \uncover<4->{
% \quad \text{implique}  \quad \left\{ \begin{array}{cc}
%        a^2 a' - ab b' = a 
%        \\
% 		ab b' + b^2 a' = 0
%      \end{array} \right.
% }     
% \]
% \pause
% \pause
% \pause
% 
% %   en écrivant $a L_1 + b L_2$ (on multiplie la ligne ($L_1$) par $a$, 
% % la ligne ($L_2$) par $b$ et on additionne)
% %  et $- bL_1 + aL_2$ on obtient
%   \[\text{ d'où }\quad \left\{ \begin{aligned}
%        &a'  \left( a^2 + b^2 \right) = a\\
%        &b'  \left( a^2 + b^2 \right) = - b
%      \end{aligned} \right. \quad \text{ donc }\quad \left\{ \begin{aligned}
%        &a' = \frac{a}{a^2 + b^2}\\
%        &b' = - \frac{b}{a^2 + b^2}
%      \end{aligned} \right. \]
% \vspace{0.5em}
% 
% \pause
% 
% L'inverse de $z$ est donc
% \[ z' = \frac{1}{z} = \frac{a}{a^2 + b^2} + \ii  \frac{- b}{a^2 + b^2} =
%      \frac{a - \ii b}{a^2 + b^2}  \]
% \end{proof}
% }

\vspace{4cm}

\end{frame}

%%%%%%%%%%%%%%%%%%%%%%%%%%%%%%%%%%%%%%%%%%%%%%%%%%%%%%%%%%%%%%%%
\begin{frame}

\begin{proposition}
\label{prop:somme}
 Pour tout $z \in \Cc$ différent de $1$\!:\!\!\!\!
\myboxinline{$1 + z + z^2 + \cdots + z^n = \dfrac{1 - z^{n + 1}}{1 - z}$}
\end{proposition}
\pause
$$\text{Ou encore} \qquad \displaystyle \sum_{k=0}^n z^k = \dfrac{1 - z^{n + 1}}{1 - z}$$

\pause

\vspace{\stretch{1}}

\begin{proof}
\normalsize
$\begin{array}{rl}
(1 + z + z^2 + \cdots + z^n) \times (1 - z) 
  \pause
  &= 1 + z + z^2 + \cdots + z^n \\
  &\hphantom{= 1 } -z -z^2 - \cdots - z^n - z^{n+1} \\
  \pause
  &= 1 - z^{n + 1}  \\
\end{array}$
%\vspace*{-1em}
\end{proof}

\end{frame}


%%%%%%%%%%%%%%%%%%%%%%%%%%%%%%%%%%%%%%%%%%%%%%%%%%%%%%%%%%%%%%%%
\section{Conjugué, module}

\begin{frame}
   Le \defi{conjugué} de $z = a + \ii b$ est \myboxinline{$\overline{z} = a - \ii b$}\\[0.5em]
%, autrement dit   $\Re(\overline{z}) = \Re(z)$ et   $\Im(\overline{z}) = - \Im (z)$



  \uncover<2->{Le \defi{module} de $z = a + \ii b$ est \myboxinline{$|z| = \sqrt{a^2 + b^2} = \sqrt{z\overline z}$}\\[0.5em]}
 
  \uncover<3-4>{\centerline{$z \times \overline z = (a+\ii b)(a-\ii b) = a^2+b^2$} }

\uncover<1-4>{
\myfigure{1}{
\uncover<1->{\tikzinput{fig_complexes07} \hspace{2em} }
\uncover<4->{\tikzinput{fig_complexes08}}
}
}
\pause
\pause
\pause
\pause 

\vspace{-11em}

\begin{itemize}
  \item $\overline{z + z'} = \overline{z} + \overline{z'}$,\quad $\overline{\overline{z}} =
  z$,\quad $\overline{zz'} = \overline{z}  \overline{z'}$
  
  \item $z = \overline{z} \Longleftrightarrow z \in \Rr$
  
  \item $\left| z \right|^2 = z \times \overline{z}$,\quad $\left| \overline{z} \right| =
  \left| z \right|$, \quad $\left| zz' \right| = \left| z \right|  \left| z'
  \right|$
  
  \item $\left| z \right| = 0 \Longleftrightarrow z = 0$
  
  \item Inégalité triangulaire  $\left| z + z' \right| \leqslant \left| z
  \right| + \left| z' \right|$
\end{itemize}

\end{frame}

%%%%%%%%%%%%%%%%%%%%%%%%%%%%%%%%%%%%%%%%%%%%%%%%%%%%%%%%%%%%%%%%
\begin{frame}


\begin{exemple}
  Dans un parallélogramme, la somme des carrés des diagonales égale la
  somme des carrés des c\^otés
\end{exemple}


\pause

\vspace{1em}

\myfigure{1}{
\tikzinput{fig_complexes10}
}
$$D^2+d^2 = 2\ell^2+2L^2$$
\end{frame}

%%%%%%%%%%%%%%%%%%%%%%%%%%%%%%%%%%%%%%%%%%%%%%%%%%%%%%%%%%%%%%%%
\begin{frame}


\begin{proof}

\begin{minipage}{0.5\linewidth}
\myfigure{0.8}{
\tikzinput{fig_complexes09}
}
\end{minipage}%
%
\begin{minipage}{0.7\linewidth}
Sommets : $0$, $z$, $z'$, $z+z'$\\
Longueur des c\^ot\'es : $|z|$, $|z'|$\\
\uncover<2->{%
Celle des diagonales : $|z+z'|$, $|z-z'|$
}
\end{minipage}
\pause\pause
\vspace{-2em}
\begin{center}
\uncover<3->{
\[  \begin{aligned}
  \qquad  \alert{D^2 + d^2} & = \left| z + z' \right|^2 + \left| z - z' \right|^2  \\
   \pause
   & =  \left( z + z'     \right)  \overline{\left( z + z' \right)} + \left( z - z' \right) 
    \overline{\left( z - z' \right)} \\
    \pause
    & =  z \overline{z} + z \overline{z'} + z'  \overline{z} + z'  \overline{z'} + z
    \overline{z} - z \overline{z'} - z'  \overline{z} + z'  \overline{z'}\\
    \pause
    & =  2 z \overline{z} + 2 z'  \overline{z'} \pause = 2 \left| z \right|^2 + 2 \left|    z' \right|^2 \\
    \pause
    & =  \alert<4->{2\ell^2+2L^2}
  \end{aligned}
\]
\vspace{-2em}
}
\end{center}

\end{proof}


\end{frame}

%%%%%%%%%%%%%%%%%%%%%%%%%%%%%%%%%%%%%%%%%%%%%%%%%%%%%%%%%%%%%%%%
\section*{Mini-exercices}

\begin{frame}
\begin{miniexercice}
 \begin{enumerate}
  \setlength{\itemsep}{1em}
  \vspace{1em}
  \item Calculer $1 - 2\ii + \frac{\ii}{1 - 2\ii}$.
  \item \'Ecrire sous la forme $a+\ii b$ les nombres complexes $(1+\ii)^2$,
$(1+\ii)^3$, $(1+\ii)^4$, $(1+\ii)^8$. 
  \item En déduire $1+(1+\ii)+(1+\ii)^2+\cdots +(1+\ii)^7$.
  \item Soit $z\in \Cc$ tel que $|1+ \ii z| = |1-\ii z|$, montrer que $z \in \Rr$. 
  \item Montrer que si $|\Re z| \le |\Re z'|$ et $|\Im z| \le |\Im z'|$ alors
$|z| \le |z'|$, mais que la réciproque est fausse.
  \item Montrer que $1 / \overline{z} = z/\left| z \right|^2$ (pour $z\neq 0$).
\end{enumerate}
\end{miniexercice}
\end{frame}



\end{document}
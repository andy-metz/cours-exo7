
%%%%%%%%%%%%%%%%%% PREAMBULE %%%%%%%%%%%%%%%%%%


\documentclass[12pt]{article}

\usepackage{amsfonts,amsmath,amssymb,amsthm}
\usepackage[utf8]{inputenc}
\usepackage[T1]{fontenc}
\usepackage[francais]{babel}


% packages
\usepackage{amsfonts,amsmath,amssymb,amsthm}
\usepackage[utf8]{inputenc}
\usepackage[T1]{fontenc}
%\usepackage{lmodern}

\usepackage[francais]{babel}
\usepackage{fancybox}
\usepackage{graphicx}

\usepackage{float}

%\usepackage[usenames, x11names]{xcolor}
\usepackage{tikz}
\usepackage{datetime}

\usepackage{mathptmx}
%\usepackage{fouriernc}
%\usepackage{newcent}
\usepackage[mathcal,mathbf]{euler}

%\usepackage{palatino}
%\usepackage{newcent}


% Commande spéciale prompteur

%\usepackage{mathptmx}
%\usepackage[mathcal,mathbf]{euler}
%\usepackage{mathpple,multido}

\usepackage[a4paper]{geometry}
\geometry{top=2cm, bottom=2cm, left=1cm, right=1cm, marginparsep=1cm}

\newcommand{\change}{{\color{red}\rule{\textwidth}{1mm}\\}}

\newcounter{mydiapo}

\newcommand{\diapo}{\newpage
\hfill {\normalsize  Diapo \themydiapo \quad \texttt{[\jobname]}} \\
\stepcounter{mydiapo}}


%%%%%%% COULEURS %%%%%%%%%%

% Pour blanc sur noir :
%\pagecolor[rgb]{0.5,0.5,0.5}
% \pagecolor[rgb]{0,0,0}
% \color[rgb]{1,1,1}



%\DeclareFixedFont{\myfont}{U}{cmss}{bx}{n}{18pt}
\newcommand{\debuttexte}{
%%%%%%%%%%%%% FONTES %%%%%%%%%%%%%
\renewcommand{\baselinestretch}{1.5}
\usefont{U}{cmss}{bx}{n}
\bfseries

% Taille normale : commenter le reste !
%Taille Arnaud
%\fontsize{19}{19}\selectfont

% Taille Barbara
%\fontsize{21}{22}\selectfont

%Taille François
\fontsize{25}{30}\selectfont

%Taille Pascal
%\fontsize{25}{30}\selectfont

%Taille Laura
%\fontsize{30}{35}\selectfont


%\myfont
%\usefont{U}{cmss}{bx}{n}

%\Huge
%\addtolength{\parskip}{\baselineskip}
}


% \usepackage{hyperref}
% \hypersetup{colorlinks=true, linkcolor=blue, urlcolor=blue,
% pdftitle={Exo7 - Exercices de mathématiques}, pdfauthor={Exo7}}


%section
% \usepackage{sectsty}
% \allsectionsfont{\bf}
%\sectionfont{\color{Tomato3}\upshape\selectfont}
%\subsectionfont{\color{Tomato4}\upshape\selectfont}

%----- Ensembles : entiers, reels, complexes -----
\newcommand{\Nn}{\mathbb{N}} \newcommand{\N}{\mathbb{N}}
\newcommand{\Zz}{\mathbb{Z}} \newcommand{\Z}{\mathbb{Z}}
\newcommand{\Qq}{\mathbb{Q}} \newcommand{\Q}{\mathbb{Q}}
\newcommand{\Rr}{\mathbb{R}} \newcommand{\R}{\mathbb{R}}
\newcommand{\Cc}{\mathbb{C}} 
\newcommand{\Kk}{\mathbb{K}} \newcommand{\K}{\mathbb{K}}

%----- Modifications de symboles -----
\renewcommand{\epsilon}{\varepsilon}
\renewcommand{\Re}{\mathop{\text{Re}}\nolimits}
\renewcommand{\Im}{\mathop{\text{Im}}\nolimits}
%\newcommand{\llbracket}{\left[\kern-0.15em\left[}
%\newcommand{\rrbracket}{\right]\kern-0.15em\right]}

\renewcommand{\ge}{\geqslant}
\renewcommand{\geq}{\geqslant}
\renewcommand{\le}{\leqslant}
\renewcommand{\leq}{\leqslant}

%----- Fonctions usuelles -----
\newcommand{\ch}{\mathop{\mathrm{ch}}\nolimits}
\newcommand{\sh}{\mathop{\mathrm{sh}}\nolimits}
\renewcommand{\tanh}{\mathop{\mathrm{th}}\nolimits}
\newcommand{\cotan}{\mathop{\mathrm{cotan}}\nolimits}
\newcommand{\Arcsin}{\mathop{\mathrm{Arcsin}}\nolimits}
\newcommand{\Arccos}{\mathop{\mathrm{Arccos}}\nolimits}
\newcommand{\Arctan}{\mathop{\mathrm{Arctan}}\nolimits}
\newcommand{\Argsh}{\mathop{\mathrm{Argsh}}\nolimits}
\newcommand{\Argch}{\mathop{\mathrm{Argch}}\nolimits}
\newcommand{\Argth}{\mathop{\mathrm{Argth}}\nolimits}
\newcommand{\pgcd}{\mathop{\mathrm{pgcd}}\nolimits} 

\newcommand{\Card}{\mathop{\text{Card}}\nolimits}
\newcommand{\Ker}{\mathop{\text{Ker}}\nolimits}
\newcommand{\id}{\mathop{\text{id}}\nolimits}
\newcommand{\ii}{\mathrm{i}}
\newcommand{\dd}{\mathrm{d}}
\newcommand{\Vect}{\mathop{\text{Vect}}\nolimits}
\newcommand{\Mat}{\mathop{\mathrm{Mat}}\nolimits}
\newcommand{\rg}{\mathop{\text{rg}}\nolimits}
\newcommand{\tr}{\mathop{\text{tr}}\nolimits}
\newcommand{\ppcm}{\mathop{\text{ppcm}}\nolimits}

%----- Structure des exercices ------

\newtheoremstyle{styleexo}% name
{2ex}% Space above
{3ex}% Space below
{}% Body font
{}% Indent amount 1
{\bfseries} % Theorem head font
{}% Punctuation after theorem head
{\newline}% Space after theorem head 2
{}% Theorem head spec (can be left empty, meaning ‘normal’)

%\theoremstyle{styleexo}
\newtheorem{exo}{Exercice}
\newtheorem{ind}{Indications}
\newtheorem{cor}{Correction}


\newcommand{\exercice}[1]{} \newcommand{\finexercice}{}
%\newcommand{\exercice}[1]{{\tiny\texttt{#1}}\vspace{-2ex}} % pour afficher le numero absolu, l'auteur...
\newcommand{\enonce}{\begin{exo}} \newcommand{\finenonce}{\end{exo}}
\newcommand{\indication}{\begin{ind}} \newcommand{\finindication}{\end{ind}}
\newcommand{\correction}{\begin{cor}} \newcommand{\fincorrection}{\end{cor}}

\newcommand{\noindication}{\stepcounter{ind}}
\newcommand{\nocorrection}{\stepcounter{cor}}

\newcommand{\fiche}[1]{} \newcommand{\finfiche}{}
\newcommand{\titre}[1]{\centerline{\large \bf #1}}
\newcommand{\addcommand}[1]{}
\newcommand{\video}[1]{}

% Marge
\newcommand{\mymargin}[1]{\marginpar{{\small #1}}}



%----- Presentation ------
\setlength{\parindent}{0cm}

%\newcommand{\ExoSept}{\href{http://exo7.emath.fr}{\textbf{\textsf{Exo7}}}}

\definecolor{myred}{rgb}{0.93,0.26,0}
\definecolor{myorange}{rgb}{0.97,0.58,0}
\definecolor{myyellow}{rgb}{1,0.86,0}

\newcommand{\LogoExoSept}[1]{  % input : echelle
{\usefont{U}{cmss}{bx}{n}
\begin{tikzpicture}[scale=0.1*#1,transform shape]
  \fill[color=myorange] (0,0)--(4,0)--(4,-4)--(0,-4)--cycle;
  \fill[color=myred] (0,0)--(0,3)--(-3,3)--(-3,0)--cycle;
  \fill[color=myyellow] (4,0)--(7,4)--(3,7)--(0,3)--cycle;
  \node[scale=5] at (3.5,3.5) {Exo7};
\end{tikzpicture}}
}



\theoremstyle{definition}
%\newtheorem{proposition}{Proposition}
%\newtheorem{exemple}{Exemple}
%\newtheorem{theoreme}{Théorème}
\newtheorem{lemme}{Lemme}
\newtheorem{corollaire}{Corollaire}
%\newtheorem*{remarque*}{Remarque}
%\newtheorem*{miniexercice}{Mini-exercices}
%\newtheorem{definition}{Définition}




%definition d'un terme
\newcommand{\defi}[1]{{\color{myorange}\textbf{\emph{#1}}}}
\newcommand{\evidence}[1]{{\color{blue}\textbf{\emph{#1}}}}



 %----- Commandes divers ------

\newcommand{\codeinline}[1]{\texttt{#1}}

%%%%%%%%%%%%%%%%%%%%%%%%%%%%%%%%%%%%%%%%%%%%%%%%%%%%%%%%%%%%%
%%%%%%%%%%%%%%%%%%%%%%%%%%%%%%%%%%%%%%%%%%%%%%%%%%%%%%%%%%%%%



\begin{document}

\debuttexte



%%%%%%%%%%%%%%%%%%%%%%%%%%%%%%%%%%%%%%%%%%%%%%%%%%%%%%%%%%%
\diapo

\change

Dans cette première partie sur les nombres complexes

\change

nous rappelons d'abord la définition d'un nombre complexe

\change

puis celle de la partie réelle et de la partie imaginaire.

\change 

Dans la partie calcul nous verrons comment trouver l'inverse
d'un nombre complexe

\change

Nous aborderons ensuite les notions de conjugué et module.

Nous terminons par une application géométrique.


%%%%%%%%%%%%%%%%%%%%%%%%%%%%%%%%%%%%%%%%%%%%%%%%%%%%%%%%%%%
\diapo


Commençons d'abord par motiver les nombres complexes.

L'équation $x+5=2$ a ses coefficients dans l'ensemble $\Nn$ des entiers naturels
et a pour solution $x=-3$.

Nous sommes obligés de considérer un ensemble plus grand pour la solution.

\change

De même l'équation $2x=-3$ a ses coefficients dans l'ensemble $\Zz$ mais sa solution
$x= -\frac 32$ appartient aux nombres rationnels $\Qq$.

\change

On continue avec une équation à coefficients rationnels $x^2= \frac 12$
mais dont les solutions $x = + \frac 1 {\sqrt 2}$ et $x = - \frac 1 {\sqrt 2}$
sont des nombres réels.

\change

Enfin les solutions de l'équation réelle $x^2 = - \sqrt 2$ sont des nombres complexes


\change

(pause)

Est-ce que cela continue indéfiniment ?

Non ! Le processus s'arrête ici !

Le miracle des nombres complexes est le suivant : 

Toute solution d'une équation à coefficients complexe
est aussi un nombre complexe.

\change

C'est le fameux théorème de d'Alembert-Gauss

Pour n'importe quelle équation polynomiale d'inconnue $x$

$a_nx^n+a_{n-1}x^{n-1}+\cdots + a_2 x^2 + a_1x+a_0=0$

où les $a_i$ sont des nombres complexes

Alors toutes les solutions $x_1$ jusqu’à $x_n$ sont des nombres complexes.


%%%%%%%%%%%%%%%%%%%%%%%%%%%%%%%%%%%%%%%%%%%%%%%%%%%%%%%%%%%
\diapo

Un nombre complexe est par définition un couple $(a,b)$ de réels
que l'on note $a+\ii b$.

\change

Avec $\ii$ qui est un nouveau nombre, appelé nombre imaginaire,


et qui vérifie la relation $\ii^2=-1$

\change

Voici la représentation graphique d'un nombre complexe $a+\ii b$

qui correspond au point d'abscisse $a$ et d'ordonnée $b$

\change

on peut additionner deux nombres complexes

si $z$ s'écrit $a+\ii b$ et $z'$ s'écrit $a'+\ii b'$

alors $z+z'$ s'écrit $a+a' + \ii (b+b')$.


\change

Cela correspond à l'addition de deux vecteurs

on ajoute les abscisses de $z$ et de $z'$
pour obtenir celle de $z+z'$

de même on ajoute les ordonnées

\change

Enfin, très important, 

la formule pour la multiplication découle de notre 
convention $\ii^2 = -1$

le produit $a+\ii b$ fois $a'+\ii b'$ s'obtient en développant cette
expression

c'est donc $aa'$

plus $\ii a b'$

plus $\ii ba'$

plus $\ii^2 bb'$ qui est donc $-bb'$


%%%%%%%%%%%%%%%%%%%%%%%%%%%%%%%%%%%%%%%%%%%%%%%%%%%%%%%%%%%
\diapo

Si $z = a+ \ii b$ 
alors

le réel $a$ est la partie réelle de $z$

et le réel $b$ sa partie imaginaire

\change

Réel(z) est donc l'abscisse de notre point $z$

et Imaginaire(z) son ordonnée

\change

L'écriture sous forme partie réelle 
plus $\ii$ fois partie imaginaire est unique

Plus précisément si deux nombres complexes sont égaux
alors ils ont la même partie réelle et la même partie imaginaire

Réciproquement si deux nombres complexes ont la même partie réelle et la même partie imaginaire
alors il son égaux.



%%%%%%%%%%%%%%%%%%%%%%%%%%%%%%%%%%%%%%%%%%%%%%%%%%%%%%%%%%%
\diapo

L'opposé du nombre complexe $z=a+\ii b$ est le nombre complexe $-a - \ii b$

C'est le symétrique de $z$ par rapport à l'origine

Pour un réel $\lambda$  le nombre complexe $\lambda \times z= \lambda a + \ii \lambda b$
est l'image de $z$ par l'homothétie de centre $O$
et de rapport $\lambda$.

%%%%%%%%%%%%%%%%%%%%%%%%%%%%%%%%%%%%%%%%%%%%%%%%%%%%%%%%%%%
\diapo

Tout nombre complexe non nul $z$ admet un inverse

c'est-à-dire qu'il existe $z'$ tel que $z \times z' =1$

\change

Il est facile de vérifier que cet inverse, noté $\frac 1 z$,
est $\frac{a - \ii b}{a^2 + b^2}$

\change

Une fois que l'on sait calculer l'inverse
le quotient de $z$ par $z'$ est défini comme le produit $z \times \frac 1 {z'}$

(pause)

\change


D'autre part si un produit $z \times z'$ est nul alors l'un des facteurs $z$ ou $z'$
est nul

\change

Enfin on définit $z^n$ comme étant le produit de $z$ par $z$ par $z$, $n$ fois

Notez aussi que par définition $z^0$ égale $1$

Et que $z^{-n} = (1/z)^n$ pour tout entier $n$ positif

\change

Enfin une remarque importante

il n'y a pas d'ordre naturel sur $\Cc$

donc il ne faut jamais écrire quelque chose du genre
$z\ge 0$ ou $z \le z'$


%%%%%%%%%%%%%%%%%%%%%%%%%%%%%%%%%%%%%%%%%%%%%%%%%%%%%%%%%%
\diapo

Nous allons démontrer une formule remarquable : la valeur de la somme 
des termes d'une suite géométrique

Pour tout nombre complexe $z$ différent de $1$
la somme 

$1 + z + z^2 + \cdots + z^n$ vaut $\dfrac{1 - z^{n + 1}}{1 - z}$

\change

Cela s'écrit aussi à l'aide d'une somme formelle
$$\sum_{k=0}^n z^k = \dfrac{1 - z^{n + 1}}{1 - z}$$

\change

La preuve est facile

on multiplie $1 + z + z^2 + \cdots + z^n$ par $1-z$

\change 

et on développe

en multipliant par $1$
on obtient d'abord 
$1 + z + z^2 + \cdots + z^n$


en multipliant par $-z$ on obtient
$-z -z^2 - \cdots - z^n - z^{n+1} $

\change

La plupart des termes s'éliminent 

Il reste $1-z^{n+1}$


%%%%%%%%%%%%%%%%%%%%%%%%%%%%%%%%%%%%%%%%%%%%%%%%%%%%%%%%%%%
\diapo

Le conjugué de $z=a+\ii b$ est le complexe 
$\bar z = a - \ii b$

C'est le symétrique de $z$ par rapport à l'axe des abscisses

\change

Le module de $z = a+\ii b$ est le réel $\sqrt{a^2 + b^2}$
qui vaut aussi $\sqrt{z\overline z}$

\change

En effet $z \times \overline z = (a+\ii b)(a-\ii b) = a^2+b^2$

\change

Le module mesure la distance entre le point $z$ et l'origine

(pause)

\change


Voici quelques propriétés élémentaires sur les conjugués et les modules

\begin{itemize}
  \item $\overline{z + z'} = \overline{z} + \overline{z'}$,\quad $\overline{\overline{z}} =
  z$,\quad $\overline{zz'} = \overline{z}  \overline{z'}$
  
  \item $z = \overline{z} \Longleftrightarrow z \in \Rr$
  
  \item $\left| z \right|^2 = z \times \overline{z}$,\quad $\left| \overline{z} \right| =
  \left| z \right|$, \quad $\left| zz' \right| = \left| z \right|  \left| z'
  \right|$ (le module d'un produit égale le produit des modules)
  
  \item $\left| z \right| = 0 \Longleftrightarrow z = 0$
  
  \item 
Enfin le module de $z+z'$ n'est pas égale au module de $z$ plus le module de $z'$

Mais nous avons une relation importante : 

l'inégalité triangulaire  $\left| z + z' \right| \leqslant \left| z
  \right| + \left| z' \right|$
\end{itemize}


%%%%%%%%%%%%%%%%%%%%%%%%%%%%%%%%%%%%%%%%%%%%%%%%%%%%%%%%%%%
\diapo

Terminons cette première partie avec une application géométrique

``Dans un parallélogramme, la somme des carrés des diagonales égale la
  somme des carrés des c\^otés''



\change

Autrement dit si on note grand $D$ et petit $d$
les longueurs des diagonales

et grand $L$ et petit $\ell$ les longueurs des cotés

[grand/petit]
Alors $D^2+d^2 = 2\ell^2+2L^2$





%%%%%%%%%%%%%%%%%%%%%%%%%%%%%%%%%%%%%%%%%%%%%%%%%%%%%%%%%%%
\diapo


Avec l'aide des nombres complexes la preuve n'est pas difficile

On peut supposer que l'un des sommets à pour coordonnées $0$

le deuxième $z$, et le troisième $z'$

comme la figure est un parallélogramme alors le quatrième a pour
coordonnée $z+z'$

Les longueurs sont mesurés par le modules, ici $|z|$
pour les grand cotés et $|z'|$ pour les petits cotés

\change

On calcule aussi la longueur des diagonales

c'est $|z+z'|$ pour la grande diagonale

Il faut se convaincre que la longueur de la petite diagonale est
$|z-z'|$. Pour cela placer le point $z-z'$.

\change

Ensuite place au calcul...

$D^2 + d^2 = \left| z + z' \right|^2 + \left| z - z' \right|^2$

\change

on utilise ensuite que le module d'un nombre au carré égale ce nombre par son conjugué

\change

on développe, 

\change

on simplifie 

\change 

et on reconnaît $2 \left| z \right|^2$ et $2 \left|    z' \right|^2$

\change

or $|z|$ et $|z'|$ sont les longueurs des cotés

donc on obtient ce que l'on voulait démontrer

$2\ell^2+2L^2$



%%%%%%%%%%%%%%%%%%%%%%%%%%%%%%%%%%%%%%%%%%%%%%%%%%%%%%%%%%%
\diapo

Vérifiez maintenant si vous avez bien compris le cours en répondant aux questions suivantes

\end{document}
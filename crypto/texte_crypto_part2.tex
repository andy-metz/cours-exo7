
%%%%%%%%%%%%%%%%%% PREAMBULE %%%%%%%%%%%%%%%%%%


\documentclass[12pt]{article}

\usepackage{amsfonts,amsmath,amssymb,amsthm}
\usepackage[utf8]{inputenc}
\usepackage[T1]{fontenc}
\usepackage[francais]{babel}


% packages
\usepackage{amsfonts,amsmath,amssymb,amsthm}
\usepackage[utf8]{inputenc}
\usepackage[T1]{fontenc}
%\usepackage{lmodern}

\usepackage[francais]{babel}
\usepackage{fancybox}
\usepackage{graphicx}

\usepackage{float}

%\usepackage[usenames, x11names]{xcolor}
\usepackage{tikz}
\usepackage{datetime}

\usepackage{mathptmx}
%\usepackage{fouriernc}
%\usepackage{newcent}
\usepackage[mathcal,mathbf]{euler}

%\usepackage{palatino}
%\usepackage{newcent}


% Commande spéciale prompteur

%\usepackage{mathptmx}
%\usepackage[mathcal,mathbf]{euler}
%\usepackage{mathpple,multido}

\usepackage[a4paper]{geometry}
\geometry{top=2cm, bottom=2cm, left=1cm, right=1cm, marginparsep=1cm}

\newcommand{\change}{{\color{red}\rule{\textwidth}{1mm}\\}}

\newcounter{mydiapo}

\newcommand{\diapo}{\newpage
\hfill {\normalsize  Diapo \themydiapo \quad \texttt{[\jobname]}} \\
\stepcounter{mydiapo}}


%%%%%%% COULEURS %%%%%%%%%%

% Pour blanc sur noir :
%\pagecolor[rgb]{0.5,0.5,0.5}
% \pagecolor[rgb]{0,0,0}
% \color[rgb]{1,1,1}



%\DeclareFixedFont{\myfont}{U}{cmss}{bx}{n}{18pt}
\newcommand{\debuttexte}{
%%%%%%%%%%%%% FONTES %%%%%%%%%%%%%
\renewcommand{\baselinestretch}{1.5}
\usefont{U}{cmss}{bx}{n}
\bfseries

% Taille normale : commenter le reste !
%Taille Arnaud
%\fontsize{19}{19}\selectfont

% Taille Barbara
%\fontsize{21}{22}\selectfont

%Taille François
\fontsize{25}{30}\selectfont

%Taille Pascal
%\fontsize{25}{30}\selectfont

%Taille Laura
%\fontsize{30}{35}\selectfont


%\myfont
%\usefont{U}{cmss}{bx}{n}

%\Huge
%\addtolength{\parskip}{\baselineskip}
}


% \usepackage{hyperref}
% \hypersetup{colorlinks=true, linkcolor=blue, urlcolor=blue,
% pdftitle={Exo7 - Exercices de mathématiques}, pdfauthor={Exo7}}


%section
% \usepackage{sectsty}
% \allsectionsfont{\bf}
%\sectionfont{\color{Tomato3}\upshape\selectfont}
%\subsectionfont{\color{Tomato4}\upshape\selectfont}

%----- Ensembles : entiers, reels, complexes -----
\newcommand{\Nn}{\mathbb{N}} \newcommand{\N}{\mathbb{N}}
\newcommand{\Zz}{\mathbb{Z}} \newcommand{\Z}{\mathbb{Z}}
\newcommand{\Qq}{\mathbb{Q}} \newcommand{\Q}{\mathbb{Q}}
\newcommand{\Rr}{\mathbb{R}} \newcommand{\R}{\mathbb{R}}
\newcommand{\Cc}{\mathbb{C}} 
\newcommand{\Kk}{\mathbb{K}} \newcommand{\K}{\mathbb{K}}

%----- Modifications de symboles -----
\renewcommand{\epsilon}{\varepsilon}
\renewcommand{\Re}{\mathop{\text{Re}}\nolimits}
\renewcommand{\Im}{\mathop{\text{Im}}\nolimits}
%\newcommand{\llbracket}{\left[\kern-0.15em\left[}
%\newcommand{\rrbracket}{\right]\kern-0.15em\right]}

\renewcommand{\ge}{\geqslant}
\renewcommand{\geq}{\geqslant}
\renewcommand{\le}{\leqslant}
\renewcommand{\leq}{\leqslant}

%----- Fonctions usuelles -----
\newcommand{\ch}{\mathop{\mathrm{ch}}\nolimits}
\newcommand{\sh}{\mathop{\mathrm{sh}}\nolimits}
\renewcommand{\tanh}{\mathop{\mathrm{th}}\nolimits}
\newcommand{\cotan}{\mathop{\mathrm{cotan}}\nolimits}
\newcommand{\Arcsin}{\mathop{\mathrm{Arcsin}}\nolimits}
\newcommand{\Arccos}{\mathop{\mathrm{Arccos}}\nolimits}
\newcommand{\Arctan}{\mathop{\mathrm{Arctan}}\nolimits}
\newcommand{\Argsh}{\mathop{\mathrm{Argsh}}\nolimits}
\newcommand{\Argch}{\mathop{\mathrm{Argch}}\nolimits}
\newcommand{\Argth}{\mathop{\mathrm{Argth}}\nolimits}
\newcommand{\pgcd}{\mathop{\mathrm{pgcd}}\nolimits} 

\newcommand{\Card}{\mathop{\text{Card}}\nolimits}
\newcommand{\Ker}{\mathop{\text{Ker}}\nolimits}
\newcommand{\id}{\mathop{\text{id}}\nolimits}
\newcommand{\ii}{\mathrm{i}}
\newcommand{\dd}{\mathrm{d}}
\newcommand{\Vect}{\mathop{\text{Vect}}\nolimits}
\newcommand{\Mat}{\mathop{\mathrm{Mat}}\nolimits}
\newcommand{\rg}{\mathop{\text{rg}}\nolimits}
\newcommand{\tr}{\mathop{\text{tr}}\nolimits}
\newcommand{\ppcm}{\mathop{\text{ppcm}}\nolimits}

%----- Structure des exercices ------

\newtheoremstyle{styleexo}% name
{2ex}% Space above
{3ex}% Space below
{}% Body font
{}% Indent amount 1
{\bfseries} % Theorem head font
{}% Punctuation after theorem head
{\newline}% Space after theorem head 2
{}% Theorem head spec (can be left empty, meaning ‘normal’)

%\theoremstyle{styleexo}
\newtheorem{exo}{Exercice}
\newtheorem{ind}{Indications}
\newtheorem{cor}{Correction}


\newcommand{\exercice}[1]{} \newcommand{\finexercice}{}
%\newcommand{\exercice}[1]{{\tiny\texttt{#1}}\vspace{-2ex}} % pour afficher le numero absolu, l'auteur...
\newcommand{\enonce}{\begin{exo}} \newcommand{\finenonce}{\end{exo}}
\newcommand{\indication}{\begin{ind}} \newcommand{\finindication}{\end{ind}}
\newcommand{\correction}{\begin{cor}} \newcommand{\fincorrection}{\end{cor}}

\newcommand{\noindication}{\stepcounter{ind}}
\newcommand{\nocorrection}{\stepcounter{cor}}

\newcommand{\fiche}[1]{} \newcommand{\finfiche}{}
\newcommand{\titre}[1]{\centerline{\large \bf #1}}
\newcommand{\addcommand}[1]{}
\newcommand{\video}[1]{}

% Marge
\newcommand{\mymargin}[1]{\marginpar{{\small #1}}}



%----- Presentation ------
\setlength{\parindent}{0cm}

%\newcommand{\ExoSept}{\href{http://exo7.emath.fr}{\textbf{\textsf{Exo7}}}}

\definecolor{myred}{rgb}{0.93,0.26,0}
\definecolor{myorange}{rgb}{0.97,0.58,0}
\definecolor{myyellow}{rgb}{1,0.86,0}

\newcommand{\LogoExoSept}[1]{  % input : echelle
{\usefont{U}{cmss}{bx}{n}
\begin{tikzpicture}[scale=0.1*#1,transform shape]
  \fill[color=myorange] (0,0)--(4,0)--(4,-4)--(0,-4)--cycle;
  \fill[color=myred] (0,0)--(0,3)--(-3,3)--(-3,0)--cycle;
  \fill[color=myyellow] (4,0)--(7,4)--(3,7)--(0,3)--cycle;
  \node[scale=5] at (3.5,3.5) {Exo7};
\end{tikzpicture}}
}



\theoremstyle{definition}
%\newtheorem{proposition}{Proposition}
%\newtheorem{exemple}{Exemple}
%\newtheorem{theoreme}{Théorème}
\newtheorem{lemme}{Lemme}
\newtheorem{corollaire}{Corollaire}
%\newtheorem*{remarque*}{Remarque}
%\newtheorem*{miniexercice}{Mini-exercices}
%\newtheorem{definition}{Définition}




%definition d'un terme
\newcommand{\defi}[1]{{\color{myorange}\textbf{\emph{#1}}}}
\newcommand{\evidence}[1]{{\color{blue}\textbf{\emph{#1}}}}



 %----- Commandes divers ------

\newcommand{\codeinline}[1]{\texttt{#1}}

%%%%%%%%%%%%%%%%%%%%%%%%%%%%%%%%%%%%%%%%%%%%%%%%%%%%%%%%%%%%%
%%%%%%%%%%%%%%%%%%%%%%%%%%%%%%%%%%%%%%%%%%%%%%%%%%%%%%%%%%%%%



\begin{document}

\debuttexte

%%%%%%%%%%%%%%%%%%%%%%%%%%%%%%%%%%%%%%%%%%%%%%%%%%%%%%%%%%%
\diapo

Nous avons vu que le chiffrement de César présente une sécurité très faible, 
la principale raison étant que l'espace des clés est trop petit : il n'y a que $26$
clés de chiffrement possibles. Avec un nombre si petit, il est possible de tester toutes les clés.

\change

Nous allons voir deux nouvelles techniques de chiffrement qui offrent une meilleure sécurité :

\change

Le chiffrement par substitution

\change

et le chiffrement de Vigenère.


%%%%%%%%%%%%%%%%%%%%%%%%%%%%%%%%%%%%%%%%%%%%%%%%%%%%%%%%%%%
\diapo

Commençons par le chiffrement par substitution.

Au lieu de décaler les lettres de l'alphabet comme dans le chiffrement de César, 

on associe maintenant à chaque lettre une autre lettre (sans ordre fixe ou règle générale).

\change

De cette façon par exemple :

{\small
\begin{tabular}{|*{26}{c|}}
\hline
A&B&C&D&E&F&G&H&I&J&K&L&M&N&O&P&Q&R&S&T&U&V&W&X&Y&Z\\
\hline
F&Q&B&M&X&I&T&E&P&A&L&W&H&S&D&O&Z&K&V&G&R&C&N&Y&J&U\\
\hline
\end{tabular}  
}
\medskip

\change

Pour crypter le message \\
{ETRE \ OU \ NE \ PAS \ ETRE \ TELLE \ EST \ LA \ QUESTION}

\change


on regarde la correspondance et on remplace 

$\equiv \equiv \equiv $ la lettre E par la lettre X, [sur le tableau]

puis la lettre T par la lettre G, 

puis la lettre R par la lettre K...

\change


Le message crypté est alors : \\
{XGKX \ DR \ SX \ OFV \ XGKX \ GXWWX \ XVG \ WF \ ZRXVGPDS}

\change

||| Pour le décrypter, en connaissant les substitutions, c'est-à-dire la clé secrète de chiffrement, on fait l'opération inverse.

\change

Nous allons voir que le principal avantage est ici que l'espace des clés est gigantesque.

Il ne sera donc plus question d'énumérer et de tester toutes les possibilités.


\change

Il y a cependant des inconvénients :

\change

tout d'abord : la clé à retenir est beaucoup plus longue, elle est constituée des $26$ lettres de l'alphabet : "FQBMX...".

\change

Mais surtout, nous allons voir que malgré ce grand nombre de clés, il est finalement assez simple de déchiffrer les messages interceptés.


%%%%%%%%%%%%%%%%%%%%%%%%%%%%%%%%%%%%%%%%%%%%%%%%%%%%%%%%%%%
\diapo

Mathématiquement, choisir une clé de chiffrement revient à choisir une bijection de
l'ensemble $\big\{ A,B,\ldots,Z \big\}$ sur lui-même

\change

Il y a $26!$ choix possibles. 

\change

En effet, pour la lettre A de l'ensemble de départ, il y a
$26$ choix possibles dans l'ensemble d'arrivée (nous avions choisi F), 

\change

pour B il reste $25$ choix possibles
(tout sauf F qui est déjà choisi), 

\change

pour C il reste $24$ choix...

\change

enfin pour Z il ne reste qu'une seule possibilité, la seule lettre qui n'a pas encore été choisie.

\change

$\equiv \equiv \equiv $ Au final il y a : $26\times 25 \times 24 \times \cdots \times 2 \times 1$ 

\change

c'est-à-dire $26!$ choix de clés.

\change

||| Ce qui fait environ $4 \times 10^{26}$ clés. C'est un nombre immense.

Il y a plus de clés différentes que de grains de sable
sur Terre ! 

\change

Si un ordinateur pouvait tester $1 \; 000\; 000$ de clés par seconde 

(c'est-à-dire déchiffrer un message et tester son éventuelle signification), 

il lui faudrait alors plus de $12$ millions d'années pour les essayer toutes !!



%%%%%%%%%%%%%%%%%%%%%%%%%%%%%%%%%%%%%%%%%%%%%%%%%%%%%%%%%%%
\diapo

\change

La principale faiblesse du chiffrement par substitution est qu'une même lettre
est toujours chiffrée de la même façon. 

Dans l'exemple précédent E est toujours transformé en X.

\change

Or, dans les textes longs, les lettres %de la langue française %, et ceci est vrai pour toutes les langues, 
n'apparaissent pas toutes avec la même fréquence.

Mais ces fréquences sont à peu de choses près les mêmes quels que soient les textes choisis. 

\change

En français, les lettres les plus rencontrées sont dans l'ordre : 

$\equiv \equiv \equiv $ 
$${E S A I N T R U L O D C P M V Q G F H B X J Y Z K W}$$

%En fait ce classement peut varier en fonction de la taille ou du style des textes choisis. 

\change

Voici quelques fréquences, la lettre $E$ est de loin la plus représentée, mais ici les lettres N et T possèdent des fréquences très très proches.


\change

Ces spécificités liées à chaque langue fournissent une méthode d'attaque que voici : 

\change
dans le texte crypté, on cherche la lettre qui apparaît le plus souvent

\change
si le texte est assez long cela devrait être le $E$ ou plus exactement l'image de la lettre E par la fonction de chiffrement.

\change
la lettre qui apparaît ensuite dans l'étude des fréquences devrait être l'image de la lettre S, 

\change
puis de la lettre A... 

\change
On obtient ainsi par approches successives le message initial sous la forme d'un texte à trous 

\change
||| dans lequel il faut deviner les lettres manquantes.


%%%%%%%%%%%%%%%%%%%%%%%%%%%%%%%%%%%%%%%%%%%%%%%%%%%%%%%%%%%
\diapo

Par exemple, tentons de déchiffrer la phrase suivante : \\
{LHLZ \  HFQ  \ BC HFFPZ \  WH \  YOUPFH  \ MUPZH}

\change
$\equiv \equiv \equiv $ 
On compte les apparitions des lettres : \\
\centerline{H : 6 \quad F : 4 \quad P : 3 \quad Z : 3}

\change
ce qui nous permet de faire dans un premier temps l'hypothèse que la lettre H crypte la lettre E, 
et la lettre F crypte la lettre S,

\change
ce qui donne le texte à trous suivant : 

\centerline{*E** \ ES* \ ** \ ESS** \ *E \ ***SE \ ****E}

\change

D'après les statistiques P et Z devraient se décrypter en (A et I) ou (I et A), ce n'est pas très clair.

\change

Le quatrième mot "HFFPZ", pour l'instant décrypté en "ESS**",

\change

 se complèterait alors en
"ESSAI" ou "ESSIA". La première solution semble donc correcte ! 

\change

Ainsi P crypte A, et
Z crypte I.

\change

La phrase est maintenant celle-ci : 

\centerline{*E*I \  ES* \  ** \  ESSAI \  *E \ ***ASE \ **AIE}


||| En réfléchissant un petit peu, et après quelques essais, on décrypte le message : 

\change

\centerline{CECI \ EST \ UN \ ESSAI \ DE \ PHRASE \ VRAIE}

%%%%%%%%%%%%%%%%%%%%%%%%%%%%%%%%%%%%%%%%%%%%%%%%%%%%%%%%%%%
\diapo


L'espace des clés du chiffrement par substitution est immense, mais le
fait qu'une lettre soit toujours chiffrée de la même façon représente comme on vient de le voir une trop grande faiblesse.
Le chiffrement de Vigenère remédie à ce problème. Voyons comment !


Nous allons nous concentrer non plus sur des lettres mais sur des blocs de lettres.

\change

$\equiv \equiv \equiv $ 
Par exemple on découpe ce message en blocs de longueur $4$ : 
{CETTE \ PHRASE \  NE \ VEUT \ RIEN \ DIRE}

\change

il devient alors ceci :

{CETT \ EPHR \ ASEN \ EVEU \ TRIE \ NDIR \ E}

(les espaces sont purement indicatifs, dans la première phrase ils séparent les mots,
dans la seconde ils séparent les blocs).

\change

Si $k$ est la longueur d'un bloc, alors on choisit une clé constituée 
de $k$ nombres entiers $(n_1,n_2,\ldots,n_k)$ compris entre $0$ à $25$ :


\change

||| Le chiffrement consiste maintenant à effectuer un chiffrement de type César, c'est-à-dire un décalage, 

mais qui dépend de la position de chaque lettre dans les blocs, c'est-à-dire : 

\change

$\equiv \equiv \equiv $ 
un décalage de $n_1$ positions pour la première lettre de chaque bloc,

\change 

un décalage de $n_2$ positions pour la deuxième lettre de chaque bloc,

\change

etc. jusqu'à un décalage de $n_k$ positions pour la $k$-ème et dernière lettre de chaque bloc. 


\change

Pour notre exemple, on choisit comme clé $(3,1,5,2)$ 

alors pour le premier bloc "CETT", on effectue  :

  \change
  un décalage de $3$ pour C qui donne F,
  
  \change
  un décalage de $1$ pour E qui donne F,
  
  \change
  un décalage de $5$ pour le premier T qui donne Y,
  
  \change
  et enfin un décalage de $2$ pour le deuxième T qui donne V. 

\change

Ainsi "CETT" devient "FFYV".

||| Vous remarquerez que les deux lettres T ne sont pas cryptées par la même lettre
et que les deux F ne cryptent pas la même lettre. 

Pour crypter le message en entier, il suffit ensuite de répéter l'opération pour tous les blocs.


%%%%%%%%%%%%%%%%%%%%%%%%%%%%%%%%%%%%%%%%%%%%%%%%%%%%%%%%%%%
\diapo


L'élément de base du chiffrement de Vigenère n'est plus, comme pour le chiffrement de César, une lettre mais un \defi{bloc}, 
c'est-à-dire un groupement de lettres.

La fonction de chiffrement associe maintenant à un bloc de lon\-gueur $k$, un autre bloc de longueur $k$, 

\change

$\equiv \equiv \equiv $ 
ce qui donne en mathématisant les choses :

une fonction $C_{n_1,n_2,\ldots,n_k}$ qui va de
$\Zz/26\Zz \times \Zz/26\Zz \times \cdots \times \Zz/26\Zz$ dans lui même.

et qui au k-uplet $(x_1,x_2,\ldots,x_k)$ associe le k-uplet $(x_1+n_1,x_2+n_2,\ldots,x_k+n_k)$
dans lequel les calculs sont effectués modulo $26$.

\change

||| Chacune des composantes de cette fonction est donc un chiffrement de César,

mais le décalage dépend du rang dans le bloc.

\change

Enfin, la fonction de déchiffrement est simplement $C_{-n_1,-n_2,\ldots,-n_k}$.


%%%%%%%%%%%%%%%%%%%%%%%%%%%%%%%%%%%%%%%%%%%%%%%%%%%%%%%%%%%
\diapo

\change

Pour une clé de longueur $k$, il y a $26^k$ choix possibles.

\change


Par exemple, pour des blocs de longueur $k=4$ cela en donne déjà $456\;976$ clés possibles, et même si un ordinateur teste
toutes les combinaisons possibles sans problème, il n'est pas question pour un humain de parcourir 
cette liste pour trouver le message initial, c'est-à-dire celui qui est compréhensible !

\change

Il persiste tout de même une faiblesse du même ordre que celle rencontrée dans le chiffrement par substitution : la lettre A
n'est pas toujours cryptée par la même lettre, mais si deux lettres A sont situées à la même position dans deux blocs différents 
alors elles seront cryptées par la même lettre.

\change 
$\equiv \equiv \equiv $ 
Par exemple dans "ALPH \ ABET" les deux A des deux blocs ont la même position, 

\change

elles seront donc cryptées par la même lettre, ici, $D$

\change


Une attaque possible est donc la suivante : 

\change

on découpe notre message en plusieurs listes,

\change


les premières lettres de chaque bloc, les deuxièmes lettres de chaque bloc...

\change


et on fait une attaque statistique sur chacun de ces regroupements. En effet, ces regroupements seront chiffrés par un même décalage, c'est-à-dire un cas particulier de chiffrement par substitution.

\change

||| Mais ce type d'attaque n'est bien s\^ur possible que si la taille des blocs 
est petite devant la longueur du texte.


\end{document}

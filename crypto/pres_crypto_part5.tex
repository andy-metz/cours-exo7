
%%%%%%%%%%%%%%%%%% PREAMBULE %%%%%%%%%%%%%%%%%%

\documentclass[aspectratio=169,utf8]{beamer}
%\documentclass[aspectratio=169,handout]{beamer}

\usetheme{Boadilla}
%\usecolortheme{seahorse}
%\usecolortheme[RGB={245,66,24}]{structure}
\useoutertheme{infolines}

% packages
\usepackage{amsfonts,amsmath,amssymb,amsthm}
\usepackage[utf8]{inputenc}
\usepackage[T1]{fontenc}
\usepackage{lmodern}

\usepackage[francais]{babel}
\usepackage{fancybox}
\usepackage{graphicx}

\usepackage{float}
\usepackage{xfrac}

%\usepackage[usenames, x11names]{xcolor}
\usepackage{pgfplots}
\usepackage{datetime}


% ----------------------------------------------------------------------
% Pour les images
\usepackage{tikz}
\usetikzlibrary{calc,shadows,arrows.meta,patterns,matrix}

\newcommand{\tikzinput}[1]{\input{figures/#1.tikz}}
% --- les figures avec échelle éventuel
\newcommand{\myfigure}[2]{% entrée : échelle, fichier(s) figure à inclure
\begin{center}\small%
\tikzstyle{every picture}=[scale=1.0*#1]% mise en échelle + 0% (automatiquement annulé à la fin du groupe)
#2%
\end{center}}



%-----  Package unités -----
\usepackage{siunitx}
\sisetup{locale = FR,detect-all,per-mode = symbol}

%\usepackage{mathptmx}
%\usepackage{fouriernc}
%\usepackage{newcent}
%\usepackage[mathcal,mathbf]{euler}

%\usepackage{palatino}
%\usepackage{newcent}
% \usepackage[mathcal,mathbf]{euler}



% \usepackage{hyperref}
% \hypersetup{colorlinks=true, linkcolor=blue, urlcolor=blue,
% pdftitle={Exo7 - Exercices de mathématiques}, pdfauthor={Exo7}}


%section
% \usepackage{sectsty}
% \allsectionsfont{\bf}
%\sectionfont{\color{Tomato3}\upshape\selectfont}
%\subsectionfont{\color{Tomato4}\upshape\selectfont}

%----- Ensembles : entiers, reels, complexes -----
\newcommand{\Nn}{\mathbb{N}} \newcommand{\N}{\mathbb{N}}
\newcommand{\Zz}{\mathbb{Z}} \newcommand{\Z}{\mathbb{Z}}
\newcommand{\Qq}{\mathbb{Q}} \newcommand{\Q}{\mathbb{Q}}
\newcommand{\Rr}{\mathbb{R}} \newcommand{\R}{\mathbb{R}}
\newcommand{\Cc}{\mathbb{C}} 
\newcommand{\Kk}{\mathbb{K}} \newcommand{\K}{\mathbb{K}}

%----- Modifications de symboles -----
\renewcommand{\epsilon}{\varepsilon}
\renewcommand{\Re}{\mathop{\text{Re}}\nolimits}
\renewcommand{\Im}{\mathop{\text{Im}}\nolimits}
%\newcommand{\llbracket}{\left[\kern-0.15em\left[}
%\newcommand{\rrbracket}{\right]\kern-0.15em\right]}

\renewcommand{\ge}{\geqslant}
\renewcommand{\geq}{\geqslant}
\renewcommand{\le}{\leqslant}
\renewcommand{\leq}{\leqslant}
\renewcommand{\epsilon}{\varepsilon}

%----- Fonctions usuelles -----
\newcommand{\ch}{\mathop{\text{ch}}\nolimits}
\newcommand{\sh}{\mathop{\text{sh}}\nolimits}
\renewcommand{\tanh}{\mathop{\text{th}}\nolimits}
\newcommand{\cotan}{\mathop{\text{cotan}}\nolimits}
\newcommand{\Arcsin}{\mathop{\text{arcsin}}\nolimits}
\newcommand{\Arccos}{\mathop{\text{arccos}}\nolimits}
\newcommand{\Arctan}{\mathop{\text{arctan}}\nolimits}
\newcommand{\Argsh}{\mathop{\text{argsh}}\nolimits}
\newcommand{\Argch}{\mathop{\text{argch}}\nolimits}
\newcommand{\Argth}{\mathop{\text{argth}}\nolimits}
\newcommand{\pgcd}{\mathop{\text{pgcd}}\nolimits} 


%----- Commandes divers ------
\newcommand{\ii}{\mathrm{i}}
\newcommand{\dd}{\text{d}}
\newcommand{\id}{\mathop{\text{id}}\nolimits}
\newcommand{\Ker}{\mathop{\text{Ker}}\nolimits}
\newcommand{\Card}{\mathop{\text{Card}}\nolimits}
\newcommand{\Vect}{\mathop{\text{Vect}}\nolimits}
\newcommand{\Mat}{\mathop{\text{Mat}}\nolimits}
\newcommand{\rg}{\mathop{\text{rg}}\nolimits}
\newcommand{\tr}{\mathop{\text{tr}}\nolimits}


%----- Structure des exercices ------

\newtheoremstyle{styleexo}% name
{2ex}% Space above
{3ex}% Space below
{}% Body font
{}% Indent amount 1
{\bfseries} % Theorem head font
{}% Punctuation after theorem head
{\newline}% Space after theorem head 2
{}% Theorem head spec (can be left empty, meaning ‘normal’)

%\theoremstyle{styleexo}
\newtheorem{exo}{Exercice}
\newtheorem{ind}{Indications}
\newtheorem{cor}{Correction}


\newcommand{\exercice}[1]{} \newcommand{\finexercice}{}
%\newcommand{\exercice}[1]{{\tiny\texttt{#1}}\vspace{-2ex}} % pour afficher le numero absolu, l'auteur...
\newcommand{\enonce}{\begin{exo}} \newcommand{\finenonce}{\end{exo}}
\newcommand{\indication}{\begin{ind}} \newcommand{\finindication}{\end{ind}}
\newcommand{\correction}{\begin{cor}} \newcommand{\fincorrection}{\end{cor}}

\newcommand{\noindication}{\stepcounter{ind}}
\newcommand{\nocorrection}{\stepcounter{cor}}

\newcommand{\fiche}[1]{} \newcommand{\finfiche}{}
\newcommand{\titre}[1]{\centerline{\large \bf #1}}
\newcommand{\addcommand}[1]{}
\newcommand{\video}[1]{}

% Marge
\newcommand{\mymargin}[1]{\marginpar{{\small #1}}}

\def\noqed{\renewcommand{\qedsymbol}{}}


%----- Presentation ------
\setlength{\parindent}{0cm}

%\newcommand{\ExoSept}{\href{http://exo7.emath.fr}{\textbf{\textsf{Exo7}}}}

\definecolor{myred}{rgb}{0.93,0.26,0}
\definecolor{myorange}{rgb}{0.97,0.58,0}
\definecolor{myyellow}{rgb}{1,0.86,0}

\newcommand{\LogoExoSept}[1]{  % input : echelle
{\usefont{U}{cmss}{bx}{n}
\begin{tikzpicture}[scale=0.1*#1,transform shape]
  \fill[color=myorange] (0,0)--(4,0)--(4,-4)--(0,-4)--cycle;
  \fill[color=myred] (0,0)--(0,3)--(-3,3)--(-3,0)--cycle;
  \fill[color=myyellow] (4,0)--(7,4)--(3,7)--(0,3)--cycle;
  \node[scale=5] at (3.5,3.5) {Exo7};
\end{tikzpicture}}
}


\newcommand{\debutmontitre}{
  \author{} \date{} 
  \thispagestyle{empty}
  \hspace*{-10ex}
  \begin{minipage}{\textwidth}
    \titlepage  
  \vspace*{-2.5cm}
  \begin{center}
    \LogoExoSept{2.5}
  \end{center}
  \end{minipage}

  \vspace*{-0cm}
  
  % Astuce pour que le background ne soit pas discrétisé lors de la conversion pdf -> png
\begin{tikzpicture}
        \fill[opacity=0,green!60!black] (0,0)--++(0,0)--++(0,0)--++(0,0)--cycle; 
\end{tikzpicture}

% toc S'affiche trop tot :
% \tableofcontents[hideallsubsections, pausesections]
}

\newcommand{\finmontitre}{
  \end{frame}
  \setcounter{framenumber}{0}
} % ne marche pas pour une raison obscure

%----- Commandes supplementaires ------

% \usepackage[landscape]{geometry}
% \geometry{top=1cm, bottom=3cm, left=2cm, right=10cm, marginparsep=1cm
% }
% \usepackage[a4paper]{geometry}
% \geometry{top=2cm, bottom=2cm, left=2cm, right=2cm, marginparsep=1cm
% }

%\usepackage{standalone}


% New command Arnaud -- november 2011
\setbeamersize{text margin left=24ex}
% si vous modifier cette valeur il faut aussi
% modifier le decalage du titre pour compenser
% (ex : ici =+10ex, titre =-5ex

\theoremstyle{definition}
%\newtheorem{proposition}{Proposition}
%\newtheorem{exemple}{Exemple}
%\newtheorem{theoreme}{Théorème}
%\newtheorem{lemme}{Lemme}
%\newtheorem{corollaire}{Corollaire}
%\newtheorem*{remarque*}{Remarque}
%\newtheorem*{miniexercice}{Mini-exercices}
%\newtheorem{definition}{Définition}

% Commande tikz
\usetikzlibrary{calc}
\usetikzlibrary{patterns,arrows}
\usetikzlibrary{matrix}
\usetikzlibrary{fadings} 

%definition d'un terme
\newcommand{\defi}[1]{{\color{myorange}\textbf{\emph{#1}}}}
\newcommand{\evidence}[1]{{\color{blue}\textbf{\emph{#1}}}}
\newcommand{\assertion}[1]{\emph{\og#1\fg}}  % pour chapitre logique
%\renewcommand{\contentsname}{Sommaire}
\renewcommand{\contentsname}{}
\setcounter{tocdepth}{2}



%------ Encadrement ------

\usepackage{fancybox}


\newcommand{\mybox}[1]{
\setlength{\fboxsep}{7pt}
\begin{center}
\shadowbox{#1}
\end{center}}

\newcommand{\myboxinline}[1]{
\setlength{\fboxsep}{5pt}
\raisebox{-10pt}{
\shadowbox{#1}
}
}

%--------------- Commande beamer---------------
\newcommand{\beameronly}[1]{#1} % permet de mettre des pause dans beamer pas dans poly


\setbeamertemplate{navigation symbols}{}
\setbeamertemplate{footline}  % tiré du fichier beamerouterinfolines.sty
{
  \leavevmode%
  \hbox{%
  \begin{beamercolorbox}[wd=.333333\paperwidth,ht=2.25ex,dp=1ex,center]{author in head/foot}%
    % \usebeamerfont{author in head/foot}\insertshortauthor%~~(\insertshortinstitute)
    \usebeamerfont{section in head/foot}{\bf\insertshorttitle}
  \end{beamercolorbox}%
  \begin{beamercolorbox}[wd=.333333\paperwidth,ht=2.25ex,dp=1ex,center]{title in head/foot}%
    \usebeamerfont{section in head/foot}{\bf\insertsectionhead}
  \end{beamercolorbox}%
  \begin{beamercolorbox}[wd=.333333\paperwidth,ht=2.25ex,dp=1ex,right]{date in head/foot}%
    % \usebeamerfont{date in head/foot}\insertshortdate{}\hspace*{2em}
    \insertframenumber{} / \inserttotalframenumber\hspace*{2ex} 
  \end{beamercolorbox}}%
  \vskip0pt%
}


\definecolor{mygrey}{rgb}{0.5,0.5,0.5}
\setlength{\parindent}{0cm}
%\DeclareTextFontCommand{\helvetica}{\fontfamily{phv}\selectfont}

% background beamer
\definecolor{couleurhaut}{rgb}{0.85,0.9,1}  % creme
\definecolor{couleurmilieu}{rgb}{1,1,1}  % vert pale
\definecolor{couleurbas}{rgb}{0.85,0.9,1}  % blanc
\setbeamertemplate{background canvas}[vertical shading]%
[top=couleurhaut,middle=couleurmilieu,midpoint=0.4,bottom=couleurbas] 
%[top=fondtitre!05,bottom=fondtitre!60]



\makeatletter
\setbeamertemplate{theorem begin}
{%
  \begin{\inserttheoremblockenv}
  {%
    \inserttheoremheadfont
    \inserttheoremname
    \inserttheoremnumber
    \ifx\inserttheoremaddition\@empty\else\ (\inserttheoremaddition)\fi%
    \inserttheorempunctuation
  }%
}
\setbeamertemplate{theorem end}{\end{\inserttheoremblockenv}}

\newenvironment{theoreme}[1][]{%
   \setbeamercolor{block title}{fg=structure,bg=structure!40}
   \setbeamercolor{block body}{fg=black,bg=structure!10}
   \begin{block}{{\bf Th\'eor\`eme }#1}
}{%
   \end{block}%
}


\newenvironment{proposition}[1][]{%
   \setbeamercolor{block title}{fg=structure,bg=structure!40}
   \setbeamercolor{block body}{fg=black,bg=structure!10}
   \begin{block}{{\bf Proposition }#1}
}{%
   \end{block}%
}

\newenvironment{corollaire}[1][]{%
   \setbeamercolor{block title}{fg=structure,bg=structure!40}
   \setbeamercolor{block body}{fg=black,bg=structure!10}
   \begin{block}{{\bf Corollaire }#1}
}{%
   \end{block}%
}

\newenvironment{mydefinition}[1][]{%
   \setbeamercolor{block title}{fg=structure,bg=structure!40}
   \setbeamercolor{block body}{fg=black,bg=structure!10}
   \begin{block}{{\bf Définition} #1}
}{%
   \end{block}%
}

\newenvironment{lemme}[0]{%
   \setbeamercolor{block title}{fg=structure,bg=structure!40}
   \setbeamercolor{block body}{fg=black,bg=structure!10}
   \begin{block}{\bf Lemme}
}{%
   \end{block}%
}

\newenvironment{remarque}[1][]{%
   \setbeamercolor{block title}{fg=black,bg=structure!20}
   \setbeamercolor{block body}{fg=black,bg=structure!5}
   \begin{block}{Remarque #1}
}{%
   \end{block}%
}


\newenvironment{exemple}[1][]{%
   \setbeamercolor{block title}{fg=black,bg=structure!20}
   \setbeamercolor{block body}{fg=black,bg=structure!5}
   \begin{block}{{\bf Exemple }#1}
}{%
   \end{block}%
}


\newenvironment{miniexercice}[0]{%
   \setbeamercolor{block title}{fg=structure,bg=structure!20}
   \setbeamercolor{block body}{fg=black,bg=structure!5}
   \begin{block}{Mini-exercices}
}{%
   \end{block}%
}


\newenvironment{tp}[0]{%
   \setbeamercolor{block title}{fg=structure,bg=structure!40}
   \setbeamercolor{block body}{fg=black,bg=structure!10}
   \begin{block}{\bf Travaux pratiques}
}{%
   \end{block}%
}
\newenvironment{exercicecours}[1][]{%
   \setbeamercolor{block title}{fg=structure,bg=structure!40}
   \setbeamercolor{block body}{fg=black,bg=structure!10}
   \begin{block}{{\bf Exercice }#1}
}{%
   \end{block}%
}
\newenvironment{algo}[1][]{%
   \setbeamercolor{block title}{fg=structure,bg=structure!40}
   \setbeamercolor{block body}{fg=black,bg=structure!10}
   \begin{block}{{\bf Algorithme}\hfill{\color{gray}\texttt{#1}}}
}{%
   \end{block}%
}


\setbeamertemplate{proof begin}{
   \setbeamercolor{block title}{fg=black,bg=structure!20}
   \setbeamercolor{block body}{fg=black,bg=structure!5}
   \begin{block}{{\footnotesize Démonstration}}
   \footnotesize
   \smallskip}
\setbeamertemplate{proof end}{%
   \end{block}}
\setbeamertemplate{qed symbol}{\openbox}


\makeatother
\usecolortheme[RGB={127,0,0}]{structure}

% Commande spécifique à ce chapitre

\newcommand{\Python}{\texttt{Python}}
\renewcommand{\evidence}[1]{{\color{blue}\textbf{#1}}}

\usepackage{textcomp}

\usepackage{listings}
\lstset{
  upquote=true,
  columns=flexible,
  keepspaces=true,
  basicstyle=\ttfamily,
  commentstyle=\color{gray},
  language=Python,
  showstringspaces=false,
  aboveskip=0em,  
  belowskip=0em,
  escapeinside=||
}

\lstset{
  literate={é}{{\'e}}1
           {è}{{\`e}}1
           {à}{{\`a}}1
}


\newcommand{\codeinline}[1]{\lstinline!#1!}

%%%%%%%%%%%%%%%%%%%%%%%%%%%%%%%%%%%%%%%%%%%%%%%%%%%%%%%%%%%%%
%%%%%%%%%%%%%%%%%%%%%%%%%%%%%%%%%%%%%%%%%%%%%%%%%%%%%%%%%%%%%


\begin{document}


\title{{\bf Cryptographie}}
\subtitle{L'arithmétique pour RSA}

\begin{frame}
  
  \debutmontitre

  \pause

{\footnotesize
\hfill
\setbeamercovered{transparent=50}
\begin{minipage}{0.6\textwidth}
  \begin{itemize}
    \item<3-> Le petit théorème de Fermat amélioré
    \item<4-> L'algorithme d'Euclide étendu
    \item<5-> Inverse modulo $n$
    \item<6-> L'exponentiation rapide
  \end{itemize}
\end{minipage}
}

\end{frame}

\setcounter{framenumber}{0}


%%%%%%%%%%%%%%%%%%%%%%%%%%%%%%%%%%%%%%%%%%%%%%%%%%%%%%%%%%%%%%%%
\section{Le petit théorème de Fermat amélioré}

\begin{frame}

\begin{theoreme}[Petit théorème de Fermat]
Si $p$ est un nombre premier et $a \in \Zz$ alors
\mybox{$a^p \equiv a \pmod p$}
\end{theoreme}


\pause
\bigskip

\begin{corollaire}
Si $p$ ne divise pas $a$ alors
\mybox{$a^{p-1} \equiv 1 \pmod p$} 
\end{corollaire}
\end{frame}


\begin{frame}

\begin{theoreme}[Petit théorème de Fermat amélioré]
Soient $p$ et $q$ deux nombres premiers distincts et soit $n = p q$. Pour tout
$a\in \Zz$ tel que $\pgcd(a,n)=1$ alors :
\mybox{$a^{(p-1)(q-1)} \equiv 1 \pmod n$}
\end{theoreme}

\pause
\bigskip

\begin{itemize}
%  \item On note $\varphi(n)=(p-1)(q-1)$ la \defi{fonction d'Euler}
%  \pause
  \item $\pgcd(a,n)=1 \iff $ $p$ et $q$ ne divisent par $a$

  \pause
  \item Exemple : $p=5$, $q=7$
  \begin{itemize}
    \item $n = p \times q=35$
    \item $%\varphi(n) =
 (p-1)\times (q-1) =4\times 6 = 24$
    \item Pour $a\!=\!1,2,3,4,6,8,9,11,12,13,16,17,18,...$ \quad $a^{24}\! \equiv \!1 \!\pmod{\!35}$
  \end{itemize}
 
\end{itemize}
\end{frame}


\begin{frame}

\begin{theoreme}[Petit théorème de Fermat amélioré]
Soient $p$ et $q$ deux nombres premiers distincts et soit $n = p q$. Pour tout
$a\in \Zz$ tel que $\pgcd(a,n)=1$ alors :
%\vspace*{-2ex}
\myboxinline{$a^{(p-1)(q-1)} \equiv 1 \pmod n$}
\end{theoreme}


\begin{proof}
\pause
\vspace*{-1ex}
Notons $c = a^{(p-1)(q-1)}$
\begin{itemize}
\pause
  \item Calculons $c$ modulo $p$ :  $c \equiv a^{(p-1)(q-1)} \equiv (a^{(p-1)})^{q-1} \equiv 1^{q-1} \equiv 1 \pmod{p}$
\pause
  \item Calculons $c$ modulo $q$ :  $c \equiv a^{(p-1)(q-1)} \equiv (a^{(q-1)})^{p-1} \equiv 1^{p-1} \equiv 1 \pmod{q}$
\pause
  \item Nous allons en déduire que $c \equiv 1 \pmod{pq}$  
  \pause
  \begin{itemize}[<+->]
    \item comme $c \equiv 1 \pmod{p}$ alors il existe $\alpha \in \Zz$ tel que $c = 1 + \alpha p$
    \item comme $c \equiv 1 \pmod{q}$ alors il existe $\beta \in \Zz$ tel que $c = 1 + \beta q$
    \item donc $c-1 = \alpha p = \beta q$ donc $p | \beta q$
    \item comme $p$ et $q$ sont premiers entre eux $p | \beta$
    \item il existe donc $\beta'\in \Zz$ tel que $\beta = \beta' p $
    \item ainsi $c = 1 + \beta q = 1 + \beta' pq$
    \item d'où $c \equiv 1 \pmod{pq}$ \qedhere
  \end{itemize}
  
\end{itemize}
\vspace*{-1ex}
\end{proof}
 
\end{frame}



%%%%%%%%%%%%%%%%%%%%%%%%%%%%%%%%%%%%%%%%%%%%%%%%%%%%%%%%%%%%%%%%
\section{L'algorithme d'Euclide étendu}

\begin{frame}[fragile]
Principe de l'\evidence{algorithme d'Euclide}
$$\pgcd(a,b)= \pgcd(b, a \pmod{b})$$

\pause
\bigskip

$$a_0=a, b_0=b \qquad \text{puis} \qquad \left\{\begin{array}{rcl}
           a_{i+1} & = & b_i \\
           b_{i+1} & = & a_i \pmod {b_i} \\
         \end{array}
 \right.$$

\pause
\bigskip

\begin{algo}[euclide.py (1)]
\begin{lstlisting}
def euclide(a,b):
    while b !=0 :
        a , b = b , a % b
    return a
\end{lstlisting}  
\end{algo}    
   
\end{frame}


\begin{frame}[fragile]

L'\defi{algorithme d'Euclide étendu} pour obtenir les coefficients de Bézout $u,v$ tels que $au+bv=\pgcd(a,b)$

\pause
\medskip

$$x_0=1 \qquad  x_1=0 \qquad y_0=0 \qquad y_1 = 1$$

Puis pour $i\geq 1$
$$x_{i+1} = x_{i-1} - q_i x_i \qquad y_{i+1} = y_{i-1} - q_i y_i$$
 où $q_i$ est le quotient de la division euclidienne de $a_i$ par $b_i$

 
\pause

{\small
\begin{algo}[euclide.py (2)]
\vspace*{-1ex}
\begin{lstlisting}
def euclide_etendu(a,b):
    x = 1 ; xx = 0
    y = 0 ; yy = 1 
    while b !=0 :
        q = a // b
        a , b = b , a % b
        xx , x = x - q*xx , xx
        yy , y = y - q*yy , yy
    return (a,x,y)
\end{lstlisting}
\vspace*{-1ex}
\end{algo}
}


\end{frame}


%%%%%%%%%%%%%%%%%%%%%%%%%%%%%%%%%%%%%%%%%%%%%%%%%%%%%%%%%%%%%%%%
\section{Inverse modulo $n$}

\begin{frame}

Soit $a \in \Zz$, on dit que $x \in \Zz$ est un \defi{inverse de $a$ modulo $n$}
si \mybox{$ax \equiv 1 \pmod{n}$}

\pause

\begin{proposition}
\begin{itemize}
  \item $a$ admet un inverse modulo $n$ si et seulement si $\pgcd(a,n)=1$
\pause
  \item Si $au+nv=1$ alors $u$ est un inverse de $a$ modulo $n$
\end{itemize}  
\end{proposition}

\pause

\begin{proof}
$$\begin{array}{rcl}
\pgcd(a,n)=1 
\pause
       & \iff & \exists u,v \in \Zz \qquad au+nv=1 \\
\pause
       & \iff & \exists u \in \Zz \qquad au \equiv 1 \pmod {n} \\
\end{array}$$
\end{proof}


\end{frame}
% 
% \begin{frame}[fragile]
% \begin{algo}[euclide.py (3)]
% \begin{lstlisting}
% def inverse(a,n):
%     c,u,v = euclide_etendu(a,n) # pgcd et coeff de Bézout
%     if c != 1 :        # Si pgcd différent de 1 renvoie 0
%         return 0
%     else :
%         return u % n                  # Renvoie l'inverse 
% \end{lstlisting}
% \end{algo}
% \end{frame}



%%%%%%%%%%%%%%%%%%%%%%%%%%%%%%%%%%%%%%%%%%%%%%%%%%%%%%%%%%%%%%%%
\section{L'exponentiation rapide}

\begin{frame}


\evidence{Calcul efficace de $a^k \pmod n$}

\bigskip
\pause

Exemple : calcul de $5^{11} \pmod{14}$
\pause
\begin{enumerate}
  \item On remarque que $11 = 8 + 2 + 1$
\pause
donc\vspace*{-1ex}
$$5 ^{11} = 5^8 \times 5^2 \times 5^1$$

\pause
  \item Calculons les $5^{2^i} \pmod {14}$ 
\pause\vspace*{-1ex}
$$
\begin{array}{l}
5  \equiv  5 \pmod{14} \\
\pause
5^2  \equiv  25 \equiv 11 \pmod{14} \\
\pause
5^4  \pause \equiv 5^2 \times 5^2 \pause \equiv 11 \times 11 \pause \equiv 121 \pause\equiv 9 \pmod{14} \\
\pause
5^8  \pause\equiv 5^4 \times 5^4 \pause\equiv 9 \times 9 \pause\equiv 81 \pause\equiv 11 \pmod{14} \\
\end{array}
$$

\pause  
  \item Conséquence 
  
\qquad \qquad $5^{11} \equiv 5^8 \times 5^2 \times 5^1 \pause \equiv 11 \times 11 \times 5$

\pause\qquad\qquad$\quad \ \  \equiv 11 \times 55 \pause\equiv 11 \times 13 \pause\equiv 143 \pause\equiv 3 \pmod{14}$
\end{enumerate}

\end{frame}

\begin{frame}
\evidence{Méthode générale}

\begin{enumerate}
  \item On écrit le développement de l'exposant $k$ en base $2$ 
  \begin{itemize}
    \item $(k_\ell, \ldots, k_2, k_1, k_0)$ avec $k_i \in \{0,1\}$
    
    \item $k = \sum_{i=0}^\ell k_i 2^i$

  \end{itemize}


\pause

  \item On calcule successivement les $x^{2^i}$ modulo $n$ sachant $x^{2^{i+1}} = x^{2^i} \times x^{2^i}$

\pause
  
  \item On rassemble   
  $$x^k = x^{\sum_{i=0}^\ell k_i 2^i} = \prod_{i=0}^\ell (x^{2^i})^{k_i}$$
  
\end{enumerate}

\pause
\bigskip

 Exemple
 \begin{enumerate}
   \item $11$ en base $2$ s'écrit $({\color{red}1},{\color{red}0},{\color{red}1},{\color{red}1})$
   \pause
   \item On calcule $5^{2^0}$, $5^{2^1}$, $5^{2^2}$, $5^{2^3}$   
   \pause
   \item $5^{11} = (5^{2^3})^{{\color{red}1}} \times (5^{2^2})^{{\color{red}0}}
\times (5^{2^1})^{{\color{red}1}} \times (5^{2^0})^{{\color{red}1}}$
 \end{enumerate}

\end{frame}


\begin{frame}

Calcul de $17^{154} \pmod {100}$


\begin{enumerate}

\pause  
  \item $k=154$ en base $2$ : $154 = 128 + 16 + 8 + 2 = 2^7 + 2^4 + 2^3 + 2^1$

\pause
  $154$ s'écrit donc en base $2$ : $(1,0,0,1,1,0,1,0)$
  
\pause  

  \item Calcul $17, 17^2, 17^4, 17^8,...,17^{128}$ modulo $100$
  \vspace*{-2ex}
  \pause
$$
\begin{array}{l}
17  \equiv  17 \pmod{100} \\
17^2  \equiv  17 \times 17 \equiv 289 \equiv 89 \pmod{100} \\
17^4  \equiv 17^2 \times 17^2 \equiv 89 \times 89 \equiv 7921 \equiv 21 \pmod{100} \\
17^8  \equiv 17^4 \times 17^4 \equiv 21 \times 21 \equiv 441 \equiv 41 \pmod{100} \\
17^{16}  \equiv 17^8 \times 17^8 \equiv 41 \times 41 \equiv 1681 \equiv 81 \pmod{100} \\
17^{32}  \equiv 17^{16} \times 17^{16} \equiv 81 \times 81 \equiv 6561 \equiv 61 \pmod{100} \\
17^{64}  \equiv 17^{32} \times 17^{32} \equiv 61 \times 61 \equiv 3721 \equiv 21 \pmod{100} \\
17^{128}  \equiv 17^{64} \times 17^{64} \equiv 21 \times 21 \equiv 441 \equiv 41 \pmod{100} \\
\end{array}
$$

\pause 

  \item $17^{154} \equiv 17^{128} \times 17^{16} \times 17^{8} \times 17^{2}
\equiv 41 \times 81 \times 41 \times 89 
$

$ \quad \quad \ \, \equiv 3321 \times 3649 \equiv 21 \times 49\equiv 1029 \equiv 29 \pmod {100}$
\end{enumerate}
\end{frame}


\begin{frame}[fragile]

\begin{algo}[euclide.py (3)]
\begin{lstlisting}
def puissance(x,k,n):
    puiss = 1                      # Résultat
    while (k>0):
        if k % 2 != 0 :            # Si k est impair
            puiss = (puiss*x) % n  
        x = x*x % n                # Vaut x, x^2, x^4,...
        k = k // 2 
    return(puiss)  
\end{lstlisting}
\end{algo}

\pause
\bigskip

Commande \Python\ \ \codeinline{pow(x,k,n)} \ pour le calcul de $x^k$ modulo $n$

\pause

\'Eviter : \codeinline{(x ** k) \% n}
\end{frame}

\end{document}

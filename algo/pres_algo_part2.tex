
%%%%%%%%%%%%%%%%%% PREAMBULE %%%%%%%%%%%%%%%%%%

\documentclass[aspectratio=169,utf8]{beamer}
%\documentclass[aspectratio=169,handout]{beamer}

\usetheme{Boadilla}
%\usecolortheme{seahorse}
%\usecolortheme[RGB={245,66,24}]{structure}
\useoutertheme{infolines}

% packages
\usepackage{amsfonts,amsmath,amssymb,amsthm}
\usepackage[utf8]{inputenc}
\usepackage[T1]{fontenc}
\usepackage{lmodern}

\usepackage[francais]{babel}
\usepackage{fancybox}
\usepackage{graphicx}

\usepackage{float}
\usepackage{xfrac}

%\usepackage[usenames, x11names]{xcolor}
\usepackage{pgfplots}
\usepackage{datetime}


% ----------------------------------------------------------------------
% Pour les images
\usepackage{tikz}
\usetikzlibrary{calc,shadows,arrows.meta,patterns,matrix}

\newcommand{\tikzinput}[1]{\input{figures/#1.tikz}}
% --- les figures avec échelle éventuel
\newcommand{\myfigure}[2]{% entrée : échelle, fichier(s) figure à inclure
\begin{center}\small%
\tikzstyle{every picture}=[scale=1.0*#1]% mise en échelle + 0% (automatiquement annulé à la fin du groupe)
#2%
\end{center}}



%-----  Package unités -----
\usepackage{siunitx}
\sisetup{locale = FR,detect-all,per-mode = symbol}

%\usepackage{mathptmx}
%\usepackage{fouriernc}
%\usepackage{newcent}
%\usepackage[mathcal,mathbf]{euler}

%\usepackage{palatino}
%\usepackage{newcent}
% \usepackage[mathcal,mathbf]{euler}



% \usepackage{hyperref}
% \hypersetup{colorlinks=true, linkcolor=blue, urlcolor=blue,
% pdftitle={Exo7 - Exercices de mathématiques}, pdfauthor={Exo7}}


%section
% \usepackage{sectsty}
% \allsectionsfont{\bf}
%\sectionfont{\color{Tomato3}\upshape\selectfont}
%\subsectionfont{\color{Tomato4}\upshape\selectfont}

%----- Ensembles : entiers, reels, complexes -----
\newcommand{\Nn}{\mathbb{N}} \newcommand{\N}{\mathbb{N}}
\newcommand{\Zz}{\mathbb{Z}} \newcommand{\Z}{\mathbb{Z}}
\newcommand{\Qq}{\mathbb{Q}} \newcommand{\Q}{\mathbb{Q}}
\newcommand{\Rr}{\mathbb{R}} \newcommand{\R}{\mathbb{R}}
\newcommand{\Cc}{\mathbb{C}} 
\newcommand{\Kk}{\mathbb{K}} \newcommand{\K}{\mathbb{K}}

%----- Modifications de symboles -----
\renewcommand{\epsilon}{\varepsilon}
\renewcommand{\Re}{\mathop{\text{Re}}\nolimits}
\renewcommand{\Im}{\mathop{\text{Im}}\nolimits}
%\newcommand{\llbracket}{\left[\kern-0.15em\left[}
%\newcommand{\rrbracket}{\right]\kern-0.15em\right]}

\renewcommand{\ge}{\geqslant}
\renewcommand{\geq}{\geqslant}
\renewcommand{\le}{\leqslant}
\renewcommand{\leq}{\leqslant}
\renewcommand{\epsilon}{\varepsilon}

%----- Fonctions usuelles -----
\newcommand{\ch}{\mathop{\text{ch}}\nolimits}
\newcommand{\sh}{\mathop{\text{sh}}\nolimits}
\renewcommand{\tanh}{\mathop{\text{th}}\nolimits}
\newcommand{\cotan}{\mathop{\text{cotan}}\nolimits}
\newcommand{\Arcsin}{\mathop{\text{arcsin}}\nolimits}
\newcommand{\Arccos}{\mathop{\text{arccos}}\nolimits}
\newcommand{\Arctan}{\mathop{\text{arctan}}\nolimits}
\newcommand{\Argsh}{\mathop{\text{argsh}}\nolimits}
\newcommand{\Argch}{\mathop{\text{argch}}\nolimits}
\newcommand{\Argth}{\mathop{\text{argth}}\nolimits}
\newcommand{\pgcd}{\mathop{\text{pgcd}}\nolimits} 


%----- Commandes divers ------
\newcommand{\ii}{\mathrm{i}}
\newcommand{\dd}{\text{d}}
\newcommand{\id}{\mathop{\text{id}}\nolimits}
\newcommand{\Ker}{\mathop{\text{Ker}}\nolimits}
\newcommand{\Card}{\mathop{\text{Card}}\nolimits}
\newcommand{\Vect}{\mathop{\text{Vect}}\nolimits}
\newcommand{\Mat}{\mathop{\text{Mat}}\nolimits}
\newcommand{\rg}{\mathop{\text{rg}}\nolimits}
\newcommand{\tr}{\mathop{\text{tr}}\nolimits}


%----- Structure des exercices ------

\newtheoremstyle{styleexo}% name
{2ex}% Space above
{3ex}% Space below
{}% Body font
{}% Indent amount 1
{\bfseries} % Theorem head font
{}% Punctuation after theorem head
{\newline}% Space after theorem head 2
{}% Theorem head spec (can be left empty, meaning ‘normal’)

%\theoremstyle{styleexo}
\newtheorem{exo}{Exercice}
\newtheorem{ind}{Indications}
\newtheorem{cor}{Correction}


\newcommand{\exercice}[1]{} \newcommand{\finexercice}{}
%\newcommand{\exercice}[1]{{\tiny\texttt{#1}}\vspace{-2ex}} % pour afficher le numero absolu, l'auteur...
\newcommand{\enonce}{\begin{exo}} \newcommand{\finenonce}{\end{exo}}
\newcommand{\indication}{\begin{ind}} \newcommand{\finindication}{\end{ind}}
\newcommand{\correction}{\begin{cor}} \newcommand{\fincorrection}{\end{cor}}

\newcommand{\noindication}{\stepcounter{ind}}
\newcommand{\nocorrection}{\stepcounter{cor}}

\newcommand{\fiche}[1]{} \newcommand{\finfiche}{}
\newcommand{\titre}[1]{\centerline{\large \bf #1}}
\newcommand{\addcommand}[1]{}
\newcommand{\video}[1]{}

% Marge
\newcommand{\mymargin}[1]{\marginpar{{\small #1}}}

\def\noqed{\renewcommand{\qedsymbol}{}}


%----- Presentation ------
\setlength{\parindent}{0cm}

%\newcommand{\ExoSept}{\href{http://exo7.emath.fr}{\textbf{\textsf{Exo7}}}}

\definecolor{myred}{rgb}{0.93,0.26,0}
\definecolor{myorange}{rgb}{0.97,0.58,0}
\definecolor{myyellow}{rgb}{1,0.86,0}

\newcommand{\LogoExoSept}[1]{  % input : echelle
{\usefont{U}{cmss}{bx}{n}
\begin{tikzpicture}[scale=0.1*#1,transform shape]
  \fill[color=myorange] (0,0)--(4,0)--(4,-4)--(0,-4)--cycle;
  \fill[color=myred] (0,0)--(0,3)--(-3,3)--(-3,0)--cycle;
  \fill[color=myyellow] (4,0)--(7,4)--(3,7)--(0,3)--cycle;
  \node[scale=5] at (3.5,3.5) {Exo7};
\end{tikzpicture}}
}


\newcommand{\debutmontitre}{
  \author{} \date{} 
  \thispagestyle{empty}
  \hspace*{-10ex}
  \begin{minipage}{\textwidth}
    \titlepage  
  \vspace*{-2.5cm}
  \begin{center}
    \LogoExoSept{2.5}
  \end{center}
  \end{minipage}

  \vspace*{-0cm}
  
  % Astuce pour que le background ne soit pas discrétisé lors de la conversion pdf -> png
\begin{tikzpicture}
        \fill[opacity=0,green!60!black] (0,0)--++(0,0)--++(0,0)--++(0,0)--cycle; 
\end{tikzpicture}

% toc S'affiche trop tot :
% \tableofcontents[hideallsubsections, pausesections]
}

\newcommand{\finmontitre}{
  \end{frame}
  \setcounter{framenumber}{0}
} % ne marche pas pour une raison obscure

%----- Commandes supplementaires ------

% \usepackage[landscape]{geometry}
% \geometry{top=1cm, bottom=3cm, left=2cm, right=10cm, marginparsep=1cm
% }
% \usepackage[a4paper]{geometry}
% \geometry{top=2cm, bottom=2cm, left=2cm, right=2cm, marginparsep=1cm
% }

%\usepackage{standalone}


% New command Arnaud -- november 2011
\setbeamersize{text margin left=24ex}
% si vous modifier cette valeur il faut aussi
% modifier le decalage du titre pour compenser
% (ex : ici =+10ex, titre =-5ex

\theoremstyle{definition}
%\newtheorem{proposition}{Proposition}
%\newtheorem{exemple}{Exemple}
%\newtheorem{theoreme}{Théorème}
%\newtheorem{lemme}{Lemme}
%\newtheorem{corollaire}{Corollaire}
%\newtheorem*{remarque*}{Remarque}
%\newtheorem*{miniexercice}{Mini-exercices}
%\newtheorem{definition}{Définition}

% Commande tikz
\usetikzlibrary{calc}
\usetikzlibrary{patterns,arrows}
\usetikzlibrary{matrix}
\usetikzlibrary{fadings} 

%definition d'un terme
\newcommand{\defi}[1]{{\color{myorange}\textbf{\emph{#1}}}}
\newcommand{\evidence}[1]{{\color{blue}\textbf{\emph{#1}}}}
\newcommand{\assertion}[1]{\emph{\og#1\fg}}  % pour chapitre logique
%\renewcommand{\contentsname}{Sommaire}
\renewcommand{\contentsname}{}
\setcounter{tocdepth}{2}



%------ Encadrement ------

\usepackage{fancybox}


\newcommand{\mybox}[1]{
\setlength{\fboxsep}{7pt}
\begin{center}
\shadowbox{#1}
\end{center}}

\newcommand{\myboxinline}[1]{
\setlength{\fboxsep}{5pt}
\raisebox{-10pt}{
\shadowbox{#1}
}
}

%--------------- Commande beamer---------------
\newcommand{\beameronly}[1]{#1} % permet de mettre des pause dans beamer pas dans poly


\setbeamertemplate{navigation symbols}{}
\setbeamertemplate{footline}  % tiré du fichier beamerouterinfolines.sty
{
  \leavevmode%
  \hbox{%
  \begin{beamercolorbox}[wd=.333333\paperwidth,ht=2.25ex,dp=1ex,center]{author in head/foot}%
    % \usebeamerfont{author in head/foot}\insertshortauthor%~~(\insertshortinstitute)
    \usebeamerfont{section in head/foot}{\bf\insertshorttitle}
  \end{beamercolorbox}%
  \begin{beamercolorbox}[wd=.333333\paperwidth,ht=2.25ex,dp=1ex,center]{title in head/foot}%
    \usebeamerfont{section in head/foot}{\bf\insertsectionhead}
  \end{beamercolorbox}%
  \begin{beamercolorbox}[wd=.333333\paperwidth,ht=2.25ex,dp=1ex,right]{date in head/foot}%
    % \usebeamerfont{date in head/foot}\insertshortdate{}\hspace*{2em}
    \insertframenumber{} / \inserttotalframenumber\hspace*{2ex} 
  \end{beamercolorbox}}%
  \vskip0pt%
}


\definecolor{mygrey}{rgb}{0.5,0.5,0.5}
\setlength{\parindent}{0cm}
%\DeclareTextFontCommand{\helvetica}{\fontfamily{phv}\selectfont}

% background beamer
\definecolor{couleurhaut}{rgb}{0.85,0.9,1}  % creme
\definecolor{couleurmilieu}{rgb}{1,1,1}  % vert pale
\definecolor{couleurbas}{rgb}{0.85,0.9,1}  % blanc
\setbeamertemplate{background canvas}[vertical shading]%
[top=couleurhaut,middle=couleurmilieu,midpoint=0.4,bottom=couleurbas] 
%[top=fondtitre!05,bottom=fondtitre!60]



\makeatletter
\setbeamertemplate{theorem begin}
{%
  \begin{\inserttheoremblockenv}
  {%
    \inserttheoremheadfont
    \inserttheoremname
    \inserttheoremnumber
    \ifx\inserttheoremaddition\@empty\else\ (\inserttheoremaddition)\fi%
    \inserttheorempunctuation
  }%
}
\setbeamertemplate{theorem end}{\end{\inserttheoremblockenv}}

\newenvironment{theoreme}[1][]{%
   \setbeamercolor{block title}{fg=structure,bg=structure!40}
   \setbeamercolor{block body}{fg=black,bg=structure!10}
   \begin{block}{{\bf Th\'eor\`eme }#1}
}{%
   \end{block}%
}


\newenvironment{proposition}[1][]{%
   \setbeamercolor{block title}{fg=structure,bg=structure!40}
   \setbeamercolor{block body}{fg=black,bg=structure!10}
   \begin{block}{{\bf Proposition }#1}
}{%
   \end{block}%
}

\newenvironment{corollaire}[1][]{%
   \setbeamercolor{block title}{fg=structure,bg=structure!40}
   \setbeamercolor{block body}{fg=black,bg=structure!10}
   \begin{block}{{\bf Corollaire }#1}
}{%
   \end{block}%
}

\newenvironment{mydefinition}[1][]{%
   \setbeamercolor{block title}{fg=structure,bg=structure!40}
   \setbeamercolor{block body}{fg=black,bg=structure!10}
   \begin{block}{{\bf Définition} #1}
}{%
   \end{block}%
}

\newenvironment{lemme}[0]{%
   \setbeamercolor{block title}{fg=structure,bg=structure!40}
   \setbeamercolor{block body}{fg=black,bg=structure!10}
   \begin{block}{\bf Lemme}
}{%
   \end{block}%
}

\newenvironment{remarque}[1][]{%
   \setbeamercolor{block title}{fg=black,bg=structure!20}
   \setbeamercolor{block body}{fg=black,bg=structure!5}
   \begin{block}{Remarque #1}
}{%
   \end{block}%
}


\newenvironment{exemple}[1][]{%
   \setbeamercolor{block title}{fg=black,bg=structure!20}
   \setbeamercolor{block body}{fg=black,bg=structure!5}
   \begin{block}{{\bf Exemple }#1}
}{%
   \end{block}%
}


\newenvironment{miniexercice}[0]{%
   \setbeamercolor{block title}{fg=structure,bg=structure!20}
   \setbeamercolor{block body}{fg=black,bg=structure!5}
   \begin{block}{Mini-exercices}
}{%
   \end{block}%
}


\newenvironment{tp}[0]{%
   \setbeamercolor{block title}{fg=structure,bg=structure!40}
   \setbeamercolor{block body}{fg=black,bg=structure!10}
   \begin{block}{\bf Travaux pratiques}
}{%
   \end{block}%
}
\newenvironment{exercicecours}[1][]{%
   \setbeamercolor{block title}{fg=structure,bg=structure!40}
   \setbeamercolor{block body}{fg=black,bg=structure!10}
   \begin{block}{{\bf Exercice }#1}
}{%
   \end{block}%
}
\newenvironment{algo}[1][]{%
   \setbeamercolor{block title}{fg=structure,bg=structure!40}
   \setbeamercolor{block body}{fg=black,bg=structure!10}
   \begin{block}{{\bf Algorithme}\hfill{\color{gray}\texttt{#1}}}
}{%
   \end{block}%
}


\setbeamertemplate{proof begin}{
   \setbeamercolor{block title}{fg=black,bg=structure!20}
   \setbeamercolor{block body}{fg=black,bg=structure!5}
   \begin{block}{{\footnotesize Démonstration}}
   \footnotesize
   \smallskip}
\setbeamertemplate{proof end}{%
   \end{block}}
\setbeamertemplate{qed symbol}{\openbox}


\makeatother
\usecolortheme[RGB={192,41,0}]{structure}

% Commande spécifique à ce chapitre

\newcommand{\Python}{\texttt{Python}}
\renewcommand{\evidence}[1]{{\color{blue}\textbf{#1}}}

\usepackage{textcomp}

\usepackage{listings}
\lstset{
  upquote=true,
  columns=flexible,
  keepspaces=true,
  basicstyle=\ttfamily,
  commentstyle=\color{gray},
  language=Python,
  showstringspaces=false,
  aboveskip=0em,  
  belowskip=0em,
  escapeinside=||
}

\lstset{
  literate={é}{{\'e}}1
           {è}{{\`e}}1
           {à}{{\`a}}1
}


\newcommand{\codeinline}[1]{\lstinline!#1!}

\newcounter{myi}
\newcounter{myx}

%%%%%%%%%%%%%%%%%%%%%%%%%%%%%%%%%%%%%%%%%%%%%%%%%%%%%%%%%%%%%
%%%%%%%%%%%%%%%%%%%%%%%%%%%%%%%%%%%%%%%%%%%%%%%%%%%%%%%%%%%%%


\begin{document}


\title{{\bf Algorithmes et mathématiques}}
\subtitle{\'Ecriture des entiers}

\begin{frame}
  
  \debutmontitre

  \pause

{\footnotesize
\hfill
\setbeamercovered{transparent=50}
\begin{minipage}{0.6\textwidth}
  \begin{itemize}
    \item<3-> Division euclidienne, calculs avec les modulo
    \item<4-> \'Ecriture des nombres en base $10$
    \item<5-> Module \codeinline{math}
    \item<6-> \'Ecriture des nombres en base $2$ 
  \end{itemize}
\end{minipage}
}

\end{frame}

\setcounter{framenumber}{0}


%%%%%%%%%%%%%%%%%%%%%%%%%%%%%%%%%%%%%%%%%%%%%%%%%%%%%%%%%%%%%%%%
\section{Division euclidienne et reste, calcul avec les modulo}

\begin{frame}

La division euclidienne de $a$ par $b$ ($a \in \Zz$, $b \in \Zz^*$) est 
\mybox{$a = bq+r \quad \text{ et } \quad 0 \le r < b$}

\pause

\begin{itemize}
  \item $q \in \Zz$ est le \defi{quotient} \pause  \qquad \codeinline{a // b}
\pause
  \item $r \in \Nn$ est le \defi{reste} \pause \qquad \codeinline{a \% b}
\pause
  \item Exemple
  \begin{itemize}
    \item \codeinline{14 // 3} retourne $4$
\pause
    \item \codeinline{14 \% 3} retourne $2$
\pause
    \item $14 = 3 \times 4 + 2$
\pause
    \item Test de pair/impair \codeinline{if (n\%2 == 0): ... else: ...}
\pause
    \item Discussion $\cos( n\frac\pi2)$ suivant \codeinline{n\%4}
  \end{itemize}  
\end{itemize}

\end{frame}


\begin{frame}[fragile]

\begin{tp}
Combien y-a-t-il d’occurrences du chiffre $1$ dans les nombres de $1$ à $999$ ?   
Par exemple le chiffre $1$ apparaît une fois dans $51$ mais deux fois dans $131$.
\end{tp}

\pause

\begin{algo}[nb-un.py]
\small
\begin{lstlisting}
NbDeUn = 0
for N in range(1,999+1):
    ChiffreUnite = N % 10              |\pause|
    ChiffreDizaine = (N // 10) % 10    |\pause|
    ChiffreCentaine = (N // 100) % 10  |\pause| 
    if (ChiffreUnite == 1):
        NbDeUn = NbDeUn + 1            |\pause|  
    if (ChiffreDizaine == 1):
        NbDeUn = NbDeUn + 1            |\pause|   
    if (ChiffreCentaine == 1):
        NbDeUn = NbDeUn + 1            |\pause|  
print("Nombre d'occurences du chiffre '1' :", NbDeUn)
\end{lstlisting}  
\end{algo}

\end{frame}

%%%%%%%%%%%%%%%%%%%%%%%%%%%%%%%%%%%%%%%%%%%%%%%%%%%%%%%%%%%%%%%%
\section{\'Ecriture des nombres en base $10$}

\begin{frame}
\begin{itemize}
  \item \'Ecriture décimale : associer à un entier $N$ la suite de ses chiffres $[a_0,a_1,\ldots,a_n]$
\pause  
  \item $N= a_n 10^n+ a_{n-1}10^{n-1}+\cdots + a_2 10^2 + a_1 10 + a_0, \ a_i \in \{0,1,\ldots,9\}$
\pause  
  \item $a_0$ est le chiffre des unités, $a_1$ celui des dizaines,...
\end{itemize}

\pause

\begin{tp}
\begin{enumerate}
  \item \'Ecrire une fonction qui à partir d'une liste $[a_0,a_1,\ldots,a_n]$ calcule l'entier $N$ correspondant.
  \item Pour un entier $N$ fixé, combien a-t-il de chiffres ? 
  On pourra s'aider d'une inégalité du type $10^n \le N < 10^{n+1}$.
  \item \'Ecrire une fonction qui à partir de $N$ calcule son écriture décimale $[a_0,a_1,\ldots,a_n]$.
\end{enumerate}
\end{tp}
\end{frame}


\begin{frame}[fragile]

\begin{algo}[decimale.py (1)]
\begin{lstlisting}
def chiffres_vers_entier(tab):
    N = 0
    for i in range(len(tab)):
        N = N + tab[i] * (10 ** i)
    return N
\end{lstlisting}  
\end{algo}

\pause
\bigskip

\centerline{$N= a_n 10^n+ a_{n-1}10^{n-1}+\cdots + a_2 10^2 + a_1 10 + a_0$}

\pause
\bigskip

\centerline{\codeinline{chiffres_vers_entier([4,3,2,1])} renvoie $1234$}

\end{frame}


\begin{frame}
\begin{itemize}
  \item Une \evidence{liste} est présentée entre des crochets
\pause
  \begin{itemize}
    \item Exemple : \codeinline{tab = [4,3,2,1]}
\pause
    \item On accède aux valeurs par \codeinline{tab[i]}
\pause
    \item \codeinline{tab[0]} vaut $4$,  \codeinline{tab[1]} vaut $3$,   \codeinline{tab[2]} vaut $2$,  \codeinline{tab[3]} vaut $1$
  \end{itemize}
\pause

  \item Pour parcourir les éléments d'une liste \codeinline{for x in tab}
\pause  
  \item Longueur d'une liste  \codeinline{len(tab)} (\codeinline{len([4,3,2,1])} vaut $4$)
\pause
  \item Parcourir toutes les valeurs d'une liste 
  \begin{itemize}
    \item \codeinline{for i in range(len(tab))}
    \item puis utiliser \codeinline{tab[i]} (ici $i$ variant ici de $0$ à $3$)
  \end{itemize}

\pause  
  \item La liste vide  \codeinline{[]}
\pause  
  \item Ajouter un élément à une liste
  \begin{itemize}
    \item Exemple : \codeinline{tab=[]}
\pause
    \item  \codeinline{tab.append(4)} \quad  la liste \codeinline{tab} devient $[4]$
\pause
    \item \codeinline{tab.append(3)}  \quad  la liste \codeinline{tab} devient $[4,3]$
  \end{itemize}
\end{itemize}
\end{frame}


\begin{frame}[fragile]

\begin{algo}[decimale.py (2)]
\small
\begin{lstlisting}
def entier_vers_chiffres(N):
    tab = []
    n = floor(log(N,10))    # le nombre de chiffres est n+1 
    for i in range(0,n+1):
        tab.append((N // 10 ** i) % 10)
    return tab
\end{lstlisting}  
\end{algo}

\pause

\begin{itemize}
  \item Exemple : \codeinline{entier_vers_chiffres(1234)} renvoie $[4,3,2,1]$
\pause
  \item 
  \begin{itemize}
    \item $\Nn^* =  [1,10[  \ \cup  \ [10,100[ \ \cup \  [100,1000[ \  \cup \  [1\,000,10\, 000[ \ \cup \cdots$ 
\pause
    \item Chaque intervalle est du type $[10^n,10^{n+1}[$
\pause
    \item Pour $N \in \Nn^*$ il existe donc $n\in\Nn$ tel que $10^n \le N < 10^{n+1}$
\pause
    \item Le nombre de chiffres de $N$ est $n+1$
\pause
    \item Ex : $N=1234$ donc $1 \, 000 = 10^3 \le N < 10^4 = 10\, 000$ ainsi $n=3$
  \end{itemize}
\pause
  \item Comment calculer $n$ à partir de $N$ ? 
  \begin{itemize}
\pause
    \item Logarithme décimal $\log_{10}$ :  $\log_{10}(10) = 1$ et $\log_{10}(10^i) = i$
\pause
    \item $10^n \le N < 10^{n+1}$ implique $\log_{10}(10^n) \le \log_{10}(N) < \log_{10}(10^{n+1})$
\pause
    \item $n \le \log_{10}(N) < n+1$ donc \myboxinline{$n = E(\log_{10}(N))$}
  \end{itemize}
\end{itemize}

\end{frame}






%%%%%%%%%%%%%%%%%%%%%%%%%%%%%%%%%%%%%%%%%%%%%%%%%%%%%%%%%%%%%%%%
\section{Module \codeinline{math}}

\begin{frame}
\begin{itemize}
  \item \codeinline{import math} : \codeinline{math.cos(3.14)}
\pause
  \item \codeinline{from math import *} : \codeinline{cos(3.14)}
\end{itemize}

\pause
\vspace*{-3ex}
\small
\begin{center}
\setlength{\arrayrulewidth}{0.05mm}
%\begin{tabular}{|l|l|l|} \hline
\begin{tabular}[t]{cc@{\vrule depth 1.2ex height 3ex width 0mm \ }} 
\hline \hline 
   \codeinline{abs(x)}     &   $|x|$      \\ \hline
   \codeinline{x ** n}     &   $x^n$      \\ \hline
   \codeinline{sqrt(x)}    &  $\sqrt{x}$ \\ \hline   
   \codeinline{exp(x)}     & $\exp x$    \\ \hline   
   \codeinline{log(x)}     & $\ln x$ logarithme népérien \\ \hline
   \codeinline{log(x,10)}  & $\log x$ logarithme décimal \\ \hline
   \codeinline{cos(x), sin(x), tan(x)}  & $\cos x$, $\sin x$, $\tan x$\\ \hline
   \codeinline{acos(x), asin(x), atan(x)}  & $\arccos x$, $\arcsin x$, $\arctan x$\\ \hline      
   \codeinline{floor(x)}  & partie entière $E(x)$ \\ \hline
   \codeinline{ceil(x)}   & plus petit entier $n \ge x$  \\ \hline \hline     
\end{tabular} 
\end{center}
\end{frame}


%%%%%%%%%%%%%%%%%%%%%%%%%%%%%%%%%%%%%%%%%%%%%%%%%%%%%%%%%%%%%%%%
\section{\'Ecriture des nombres en base $2$}

\begin{frame}

\myfigure{1}{
\tikzinput{fig_algo_pres01} 
}

\pause\pause\pause\pause\pause\pause\pause\pause\pause\pause

\vspace*{-1ex}
\pause

\begin{itemize}
  \item L'\defi{écriture binaire} d'un nombre c'est son écriture en base $2$
\pause
  \item 
  \begin{itemize}
    \item Le chiffre des ``dizaines'' correspond à $2$ (au lieu de $10$)
    \item Le chiffre des ``centaines'' à $4=2^2$ (au lieu de $100=10^2$)
    \item Le chiffres des ``milliers'' à $8=2^3$ (au lieu de $1000=10^3$)
    \item Pour le chiffre des unités cela correspond à $2^0 = 1$  
  \end{itemize}

\pause  

  \item ${\color{blue}10011}_b 
  \pause 
  = {\color{blue}1} \cdot 2^4 + {\color{blue}0} \cdot 2^3 + 
  {\color{blue}0} \cdot 2^2 + {\color{blue}1}\cdot 2^1 + {\color{blue}1}\cdot 2^0 
  \pause
  = 16+2+1
  \pause
  =19$
 
\pause
  \item Tout entier $N\in \Nn$ s'écrit de manière unique 
$N= a_n 2^n+ a_{n-1}2^{n-1}+\cdots + a_2 2^2 + a_1 2 + a_0 \quad \text{ et } \quad a_i \in \{0,1\}$
\pause
  \item $N= a_n a_{n-1}\ldots a_1 a_0 \ _b$

\end{itemize}
\end{frame}


\begin{frame}[fragile]
\begin{tp}
\begin{enumerate}
  \item \'Ecrire une fonction qui à partir d'une liste $[a_0,a_1,\ldots,a_n]$ calcule l'entier $N$ correspondant à l'écriture binaire 
  $a_n a_{n-1}\ldots a_1 a_0 \ _b$.
  \item \'Ecrire une fonction qui à partir de $N$ calcule son écriture binaire sous la forme $[a_0,a_1,\ldots,a_n]$.
\end{enumerate}
\end{tp}

\pause
\small

\begin{algo}[binaire.py (1) \& (2)]
\vspace*{-1ex}
\begin{lstlisting}
def binaire_vers_entier(tab):
    N = 0
    for i in range(len(tab)):
        N = N + tab[i] * (2 ** i)
    return N   |\pause\vspace*{0.5ex}|
def entier_vers_binaire(N):
    tab = []
    n = floor(log(N,2))       # le nombre de chiffres est n+1
    for i in range(0,n+1):
        tab.append((N // 2 ** i) % 2)
    return tab
\end{lstlisting} 
\vspace*{-1ex}
\end{algo}

\end{frame}


\begin{frame}

\myfigure{0.8}{
\tikzinput{fig_algo02} 
}

\pause

\vspace*{-2ex}
\centerline{Codage  $[1,0,0,1,0,1,1,1]$ \pause  nombre binaire $11101001_b = 233$}

\pause

\footnotesize
\begin{tp}

\begin{enumerate}
\setlength{\itemsep}{0pt}
  \item Faire une boucle qui affiche toutes les combinaisons possibles (pour une taille de rampe donnée).
  
  \item Quelle opération mathématique élémentaire transforme un nombre binaire 
 $a_n \ldots a_1 a_0 \ _b$ en $a_n \ldots a_1 a_0 0\ _b$ (décalage vers la gauche et ajout d'un $0$ à la fin) ?

 \item Soit $N' = a_n a_{n-1} \ldots a_1 a_0 0 \ _b$ (une écriture avec $n+2$ chiffres). 
 Quelle est l'écriture binaire de $N' \pmod {2^{n+1}}$ ? (C'est une écriture avec $n+1$ chiffres.)
 
 \item En déduire un algorithme qui pour une configuration donnée de la rampe, fait permuter cycliquement 
 (vers la droite) cette configuration. Par exemple $[1,0,1,0,1,1,1,0]$ devient $[0,1,0,1,0,1,1,1]$.
 
 \item Quelle opération mathématique élémentaire permet de passer d'une configuration à son opposée (une lampe éteinte s'allume,
 et réciproquement). Par exemple si la configuration était $[1,0,1,0,1,1,1,0]$ alors on veut $[0,1,0,1,0,0,0,1]$.
 (Indication : sur cet exemple calculer les deux nombres correspondants et trouver la relation qui les lie.)
\end{enumerate}
\end{tp}  
\end{frame}


\begin{frame}[fragile]

\begin{itemize}
  \item Avec $4$ lampes les configurations sont $[0,0,0,0]$, $[1,0,0,0]$, $[0,1,0,0]$, $[1,1,0,0]$,\ldots, $[1,1,1,1]$

\pause

  \item 
  \begin{itemize}
    \item Pour chaque lampe nous avons deux choix (allumé ou éteint)

\pause

    \item Il y a $n+1$ lampes donc $2^{n+1}$ configurations

\pause

    \item Correspond à l'énumération $0,1,2,3,\ldots, 2^{n+1}-1$
  \end{itemize}
\end{itemize}

\pause

\begin{algo}[binaire.py (3)]
\begin{lstlisting}
def configurations(n):
    for N in range(2**(n+1)):
        print(entier_vers_binaire_bis(N,n))
\end{lstlisting}  
\end{algo}

\pause

\begin{itemize}
  \item \codeinline{entier_vers_binaire_bis(N,n)} est similaire à \codeinline{entier_vers_binaire(N)},
mais affiche les zéros non significatifs

\pause

  \item Exemple $7$ en binaire s'écrit $111_b$ mais sur $8$ chiffres c'est $00000111_b$
\end{itemize}
\end{frame}



\begin{frame}[fragile]

\begin{itemize}
  \item En écriture décimale, multiplier par $10$ revient à décaler le nombre initial et rajouter un zéro.
Par exemple $10\times{\color{blue}19} = {\color{blue}19}{\color{red}0}$

\pause

  \item Multiplier un nombre par $2$ revient sur l'écriture à un décalage 
vers la gauche et ajout d'un zéro sur le chiffre des unités

\pause

Exemple : $19 = 10011_b$, $2 \times{\color{blue}10011}_b  = {\color{blue}10011}{\color{red}0}_b$ : $2 \times 19 = 38$ 

\pause

  \item 
  \begin{itemize}
    \item $N=a_n  a_{n-1}\ldots a_1 a_0 \ _b$
\pause      
    \item $N'=2N$ s'écrit $N' = a_n  a_{n-1}\ldots a_1 a_0 0 \ _b$
 \pause  
    \item $N' \pmod {2^{n+1}}$ s'écrit exactement $a_{n-1} a_{n-2} \ldots a_1 a_0 0 \ _b$
\pause
    \item  on ajoute $a_n$ qui est le quotient de $N'$ par $2^{n+1}$
\pause
    \item {\footnotesize $N'\pmod{2^{n+1}}+a_n = a_{n-1} \cdot 2^n+\cdots + a_0 \cdot 2 +a_n 
    \pause
    =a_{n-1} a_{n-2} \ldots a_1 a_0 a_n \ _b$}
  \end{itemize}

\end{itemize}

\pause

\begin{algo}[binaire.py (4)]
\begin{lstlisting}
def decalage(tab):
    N = binaire_vers_entier(tab)
    n = len(tab)-1   # le nombre de chiffres est n+1
    NN = 2*N % 2**(n+1) + 2*N // 2**(n+1)
    return entier_vers_binaire_bis(NN,n)
\end{lstlisting}  
\end{algo}

\end{frame}



\begin{frame}[fragile]

\begin{itemize}
  \item Si l'on a deux configurations opposées alors leur somme vaut $2^{n+1}-1$

\pause

  \item 
  \begin{itemize}
    \item Exemple : $[1,0,0,1,0,1,1,1]$ et $[0,1,1,0,1,0,0,0]$
\pause    
    \item $N = 11101001_b$ et $N' = 00010110_b$
\pause    
    \item $N + N' = 11101001_b + 00010110_b = 11111111_b \pause = 2^{8}-1$
  \end{itemize}

\pause
  \item $N' = 2^{n+1}-1 - N$
\end{itemize}
\pause 
\begin{algo}[binaire.py (5)]
\begin{lstlisting}
def inversion(tab):
    N = binaire_vers_entier(tab)
    n = len(tab)-1   # le nombre de chiffres est n+1
    NN = 2**(n+1)-1 - N
    return entier_vers_binaire_bis(NN,n)
\end{lstlisting}  
\end{algo}

\end{frame}

  

%%%%%%%%%%%%%%%%%%%%%%%%%%%%%%%%%%%%%%%%%%%%%%%%%%%%%%%%%%%%%%%%
\section{Mini-exercices}

\begin{frame}
\footnotesize
\vspace*{-1ex}
\begin{miniexercice}
\vspace*{-1ex}
\begin{enumerate}
\setlength{\itemsep}{-1pt}
  \item Pour un entier $n$ fixé, combien y-a-t-il d'occurrences du chiffre $1$ dans
  l'écriture des nombres de $1$ à $n$ ?
  
  \item \'Ecrire une fonction qui calcule l'écriture décimale d'un entier, sans recourir au $\log$
  (une boucle \codeinline{while} est la bienvenue).
  
  \item \'Ecrire un algorithme qui permute une configuration de rampe vers la droite.
  
  \item On dispose de $n+1$ lampes, chaque lampe peut s'éclairer de trois couleurs : vert, orange, rouge (dans cet ordre).
  Trouver toutes les combinaisons possibles. Comment passer toutes les lampes à la couleur suivante ?
  
  \item Générer toutes les matrices $4\times 4$ n'ayant que des $0$ et des $1$ comme coefficients. 
  On codera une matrice sous la forme de lignes $[ [1,1,0,1], [0,0,1,0], [1,1,1,1], [0,1,0,1] ]$.

  \item On part du point $(0,0) \in \Zz^2$. A chaque pas on choisit au hasard un direction Nord, Sud, Est, Ouest.
  Si on va au Nord alors on ajoute $(0,1)$ à sa position (pour Sud on ajoute $(0,-1)$ ; pour Est $(1,0)$ ; pour Ouest $(-1,0)$).
  Pour un chemin d'une longueur fixée de $n$ pas, coder tous les chemins possibles. Caractériser les chemins qui repassent par l'origine.
  Calculer la probabilité $p_n$ de repasser par l'origine. Que se passe-t-il lorsque $n \to +\infty$ ?
  
  \item \'Ecrire une fonction, qui pour un entier $N$, affiche son écriture en chiffres romains :
  $M = 1000$, $D=500$, $C=100$, $X=10$, $V=5$, $I=1$. Il ne peut y avoir plus de trois symboles identiques à suivre.
 
\end{enumerate}
\vspace*{-2ex}
\end{miniexercice}

\end{frame}

\end{document}